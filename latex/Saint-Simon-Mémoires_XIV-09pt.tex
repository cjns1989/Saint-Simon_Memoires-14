\PassOptionsToPackage{unicode=true}{hyperref} % options for packages loaded elsewhere
\PassOptionsToPackage{hyphens}{url}
%
\documentclass[oneside,9pt,french,]{extbook} % cjns1989 - 27112019 - added the oneside option: so that the text jumps left & right when reading on a tablet/ereader
\usepackage{lmodern}
\usepackage{amssymb,amsmath}
\usepackage{ifxetex,ifluatex}
\usepackage{fixltx2e} % provides \textsubscript
\ifnum 0\ifxetex 1\fi\ifluatex 1\fi=0 % if pdftex
  \usepackage[T1]{fontenc}
  \usepackage[utf8]{inputenc}
  \usepackage{textcomp} % provides euro and other symbols
\else % if luatex or xelatex
  \usepackage{unicode-math}
  \defaultfontfeatures{Ligatures=TeX,Scale=MatchLowercase}
%   \setmainfont[]{EBGaramond-Regular}
    \setmainfont[Numbers={OldStyle,Proportional}]{EBGaramond-Regular}      % cjns1989 - 20191129 - old style numbers 
\fi
% use upquote if available, for straight quotes in verbatim environments
\IfFileExists{upquote.sty}{\usepackage{upquote}}{}
% use microtype if available
\IfFileExists{microtype.sty}{%
\usepackage[]{microtype}
\UseMicrotypeSet[protrusion]{basicmath} % disable protrusion for tt fonts
}{}
\usepackage{hyperref}
\hypersetup{
            pdftitle={SAINT-SIMON},
            pdfauthor={Mémoires XIV},
            pdfborder={0 0 0},
            breaklinks=true}
\urlstyle{same}  % don't use monospace font for urls
\usepackage[papersize={4.80 in, 6.40  in},left=.5 in,right=.5 in]{geometry}
\setlength{\emergencystretch}{3em}  % prevent overfull lines
\providecommand{\tightlist}{%
  \setlength{\itemsep}{0pt}\setlength{\parskip}{0pt}}
\setcounter{secnumdepth}{0}

% set default figure placement to htbp
\makeatletter
\def\fps@figure{htbp}
\makeatother

\usepackage{ragged2e}
\usepackage{epigraph}
\renewcommand{\textflush}{flushepinormal}

\usepackage{indentfirst}
\usepackage{relsize}

\usepackage{fancyhdr}
\pagestyle{fancy}
\fancyhf{}
\fancyhead[R]{\thepage}
\renewcommand{\headrulewidth}{0pt}
\usepackage{quoting}
\usepackage{ragged2e}

\newlength\mylen
\settowidth\mylen{...................}

\usepackage{stackengine}
\usepackage{graphicx}
\def\asterism{\par\vspace{1em}{\centering\scalebox{.9}{%
  \stackon[-0.6pt]{\bfseries*~*}{\bfseries*}}\par}\vspace{.8em}\par}

\usepackage{titlesec}
\titleformat{\chapter}[display]
  {\normalfont\bfseries\filcenter}{}{0pt}{\Large}
\titleformat{\section}[display]
  {\normalfont\bfseries\filcenter}{}{0pt}{\Large}
\titleformat{\subsection}[display]
  {\normalfont\bfseries\filcenter}{}{0pt}{\Large}

\setcounter{secnumdepth}{1}
\ifnum 0\ifxetex 1\fi\ifluatex 1\fi=0 % if pdftex
  \usepackage[shorthands=off,main=french]{babel}
\else
  % load polyglossia as late as possible as it *could* call bidi if RTL lang (e.g. Hebrew or Arabic)
%   \usepackage{polyglossia}
%   \setmainlanguage[]{french}
%   \usepackage[french]{babel} % cjns1989 - 1.43 version of polyglossia on this system does not allow disabling the autospacing feature
\fi

\title{SAINT-SIMON}
\author{Mémoires XIV}
\date{}

\begin{document}
\maketitle

\hypertarget{chapitre-premier.}{%
\chapter{CHAPITRE PREMIER.}\label{chapitre-premier.}}

1716

~

{\textsc{Assemblées d'huguenots dissipées.}} {\textsc{- Le régent, tenté
de les rappeler, me le propose.}} {\textsc{- Aveuglement du régent sur
l'Angleterre.}} {\textsc{- Je détourne le régent de rappeler les
huguenots.}} {\textsc{- Mort de Bréauté, dernier de son nom.}}
{\textsc{- Mort de Connelaye, de Chalmazel et de Greder.}} {\textsc{-
Mort de l'archevêque de Tours\,; sa naissance et son mérite.}}
{\textsc{- Mort de La Porte, premier président du parlement de Metz, à
qui Chaseaux succède.}} {\textsc{- Anecdote curieuse sur
M\textsuperscript{lle} de Chausseraye.}} {\textsc{- Mort de Cani.}}
{\textsc{- Sa charge de grand maréchal des logis et son brevet de
retenue donnés à son fils enfant.}} {\textsc{- Mort de la duchesse de La
Feuillade.}} {\textsc{- Mort de la jeune Castries et de son mari.}}
{\textsc{- Mort d'une bâtarde non reconnue de Monseigneur.}} {\textsc{-
Mariage du comte de Croï avec M\textsuperscript{lle} de Milandon.}}
{\textsc{- Hardies prétentions de cette veuve.}} {\textsc{- Mariages de
Rothelin avec M\textsuperscript{lle} de Clèves.}} {\textsc{- Le
parlement continue à s'opposer au rétablissement de la charge des postes
et de celle des bâtiments.}} {\textsc{- Motifs de sa conduite et ses
appuis.}} {\textsc{- Il dispute la préséance au régent à la procession
de l'Assomption, et l'empêche de s'y trouver.}} {\textsc{- Audace de
cette prétention, qui se détruit d'elle-même par droit et par faits
expliqués même à l'égard de seigneurs particuliers.}} {\textsc{- Comment
le terme de gentilshommes doit être pris.}} {\textsc{- Conduite du
régent avec le parlement, du parlement avec lui, et la mienne avec ce
prince à l'égard du parlement.}} {\textsc{- Pension de six mille livres
donnée à Maisons, et un régiment de dragons à Rion.}} {\textsc{-
Pensions dites de Pontoise, dont une donnée au président Aligre.}}

~

Les huguenots, dont il était demeuré ou rentré beaucoup dans le royaume,
la plupart sous de feintes abjurations, profitaient d'un temps qui se
pouvait appeler de liberté en comparaison de celui du feu roi. Ils
s'assemblaient clandestinement d'abord et en petit nombre\,; ils prirent
courage après sur le peu de cas qu'on on fit, et bientôt on eut des
nouvelles d'assemblées considérables en Poitou, Saintonge, Guyenne et
Languedoc. On marcha même à une fort nombreuse en Guyenne, où un
prédicant faisait en pleine campagne des exhortations fort vives. Ils
n'étaient point armés et se dissipèrent d'abord\,; mais on trouva tout
près du lieu où ils s'étaient assemblés deux charrettes toutes chargées
de fusils, de baïonnettes et de pistolets. Il y eut aussi de petites
assemblées nocturnes vers les bouts du faubourg Saint-Antoine.

Le régent m'en parla, et à ce propos de toutes les contradictions et de
toutes les difficultés dont les édits et déclarations du feu roi sur les
huguenots étaient remplis, sur lesquels on ne pouvait statuer par
impossibilité de les concilier, et d'autre part de les exécuter à
l'égard de leurs mariages, testaments, etc. J'étais souvent témoin de
cette vérité au conseil de régence, tant par les procès qui y étaient
évoqués, parce qu'il n'y avait que le roi qui pût s'interpréter soi-même
dans ces diverses contradictions, que par les consultations des divers
tribunaux au chancelier sur ces matières, qu'il rapportait au conseil de
régence pour y statuer. De la plainte de ces embarras, le régent vint à
celle de la cruauté avec laquelle le feu roi avait traité les huguenots,
à la faute même de la révocation de l'édit de Nantes, au préjudice
immense que l'État en avait souffert et en souffrait encore dans sa
dépopulation, dans son commerce, dans la haine que ce traitement avait
allumée chez tons les protestants de l'Europe. J'abrège une longue
conversation où jusque-là je n'eus rien à contredire. Après bien du
raisonnement très solide et très vrai, tant sur le mal en soi que sut la
manière douce et sûre d'éteindre peu à peu le protestantisme en gagnant
les ministres, en ôtant tout exercice de cette religion, en excluant de
fait de tout emploi quel qu'il fût les huguenots, le régent se mit sur
les réflexions de l'État ruiné où le roi avait réduit et laissé la
France, et de là sur celle du gain de peuple, d'arts, d'argent et de
commerce qu'elle ferait en un moment par le rappel si désiré des
huguenots dans leur patrie, et finalement me le proposa. Je ne veux
accuser personne d'avoir suggéré au régent une telle pensée, parce que
je n'ai jamais su de qui elle lui était venue\,; mais dans l'extrême
désir où il n'avait cessé d'être de s'allier étroitement avec la
Hollande, surtout avec l'Angleterre, depuis qu'il était possédé par le
duc de Noailles, Canillac et l'abbé Dubois, et où il était plus que
jamais, les soupçons ne sont pas difficiles. Il croyait par ce rappel
flatter les puissances maritimes, leur donner la plus grande marque
d'estime, d'amitié, de complaisance et de condescendance, tout cela paré
de la persuasion de ranimer, d'enrichir, de faire refleurir le royaume
en un instant.

Stairs, conduit et appuyé de trois si bons seconds, avait eu l'adresse
de voiler au régent ce qui ne l'était à personne, ni à lui-même, quand
il y voulait faire réflexion, et de l'intimider sur les grands coups que
l'Angleterre alliée, comme il le disait, pouvait faire à tout moment
pour ou contre la France, et en particulier pour ou contre lui. Pour peu
qu'on fût instruit de la situation intérieure de l'Angleterre travaillée
de toute espèce de divisions et de fermentations, du mépris général du
gouvernement, du nombre infini de mécontents, de la jalousie de commerce
et de puissance delà les grandes mers, qui ne laissait que de beaux
dehors entre la Hollande et l'Angleterre, de tout ce que notre union
avec l'Espagne eût encore pu y influer à l'avantage commun des deux
couronnes, la sujétion, les embarras, le malaise où les affaires du
nord, les usurpations sur la Suède et tant, d'autres choses qui y
étaient relatives, tenaient le roi Georges par rapport à ses alliés du
nord et à l'empereur, on voyait à plein que la France n'avait rien à
craindre d'elle, aussi peu à en espérer\,; qu'au contraire c'était
l'Angleterre qui avait tout à craindre de la France, au dedans
d'elle-même et au dehors, et que le régent, s'il eût voulu, aurait pu y
allumer un embrasement de longues années, dont la France aurait
infiniment pu profiter en Europe et dans le nouveau monde, ou faire
naître une révolution qui aurait aussi eu ses avantages pour elle, en
opérant le renvoi de la maison d'Hanovre en Allemagne, d'où il ne lui
aurait pas été aisé de remonter sur le trône dont les Anglais eux-mêmes
l'auraient fait descendre. Une telle méprise dans un prince d'ailleurs
si éclairé me faisait gémir sans cesse sur l'État et sur lui, et
chercher souvent et toujours inutilement à lui dessiller les yeux sur
une duperie si grossière et si importante. Je lui avais plusieurs fois
tiré de l'argent pour le Prétendant à l'insu de tous ses ministres\,; je
ne m'étais pas tenu sur l'infâme affaire de Nonancourt, sur les allures
de Stairs, ni sur le malheur du mauvais succès d'Écosse. Il me croyait
trop jacobite\,; il se persuadait que ma haine pour Noailles et mon
éloignement de Canillac m'en donnait pour les Anglais qu'ils
portaient\,; et la défiance de ce prince, qui n'épargnait pas même ses
plus réitérées expériences, et qui gâtait tout, presque autant que sa
faiblesse et sa facilité ôtait toute la force à l'évidence de mes
raisons.

Je fus heureux à l'égard des huguenots. Je sentis à la préface qu'il
employa, et dont je viens de parler, que son désir était grand, mais
qu'il comprenait le poids et les suites d'une telle résolution, à
laquelle il cherchait des approbateurs, je n'ose dire des appuis. Je
profitai sur-le-champ de cette heureuse et sage timidité, et je lui dis
que, faisant abstraction de ce que la religion dictait là-dessus, je me
contenterais de lui parler un langage qui lui serait plus propre. Je lui
représentai les désordres et les guerres civiles dont les huguenots
avaient été cause en France depuis Henri II jusqu'à Louis XIII\,;
combien de ruines et de sang répandu\,; qu'à leur ombre la Ligue s'était
formée, qui avait été si près d'arracher la couronne à Henri IV\,; et
tout ce qu'il en avait coûté en tout genre aux rois et à l'État, et pour
les huguenots et pour les Ligueurs, les uns et les autres appuyés des
puissances étrangères, desquelles il fallait tout souffrir, tandis
qu'elles nous méprisaient, et savaient profiter de nos misères, au point
que Henri IV n'a dû sa couronne qu'au nombre de ceux qui prétendaient
l'emporter chacun pour soi\,: le duc de Guise, le fils du duc de
Mayenne, le marquis du Pont \footnote{Fils du duc de Lorraine et de
  Claude de France.}, l'infante fille de Philippe II, et jusqu'au duc
Charles-Emmanuel de Savoie, et ensuite à sa valeur et à sa noblesse. Je
lui fis sentir ce que c'était, dans les temps les moins tumultueux et
les plus supportables, que des sujets qui, en changeant de religion, se
donnaient le droit de ne l'être qu'en partie, d'avoir des places de
sûreté, des garnisons, des troupes, des subsides\,; un gouvernement
particulier, organisé, républicain\,; des privilèges, des cours de
justice\footnote{Ces cours de justice s'appelaient \emph{chambres de
  l'édit}, en souvenir le l'édit de Nantes qui les avait établies.}
érigées exprès pour leurs affaires, même avec les catholiques\,; une
société de laquelle tous ses membres dépendaient\,; des chefs élus par
eux, des correspondances étrangères, des députés à la cour sous la
protection du droit des gens\,; en un mot, un État dans un État, et qui
ne dépendaient du souverain que pour la forme, et autant ou si peu que,
bon leur semblait\,; toujours en plaintes et prêts à reprendre les
armes, et les reprenant toujours très dangereusement pour l'État.

Je lui remis levant les yeux toutes les peines qu'ils avaient données à
Henri IV dans ses années les plus florissantes, et après l'édit de
Nantes, et les inquiétudes que lui avait causées jusqu'à sa mort
l'ingratitude et l'ambition du maréchal de Bouillon, depuis qu'il lui
eut deux fois procuré Sedan, qui machina sans cesse contre lui et contre
Louis XIII, et dont le but était de se faire le chef des huguenots de
France sous la protection déclarée d'une puissance étrangère, à quoi, au
moins pour le nom et le commandement militaire, le duc de Rohan parvint
depuis. Je lui retraçai les travaux héroïques du roi son grand-père, qui
abattit enfin cette hydre à force de courage, et qui a mis le feu roi en
état de s'en délivrer tout à fait et pour jamais, sans autre combat que
l'exécution tranquille de ses volontés, qui n'ont pu trouver la moindre
résistance. Je priai le régent de réfléchir qu'il jouissait maintenant
du bénéfice d'un si grand repos domestique, que c'était à lui à le
comparer avec tout ce que je venais de lui retracer\,; que c'était de
cette douce et paisible position qu'il fallait partir pour raisonner
utilement sur une affaire, ou plutôt pour être convaincu qu'il n'était
pas besoin d'en raisonner, ni de balancer s'il fallait faire ou non,
dans un temps de paix où nulle puissance ne demandait rien là-dessus, ce
que le feu roi avait eu le courage et la force de rejeter avec
indignation, quoi qu'il en pût arriver, quand épuisé de blés, d'argent,
de ressources et presque de troupes, ses frontières conquises et
ouvertes, et à la veille des plus calamiteuses extrémités, ses nombreux
ennemis voulurent exiger le retour des huguenots en France comme l'une
des conditions sans laquelle ils ne voulaient point mettre de bornes à
leurs conquêtes ni à leurs prétentions, pour finir une guerre que ce
monarque n'avait plus aucun moyen de soutenir.

Je fis après sentir au régent un autre danger de ce rappel. C'est
qu'après la triste et cruelle expérience que les huguenots avaient faite
de l'abattement de leur puissance par Louis XIII, de la révocation de
l'édit de Nantes par le feu roi, et des rigoureux traitements qui
l'avaient suivie et qui duraient encore, il ne fallait pas s'attendre
qu'ils s'exposassent à revenir en France sans de fortes et d'assurées
précautions, qui ne pouvaient être que les mêmes sous lesquelles ils
avaient fait gémir cinq de nos rois, et plus grandes encore,
puisqu'elles n'avaient pu empêcher le cinquième de les assujettir enfin,
et de les livrer pieds et poings liés à la volonté de son successeur,
qui les avait confisqués, chassés, expatriés. Je finis par supplier le
régent de peser l'avantage qu'il se représentait de ce retour, avec les
désavantages et les dangers infinis dont il était impossible qu'il ne
fût pas accompagné\,; que ces hommes, cet argent, ce commerce, dont il
croyait en accroître au royaume, seraient hommes, argent, commerce
ennemis et contre le royaume\,; et que la complaisance et le gré qu'en
sentiraient les puissances maritimes et les autres protestantes, serait
uniquement de la faute incomparable et irréparable qui les rendrait pour
toujours arbitres et maîtres du sort et de la conduite de la France au
dedans et au dehors. Je conclus que, puisque le feu roi avait fait la
faute beaucoup plus dans la manière de l'exécution que dans la chose
même, il y avait plus de trente ans, et que l'Europe y était maintenant
accoutumée et les protestants hors de toute raisonnable espérance
là-dessus, depuis le refus du feu roi dans la plus pressante extrémité
de ses affaires de rien écouter là-dessus, il fallait au moins savoir
profiter du calme, de la paix, de la tranquillité intérieure qui en
était le fruit, et {[}ne pas{]} de gaieté de coeur et moins encore dans
un temps de régence, se rembarquer dans les malheurs certains et sans
ressource qui avaient mis la France sens dessus dessous, et qui
plusieurs fois l'avaient pensé renverser depuis la mort d'Henri II
jusqu'à l'édit de Nantes, et qui l'avaient toujours très dangereusement
troublée depuis cet édit jusqu'à la fin des triomphes de Louis XIII à la
Rochelle et en Languedoc. À tant et de si fortes raisons le régent n'en
eut aucunes à opposer qui pussent les balancer en aucune sorte. La
conversation ne laissa pas de durer encore\,; mais depuis ce jour-là il
ne fut plus question de songer à rappeler les huguenots, ni de se
départir de l'observation de ce que le feu roi avait statué à leur
égard, autant que les contradictions et quelques impossibilités
effectives de la lettre de ces diverses ordonnances on rendirent
l'exécution possible.

Bréauté mourut jeune et sans alliance, en qui finit une dos meilleures
maisons de Normandie. Il était fils du cousin germain du gros Bréauté,
mort en 1708, dont j'ai parlé en son temps, que j'avais fort connu à
l'hôtel de Lorges, lequel était fils du frère cadet de Pierre de
Bréauté, qui se rendit célèbre avant l'âge de vingt ans, par son combat
de vingt-deux contre vingt-deux, sous Bois-le-Duc, où il acquit tant de
gloire, et ses ennemis tant de honte par leurs supercheries, que
Grobendunck, gouverneur de Bois-le-Duc, couronna en le faisant
assassiner entre les portes de sa place en 1600. Le père de Bréauté, de
la mort duquel je parle, était mort assez jeune, en 1711, maître de la
garde-robe de M. le duc d'Orléans, dont je fis donner la charge à son
fils.

La Connelaye et Chalmazel moururent en ce même temps, tous deux
lieutenants généraux qui s'étaient fort distingués. L'un avait été
capitaine aux gardes, et fort du grand monde\,; il était gouverneur de
Belle-Ile\,; l'autre avait commandé le régiment de Picardie avec grande
estime et considération\,; c'était la douceur et la vertu même. Il était
fort vieux, et avait le commandement de Toulon. Chalmazel, premier
maître d'hôtel de la reine, est son neveu. Des Fourneaux, homme de
fortune, mais de valeur et de mérite, officier général et lieutenant des
gardes du corps, eut le gouvernement de Belle-Ile. Greder, lieutenant
général fort estimé, mourut aux eaux de Bourbonne. Il avait un régiment
allemand qui lui valait beaucoup, et qui fut donné au neveu du baron
Spaar, qui avait longtemps servi en France, qui y fut depuis ambassadeur
de Suède, et qui y est mort sénateur, toujours le coeur français, un des
plus galants hommes et des mieux faits qu'on pût voir, avec l'air le
plus doux et le plus militaire.

L'archevêque de Tours mourut aussi à Paris, où les affaires de la
constitution l'avaient retenu malgré lui. Il était un des prélats de
France les plus estimés pour son savoir, sa vertu, sa résidence et son
application épiscopale. Il avait été longtemps auditeur de rote avec
beaucoup de réputation, et connaissait parfaitement la cour de Rome.
C'était un homme doux et d'esprit, fort attaché aux libertés de l'Église
gallicane, étroitement uni au cardinal de Noailles dans l'affaire de la
bulle, qui y perdit un excellent conseil et un ferme appui, en un mot un
vrai gentilhomme de bien et d'honneur, et un excellent et courageux
évêque. Il s'appelait Isoré d'Hervault, de maison ancienne et bien
alliée, et qui avait eu en divers temps des emplois distingués. Il était
issu de germain du duc de Beauvilliers, qui, malgré la différence des
sentiments, en faisait grand cas et l'aimait fort.

La Porte, premier président du parlement de Metz, mourut à
quatre-vingt-six ans. Il avait été premier président du parlement de
Chambéry. Il était du pays, et s'attacha à la France quand le maréchal
Catinat prit la Savoie. Il eut divers emplois. Le feu roi l'aimait et le
considérait. Chaseaux, président à Metz, eut sa place. Il était neveu du
célèbre Bossuet, évêque de Meaux. M. le duc d'Orléans, je ne sais pas
où, avait pris anciennement de l'amitié pour lui\,; et comme il était
assez pauvre et point marié, il lui donna peu après une fort bonne
abbaye dans Metz.

Le maréchal de Villeroy mena promener le roi chez M\textsuperscript{lle}
de Chausseraye, qui s'était fait donner, puis fort ajuster et accroître
une petite maison au bois de Boulogne\,; tout près du château de Madrid,
dont les promenades étaient charmantes, et où elle amusa le roi de mille
choses qu'elle avait curieusement rassemblées\,; car elle était fort
riche et avait un goût exquis. Quoique j'aie parlé ailleurs de cette
singulière fille et de son caractère, il s'en faut bien que j'en aie
tout dit. Elle avait plu au feu roi autrefois, et en petit était devenue
une autre M\textsuperscript{me} de Soubise. Il y paraissait encore bien
moins au dehors\,; mais les particuliers étaient plus intimes\,; quoique
moins utiles pour elle, parce qu'elle n'était pas dans une position à
cela, sans famille, et à peu près sans nom. Le roi et elle s'écrivaient
souvent, et souvent il la faisait venir à Versailles, sans que personne
s'en doutât, ni qu'on sût ce qu'elle y faisait. Le prétexte était de
venir voir la duchesse de Ventadour et Madame. Bloin était celui par qui
passaient les lettres et les messages, et qui l'introduisait chez le roi
par les derrières dans le plus grand secret.

Le roi se plaisait fort avec elle, parce qu'elle était fort amusante et
divertissante quand il lui plaisait, qu'elle avait l'art de lui cacher
son esprit, qui était son soin le plus attentif et le plus continuel, et
qu'elle faisait très bien l'ingénue et la personne indifférente qui ne
prenait part à rien, ni parti pour personne. Par cet artifice elle avait
accoutumé le roi à ne se défier point d'elle, à se mettre à son aise, à
lui parler de tout avec confiance, à goûter même ses conseils, car ils
en étaient là ensemble, et il est incroyable combien elle a su par là
servir et nuire à quantité de gens, sans que le roi s'aperçût qu'elle se
souciât le moins du monde des personnes dont ils se parlaient. Les
ordres qu'il donna souvent en sa faveur aux contrôleurs généraux les uns
après les autres, et qui l'enrichirent extrêmement, n'ayant rien d'elle,
dont elle sut bien profiter pour se les rendre souples, sans toujours
recourir au roi, firent bien douter de quelque chose dans l'intérieur du
ministère et de la plus intrinsèque cour, mais non pas de toute
l'étendue de sa faveur, qui a duré autant que la vie du roi.

Elle était amie du cardinal de Noailles\,; et parmi bien de fort
mauvaises choses, elle en avait quelques bonnes. Les scélératesses qui
se faisaient pour l'opprimer la révoltaient en secret. Elle avait la
force d'y paraître au moins indifférente pour en découvrir davantage, et
de cacher avec grand soin son amitié et son commerce avec le cardinal de
Noailles. Le prince de Rohan, pour qui son frère n'avait point de
secret, et qui était son conseil intime, ne bougeait de chez la duchesse
de Ventadour, le cardinal de Rohan aussi tant qu'il pouvait. Ils la
ménageaient infiniment pour leurs vues, et comme on ne peut avoir moins
d'esprit et de sens qu'elle en avait, qui se réduisait à Pair, à
l'habitude, au langage et aux manières du grand monde et de la cour dont
elle était esclave, elle était aisément entrée dans tout avec eux par
amitié, et par être touchée de leur confidence sur les affaires de la
constitution, qui était la grande, la supérieure, celle de tous les
jours, et qui influait puissamment sur toutes les autres en ce temps-là.
Les Rohan, accoutumés à l'intimité qui était de tous les temps entre
M\textsuperscript{me} de Ventadour et M\textsuperscript{lle} de
Chausseraye, et qui recevaient d'elle toutes sortes de flatteries, ne se
cachaient point d'elle pour parler à M\textsuperscript{me} de Ventadour
de leurs succès et de leurs projets. Ils eurent l'imprudence de parler
devant elle de celui de faire enlever le cardinal de Noailles allant à
Conflans, par ordre du roi, et de l'envoyer tout de suite à Rome, qui
n'attendait que cela pour le déposer de son siège et le priver de la
pourpre, mais qui autrement n'osait entreprendre ni l'un ni l'autre,
quoi que les cardinaux de Rohan et Bissy, le P. Tellier et toute leur
cabale eût pu faire pour y déterminer le pape. C'était donc pour eux un
coup de partie, quoiqu'un parti forcé. La mine était chargée, où chacun
devait faire son personnage, et le P. Tellier le principal, qui avait
déjà commencé à en parler au roi.

Chausseraye, de providence, fut le lendemain longtemps avec le roi qui
avait travaillé le matin avec le P. Tellier sur cette affaire. Elle
trouva le roi triste et rêveur\,; elle affecta de lui trouver mauvais
visage et d'être inquiète de sa santé. Le roi, sans lui parler de
l'enlèvement proposé du cardinal de Noailles, lui dit qu'il était vrai
qu'il se trouvait extrêmement tracassé de cette affaire de la
constitution\,; qu'on lui proposait des choses auxquelles il avait peine
à se résoudre\,; qu'il avait disputé tout le matin là-dessus\,; que
tantôt les uns et tantôt les autres le relayaient sur les mêmes choses,
et qu'il n'avait point de repos. L'adroite Chausseraye saisit le moment,
répondit au roi qu'il était bien bon de se laisser tourmenter de la
sorte à faire chose contre son gré, son sens, sa volonté\,; que ces bons
messieurs ne se souciaient que de leur affaire, et point du tout de sa
santé, aux dépens de laquelle ils voulaient l'amener à tout ce qu'ils
désiraient\,; qu'en sa place, content de ce qu'il avait fait, elle ne
songerait qu'à vivre, et à vivre en repos, les laisserait battre tant
que bon leur semblerait sans s'en mêler davantage, ni en prendre un
moment de souci, bien loin de s'agiter comme il faisait, d'en perdre son
repos, et d'altérer sa santé, comme il n'y paraissait que trop à son
visage\,; que pour elle, elle n'entendait rien, ni ne voulait entendre à
toutes ces questions d'école\,; qu'elle ne se souciait pas plus d'un des
deux partis que de l'autre\,; qu'elle n'était touchée que de sa vie, de
sa tranquillité, de sa santé qu'il ne conserverait jamais qu'en les
laissant entre-battre tant qu'ils voudraient, sans plus s'en embarrasser
ni s'en mêler. Elle en dit tant, et avec un air si simple, si
indifférent sur les partis, et si touchant sur l'intérêt qu'elle prenait
au roi, qu'il lui répondit qu'elle avait raison\,; qu'il suivrait son
conseil en tout ce qu'il pourrait là-dessus, parce qu'il sentait que ces
gens-là le feraient mourir\,; et que, pour commencer, il leur défendrait
dès le lendemain de lui plus parler de quelque chose qui le peinait au
dernier point, à quoi ils revenaient sans cesse, qu'il avait été sur le
point de leur accorder malgré lui, et qu'il ne permettrait pas, et pour
cela comme le plus court, leur fermerait dès le lendemain la bouche
là-dessus pour toujours. Chausseraye, ravie, et qui entendait mieux de
quoi il s'agissait que le roi ne le pouvait imaginer, toujours pressante
sur sa santé, vie, repos, confirma le roi dans cette résolution, le
piqua d'honneur d'être leur dupe et leur victime, et fit tant que le roi
lui donna parole positive d'exécuter si bien dès le lendemain ce qu'il
venait de projeter et de lui dire, sans s'en expliquer davantage avec
elle, que la chose serait rompue sans retour, et sans que pas un d'eux
osât jamais lui en parler.

Elle avait averti le cardinal de Noailles du danger qu'il courait, et
d'éviter de sortir de Paris où il était adoré, et où on n'aurait osé
tenter de l'enlever, dont il y avait déjà quelque temps qu'elle était
informée par l'inconsidérée confiance de la duchesse de Ventadour, qui
lui avait appris le projet et ses machines, en y applaudissant, et
ensuite par les Rohan mêmes. Elle fut, au sortir de chez le roi, passer
sa soirée chez la duchesse de Ventadour\,; elle y trouva la joie peinte
sur son visage et sur celui des Rohan. Elle soupa, joua et se retira le
plus tôt qu'elle put. Le lendemain elle monta en chaise à quatre heures
du matin, se mit à pied à distance, et par l'église de Notre-Dame entra
dans un recoin de la cour de l'archevêché, où elle fit descendre le
cardinal de Noailles par un petit degré\,; car il se levait toujours
extrêmement matin. Ils entrèrent dans un méchant lieu nu et ouvert, où
il n'y avait rien, et où on n'entrait point, parce que cela n'allait à
rien\,; et là lui conta sa conversation et son succès de la veille, et
l'assura qu'il n'avait plus de violence à craindre. Elle ne fut guère
plus d'un quart d'heure avec lui, regagna sa chaise de poste et
Versailles d'où il ne parut pas qu'elle fût sortie. Elle alla dîner chez
la duchesse de Ventadour, et y passa tout le jour et tout le soir pour
tâcher à découvrir si le roi lui avait tenu parole\,; elle n'eut
satisfaction que tout au soir.

Le prince de Rohan vint avec un air triste et déconcerté qu'il
communiqua à sa belle-mère, qu'il tira à part un moment. Il ne joua
point, et demeura seul à rêver dans un coin de la chambre. Chausseraye,
qui jouait, et qui remarquait tout avec sa lorgnette, quitta le jeu,
l'alla trouver, et s'assit auprès de lui, disant qu'elle venait lui
tenir compagnie. Elle se garda bien de lui parler de rien, mais peu à
peu conduisit la conversation sur la santé, les vapeurs, les tristesses
involontaires, pour lui pouvoir parler de celle où elle le trouvait.
L'hameçon prit dans le moment. Il lui dit que ce n'était pas sans cause
qu'il était triste\,; de là à déclamer contre la faiblesse du roi, qui
plusieurs fois avait été sur le point de consentir à l'enlèvement du
cardinal de Noailles, qui la veille au matin, en résistant là-dessus au
P. Tellier, avait été dix fois près à lâcher la parole, s'était tout à
coup ravisé, et ce matin avait pris à part un moment le P. Tellier, et à
quelque distance le cardinal de Rohan, leur avait dit qu'il avait pensé
et repensé à l'enlèvement qu'ils lui avaient proposé, et dont ils le
pressaient sans cesse, et d'un ton de maître avait ajouté qu'il voulait
bien leur dire qu'il n'y consentirait jamais, et que de plus il leur
défendait d'y plus songer et de lui en jamais parler\,; après quoi, sans
laisser un instant d'intervalle, il avait tourné le dos à l'un et à
l'autre. De là le prince de Rohan à déclamer et à dire rage. Voilà
Chausseraye bien étonnée (car elle faisait d'elle tout ce qu'elle
voulait) et bien appliquée à n'oublier aucun langage qui pût tirer du
prince de Rohan les expédients, s'ils en imaginaient quelqu'un, qui
pussent redresser l'affaire, et la conduite qu'ils y allaient tenir, et
cependant se délectait et se moquait d'eux en elle-même. Elle eut une
nouvelle joie de les découvrir effrayés du ton absolu que le roi avait
pris, découragés et persuadés que ce serait se perdre inutilement que de
tenter plus rien sur cet enlèvement.

J'avoue ingénument que j'avais ignoré ces particuliers du roi, et cette
confiance qu'il avait prise en M\textsuperscript{lle} de Chausseraye,
conséquemment cette curieuse anecdote touchant le cardinal de Noailles.
Son esprit tout tourné à l'intrigue n'en eut pas moins depuis la mort du
roi avec M. le duc d'Orléans, qu'on a vu en son lieu qu'elle avait fort
connu et pratiqué étant à Madame, et toujours depuis, et avec tous les
personnages qui lui parurent mériter de s'en occuper. On dit que quand
le diable fut vieux il se fit ermite\,; aussi fit M\textsuperscript{lle}
de Chausseraye. Elle se mit dans la dévotion. Ses moeurs, sa vie, ses
richesses l'effrayèrent. Elle ne sortit plus de son bois de Boulogne, et
n'y reçut presque plus personne, quelques instances que ses amis fissent
pour la voir. On a vu en son lieu que sa mère, qui était Brissac, avait
épousé en premières noces le marquis de La Porte-Vezins, dont elle avait
eu des enfants, et en secondes noces, par amour, le sieur Petit, dont
elle eut M\textsuperscript{lle} de Chausseraye, qui fut longtemps, même
après la mort de sa mère, à ne pouvoir être reçue chez ses parents. Elle
s'honorait fort des La Porte, dont elle était soeur utérine, et dans sa
retraite elle vit beaucoup l'abbé d'Andigné, qui leur était fort proche,
homme de beaucoup de monde, de savoir et de piété, peu accommodé, fort
retiré, ami intime de tout ce que faussement on traite de jansénistes,
et demeurant à la porte des pères de l'Oratoire de Saint-Honoré. Elle
lui a conté tout ce que je viens de rapporter, et bien d'autres choses,
et lui a dit que toute son application et tout son savoir-faire auprès
du roi, et qui la mettait avec lui dans une gêne continuelle, était de
faire l'idiote, l'ignorante, l'indifférente à tout, et de lui procurer
le bien-aise d'entière supériorité d'esprit sur elle\,; que c'était
uniquement par là qu'elle entretenait sa faveur et sa confiance, et
qu'elle avait moyen de le conduire souvent où elle voulait\,; mais que
pour y parvenir sans qu'il s'en aperçût, et sans se démentir de toute sa
conduite avec lui, il fallait un temps, des tours, une délicatesse et un
art qui lui réussit souvent à bien des choses, dont elle en abandonnait
aussi d'autres, mais qui toutes lui faisaient suer sang et eau. Elle
consultait fort cet abbé sur sa conscience, qui lui laissa brûler par
scrupule des Mémoires très curieux qu'elle avait faits, et dont elle lui
montra quelque chose. Elle passa les dernières années de sa vie en
macérations, en aumônes, en prières, vendit une infinité de bijoux pour
en donner l'argent aux pauvres, priva ses héritiers de sa riche
succession, à qui elle l'avait franchement annoncé, et donna tout par
testament à l'hôpital général. Bien des années après sa mort, je connus
par des amis communs cet abbé d'Andigné, qui nous conta tout ce que je
viens d'écrire, parce que cela m'a semblé digne d'être arraché à
l'oubli. Ce ne fut pas sans le quereller, avec dépit d'avoir brûlé avec
elle de si précieux Mémoires.

Cani, fils unique de Chamillart, mourut à Paris, fort jeune, de la
petite vérole, laissant plusieurs enfants tous en bas âge de la soeur du
duc de Mortemart. Il fut regretté de tout le monde par la modestie avec
laquelle il avait supporté la fortune de son père et la sienne, par son
égalité dans la disgrâce, son courage et son application à la tête du
régiment de la marine dont il s'était fait beaucoup aimer, qui n'était
pas chose aisée avec ce corps. Il avait une pension particulière de
douze mille livres et un brevet de trois cent mille livres sur sa charge
de grand maréchal des logis de la maison du roi, dont il ne jouissait
que depuis la mort de Cavoye, duquel il avait acheté la survivance. Ce
fut une grande affliction pour Chamillart et sa femme qui étaient à
Courcelles. M. le duc d'Orléans donna la charge et le même brevet de
retenue en même temps au fils aîné, qui n'avait que sept ans. L'âge du
roi ne pouvait de longtemps donner beaucoup d'exercice à cette charge.
Dreux y fut commis jusqu'à ce que son neveu fût en âge. Ce fut bien la
plus grande douleur qui pût arriver à Chamillart\,; mais ce ne fut pas
la seule. Six semaines après, la petite vérole prit à la duchesse de La
Feuillade, qui l'emporta en trois jours, dans le dernier abandon de son
mari, qui prétexta qu'il ne pouvait se séquestrer du Palais-Royal, où
alors on ne le voyait presque jamais. Elle n'eut jamais d'enfants, non
plus que la première femme d'un si bon mari et d'un si honnête homme.

En ce même temps mourut la belle-fille {[}de M. de Castries{]}, fort
belle, fort jeune, fort sage et parfaitement au gré de la famille où
elle était entrée et de tout le monde\,; et son mari, qui n'y était pas
moins, et fils unique, {[}mourut{]} sept semaines après, qui fut une
affliction à M. et à M\textsuperscript{me} de Castries dont ils ne se
consolèrent jamais. J'ai assez parlé d'eux à l'occasion de leur mariage
pour n'avoir rien à y ajouter, sinon qu'ils ne laissèrent point
d'enfants.

La bâtarde, non reconnue, de Monseigneur et de la comédienne Raisin, que
M\textsuperscript{me} la princesse de Conti avait mariée depuis sa mort
à M. d'Avaugour, qui était de Touraine, et non des bâtards de Bretagne,
mourut aussi sans enfants.

Le comte de Croï, fils du comte de Solre, épousa en Flandre une riche
héritière, sa parente, qui s'appelait M\textsuperscript{lle} de
Milandon, et quitta le service. Il passa le reste de sa vie chez lui à
accumuler, et prit le nom de prince de Croï, après la mort de son père
arrivée en 1718, sans aucun titre, droit ni apparence. Son père n'a
jamais porté que le nom de comte de Solre, fut chevalier de l'ordre en
1688, le cinquante-neuvième parmi les gentilshommes, sans nulle
difficulté. Sa femme, qui était Bournonville, cousine germaine de la
maréchale de Noailles, était fort assidue à la cour, sans tabouret ni
prétention. Depuis la mort du fils, la veuve est venue s'établir à Paris
sous le nom de princesse de Croï, a prétendu être assise sans avoir pu
montrer pourquoi, ne la pouvant être n'a pas mis le pied à la cour, a eu
du cardinal Fleury des régiments pour ses deux fils de préférence à tout
le monde, en a marié un à une fille du duc d'Harcourt, et se promet
bien, à force d'intrigue, d'opiniâtreté et d'effronterie, de se faire
princesse effective pour le rang, dans un pays où il n'y a qu'à
prétendre et tenir bon pour réussir, à condition toutefois que ce soit
contre tout droit, ordre, justice et raison.

Rothelin épousa en même temps avec dispense la fille de sa soeur la
comtesse de Clères.

M. le duc d'Orléans donna une longue audience au premier président et
aux députés du parlement, sur les remontrances contre l'édit de
rétablissement des charges de surintendant des bâtiments et de grand
maître des postes, pour le duc d'Antin et Torcy. Rien plus en la main du
roi que ces grâces, rien plus étranger à la foule du peuple, de moins
contraire au bon ordre et à la police du royaume, rien enfin de moins
susceptible de l'opposition du parlement\,; mais cette compagnie, qui
avait dès le commencement senti la faiblesse du régent, et qui
l'environnait de ses émissaires, lesquels, comme il a été expliqué,
trouvaient leur compte au métier qu'ils faisaient, sut tourner sa
faiblesse en frayeur, lui contester tout avec avantage, et ne perdre
aucune occasion de profiter de sa facilité pour établir l'autorité de la
compagnie sur la sienne. Il était visible qu'ils ne pouvaient avoir que
ce but en celle-ci, qui ne touchait ni ne blessait personne, et de se
rendre ainsi redoutables au régent et à tout le monde.

Peu de temps après, non contents de lui embler son pouvoir, ils osèrent
disputer de rang avec lui, petit-fils de France et régent du royaume,
et, l'emporter sur ce prince faible et timide. Ces messieurs, que j'ai
nommés ailleurs, qu'il croyait entièrement attachés à lui, et dont il
admirait l'esprit et les conseils, mais qui se jouaient de lui avec tout
son esprit, sa pénétration, sa défiance, et le vendaient continuellement
au parlement, lui mirent en tête qu'il ferait chose fort décente et fort
agréable au peuple d'aller à la procession de Notre-Dame, le jour de
l'Assomption, instituée par le voeu de Louis XIII, à laquelle, assistent
le parlement et les autres compagnies. Ce prince n'aimait ni les
processions ni les cérémonies\,; il fallait un grand ascendant sur son
esprit pour lui persuader de perdre toute une après-dînée à l'ennui de
celle-là. Il y consentit, le déclara, manda toute sa maison pour l'y
accompagner en pompe, mais deux jours devant l'Assomption, il eut lieu
d'être bien surpris quand le premier président lui vint déclarer qu'il
croyait qu'il était de son respect, sur ce qu'il avait appris qu'il
comptait assister à la procession de Notre-Dame, de l'avertir que le
parlement, s'y trouvant en corps, ne pouvait lui céder, et que tout ce
qu'ils pouvaient de plus pour lui marquer leur respect était de prendre
la droite et de lui laisser la gauche. Il ajouta que leurs registres
portaient que M. Gaston, fils de France, oncle du feu roi, étant
lieutenant général de l'État, s'était trouvé à cette procession dans la
minorité du feu roi, et y avait marché à la gauche du parlement, qui
avait eu la droite. Ces messieurs prétendent tout ce qu'il leur plaît,
et maîtres de leurs registres y mettent tout ce qu'il leur convient\,;
c'est pour cela qu'ils en ont de secrets, d'où ils font passer dans les
publics ce qu'ils jugent à propos en temps convenables. La simple
proposition de précéder un petit-fils de France, régent du royaume, en
procession publique, et par respect croire s'abaisser beaucoup que se
contenter de prendre sur lui la droite, dispense de toutes réflexions.
Ce sont les mêmes qui ont osé opiner longtemps aux lits de justice avant
les pairs, puis avant les fils de France, enfin entre la reine lors
régente et le roi Louis XIV son fils, et qui contestèrent
contradictoirement et crièrent si haut lorsqu'en 1664 Louis XIV les
remit juridiquement, étant en son conseil, par arrêt, en leur ancien
rang naturel d'opiner après les pairs et les officiers de la couronne.

Le parlement est, comme on l'a vu à l'occasion du bonnet, une simple
cour de juridiction pour rendre aux sujets du roi justice, suivant le
droit, les coutumes et les ordonnances des rois, en leur nom, et dont
les officiers sont si bien, à titre de leurs offices, du corps du tiers
état, que s'il se trouvait entre eux un noble de race député aux états
généraux, sa noblesse ne lui servirait de rien, mais son office
l'emporterait et le placerait dans la chambre du tiers état, de l'ordre
duquel il serait. Le parlement fait donc partie du tiers état, il est,
par conséquent, bien moindre que son tout. Les états généraux tenant, le
parlement oserait-il imaginer, non pas de précéder, mais de marcher à
gauche et sur la même ligne du tiers état\,? et le même tiers état, je
dis plus, l'ordre de la noblesse, si distingué du tiers état aux états
généraux, oserait-il disputer la préséance en quelque lieu, cérémonie ou
occasion que ce soit à un petit-fils de France, régent du royaume\,?
Cette gradation si naturelle saute aux yeux, et je ne pense pas même que
les trois ordres du royaume assemblés en fissent la difficulté à un
petit-fils de France, qui même ne serait pas régent, bien moins encore
l'étant. Que si le parlement allègue que les grandes sanctions se font
maintenant dans son assemblée, on a montré comment cela est arrivé, et
qu'encore aujourd'hui elle est incompétente si les pairs n'y sont
appelés et présents. Mais sans recourir à l'évidence du droit, et s'en
tenant au simple fait, le \emph{Cérémonial français}, imprimé il y a
longtemps\footnote{La première édition du \emph{Cérémonial de France} de
  Théodore Godefroy, parut en 1619 (Paris, in-4°). Denis Godefroy, fils
  de Théodore, donna, en 1649, une nouvelle édition de cet ouvrage
  (Paris, 2 vol.~in-fol.). C'est de cette seconde édition que surit
  tirées les citations de Saint-Simon.}, rapporte «\,1° que Henri II, la
reine après lui, puis plusieurs princes, barons, chevaliers de l'ordre,
gentilshommes et dames, {[}marchaient{]} portant tous un cierge allumé à
la procession\,; puis venaient ceux de la cour de parlement, vêtus de
leurs mortiers et robes d'écarlate\,; à côté d'eux, messieurs des
comptes, etc.\,» (P. 951, t. II.)

2° À la procession pour la prise de Calais, depuis la Sainte-Chapelle
jusqu'à Notre-Dame, le dimanche 26 janvier 1557\footnote{La prise de
  Calais eut lieu le 8 janvier 1557 (1558). La date de la procession ne
  peut donc être le 2 janvier, comme le portent les anciennes éditions
  de Saint-Simon.} (p 955)\,:

«\,\ldots. Puis marchèrent (prélats, cardinaux, etc.)\ldots. le roi
portant à ses côtés le prince de Condé, prince du sang et le duc de
Nevers, pair de France\,; la reine après ledit seigneur\,; après elle la
reine d'Écosse et Mesdames, filles dudit seigneur, les duchesses,
comtesses, etc., au milieu de la rue\,; à la dextre, ladite cour de
parlement\,; à la senestre, au-dessous des présidents, et d'aucuns
anciens conseillers (c'est-à-dire non vis-à-vis de ceux-là) la chambre
des comptes.\,»

3° À la procession en réparation d'un sacrilège, faite à
Sainte-Geneviève, 27 décembre 1563 (p.~956)\,:

«\,\ldots. Tôt après y sont arrivés (à la Sainte-Chapelle, où on
s'assemblait) le roi, la reine\,; Monseigneur, frère du roi\,; Madame,
soeur du roi et leur suite (trois cardinaux, cinq évêques)\,; les
princes dauphin d'Auvergne et de La Roche-sur-Yon (princes du sang)\,;
les ducs de Guise, Nemours, Aumale, le marquis d'Elboeuf, la princesse
de La Roche-sur-Yon, la duchesse de Guise, plusieurs autres chevaliers
de l'ordre, seigneurs, dames et demoiselles\ldots, l'archevêque de Sens
portant l'hostie sacrée sous un poêle, dont les bâtons de devant étaient
soutenus devant par les ducs de Nemours, Aumale et marquis d'Elboeuf,
derrière par le prince dauphin d'Auvergne et le duc de Guise. Après les
roi et reine et leur suite marchait ladite cour (de parlement), à
dextre\,; les prévôt, échevins et officiers de la ville, à senestre,
etc.\,»

4° À la procession de Sainte-Geneviève faite le dimanche 10 septembre
1570, où le roi voulut assister avec tous, et où ni lui ni la reine ne
se trouvèrent (p.~960 et 961)\,:

«\,\ldots. Les châsses (et leur accompagnement). Suivaient immédiatement
lesdits évêques (de Paris) et abbé (de Sainte-Geneviève), MM. les ducs
de Montpensier, prince-dauphin (son fils), duc d'Uzès, maréchal de
Vieuville, comte de Retz et de Chavigny, etc., et plusieurs seigneurs et
gentilshommes. Après suivaient les huissiers de la cour, greffier et
quatre notaires\,; de Thou, premier président, les présidents Baillet,
Séguier, Prevost et Hennequin, leurs mortiers dessus leurs têtes (et
tout le parlement), tenant l'un des côtés à dextre, etc. (Ne dit de la
séance de l'église où il n'y avait ni chambre des comptes ni autre cour
que la ville et l'université, ni à la procession, que ces deux mots\,:
La messe célébrée dans Sainte-Geneviève (par l'évêque de Paris), étant
l'abbé de Sainte-Geneviève en une chaire en bas du rang des présidents,
et ayant le premier lieu\ldots. Parce que la messe étant dite, les
susdits de Montpensier, princes, ducs, comtes et chevaliers de l'ordre,
ensemble la cour de parlement se retirèrent chacun où bon lui sembla.\,»

5° À la procession à Saint-Denis pour la remise des corps saints en
leurs places, descendus au commencement des troubles, faite le jeudi 8
mars 1571 (p. 964)\,:

«\,Premièrement marchaient les religieux de Saint-Denis\ldots.
Monseigneur le duc d'Anjou portant la couronne, le roi, les sieurs
d'Aumale et de Nevers, suivis de plusieurs autres seigneurs. Suivant
laquelle déclaration de la volonté du roi (touchant la préséance de la
ville sur la cour des monnaies), elle marcha après la chambre des
comptes et deux à deux, du côté senestre, la cour de parlement et des
aides tenant la dextre.\,»

Je n'ai copié que les endroits qui font à la chose, marqué de points ce
qui n'y sert de rien sans le copier, et mis entre deux crochets de
parenthèse quelques mots qui ne sont pas dans le \emph{Cérémonial}, pour
lier ou expliquer ce qui en est. On voit donc ici cinq processions, dont
les jours et les années sont marqués, les occasions qui les causèrent,
et les lieux où elles se firent. Rien de plus net que l'énoncé de la
première. On y voit après le roi et la reine, \emph{plusieurs princes,
barons, chevaliers de l'ordre, gentilshommes et dames portant un cierge
allumé à la procession\,; puis venaient ceux de la cour de parlement,
vêtus de leurs mortiers et robes d'écarlate}. Ce \emph{puis venaient}
décide bien clairement que le parlement était précédé par tous ces
seigneurs et dames, et qu'ils étaient bien en rang et en cérémonie,
puisqu'ils portaient des cierges. À l'égard du terme de
\emph{gentilhomme}, il ne doit pas être entendu de simples,
gentilshommes comme il s'entend communément aujourd'hui. Alors n'était
pas marquis, comte, baron qui voulait, et gentilhomme signifiait alors
des seigneurs aussi qualifiés, et souvent plus en grandes charges, que
les marquis, comtes, et souvent leurs frères, oncles, neveux et enfants.
Cet usage ancien d'appeler de tels seigneurs du nom de gentilshommes est
encore demeuré dans l'ordre du Saint-Esprit, où on nomme de ce nom tous
les chevaliers non princes ni ducs\,; et on y dit marcher ou seoir, ou
être reçu parmi les gentilshommes\,; ce qui est un reste du style
d'autrefois.

La seconde est mal expliquée. On y voit seulement le prince de Condé et
le duc de Nevers aux côtés du roi. L'un y est énoncé prince du sang,
l'autre pair de France. Ni l'un ni l'autre n'avait de charge\,; ce
n'était donc, qua l'un par naissance, l'autre par dignité qu'ils
marchaient ainsi. Or ils n'étaient pas seuls à accompagner le roi, et il
n'est pas dit un mot d'aucun autre. Les princesses, duchesses, etc.,
sont marquées marcher au milieu de la rue, entre le parlement à droite,
et la chambre des comptes à gauche. Elles avaient donc le milieu, par
conséquent le meilleur lieu, puisqu'il n'est pas douteux que, qui est au
milieu entre deux autres, en cérémonie, précède celui qui est à sa
droite comme celui qui est à sa gauche. Il n'est donc pas douteux, par
l'énoncé, que le prince de Coudé, et le duc de Nevers côtoyant le roi,
sans fonction nécessaire de charges, précédaient le parlement, et que
les dames, qui marchaient entre cette compagnie et la chambre des
comptes, ne les précédassent aussi toutes les deux. Quoiqu'on ne voie
rien dans l'énoncé des autres seigneurs de la suite du roi, ce rang des
dames empêche d'imaginer qu'ils en aient en un inférieur.

La troisième ne s'explique que collectivement. \emph{Après lesdits roi
et reine et leur suite marchait ladite cour de parlement}. Il est au
moins clair que cette suite le précéda\,; et que si le roi seul le
pouvait précéder, il aurait eu son capitaine des gardes et tout au plus
son grand chambellan, ou en son absence le premier gentilhomme de la
chambre en année derrière lui, et personne autre avant le parlement.

La quatrième est bien décisive. Le roi et la reine ne s'y trouvèrent
point\,; par conséquent point de suite, ni personne qu'on pût dire
marcher entre eux et le parlement par raison de charge près d'eux, ou
par accompagnement, quoique ce n'en soit pas une. Or voici ce que porte
le \emph{Cérémonial\,: Suivaient immédiatement lesdites évêque et abbé.
MM. les duc de Montpensier, prince-dauphin, duc d'Uzès, maréchal de
Vieuville, comtes de Retz et de Chavigny}, etc., \emph{et plusieurs
seigneurs et gentilshommes\,; après suivaient les huissiers de la cour,
greffier et quatre notaires\,; de Thou, premier président, les
présidents Baillet, Séguier, Prevost et Hennequin, leurs mortiers dessus
leurs têtes} (\emph{et tout le parlement}, etc.). Le commentaire est ici
superflu\,; tout est clair, littéral, précis, net\,: la noblesse
précède\,; le parlement la suit, et sans la moindre difficulté.

La cinquième enfin ne prouve pas moins évidemment la même chose que la
précédente, nonobstant la parenthèse qui regarde la préséance de la
ville sur la cour des monnaies, que je ne fais que supprimer ici pour
une plus grande clarté\,: \emph{le roi, les sieurs d'Aumale et de
Nevers, suivis de plusieurs autres seigneurs}. Il est donc clair que
toute cette noblesse précéda le parlement, puisqu'elle est mise
nécessairement de suite avant de parler des compagnies, et que la
dispute de la ville avec les monnaies fait que le \emph{Cérémonial}
vient incontinent à sa marche après la chambre des comptes, qu'il dit
avoir eu la gauche et le parlement la droite.

La vérité de la préséance de fait de la noblesse sur le parlement en ces
processions saute tellement aux yeux, que ce serait vouloir perdre du
temps que de s'y arrêter davantage. Le droit et le fait sont certains.
Sauter de là à précéder un petit-fils de France régent du royaume, en
cérémonie toute pareille, il faut avoir les jarrets bons. C'est le
second tome d'avoir opiné avant la reine régente, mère de Louis XIV, au
lit de justice, après avoir escaladé les pairs, les princes du sang, les
fils de France. Ces messieurs sont l'image de la justice. Les images
portées ou menées en procession précèdent le roi, encore un tour
d'épaule et ils prétendront le précéder, comme ils prétendent tenir la
balance entre lui et ses sujets, brider son autorité par la leur, et que
celle du roi n'a de force, et ne doit trouver d'obéissance que par celle
que lui prêtent leurs enregistrements, qu'ils accordent ou refusent à
leur volonté. Je pourrais ajouter d'autres remarques sur les processions
et aussi sur les \emph{Te Deum\,;} mais ce n'est pas ici le lieu de
traiter expressément des préséances, du droit et des abus\,; je n'ai
touché cette matière que par la nécessité du récit qui doit s'arrêter
ici dans ces bornes.

Je ne dissimulerai pas que, quelle que fût mon indignation d'une
prétention qui ne peut être assez qualifiée, je riais un peu dans mes
barbes de voir le régent si bien payé par le parlement, auquel il avait
si étrangement sacrifié les pairs et ses paroles les plus solennellement
données et réitérées, et l'engagement pris avec eux en pleine séance du
parlement le lendemain de la mort du roi, comme je l'ai raconté en son
lieu. Cette compagnie, non contente de ventiler son autorité, de le
barrer dans les choses les plus indifférentes pour lui faire peur de sa
puissance, qui n'existait que par la faiblesse et la facilité du régent
qu'ils avaient bien reconnue, lui voulut étaler sa supériorité sur lui
jusque dans le rang.

M. le duc d'Orléans, ensorcelé par Noailles, Effiat, Canilla, jusque par
cette mâchoire de Besons, gémissait sous le poids de ces entreprises de
toute espèce, négociait avec le parlement par ces infidèles amis, comme
il aurait fait avec une puissance étrangère, lâchait tout, et en sa
manière imitait la déplorable conduite de Louis le Débonnaire, d'Henri
III et de Charles Ier d'Angleterre, dont je lui avais si souvent proposé
d'avoir toujours les portraits devant ses yeux, pour réfléchir à leurs
malheurs, à ce qui les y avait conduits, et à éviter une imitation si
funeste. Il avait peine dans les courts moments d'impatience à se
contenir de médire quelque mot de ce qui en faisait le sujet, mais à la
manière d'un pot qui bout et qui répand, non comme un homme qui
consulte. Jamais depuis plusieurs mois je ne lui en parlais le premier,
suivant la résolution qu'on a vu que j'en avais prise, et quand il m'en
lâchait quelque mot, je glissais par des lieux communs, vagues et
courts, et changeais subitement de propos. On a vu quelles en étaient
mes raisons. Quand je le voyais venir d'assez loin là-dessus pour
prendre mon tournant, je ne manquais pas de le faire par quelque
disparate de discours qui rompit ce que je voyais qu'il m'allait dire,
et je n'étais pas fâché de le faire assez grossièrement pour qu'il
s'aperçût que je ne voulais plus parler ni lui entendre parler du
parlement, ni de rien qui pût avoir aucun trait à cette compagnie. J'en
usai encore plus sèchement en cette occasion. Il m'avait parlé de la
procession comme en passant, et je m'étais tu pour n'entrer en aucun
discours qui pût amener détail de rang et de cérémonie\,; il le sentit
et n'alla pas plus loin. Après il ne put se tenir de me dire qu'il
n'irait point, et sans oser m'expliquer là rare prétention qui lors
était devenue publique par le premier président et ses amis, il ajouta
qu'il y avait quelque difficulté avec le parlement, et qu'il aimait
mieux laisser tout cela là. Je me mis à sourire un peu malignement, et
lui répandis que ce serait autant d'ennui et de fatigue épargnés. Nous
nous connaissions tous deux depuis bien des années. Il sentit mon
sourire et l'indifférence de ma réponse\,; il rougit, et me parla
d'autre chose, à quoi je pris avidement. Je n'en fus pas moins bien avec
lui, et j'ai bien vu depuis qu'il sentait ses torts avec moi sur le
parlement et l'injustice de ses défiances\,; mais alors il n'était pas
encore en liberté. Il céda donc au parlement en s'abstenant d'assister à
la procession, après avoir déclaré qu'il y irait, et avoir tout fait
préparer pour y assister dans toute la pompe d'un régent, petit-fils de
France.

Le rare est qu'il n'examina rien, et qu'il en crut le premier président
sur sa très périlleuse parole. L'exemple de Gaston, vrai où faux, le
frappa\,; il ne le vérifia seulement pas\,; et de plus la faute de
Gaston ne devait pas être le titre de la sienne. Gaston était le plus
faible de tous les hommes\,: il ménageait le parlement avec la dernière
bassesse, qui sut tout entreprendre dans la minorité de Louis XIV où on
était pour lors. Gaston, mené tantôt par l'abbé de La Rivière, tantôt
par le coadjuteur, tantôt contre M. le Prince, tantôt pour lui, et
levant l'étendard contre le cardinal Mazarin, voulait être le maître, et
comptait ne le pouvoir être que par le parlement, qui avait pris le
dessus jusqu'à faire la guerre au roi et le chasser nocturnement de
Paris. Ainsi cet exemple n'en était un que des monstrueuses entreprises
d'une compagnie qui pour dominer tout s'était jetée dans la sédition et
la révolte ouverte\,: belle leçon pour les rois et pour les régents.

Huit ou dix jours après, M. le duc d'Orléans fit donner une pension de
six mille livres au jeune président de Maisons, avec la jouissance à sa
mère sa vie durant, l'un et l'autre pourtant fort riches. Le duc de
Noailles et Canillac, qui était le tenant de cette maison, procurèrent
cette grâce si mal placée, et ce comble de faiblesse si proche de celle
de la procession, à des gens dont le logis était le lieu d'assemblée des
cabales du parlement et des ennemis de la régence. Ce prince, pour
rendre tout le monde content, donna en même temps, et paya, lui ou le
roi, un beau régiment de dragons à Rion, dont M\textsuperscript{me} la
duchesse de Berry fut fort satisfaite.

Pour rendre la chose complète, ces messieurs obtinrent que cette pension
donnée à Maisons ne fût pas celle qu'avait son père, parce qu'elle lui
aurait été moins propre et personnelle, et qu'il y aurait peut-être eu
quelque ombre de difficulté d'en faire jouir sa mère sa vie durant.
Cette pension du père était de celles appelées de Pontoise, et fut
donnée en même temps au président Aligre, pour mieux gratifier le
parlement qui traitait si bien le régent en son autorité et en son rang,
et dans l'instant même qu'il l'empêcha avec cet éclat d'assister à cette
procession, où ils lui déclarèrent si nettement que le parlement le
précéderait. Voici quelles étaient ces pensions dites de Pantoise.
Pendant les troubles de la minorité de Louis XIV, où le parlement
commençait à prêter l'oreille à des unions qui causèrent depuis, des
guerres civiles, on crut dans le conseil du roi rompre cours à ces
dangereuses menées en éloignant de Paris le parlement, et il fut
transféré à Pontoise. Un très petit nombre des officiers de cette
compagnie obéit, l'autre demeura à Paris et y leva bientôt le masque.
Les chefs de ceux qui avaient obéi et entraîné d'autres à Pantoise, où
ils les maintinrent dans la fidélité et dans l'exercice de leurs charges
comme le parlement y séant, en furent récompensés de six mille livres de
pension chacun. Depuis ce temps-là ces pensions se sont continuées et
sont connues sous le nom de pensions de Pontoise. Le roi les donne, à
qui il lui plaît, lorsqu'elles vaquent, d'entre les présidents à
mortier. On a cru que cette continuation de grâces rendrait les uns
reconnaissants, les autres soumis par l'espérance. Que de gens qui
perdent bras et jambes, et qui se ruinent au service du roi, à qui on ne
donne rien ou bien peu de chose, mais ils ne portent ni robe ni rabat\,!

\hypertarget{chapitre-ii.}{%
\chapter{CHAPITRE II.}\label{chapitre-ii.}}

1716

~

{\textsc{Bataille de Salankemen gagnée sur les Turcs par le prince
Eugène.}} {\textsc{- Jésuites encore interdits.}} {\textsc{- Comte
d'Évreux entre singulièrement au conseil de guerre.}} {\textsc{- Coigny,
mal avec le régent, se bat avec le duc de Mortemart\,; refusé d'entrer
au conseil de guerre, veut tout quitter.}} {\textsc{- Je le
raccommode.}} {\textsc{- Il entre au conseil de guerre.}} {\textsc{- Il
ne l'oublie jamais.}} {\textsc{- Les princes du sang présentent une
requête au roi contre le nom, le rang et les honneurs de princes du
sang, et l'habilité de succéder à la couronne, donnée par le feu roi à
ses bâtards.}} {\textsc{- Les pairs présentent une requête au roi pour
la réduction des bâtards au rang, honneurs et ancienneté de leurs
pairies parmi les autres pairs.}} {\textsc{- Bout de l'an du roi à
Saint-Denis.}} {\textsc{- Le duc de Berwick établit son fils aîné en
Espagne, qui y épouse la soeur du duc de Veragua et prend le nom de duc
de Liria.}} {\textsc{- Valentinois de nouveau enregistré au parlement,
lequel se réserve des remontrances en enregistrant un nouvel édit pour
la chambre de justice, et refuse une seconde fois les deux charges des
bâtiments et des postes.}} {\textsc{- Caractère du duc de Brancas.}}
{\textsc{- Caractère de son fils et de sa belle-fille.}} {\textsc{- Ils
désirent de nouvelles lettres de duché-pairie à faire enregistrer au
parlement de Paris.}} {\textsc{- État de leur dignité.}} {\textsc{-
Brancas trompé par Canillac, à qui il s'était adressé, s'en venge en
bons mots et a recours à moi.}} {\textsc{- Condition dont Villars me
donne toute assurance, sa foi et sa parole sous laquelle je m'engage à
le servir.}} {\textsc{- J'y réussis avec peine.}} {\textsc{- Longtemps
après, il me manque infâmement de parole et en jouit.}} {\textsc{- Le
parlement enregistre enfin l'édit de création des charges de
surintendant des bâtiments et de grand maître des postes.}} {\textsc{-
Les princes du sang et bâtards n'assistent point à la réception du duc
de Villars-Brancas.}} {\textsc{- Mort de l'abbé de Brancas.}} {\textsc{-
Mort de la princesse de Chimay.}} {\textsc{- Abbé de Pomponne chancelier
de l'ordre par démission de Torcy.}} {\textsc{- Arrivée des galions
richement chargés.}} {\textsc{- Voyage de Laffiteau\,; quel était ce
jésuite.}} {\textsc{- Mort du fils unique de Chamarande, et du comte de
Beuvron.}} {\textsc{- Mort de M\textsuperscript{me} de Lussan et de
l'abbé Servien.}} {\textsc{- Mort de M\textsuperscript{me} de
Manneville.}} {\textsc{- Mort d'Angennes.}} {\textsc{- Mort de la
duchesse d'Olonne.}} {\textsc{- M. le duc de Chartres, malade de la
petite vérole, cause un dégoût de ma façon au duc de Noailles.}}
{\textsc{- Te Deum au pillage.}} {\textsc{- Mort du maréchal de
Montrevel, de peur d'une salière renversée sur lui.}} {\textsc{- Mort du
prince de Fürstemberg.}} {\textsc{- Mort du prince de Robecque.}}
{\textsc{- Le régiment des gardes wallonnes donné au marquis de
Risbourg.}} {\textsc{- La duchesse d'Albe épouse le duc de Solferino.}}

~

La guerre s'était enfin déclarée entre les deux empires. Les deux armées
se trouvèrent fort proches au commencement d'août. Le prince Eugène, qui
commandait l'impériale, détacha le 4 le comte Palfi avec le comte
Brenner, pour aller reconnaître les Turcs avec deux mille chevaux. Les
Turcs en avaient fait un autre qui les rencontra. L'action fut vive.
Brenner fut pris, à qui en arrivant le grand-vizir fit inhumainement
couper la tête devant sa tente, où on la trouva encore avec le corps
auprès le lendemain 5. Ce même jour les deux armées s'ébranlèrent l'une
contre l'autre. La bataille dura sept heures avec beaucoup
d'opiniâtreté. Enfin les Turcs furent battus et mis en fuite, perdirent
près de trente mille hommes, toute leur artillerie, leurs tentes et
leurs bagages. La victoire du prince Eugène fut complète, à qui il n'en
coûta que quatre ou cinq mille hommes. Cette bataille fut donnée près de
Salankemen\footnote{Cette bataille est ordinairement désignée sous le
  nom de bataille de Peterwaradin. Elle fut gagnée par le prince Eugène,
  le 5 août 1716. Il y avait eu, en 1691, comme le rappelle Saint-Simon,
  une victoire remportée sur les Turcs à Salankemen (Esclavonie) par le
  prince Louis de Bade.}, où le prince Louis de Bade en avait gagné une.

La guerre de la constitution n'était pas moins animée du côté des
agresseurs, c'est-à-dire de ceux qui voulaient la faire recevoir à leur
mot, ni plus honnêtement menée que le traitement fait par le grand vizir
à un prisonnier de guerre fort distingué, qu'on vient de voir. Les
jésuites continuaient à intriguer, à écrire, à parler plus violemment
que jamais, en sorte que le cardinal de Noailles, qui avait laissé les
pouvoirs à un petit nombre d'entre eux lorsqu'il les ôta au gros, se
trouva à bout de ménagements avec eux, et interdit la totalité, excepté
les PP. Gaillard, entraîné malgré lui par sa compagnie, La Rue,
Lignières et du Trévoux, confesseurs de la reine d'Angleterre, de Madame
et de M. le duc d'Orléans. Ce dernier n'avait pas grand besoin de cette
grâce pour l'usage qu'il avait à en faire. Lignières fut depuis
confesseur du roi, mais sans feuille ni crédit\,; La Rue, qui l'avait
été de M\textsuperscript{me} la Dauphine, ne l'était plus que de
quelques personnes distinguées, à qui et pour elles seulement, le
cardinal de Noailles voulut bien ne le pas refuser.

Le comte d'Évreux, colonel général de la cavalerie, mourait d'envie de
se servir de ce temps facile pour reprendre l'autorité de sa charge, que
le comte d'Auvergne, son oncle, n'avait jamais eue, ni lui non plus. Il
ne se mêlait en aucune sorte de la cavalerie\,; tout se faisait dans le
conseil de guerre, où MM\hspace{0pt}. de Lévi et de Joffreville en
avaient le département. Dépouiller le conseil de guerre de cette partie
était chose impossible\,; y entrer, qui lui aurait cédé\,? Cet embarras
le retint longtemps dans l'inaction. À la fin le désir de prendre
l'autorité sur la cavalerie, et par là d'aller plus loin, lui parut
mériter quelque sacrifice, mais toujours conservant un coin de précieuse
chimère. Il demanda au régent la dernière place fixe au conseil de
guerre, qui que ce soit qui y pût entrer, de n'avoir ni le nom ni les
appointements de conseiller de ce conseil, et d'y être seulement chargé
du département de la cavalerie, au lieu de ceux qui l'avaient, à
condition d'y rapporter tout, et de faire comme eux faisaient sur la
cavalerie à l'égard du conseil. Il sentait que par là il acquerrait
connaissance de la cavalerie, du crédit sur elle, et de la
considération, qui s'augmenterait toujours par l'exercice, et qu'avec
cette possession subalterne au conseil de guerre, il serait difficile
qu'elle ne lui revînt pas entière et indépendante, si ce conseil venait
à cesser, et la forme du gouvernement à changer, comme l'un et l'autre
arriva en effet\,; et par cette dernière place fixe, sans titre ni
appointement de conseiller, il comptait ôter toute difficulté, faire
porter cette place sur sa charge, et mettre sa princerie à couvert. Ce
projet lui réussit\,; le régent le trouva bon, et le comte d'Évreux
entra ainsi au conseil de guerre, et y demeura sus ce pied-là tant que
ce conseil dura.

Coigny, colonel général des dragons, qui était bien éloigné des raisons
qui avaient si longtemps combattu le comte d'Évreux en lui-même sur le
conseil de guerre, avait tenté tout ce qu'il avait pu pour y entrer
depuis qu'il était formé. Il était ancien lieutenant général. Nulle
difficulté d'aucune sorte. Il était mal sur les papiers du régent, en
cela plus malheureux que ceux qui le méritaient le plus. Il s'était
insinué assez avant par la chasse avec M. le comte de Toulouse, du temps
du roi\,; il avait été depuis de tous ses voyages de Rambouillet. La
querelle des princes du sang et des bâtards excita des propos. Le duc de
Mortemart, peu d'accord avec lui-même, en tint de forts contre les
bâtards, en présence de Coigny. Celui-ci, qui y sentit le comte de
Toulouse mêlé et désigné comme le duc du Maine, voulut faire entendre au
duc de Mortemart que ses discours ne convenaient pas à sa proximité avec
eux. Cela fut mal reçu, ils se querellèrent, et pour le faire court ils
se battirent. Je ne sais qui l'emporta\,; mais le duc n'eut rien, et
Coigny en emporta une marque très visible sur le visage qui lui est
demeurée toute sa vie, et dont on ne lui fait pas plaisir de lui parler.
L'affaire fut étouffée avec grand soin pour sa cause, et Coigny fut
quelque temps sans paraître pour se laisser guérir. Tout cela avait
persuadé le régent, et confirmé depuis, que Coigny était tout aux
bâtards, et au duc du Maine autant qu'au comte de Toulouse. Ses refus
réitérés résolurent Coigny à vendre sa charge qui faisait toute son
existence et toutes ses espérances qu'il voyait évanouies\,; il en
traita. Ses amis qui par là le voyaient tomber dans un puits en
retardèrent la conclusion, sa femme surtout qui avait beaucoup de sens,
de raison, de modestie, et qui vivait fort retirée, et toute sa vie
d'une grande vertu, quoiqu'elle eût été belle, et toujours dans une
solide piété. L'entrée du comte d'Évreux dans le conseil de guerre lui
fit perdre toute patience. Il voulut finir son marché, et s'en aller
pour toujours en Normandie, où il avait beaucoup de biens. À ce coup,
personne ne put le retenir. C'était un homme au désespoir qui se voyait
perdu auprès du régent sans ressource, et sans avoir pu deviner
pourquoi.

En cette extrémité je ne sais qui avisa sa femme de me venir trouver.
Jamais je ne l'avais vue, ni M\textsuperscript{me} de Saint-Simon non
plus\,; Coigny et moi n'avions jamais mené la même vie\,; je ne le
connaissais point du tout, et ne le rencontrais presque jamais.
M\textsuperscript{me} de Coigny était soeur du Bordage que nous ne
voyions jamais non plus\,; leur mère était Goyon-Matignon d'une autre
branche que les Matignon, fille du marquis de La Moussaye et d'une soeur
de M. de Turenne, tellement qu'elle était cousine issue de germaine de
M\textsuperscript{me} de Saint-Simon, petites-filles des deux soeurs.
Elle s'en vint franchement un matin toute seule chez moi réclamer
parenté, secours, et me conter rondement le désespoir de son mari, et le
sien, de lui voir se couper la gorge résolument sans que rien l'en pût
empêcher, s'il ne parvenait à entrer au conseil de guerre, et à fondre
les glaces de M. le duc d'Orléans à son égard, qu'il ne savait pas avoir
jamais méritées. Sa franchise, sa confiance, sa situation me touchèrent.
Je savais d'où le mal venait\,; mais comme je ne m'y intéressais ni en
bien ni en mal, je n'en avais tenu nul compte. Je convins avec elle
qu'avant tout il fallait arrêter la vente de la charge, et me donner
après le temps de faire ce que je pourrais. Je la priai de m'envoyer son
mari, et je la renvoyai toute consolée de se flatter d'une ressource,
sans néanmoins m'être fait fort de rien. Dès le lendemain je vis arriver
Coigny dans un état de désespoir, qu'il ne me cacha point, d'un homme
qui voit perdus tous les travaux de sa vie pour soi et pour sa famille,
et qui se va enterrer tout vivant. Je lui dis ce que je pus pour le
remettre un peu\,; je ne laissai pas de le promener assez sans faire
semblant de rien, pour découvrir en quel état il était avec M. du Maine,
et je trouvai qu'il n'y avait rien du tout. Je lui dis que présentement
je ne lui répondais de rien, parce que j'ignorais, comme il était vrai,
jusqu'à quel point était pour lui l'éloignement de M. le duc
d'Orléans\,; que je lui demandais quinze jours pour me tourner, et voir
à traiter ce qui le regardait avec Son Altesse Royale\,; que je lui
promettais de faire tout de mon mieux pour le raccommoder, et pour le
faire entrer au conseil de guerre, mais sous une condition, sans
laquelle je ne pouvais me mêler de lui, qui était sa parole d'honneur de
surseoir le marché de sa charge pendant ces quinze jours, et qu'après
nous verrions, et qu'au cas qu'il entrât au conseil de guerre, il
romprait le marché et ne s'en déferait point. Il me le promit. Je le
priai de ne se point donner la peine de revenir chez moi, ni de se
donner aucun autre mouvement, et d'attendre pendant ces quinze jours
qu'il eût de mes nouvelles. Je le renvoyai un peu calmé.

Je n'eus pas besoin de tant de temps. Je parlai au régent\,; je le
détrompai sur la liaison de M. du Maine\,; je lui fis honte de grêler
sur le persil, tandis qu'il comblait de faveurs tant de grands coupables
à son égard, dont il ne faisait que des ingrats, et de désespérer un
ancien lieutenant général distingué dans son métier, estimé dans le
monde, qu'il s'acquerrait sûrement en ne l'excluant pas d'un agrément où
le portait sa charge et l'exemple du comte d'Évreux tout récent.
J'obtins donc tout ce que je m'étais proposé, dans les premiers huit
jours des quinze que j'avais demandés. J'envoyai prier Coigny de passer
chez moi. Il vint aussitôt\,: Je lui dis ce que j'avais fait\,; que les
préventions étaient tombées\,; qu'il s'en apercevrait dans le courant\,;
que j'avais permission de lui dire que rentrée au conseil de guerre lui
était accordée\,; qu'il pouvait en aller sur ma parole remercier le
régent\,; mais sans entrer en autre discours, parce que n'y ayant rien
eu de marqué, il n'y avait ni justification ni explication à faire. Il
est difficile de voir un homme plus aise qu'il fut. Il me dit que je le
faisais passer de la mort à la vie. Il alla au Palais-Royal, où il fut
bien reçu, et entra deux jours après au conseil de guerre, où il eut le
détail dès dragons. Sa femme me vint remercier l'après-dînée. Je leur
dois la justice qu'ils ne l'ont jamais oublié en aucun temps, et qu'ils
vivent encore aujourd'hui avec moi avec toutes les recherches, les
attentions et l'amitié possible, et la plus déclarée, sans aucun des
ménagements que les changements des temps et des choses ont produits, et
qui en ont tant changé d'autres. Il est vrai que ce que je fis alors le
remit à flot, conserva sa charge, et de l'un à l'autre a conduit lui et
son fils à la fortune qu'ils ont faite, et qui n'est peut-être pas au
bout\,; mais leur reconnaissance n'en est pas moins estimable et rare.

Enfin la querelle des princes du sang et des bâtards éclata après avoir
été longtemps couvée, aigrie, suspendue, par une requête signée de M. le
Duc, M. le comte de Charolais, et M. le prince de Conti, contre M. du
Maine et M. le comte de Toulouse, que M. le Duc présenta à M. le duc
d'Orléans, adressée au roi, le 22 août, et que le 29 du même mois M. le
duc d'Orléans donna en communication au duc du Maine, au sortir du
conseil de régence de l'après-dînée, pour y répondre. Davisard, fort
attaché à lui, avocat général au parlement de Toulouse, fut celui qui y
répondit, et qui fit toutes les autres pièces que les deux frères
produisirent ou publièrent dans le cours de ce fameux procès, dont le
curieux recueil est entre les mains de tout le monde, ainsi que l'autre
recueil de tout ce que les princes du sang y produisirent ou publièrent.
Je ne chargerai donc point ces Mémoires des raisons des uns ni des
autres, si tant est qu'à l'égard des bâtards on puisse appeler raisons
des usurpations sans nombre, toutes plus monstrueuses les unes que les
autres, et qui renversent l'ordre du royaume et toutes les lois divines
et humaines. Je ne suivrai même le cours de ce procès que sur les
événements importants, et j'en abandonnerai un inutile et ennuyeux
détail. Je me renfermerai là-dessus aux démarches que les ducs ne purent
se refuser en cette occasion, et à la part que j'ai pu y prendre.

Les princes du sang attaquant les bâtards dans l'usurpation de leur
qualité de princes du sang et de succession à la couronne, les pairs
tombaient nécessairement dans le cas de disputer à ces mêmes bâtards
l'usurpation du rang au-dessus d'eux. Ils avaient résolu de présenter
leur requête en même temps que les princes du sang présenteraient la
leur. Je ne l'avais pas laissé ignorer, comme on l'a vu, à M. du Maine
ni à M\textsuperscript{me} la duchesse d'Orléans, dès le règne du feu
roi et depuis il ne fut donc plus question que de l'exécuter. On
s'assembla\,; on la résolut\,; on la dressa\,; tous signèrent, hors cinq
ou six absents, le duc de Rohan, toujours étrange en tout, et d'Antin
qui nous pria de le dispenser de se trouver à ces assemblées. La
dernière ne fut que pour signer, et députer sur-le-champ quatre pairs
pour la porter au régent. MM. de Laon, de Sully, de La Force et de
Villeroy en furent chargés. Je refusai opiniâtrement d'en être, par
considération pour M\textsuperscript{me} la duchesse d'Orléans. En même
temps que nous sortîmes de chez M. de Laon, où en l'absence de M. de
Reims nous nous étions assemblés, les quatre députés allèrent présenter
au régent notre requête au roi, et en même temps j'allai chez
M\textsuperscript{me} la duchesse d'Orléans Je lui dis que je ne voulais
pas qu'elle apprît par M. le duc d'Orléans, moins encore par le public,
la démarche que nous faisions au moment que je lui parlais\,; que je la
suppliais de se souvenir que nous avions attendu à l'extrémité à la
faire\,; de ne point oublier ce que je lui avais dit là-dessus du vivant
du roi, et répété depuis sa mort plus d'une fois, et à M. le duc du
Maine, même à M\textsuperscript{me} du Maine, la seule fois que je
l'avais vue, lorsque M. du Maine m'y mena, rue Saint-Avoye, dans la
maison d'emprunt du premier président où ils logeaient au retour du roi
de Vincennes à Paris, et depuis encore à M. le comte de Toulouse.
M\textsuperscript{me} la duchesse d'Orléans me parut étonnée, néanmoins
reçut bien mon compliment, avoua se souvenir très bien de tout ce que je
lui alléguais, et n'osant trop s'émouvoir contre nous en ma présence, se
lâcha contre les princes du sang. Je n'étais pas là pour la contredire,
moins encore pour approuver sa déclamation\,; je pris le parti du
silence. Après qu'elle se fut exhalée, nous ne laissâmes pas de causer
d'autre chose à l'ordinaire\,; il lui vint du monde, j'en pris occasion
de me retirer.

Les députés à M. le duc d'Orléans nous rapportèrent qu'ils en avaient
été fort bien reçus. Je ne sais plus qui de nous se chargea de rendre
compte à M. le Duc de ce que nous venions de faire, qui en parut fort
aise. Nous ne fîmes là-dessus aucune civilité aux bâtards\,; mais comme
mon rang me plaçait nécessairement en tous les conseils auprès du comte
de Toulouse, avec qui j'étais là et chez M\textsuperscript{me} la
duchesse d'Orléans fort librement, où je le rencontrais souvent, je lui
en fis, en entrant au premier conseil, une civilité personnelle qu'il
reçut honnêtement. Je n'en fis aucune au duc du Maine, qui néanmoins me
salua fort civilement à son ordinaire, et moi lui, sans nous approcher.
Pour M. le duc d'Orléans, je lui parlai fortement, tant sur les princes
du sang que sur les pairs contre les bâtards. Je lui
ramenteus\footnote{Rappelai.} tout ce que lui-même m'avait dit du temps
du feu roi sur leurs différentes apothéoses, à mesure que le feu roi les
avait déifiés par degrés, et je ne lui laissai pas oublier les horreurs
inventées, et sans cesse répandues et renouvelées, contre lui par le duc
du Maine, où il avait fait entrer M\textsuperscript{me} de Maintenon, et
par elle en avait persuadé le roi et tout ce qu'il avait pu à la cour, à
Paris, dans les provinces, et jusque dans les pays étrangers. La
bénignité, pour ne pas dire l'incurie et l'insensibilité de M. le duc
d'Orléans, était inébranlable\,; mais il ne put disconvenir que nous
n'eussions raison d'avoir fait notre requête, et de la lui avoir
présentée. Les princes du sang y applaudirent fort\,; les bâtards n'en
sonnèrent mot. M\textsuperscript{me} la duchesse du Maine ne put se
contenir comme eux, mais elle n'osa pourtant se laisser {[}aller{]} au
delà des plaintes emportées pour une autre, mesurées pour elle. Nous la
laissâmes dire sans lui faire la moindre honnêteté là-dessus. La vérité
est que, après ce qui s'était passé, nous n'en devions aucune à M. ni à
M\textsuperscript{me} du Maine.

Je fus surpris de la façon dont le maréchal de Villeroy se comporta dans
cette affaire avec tout ce dont il se piquait pour le feu roi, qui ne
l'avait mis auprès de son successeur qu'en faveur des bâtards, et avec
toutes ses liaisons avec le duc du Maine. Il fût un des plus ardents
pour cette requête, et ne faiblit point dans toute la suite à cet égard.
Je ne dissimulerai pas qu'elle me fit peut-être commettre une simonie.
Quelques-uns de nous craignaient de signer la requête contre les
bâtards, et Rochebonne, évêque-comte de Noyon, plus que pas un. Il me
l'avoua, et passa jusqu'à me dire qu'il ne la signerait point. Il était
pauvre, jeune, aimait à dépenser\,; je le pris par ce faible. Je lui
promis de faire l'impossible, s'il la signait, pour lui obtenir une
grosse abbaye. Il fut combattu\,; à la fin il signa, mais sur cette
parole Il sut bien m'en sommer depuis\,; je la lui tins. Il eut l'abbaye
de Saint-Riquier, que j'arrachai du régent à là sueur de mon front. Il
me disait qu'on se moquerait de lui de donner un si gros morceau à un
homme comme M. de Noyon. Je me gardai bien de lui faire confidence de
notre marché\,; mais j'y mis tout mon crédit, et jamais je n'eus tant de
peine. J'en fus récompensé par la satisfaction de m'acquitter, et par la
joie de M. de Noyon, qui n'osait espérer une si forte abbaye, et de tous
points si fort à sa bienséance.

On fit, le 1er septembre, le bout de l'an du feu roi à l'ordinaire, mais
à petite et courte cérémonie. Il n'y eut de révérences que celles des
hérauts. Les princes du deuil furent M. le duc d'Orléans, M. le Duc et
M. le comte de Charolais\,; le duc du Maine, ses deux fils et le comte
de Toulouse y assistèrent, et presque personne. Les compagnies y
étaient. Moins de deux heures finirent tout à Saint-Denis.

Le duc de Berwick, dont on a expliqué en son temps l'érection d'un
duché-pairie avec des clauses si singulières, par l'espérance qu'il
avait du rétablissement de ses établissements en Angleterre, et d'en
revêtir le comte de Tinrnouth son fils aîné, unique de son premier
mariage, vit enfin qu'il n'y avait plus à se flatter de ce côté-là. Il
prit le parti de l'établir en Espagne, de lui céder sa grandesse suivant
le privilège insolite que le roi d'Espagne lui en avait accordé en le
faisant grand, comme il a été remarqué alors, et de l'établir pour
toujours en Espagne, où il fut gentilhomme de la chambre, prit le nom de
duc de Liria, et possession des terres que le roi d'Espagne avait
données à son père dans le royaume de Valence, qu'il lui céda, et il le
maria à la soeur unique du duc de Veragua, lequel était fort riche, sans
enfants ni volonté de se marier.

On a vu, en son temps, l'engagement pris et déclaré par le roi
d'accorder au fils unique de Matignon une érection nouvelle de
Valentinois en duché-pairie, en épousant la fille aînée de M. de Monaco,
qui n'avait point de garçons, les singulières clauses qui y furent
obtenues, et ce qui causa une grâce qui n'avait point d'exemple. Le peu
que le roi vécut depuis ne permit pas aux deux familles de la consommer,
par tous les ajustements d'intérêts qu'il fallut faire\,; mais comme la
grâce était publique, dès que les deux familles furent en état de faire
le mariage les lettres d'érection furent expédiées, en décembre 1715. Le
nouveau duc s'alla marier à Monaco, et quand il en revint, il trouva les
princes du sang et les bâtards aux prises sur le traversement du parquet
prétendu par les derniers, tellement que, pour éviter des inconvénients
personnels, M. le duc d'Orléans suspendit l'enregistrement de
Valentinois, où les uns et les autres vivaient résolu de se trouver. La
querelle grossit\,; comme on vient de le rapporter, par la requête des
princes du sang pour dépouiller les bâtards de bien d'autres choses\,;
ainsi, il ne fut plus question de se trouver au parlement, et M. de
Valentinois finit son affaire\,; mais les autres pairs s'y trouvèrent.
Dans cette séance, il y eut deux événements\,: le premier fut
l'enregistrement d'un nouvel édit pour la chambre de justice\,; mais le
parlement, qui prétendit ne l'avoir pas examiné, se réserva d'y pourvoir
par des remontrances. L'autre fut le refus réitéré de l'édit de création
des charges de surintendant des bâtiments, et de surintendant des
postes. Le duc de Noailles y fit l'orateur, pour plaire au régent et
montrer en public sa belle éloquence. Elle échoua, et les voix
contraires se trouvèrent plus nombreuses qu'elles n'avaient été au
premier refus.

L'exécution de cette grâce, jusqu'alors diversement suspendue par
différentes raisons étrangères à la grâce même, avait donné lieu depuis
longtemps à des désirs. Le duc de Brancas, tout frivole qu'il était, en
devint susceptible, et son fils aussi peu solide que lui. Le père était
un homme léger, sans méchanceté, sans bonté, sans affection et sans
haine, sans suite et sans but que celui d'attraper de l'argent, pourvu
que ce fût sans grand'peine, de le dépenser promptement et de se
divertir. À qui n'avait que faire à lui, et à qui n'y prenait point de
part, aimable, amusant, plaisant, divertissant, avec des saillies
pleines d'esprit, d'une imagination ravissante, quelquefois folle, qui
ne se refusait rien, qui parlait bien et de source, avec un air naturel,
souvent un naïf inimitable. Il se faisait justice à lui-même pour se
donner liberté entière de la faire aux autres, mais sans ambition et
sans jalousie. Une débauche outrée et vilaine l'avait séparé de presque
tous les honnêtes gens, et quoiqu'il se remit par bouffées de fantaisie
par-ci par-là dans le grand monde, dont il était toujours bien reçu du
gros, l'obscurité de son goût l'en retirait bientôt dans l'obscurité de
sa déraison, où il demeurait des années sans reparaître. Quoique le
désordre de sa vie ne fût pas du même genre que celui de M. le duc
d'Orléans, ce prince s'était toujours plu avec lui, et, devenu le
maître, avait continué à l'admettre et à le désirer dans ses soupers et
dans sa familiarité. Il n'en était pourtant guère plus ménagé que les
autres. Il disait de lui qu'il gouvernait et menait les affaires comme
un espiègle\,; et pressé outre mesure par un homme de province d'obtenir
je ne sais quoi, et qui, comme ces gens-là ne manquent jamais de faire,
lut disait qu'on savait bien qu'il pouvait tout, il lui répondit
d'impatience\,: «\,Eh bien\,! monsieur, il est vrai, puisque vous le
savez, je ne vous le nierai point, M. le duc d'Orléans me comble de
bontés, et veut tout ce que je lui demande\,; mais le malheur est qu'il
a si peu de crédit auprès du régent, mais si peu, si peu, que vous en
seriez étonné, que c'est pitié, et qu'on n'en peut rien espérer par
cette voie.\,» Le premier n'était pas mal vrai, et il le dit à M. le duc
d'Orléans lui-même. Ce prince sut le second qui n'était pas tout à fait
faux, et il rit de tout son coeur de tous les deux. Brancas disait de
soi-même au régent qu'il n'avait point de secret\,; qu'il se gardât bien
de lui rien confier\,; qu'il n'avait point aussi l'esprit d'affaires,
qu'elles l'ennuieraient, qu'il ne voulait que se divertir et s'amuser.
Cela mettait M. le duc d'Orléans à l'aise avec lui, qui ne pouvait assez
l'avoir dans ses heures obscures et dans ses soupers. Il y disait de soi
et des autres tout ce qui lui passait par la tête, avec beaucoup de
cette sorte d'esprit et de liberté\,; et ses dires revenaient après par
les autres soupeurs, qui s'en divertissaient aux dépens de qui il
appartenait.

On a vu ailleurs comment et à qui il avait marié son fils aîné, ou
plutôt vendu pour de l'argent qu'il en avait tiré pour y consentir et se
démettre de son duché. On a vu aussi que ce furent M. et Mine du Maine
qui firent ce mariage, et sur quel pied M\textsuperscript{lle} de Moras
était chez eux. Devenue par eux duchesse de Villars, elle et son mari
passèrent leur vie à Sceaux, et partout à la suite de
M\textsuperscript{me} du Maine, comme leurs plus soumis domestiques,
jusque tout à la fin de la vie du roi. Le duc de Villars avait peu servi
et avec peu de réputation. Il aimait le jeu à l'excès, la parure
quoiqu'il en fût peu susceptible, les bijoux et les breloques, beaucoup
la bonne chère, encore mieux l'argent dont il n'avait guère et qu'il
dépensait dès qu'il en avait, plus que tout cela une infâme débauche
dont il se cachait encore moins que son père\,; duquel il ne tenait rien
pour l'esprit et l'agrément, mais moins obscur et très paresseux.

Lui et sa femme sans estime réciproque, qu'en effet ils ne pouvaient
avoir, vivaient fort bien ensemble dans une entière et réciproque
liberté, dont elle usait avec aussi peu de ménagement de sa part que le
mari de la sienne, qui le trouvait fort bon, et en parlait même
indifféremment quelquefois et jusqu'à elle-même devant le mande, et l'un
et l'autre sans le moindre embarras. Mais elle était méchante, adroite,
insinuante, intéressée comme une crasse de sa sorte, ambitieuse, avec
cela artificieuse, rusée, beaucoup d'esprit d'intrigue, mais désagréable
plus encore que son mari\,; et tous les deux bas, souples, rampants,
prêts à tout faire pour leurs vues, et rien de sacré pour y réussir,
sans affection, sans reconnaissance, sans honte et sans pudeur, avec un
extérieur doux, poli, prévenant, et l'usage, l'air, la connaissance et
le langage du grand monde. Tout à la fin de la vie du roi ils sentirent
le cadavre, ils comprirent que les choses ne se passeraient pas ou
doucement, ou agréablement, entre M. le duc d'Orléans et le duc du
Maine, ni entre les princes du sang et les bâtards. Ils commencèrent
donc à intriguer doucement pour être bien reçus de M. le Duc et de
M\textsuperscript{me} la Duchesse\,; et quand ils s'en crurent assurés,
ils firent comme les rats qui sentent de loin le prochain croulement
d'un logis, et l'abandonnent à temps pour aller chercher retraite dans
un autre. C'est ce que firent aussi ces rats à deux pieds\,; sans avoir
reçu le plus léger mécontentement de M. ni de M\textsuperscript{me} du
Maine, et aussi sans le plus léger ménagement pour eux. Les princes, et
plus ordinairement les princesses, s'amusent sans dégoût de ce qu'elles
méprisent, l'habitude, l'empressement bas à leur plaire y joint souvent
de la bienveillance\,; c'est à quoi le duc de Villars s'attacha auprès
de M\textsuperscript{me} la Duchesse et de ses entours, et devint un des
tenants de la maison, comme il l'avait été de celle de M. et de
M\textsuperscript{me} du Maine, qui n'entendirent plus parler d'eux.

Brouillés souvent avec le père et devenus plus souples à son égard, par
les mêmes raisons qui les avaient fait passer d'un camp à l'autre, ils
se réunirent et se mirent en tête de se tirer d'un état embarrassant qui
les excluait de tout, et d'en sortir par une érection nouvelle en
duché-pairie enregistrée au parlement de Paris. Le fils et sa femme,
trop méprisés pour y rien pouvoir, tâchèrent à mettre le père en
mouvement. Celui-ci ne se sentit pas un crédit assez sérieux pour
l'entreprendre sans aide. Le même étrange goût les avait liés, il y
avait longtemps, Canillac et lui\,; et le Palais-Royal, où ils se
voyaient assez souvent du temps du feu roi, les rassemblait fort
ordinairement ailleurs. Brancas s'adressa donc à lui et lui parla avec
confiance. L'habitude les unis, soit plus que l'amitié\,; d'estime, ils
se connaissaient trop pour en avoir l'un pour l'autre. Canillac avait
les mêmes vues pour un autre qu'il aimait véritablement, mais dont il
n'est pas encore temps de parler\,; il fut donc fâché de celles de
Brancas, embarrassé de son ouverture et du secours qu'il lui demandait,
résolu de l'amuser et de le tromper pour ne pas croiser les vues qu'il
avait pour un autre. La belle-fille, en attendant les bons offices de
Canillac, ne s'endormait pas\,; elle était venue à bout de tonnelet
d'Aguesseau, procureur général, qu'elle se doutait bien qui serait
consulté, et, sûre de lui, pressait son beau-père, qui à son tour
tourmentait Canillac. Avant d'aller plus loin, il faut expliquer le
fait.

Louis XIII érigea la terre de Villars en duché simple en septembre 1627,
en faveur de Georges de Brancas, qui les fit enregistrer en juillet
suivant au parlement d'Aix. Il était frère cadet de l'amiral de Villars,
qui traita de la réduction de Rouen et d'une partie de la Normandie avec
Henri IV\,; pour l'amirauté qu'avait le second maréchal de Biron, et à
d'autres conditions encore, en 1594, et qui fut tué l'année suivante, de
sang-froid, près de Dourlens en Picardie, où il avait été battu et pris
par les Espagnols. Il n'avait point été marié. Georges, son frère, fut
lieutenant général de Normandie et gouverneur du Havre-de-Grâce. Il
avait épousé une soeur du premier maréchal d'Estrées, et il obtint, en
1652\footnote{La date de 1652 est donnée parle manuscrit de Saint-Simon,
  mais il y a évidemment erreur, puisque Louis XIII était mort en 1643.
  D'après le \emph{Dictionnaire de Moreri} les lettres patentes qui
  confirmèrent l'érection de 1627 sont datées de juillet 1651 et par
  conséquent du règne de Louis XIV.}, de Louis XIII, des lettres
d'érection du duché de Villars en pairie, et mourut chez lui en
Provence, en janvier 1657, à quatre-vingt-neuf ans sans avoir fait
enregistrer nulle part ses lettres de pairie. Louis-François, son fils
aîné, un mois après la mort de son père, les fit enregistrer au
parlement d'Aix. C'était un petit bossu qui ne se montra guère, qui
s'enterra dans sa province, qui mourut en 1679, et qui était frère du
comte de Brancas, chevalier d'honneur de la reine mère, si connu par la
singularité de ses distractions, qui mourut en 1681 à soixante-trois
ans, et qui de la fille de Garnier, trésorier des parties casuelles, ne
laissa que la princesse d'Harcourt et la duchesse de Brancas, qu'il fit
épouser au fils aîné de son frère et de la fille de Girard, sieur de
Villetaneuse, procureur général de la chambre des comptes de Paris.
C'est cette duchesse de Brancas si malheureuse, dont on a raconté en son
temps la singulière séparation d'avec son mari, le duc de Brancas dont
il s'agit ici, et qui pour son pain se fit dame d'honneur de Madame,
comme on l'a dit ici en son temps. Par ces érections la dignité de duc
était certaine et héréditaire, l'ancienneté fort disputée, parce que
l'enregistrement n'en avait été fait qu'au parlement d'Aix, et celle de
pair nulle par la même raison, inconnue aux pairs et à la cour des
pairs. Cela faisait donc un duché fort boiteux et une pairie en idée, un
duc à qui aucun ne cédait, par conséquent exclus de toute cérémonie.
C'est donc de cet état d'embarras et d'exclusion que le père et le fils,
et plus qu'eux encore la belle-fille voulut sortir par de nouvelles
lettres d'érection en duché-pairie, enregistrées au parlement de Paris.

Canillac ne répondait point aux empressements avec lesquels Brancas
réclamait son service\,: outre la raison secrète qui retenait Canillac,
sa liaison avec Brancas n'était qu'habitude. Il fallait à l'un un
encens, une soumission, une admiration perpétuelle à son babil
doctrinal, politique, satirique, envieux et sentencieux, et à sa
singulière morale. C'était à quoi la vivacité et la liberté de Brancas
ne s'étaient pu ployer. Il s'aperçut enfin qu'il le menait sans dessein
de le servir. Piqué contre lui, il ne se contint plus de brocards, en
divertit M. le duc d'Orléans et sa compagnie les soirs. Il y dit un jour
du babil doctrinal de Canillac en sa présence, qu'il avait une perte de
morale continuelle, comme les femmes ont quelquefois des pertes de
sang\,; et la compagnie à rire, et M. le duc d'Orléans aussi. Canillac
en colère lui reprocha la futilité de son esprit et son incapacité
d'affaires et de secret, et qu'en un mot il n'était qu'une caillette.
«\,Cela est vrai, répondit Brancas en riant\,; mais la différence qu'il
y a entre moi et toi, c'est qu'au moins je suis une caillette gaie et
que tu es une caillette triste\,; j'en fais juge la compagnie.\,» Voilà
M. le duc d'Orléans et tout ce qui était avec lui aux éclats, et
Canillac dans une fureur qui lui sortit par les yeux et qui lui mastica
la bouche. Aussi ne l'a-t-il jamais pardonné au duc de Brancas qui tous
les jours le désolait et lui en donnait de nouvelles. Tout cela pourtant
ne faisait pas son affaire\,: il fallut avouer à son fils et à sa
belle-fille, qui le pressaient sans cesse, où il en était avec Canillac,
et se tourner de quelque autre côté. Ils pensèrent à moi comme à celui
qu'ils craignaient davantage et dont ils espéraient davantage aussi
s'ils pouvaient me gagner, parce que je ne les tromperais pas, parce que
je suivais ce que je voulais bien entreprendre, et par le poids que me
donnerait en leur affaire l'éloignement connu où j'étais de
l'accroissement du nombre des pairs. Le duc et la duchesse de Villars
s'étaient toujours entretenus bien avec la duchesse de Brancas. Celle-ci
était l'amie la plus intime et de tous les temps de la maréchale de
Chamilly, qui, à une vertu peu commune dans tous les temps de sa vie,
joignait toutes les qualités les plus aimables de l'esprit, du coeur et
de la plus sûre et agréable société, et qui était depuis longtemps amie
intime de M\textsuperscript{me} de Saint-Simon, par conséquent la
mienne, et nous voyait fort souvent\,; ce fut la voie qu'ils prirent.

La duchesse de Brancas par la maréchale était aussi de nos amies mais
non assez pour nous parler\,; nous ne connaissions point du tout la
belle-fille, au plutôt assez pour n'avoir aucun commerce, et je n'avais
jamais parlé au père ni au fils, pour ainsi dire. La maréchale se
chargea de nous parler, et le fit efficacement. Je considérai que M. de
Brancas n'était pas moins duc pour l'être d'une manière bizarre\,; que
son ancienneté pouvait embarrasser\,; qu'il valait mieux s'en défaire
par de nouvelles lettres, et un nouveau rang de duc et pair qui le remît
dans l'ordre naturel et commun, que de laisser subsister des prétentions
et une exclusion de toutes cérémonies éternelle. Je consentis donc à y
travailler à cette condition, mais de laquelle je voulus me bien assurer
par celui qu'elle regardait. C'était le fils, parce que, le père s'étant
démis de son duché, il n'était plus susceptible de la pairie comme il
était arrivé au maréchal de Tallard. Nous prîmes donc un jour chez la
maréchale de Chamilly, où le duc et la duchesse de Villars se trouvèrent
avec M\textsuperscript{me} de Saint-Simon et moi. Là se fit
l'explication et la convention nette et précise. Villars convint que
tout ce qu'il désirait était d'être fait duc et pair par de nouvelles
lettres enregistrées au parlement de Paris, tant pour couper racine à
toute prétention d'ancienneté, que parce que le parlement de Paris ne
connaît point d'enregistrement d'érections de ces dignités des autres
parlements, mais seulement les siennes. Qu'à ce titre, il prendrait la
queue de tous les pairs au parlement, et de plus celle de tous les ducs
en toutes cérémonies et actes, spécialement en l'ordre du Saint-Esprit,
le cas lui arrivant, et ne prendrait ni ne prétendrait jamais en aucun
acte, cérémonie, occasion quelconque, autre rang parmi les ducs que
celui de la date du rang nouveau desdites nouvelles lettres, et de sa
réception au parlement de Paris. Cela fut bien et clairement énoncé par
moi, répété par la maréchale de Chamilly, prononcé de même par Villars,
distinctement et correctement approuvé et consenti par lui, qui m'en
donna sa foi et sa parole d'honneur positive et me la réitéra, de
manière que j'eus honte de lui faire l'affront de la lui demander par
écrit. Et voilà la sottise des honnêtes gens droits et vrais avec ceux
qui ne sont rien moins, et desquels ils ne peuvent se figurer une
infamie solennelle. J'ai en depuis tout loisir de m'en repentir.

Ce qui m'empêcha de parler d'écrit fut qu'il me pria d'expliquer à M. le
duc d'Orléans ces conditions\,; qu'il me donna sa parole que lui et son
père les stipuleraient eux-mêmes en ma présence à ce prince, et qu'ils
consentaient que la foi et la parole qu'ils me donnaient de s'y tenir
devinssent publiques. Un homme d'honneur est aisément trompé par qui
n'en a point, et qui s'en joue. Ces paroles reçues, je ne pensai plus
qu'à m'acquitter de l'engagement qu'elles m'avaient fait prendre. Je
représentai au régent la convenance de mettre à flot des gens engravés
d'une manière singulière, dont il aimait le père, et dont la mère, dame
d'honneur de Madame, méritait sa considération et ses grâces, les tirer
de prétention et d'exclusion perpétuelle par une grâce très grande à la
vérité mais qui ne changeait point leur extérieur et ne blessait
personne. Je fus surpris de la résistance que j'éprouvai du régent. Il
s'amusait des pointes que faisait le duc de Brancas et de ses saillies,
mais au fond il le méprisait\,; il faisait encore moins de cas de son
fils et de sa belle-fille, à qui peut-être il n'avait jamais parlé, et
il comptait pour fort peu la vertu et la piété de la duchesse de
Brancas\,; il sentait le ridicule à l'égard du sujet, en sorte que j'eus
toutes les peines imaginables à en venir à bout à force de bras. Je lui
expliquai la condition, sans laquelle M. le duc d'Orléans n'eût jamais
accordé chose si forte contre son sens et son goût. Le père et le fils
non seulement y consentirent en sa présence, mais la lui demandèrent.
Elle fut rendue publique en même temps que la grâce sitôt que je l'eus
emportée\,; eux l'avouèrent par augmentation de droit, puisque les
nouvelles lettres portant nouvelle érection du duché et de la pairie
abolissaient les anciennes et les anéantissaient, et le rang nouveau que
leur enregistrement et la réception du duc de Villars opéra, fixa à leur
date le rang nouveau du nouveau duc et pair, tant au parlement qu'en
tous autres actes, assemblées et cérémonies d'État de cour et publiques.
Quoique les infâmes suites de ce service, de cette grâce, et de la foi
et parole si solennellement données et réitérées, portées au régent par
eux-mêmes, et de leur aveu devenues publiques, dépassent les temps que
je me suis prescrit pour ces Mémoires, je ne laisserai pas d'avoir lieu
de les placer en leur temps. Le duc de Villars ne perdit point de temps
pour son enregistrement, et il fut reçu le 7 septembre, dernier jour du
parlement.

Ce même jour avant sa réception, Effiat alla de bon matin au palais avec
une lettre de jussion dans sa poche pour l'enregistrement des charges de
surintendant des bâtiments et de grand maître des postes. Lui et son ami
le premier président qui ne songeait qu'à tirer de l'argent du régent en
se rendant difficile, mais ne s'en voulait pas tarir la source, avaient
trouvé que le jeu avait duré assez longtemps pour faire montre de
l'autorité du parlement sur chose qui n'intéressait ni le public ni
personne en particulier. Il assembla donc les chambres sur-le-champ, et
prit son temps qu'il y en avait encore peu des enquêtes arrivés, dont il
était mains le maître, et qu'il avait fort échauffés contre cet édit. Il
le proposa en aplanissant les prétendues difficultés, en faisant
craindre de s'exposer au dégoût des lettres de jussion, et en maintenant
leur rare autorité par de misérables modifications à l'édit, qui ne
faisaient rien aux charges ni à leurs fonctions. L'édit passa ainsi à la
grande pluralité des voix, et la lutte pour cette affaire demeura enfin
finie. M. le duc d'Orléans empêcha les princes du sang et les bâtards de
se trouver à l'enregistrement ni à la réception du duc de Villars, de
peur de commise. Son oncle l'abbé de Brancas, qui avait la tête fort
dérangée, se jeta dans la rivière vers ce même temps. Des bateliers le
retirèrent, mais il mourut quelques heures après.

Le cardinal Ferrari, jacobin, que sa vertu et son rare savoir avait
élevé à la pourpre, et l'avait honorée, et fort employé dans les
principales affaires, mourut à Rome.

La soeur aînée de M. de Nevers, qui avait épousé le prince de Chimay,
grand d'Espagne et chevalier de la Toison d'or, mourut aussi sans
enfants à Paris.

Torcy vendit quatre cent mille livres sa charge de chancelier de
l'ordre, avec permission de continuer à le porter, à son beau-frère
l'abbé de Pomponne, qui obtint en même temps un brevet de retenue de
trais cent mille livres dessus.

Les galions arrivèrent à Cadix, chargés de trente millions d'écus sans
les fruits et les pacotilles. Ce fut une grande et agréable nouvelle, et
en général pour tous les commerçants de l'Europe. L'arrivée du jésuite
Lafitau dans la chaise de poste du cardinal de La Trémoille fit plus de
bruit encore parmi un certain monde. Le secret et la promptitude de son
voyage, les mesures mystérieuses qu'il affecta ici, la promptitude avec
laquelle il repartit pour Rome six ou sept jours après, firent faire
bien des raisonnements. La suite montra que ce n'était qu'un fripon qui
s'était voulu faire de fête, et qui ne fit que leurrer et tromper.
Longtemps depuis le cardinal de Rohan m'a conté que ce drôle-là
entretenait une fille dans une espèce de faubourg de Rome, chez laquelle
il donnait très bien à souper à ses amis du temps que ce cardinal était
à Rome. Il se moquait de ses supérieurs pour les moeurs, mais il les
courtisait pour leur doctrine et leurs vues. Il avait beaucoup
d'intrigues qui à la fin le firent évêque de Sisteron, où il ne fut pas
moins effronté en tous genres. Le cardinal de Rohan n'eut pas honte
depuis tout cela de lui faire prêcher un carême à la cour, ni lui
d'écrire un volume de mensonges les plus grossiers et les plus reconnus
contre l'exacte et simple vérité du voyage de l'abbé Chevalier à Rome,
écrit par lui-même.

Chamarande, dont j'ai quelquefois fait mention, perdit le seul fils qui
lui restait\,; et le comte de Beuvron mourut en même temps fort jeune,
sans alliance, perdant le sang jusque par les pores, maladie fort peu
connue des médecins. Il avait reporté en Espagne la Toison de Sezanne
son oncle, où il l'avait obtenue, et le maréchal d'Harcourt lui avait
fait donner la lieutenance générale de Normandie et le gouvernement du
vieux palais de Rouen qu'il avait. Le régent en laissa la disposition au
maréchal d'Harcourt, qui les donna à un autre de ses enfants.

M\textsuperscript{me} de Lussan, de laquelle j'ai eu lieu de parler en
son temps, mourut fort vieille. Je n'ai point su si elle était devenue
moins friponne, fausse, et doucereuse impudente qu'elle avait vécu. Une
autre belle âme qui alla paraître fort subitement devant Dieu, fut celle
de l'abbé Servien, fils du surintendant et reste de tons les Servien,
duquel j'ai parlé quelquefois.

M\textsuperscript{me} de Manneville mourut en même temps d'un cancer.
Elle était fille de M. et de M\textsuperscript{me} de Montchevreuil, les
grands amis, de M\textsuperscript{me} de Maintenon, et avait une pension
du roi de six mille livres.

Les dames et les gens du bel air regrettèrent fort d'Angennes, qui
mourut de la petite vérole. La duchesse d'Olonne en mourut aussi, pour
s'en être enfermée mourant de peur avec son mari, qui ne le méritait
guère de la façon dont il vivait avec elle. Elle était fille du premier
mariage de Barbezieux, jeune, bien faite aimable, vertueuse et pleine de
ses devoirs. Ce fut grand dommage.

J'avais profité d'une quinzaine de vacances du conseil de régence pour
m'aller amuser à la Ferté et en d'autres campagnes, lorsque la petite
vérole parut à M. le duc de Chartres. Il me fâchait fort de couper un si
court intervalle, mais on m'en pressa tant que je vins passer un jour
franc à Paris pour voir M. {[}le duc{]} et M\textsuperscript{me} la
duchesse d'Orléans. J'allai donc au Palais-Royal le lendemain que je fus
arrivé. Je trouvai M. le duc d'Orléans dans son grand appartement qui me
parut touché de mon voyage. Comme je causais seul avec lui, on lui
annonça le duc de Noailles\,; je voulus dire quelque chose, M. le duc
d'Orléans m'interrompit pour me dire qu'il lui avait donné heure, et en
même temps le duc de Noailles entra et se tint en dedans sur la porte.
«\,Oh\,! pour cela, monsieur, repris-je tout haut pour que Noailles n'en
perdît rien, je fais cinquante lieues pour avoir l'honneur de vous voir,
je m'en retourne demain\,; nous étions en train de causer, vous n'avez
qu'à renvoyer M. de Noailles, il est bon pour attendre.\,» M. le duc
d'Orléans et moi étions demeurés assis sans bouger. Il fit signe avec un
peu d'embarras au duc de Noailles, qui sortit sur-le-champ et ferma la
porte sur lui. La conversation fut presque toute d'affaires étrangères.
Il y en avait une sur le tapis importante, qui regardait la négociation
de la France avec l'Angleterre et la Hollande, sur laquelle il se leva,
et me dit\,: «\,J'ai peur qu'on nous entende là dedans, car la porte
était du côté de son bureau\,; allons-nous-en dans ce cabinet.\,» Nous
étions dans ce salon, sur la rue Saint-Honoré, il me mena dans un
cabinet qui le joignait et qui donnait sur la même rue\,; et ferma la
porte sur moi. Je ne connaissais point ce cabinet, c'était une des
pièces du petit appartement des soupers. La conversation y continua près
d'une heure. Sortant de là nous trouvâmes dans le salon le duc de
Noailles, le maréchal d'Huxelles l'un auprès de l'autre, et cinq-ou six
seigneurs qui s'y étaient amassés, mais qui se tenaient éloignés de la
porte du cabinet d'où nous sortions. Je pris là congé de M. le duc
d'Orléans pour le reste de la vacance, et j'allai de là au maréchal
d'Huxelles, à qui je parlai malicieusement à l'oreille de la matière et
de l'entretien que je venais d'avoir, et lui à moi de même, et je
regardais cependant le duc de Noailles qui devenait de toutes les
couleurs. Je fis et reçus civilité de tout ce qui était là, et je passai
devant le duc de Noailles sans le saluer, qui se rangea et me fit une
grande révérence.

De bonne heure, après dîner, j'allai chez M\textsuperscript{me} la
duchesse d'Orléans qui me reçut fort bien. M. le duc d'Orléans m'avait
demandé si je ne la verrais pas, et même témoigné qu'il le désirait\,;
il était en peine qu'elle ne fût fâchée contre moi de notre requête.
Elle ne me la parut point du tout. Elle sortait de chez M. le duc de
Chartres. Mes deux fils avaient eu la petite vérole l'année précédente,
et le cadet en avait été longtemps à l'extrémité. Je m'étais servi du
frère du Soleil, jésuite, apothicaire du collège, fort habile, et
n'avais point voulu de médecins. Je m'en étais si bien trouvé que
j'avais fort conseillé à M. {[}le duc{]} et à M\textsuperscript{me} la
duchesse d'Orléans d'en user de même si M. le duc de Chartres avait la
petite vérole. Ils me crurent et cela réussit à souhait. Ce frère du
Soleil était excellent par science, par expérience et par une attention
infinie à ses malades, et habile pour toutes les maladies, avec une
simplicité et une douceur qui le faisait {[}aimer{]}\,; c'était aussi un
humble et fort bon religieux. La guérison de M. le Duc, de M. le prince
de Conti et de M. le duc de Chartres de la petite vérole produisit une
très impertinente nouveauté. Leurs maisons firent chanter des \emph{Te
Deum} dans leurs paroisses à Paris et encore ailleurs, ce qui ne s'était
jamais fait encore que pour les choses publiques ou pour le
rétablissement de la santé des rois et des reines, encore après un grand
péril, et très rarement de leurs enfants\,; mais {[}tout{]} tombait en
pillage, tellement qu'après cet exemple des princes du sang, il n'y eut
point de particulier qui ne fit après la même entreprise. On l'a
souffert, et fait encore chanter des \emph{Te Deum} qui veut et où on
veut.

Le maréchal de Montrevel, dont le nom ne se trouvera guère dans les
histoires, ce favori des sottes, des modes, du bel air, du maréchal de
Villeroy et presque du feu roi, duquel il avait tiré plus de cent mille
livres de rente en bienfaits, dont il jouissait encore, et qui n'a pu
être nommé que pour ce à quoi il avait le moins de part, une figure qui
le fit vivre presque toute sa vie aux dépens des femmes, une grande
naissance et une valeur brillante\footnote{Cette phrase assez obscure
  veut dire que, \emph{par delà} les qualités qui viennent d'être
  citées, le maréchal de Montrevel n'avait \emph{quoi que ce puisse
  être}.}, par delà, quoi que ce puisse être, mourut escroc de ses
créanciers, n'ayant rien vaillant que trois mille louis qu'on lui
trouva, et force vaisselle et porcelaines. Il avait les misères des
femmes qui l'avaient fait subsister, et il ne craignait rien tant qu'une
salière renversée. Il se préparait à aller en Alsace. Dînant chez Biron,
depuis duc, pair et maréchal de France, une salière se répandit sur lui.
Il pâlit, se trouva mal, dit qu'il était mort\,; il fallut sortir de
table et le mener chez lui. On ne put lui remettre le peu de tête qu'il
avait. La fièvre le prit le soir, et il mourut quatre jours après,
n'emportant de regrets que ceux de ses créanciers. Il n'avait point eu
d'enfants de deux femmes qu'il avait épousées, bien sucées, et fort mal
vécu avec elles. Il laissa la dernière veuve qui était Rabodanges, veuve
d'un Médavy-Grancey, chef d'escadre, dont elle avait deux filles
M\textsuperscript{me}s de Flavacourt et de Hautefeuille qui a bien fait
parler d'elle.

Le prince de Fürstemberg, qui avait toujours laissé sa femme et ses
filles à Paris, mourut en Allemagne. Il y avait des années infinies
qu'il y était retourné, et n'en était plus sorti. Il avait toute la
confiance de l'électeur de Saxe\,; et lorsque ce prince fut élu roi de
Pologne, il le laissa gouverneur de son électorat avec toute autorité,
qu'il y a conservée toute sa vie. Il était fort riche, mais en Allemagne
les filles n'héritent point.

Le prince de Robecque ne jouit pas longtemps du régiment des gardes
wallonnes qu'il avait eu à la disgrâce du duc d'Havré. Il mourut assez
subitement et assez jeune, sans enfants de la fille du comte de Salre.
Son frère le comte d'Esterres hérita de sa grandesse, prit son titre, et
obtint sa Toison. Il servait en France. Les gardes wallonnes furent
données au marquis de Risbourg.

La duchesse d'Albe épousa en ce même temps l'abbé de Castiglione qu'elle
avait emmené d'ici retournant à Madrid. J'ai assez parlé d'eux à
l'avance pour me contenter de dire ici que le pape lui permit de
conserver des pensions considérables qu'il avait sur des bénéfices, et
qu'en faveur de ce mariage, le roi d'Espagne le fit grand de la première
classe, et lui donna une place de gentilhomme de sa chambre, dont aucun
n'avait plus nul exercice depuis longtemps. Il prit le nom de duc de
Solferino.

\hypertarget{chapitre-iii.}{%
\chapter{CHAPITRE III.}\label{chapitre-iii.}}

1716

~

{\textsc{Louville envoyé secrètement en Espagne.}} {\textsc{- Sa
commission, très importante et très secrète.}} {\textsc{- Incapacité
surprenante du duc de Noailles.}} {\textsc{- Jalousie extrême du
maréchal d'Huxelles.}} {\textsc{- Craintes et manèges intérieurs
d'Albéroni en Espagne.}} {\textsc{- Insolence de l'inquisition sur les
deux frères Macañas.}} {\textsc{- Cardinal Acquaviva chargé, au lieu de
Molinez, des affaires d'Espagne à Rome.}} {\textsc{- La peur qu'Albéroni
et Aubenton ont l'un de l'autre les unit.}} {\textsc{- Giudice ôté
d'auprès du prince des Asturies et du conseil.}} {\textsc{- Popoli fait
gouverneur du prince des Asturies\,; sa figure et son caractère.}}
{\textsc{- Mécontentement réciproque entre l'Espagne et l'Angleterre.}}
{\textsc{- Fourberie d'Albéroni pour en profiter.}} {\textsc{- Les
Anglais, en peine du chagrin du roi d'Espagne sur leur traité avec
l'empereur, le lui communiquent, et en même temps les propositions que
leur fait la France, et leur réponse.}} {\textsc{- Malignité contre le
régent pour le brouiller avec le roi d'Espagne.}} {\textsc{- Adresse de
Stanhope pour se défaire de Monteléon en Angleterre, et gagner Albéroni,
qui passe tout aux Anglais.}} {\textsc{- Albéroni, gagné par la
souplesse de Stanhope, donne carte blanche aux Anglais pour signer avec
eux une alliance défensive.}} {\textsc{- Embarras et craintes diverses
de Bubb, secrétaire et seul ministre d'Angleterre à Madrid.}} {\textsc{-
Prétention des Anglais insupportable pour le commerce, qu'Albéroni ne
leur conteste seulement pas.}} {\textsc{- Bassesses et empressement pour
les Anglais.}} {\textsc{- Crainte d'Albéroni des Parmesans, qu'il
empêche de venir en Espagne.}} {\textsc{- Louville à Madrid\,; en est
renvoyé sans pouvoir être admis.}} {\textsc{- Il en coûte Gibraltar à
l'Espagne.}} {\textsc{- Impostures d'Albéroni sur Louville.}} {\textsc{-
Le régent et Albéroni demeurent toujours piqués l'un contre l'autre du
voyage de Louville.}}

~

La négociation entre la France et l'Angleterre prenait quelquefois une
face plus riante. Toutes deux désiraient y attirer l'Espagne par des
vues différentes. Le régent en sut profiter pour ménager à l'Espagne la
restitution actuelle de Gibraltar, qui était la chose du monde qui
l'intéressait davantage. Gibraltar ne laissait pas d'être à charge au
roi d'Angleterre bien comme il était avec les Barbaresques, et fort
supérieur en marine à l'Espagne. Avec le Port-Mahon, Gibraltar lui était
inférieur en usage et en importance à la dépense et à la consommation
qu'il lui en coûtait. Il consentit donc à le rendre à l'Espagne
moyennant des riens qui ne valent pas s'en souvenir\,; mais comme il ne
voulait pas s'exposer aux cris du parti qui lui était contraire, il
exigea un grand secret et une forme. Pour le secret, il voulut que rien
de cela passât par Albéroni, ni par aucun ministre espagnol ni anglais,
mais directement du régent au roi d'Espagne par un homme de confiance du
choix du régent, et de condition à être admis à parler au roi d'Espagne
tête-à-tête. La forme fut que cet homme de confiance du régent serait
chargé de sa créance, d'une lettre touchant l'affaire du traité,
c'est-à-dire d'un papier de ces riens demandés par le roi d'Angleterre
prêt à être signé, et d'un ordre positif du roi d'Angleterre, écrit et
signé de sa main, au gouverneur de Gibraltar de remettre cette place au
roi d'Espagne à l'instant que l'ordre lui serait rendu, et de se retirer
avec sa garnison, etc., à Tanger. Pour l'exécution, un général espagnol
devait marcher subitement à Gibraltar sous prétexte des courses de sa
garnison\,; et sous celui d'envoyer sommer le gouverneur, lui porter
l'ordre du roi d'Angleterre, et en conséquence être reçu et mis en
possession de la place. La couleur était faible, mais c'était l'affaire
du roi d'Angleterre.

Le duc de Noailles était alors dans la grande faveur et voulait tout
faire. Il ne faut pas être glorieux. Je ne sus rien de tout cela que du
second bond et par Louville avant que le régent m'en eût rien dit, qui
ne m'en parla qu'après. Noailles avec qui seul le choix se fit, dont le
maréchal d'Huxelles fut outré, crut faire merveilles de proposer
Louville, comme ayant eu longtemps autrefois toute la confiance du roi
d'Espagne, et le connaissant mieux qu'aucun autre qu'on y pût envoyer.
Sans être habile, je me serais défié du roi d'Angleterre proposant une
pareille mécanique. Il ne pouvait ignorer avec quel soin et quelle
jalousie la reine et Albéroni tenaient le roi d'Espagne enfermé,
inaccessible à qui que ce pût être, et que le moyen certain d'échouer
était d'entreprendre de lui parler à leur insu, ou malgré eux et sans
eux. Quant au choix, de tout ce qu'il y avait en France, Louville était
à mon avis le dernier sur qui il dût tomber. Plus il avait été bien avec
le roi d'Espagne et avant dans sa confiance\,; plus son arrivée
ferait-elle peur à la reine et à Albéroni, et plus mettraient-ils tout
en usage pour ne pas laisser rapprocher un homme dont ils craindraient
tout pour leur crédit et leur autorité. Je le dis à Louville qui n'en
disconvint pas, mais qui se contenta de me répondre que dans sa surprise
il n'avait osé refuser, et que, de plus, s'il réussissait à percer,
l'acquisition de Gibraltar était si importante qu'il y aurait bien du
malheur si elle ne lui valait de rapporter ce qui lui était dû de ses
pensions d'Espagne, qui était pour lui un gros objet. Être choisi et
parti ne fut presque que la même chose. Il eut pourtant loisir de me le
venir dire, et raisonner avec moi, et de me venir trouver le lendemain
encore, et de me conter que M. le duc d'Orléans lui ayant parlé avec
bonté et avec confiance sur ce dont il le faisait porteur, en présence
du seul duc de Noailles, les avait promptement renvoyés chez le duc de
Noailles qui lui devait faire et donner ses expéditions.

Le duc de Noailles l'emmena donc dans sa bibliothèque, l'y promena, lui
parla de ses livres, puis de son administration des finances, chercha
des louanges tant qu'il put. Louville, qui devait partir le
surlendemain, et qui n'était averti que de la veille, mourait
d'impatience. À la fin il l'interrompit pour le ramener à son fait. Ce
ne fut pas sans peine, ni sans essuyer encore d'autres disparates
entièrement étrangères à leur sujet. Enfin il fallut prendre la plume.
Noailles se mit à vouloir faire la lettre de M. le duc d'Orléans au roi
d'Espagne. Au bout de quelques mots, pauses langues et un peu de
conversation, puis une ligne ou deux, et pause encore, puis ratures et
renvois. Elle ne fut pas à moitié qu'il voulut la refondre\,; c'était
son terme favori. Il la fondit et refondit si bien qu'elle demeura
fondue, et qu'il n'en resta rien. Louville pétillait. À la fin il lui
proposa de la lui laisser faire. Il l'écrivit tout de suite. Noailles y
mit des points et des virgules, et ne trouva rien d'omis ni à changer.
Après il voulut travailler à l'instruction\,; même cérémonie. Louville
la fit tout de suite sur son bureau. Tout cela dura plus de quatre
heures. C'en était trois plus qu'il ne fallait. Cette aventure ne
m'apprit rien de nouveau. Celle de Fontainebleau, lorsque Bolingbroke y
vint pour la paix particulière de la reine Anne, et qui a été racontée
en son temps, m'avait bien prouvé la parfaite incapacité du duc de
Noailles d'écrire sur la moindre affaire, avec tout son esprit et son
jargon, et les plumes d'autrui dont avec tant d'art il sait se faire
honneur, et les donner pour siennes.

Quand la lettre fut signée du régent le lendemain matin, en présence de
Louville, en prenant congé de lui, il lui ordonna de voir le maréchal
d'Huxelles, de lui porter l'instruction à signer, qui ne disait pas un
mot de l'affaire, mais seulement de la conduite pour voir et parler au
roi d'Espagne, etc. Louville eut beau représenter l'inutilité d'une
visite où sûrement il serait mal reçu, Noailles qui voulait tout faire,
mais qui en même temps craignait tout le monde, insista croyant par là
ménager le maréchal d'Huxelles\,; il fallut donc y aller, et ce fut en
sortant de chez lui que Louville revint chez moi. Il fut reçu comme un
chien dans un jeu de quilles\,; ce fut son expression. Le maréchal,
fronçant le sourcil, lui dit qu'il n'avait qu'à lui souhaiter bon
voyage\,; qu'il n'avait rien {[}à{]} lui dire\,; qu'il ne pouvait parler
de ce qu'il ne savait point\,; qu'il n'avait rien à mander dans ce
pays-là\,; lui tourna le dos et le laissa. Il fut enragé de se voir
passer la plume par le bec, s'en prit à Louville qu'il crut avoir brassé
toute cette intrigue, et ne le lui a jamais pardonné. Je soupçonne que
le duc de Noailles ne fut pas fâché d'en laisser tomber la haine sur
Louville, et que le timide et jaloux maréchal aima mieux s'en prendre à
l'un qu'à l'autre. Le projet était que Louville, prenant la route
détournée du pays de Foix et de l'Aragon, arrivât dans Madrid sans que
personne eût pu avoir le moindre vent de son voyage. Je ne sais si le
maréchal d'Huxelles se tint bien obligé au secret, qui malgré toutes les
précautions de Louville fut très mal gardé.

Les soupçons du roi d'Espagne contre Albéroni se fortifiaient. La reine
se contentait de l'exhorter à souffrir avec patience, lui se plaignait
de sa mollesse, de sa complaisance pour le roi, de ne pas surmonter les
défiances continuelles d'un esprit faible et irrésolu, capable de se
livrer à qui s'en voudrait emparer pour en faire un mauvais usage. Il
trouvait la reine indolente, haïssant la peine et les affaires, ne
cherchant que son repos. Il l'exhortait à ne pas souffrir qu'on les
exclût l'un et l'autre du gouvernement des affaires, et à craindre,
parmi cette confusion de nations et de langues qui inondaient la cour
d'Espagne, la cabale suivie et dissimulée des Espagnols qui voulaient
tout rappeler à leur ancien gouvernement. Il l'avertissait que, si elle
cessait d'avoir l'autorité dans les affaires, elle ne devait plus
compter sur aucun crédit ni considération dans le monde, ni sur aucun
respect de ses sujets. Les désordres étaient au dernier point en
Espagne, les peuples accablés d'impôts, les seigneurs dans la crainte et
le mépris, la noblesse à la mendicité\,; ni troupes, ni finances, ni
marine, ni commerce, et personne qui pût remédier à tant de maux, et la
maison d'Autriche attentive avait encore force partisans. Albéroni
vantait ses projets, et se vantait de tout raccommoder s'il était
soutenu à les exécuter. En se louant il décriait le cardinal del
Giudice, et avait persuadé à la reine qu'il était très dangereux à
laisser auprès du prince des Asturies.

On se souviendra de l'affaire de Macañas, qui a été racontée en son
temps. Son frère, qui était dominicain, fut mis en prison par
l'inquisition, qui refusa au roi d'Espagne de lui en remettre le
procès\,; et en même temps ce tribunal déclara par un décret Macañas
hérétique, et le cita à comparaître dans quatre-vingt-dix jours. C'était
un nouvel attentat, après celui du refus du procès de son frère. Macañas
depuis le décret que Giudice fit contre lui dans Marly, et qui le retint
si longtemps à Bayonne sans pouvoir rentrer en Espagne, était en pays
étrangers connu pour être ministre du roi d'Espagne. Ce prince et la
reine s'en voulurent prendre à Giudice comme grand inquisiteur et mobile
de procédés si insolents, et le chasser. Albéroni leur fit peur de la
conjoncture, et de le faire passer pour un martyr\,; c'est qu'il
craignit que Rome ne s'en prit à lui-même, et que, quelque haine qu'il
eût contre Giudice, il avait encore plus d'affection à son chapeau qu'il
craignit d'éloigner, mais il lui donna un autre dégoût. Il fit décharger
Molinez du soin des affaires d'Espagne à Rome, comme trop vieux et
incapable de les conduire, et les fit donner au cardinal Acquaviva.
Giudice haïssait fort toute cette maison, et le cardinal Acquaviva en
particulier qu'il regardait comme l'ami d'Albéroni, et le promoteur de
son chapeau.

Aubenton, quoique appuyé directement du pape, et personnellement honoré
de toute sa confiance et d'un commerce particulier de lettres avec lui,
se sentit trop faible contre Albéroni qui n'était qu'un avec la reine,
laquelle n'aimait point les jésuites, et n'en avait jamais voulu d'aucun
pour confesseur. Albéroni, de sa part, craignait doublement Aubenton,
qui avait la confiance du roi d'Espagne, jusqu'à lui renvoyer
quelquefois des affaires à lui seul, et il ne le redoutait pas moins
pour son chapeau à Rome. Cette frayeur réciproque relia ensemble deux
ambitieux qui ne connurent jamais que l'autorité et la fortune. Le
cardinal del Giudice fut la victime de leur ralliement. La première
nouvelle qu'il en eut fut par un billet de Grimaldo qui, sous le nom de
secrétaire d'État, l'était moins que secrétaire d'Albéroni, dont il
avait ordre d'exécuter et d'expédier tous les ordres. Par ce billet, le
cardinal eut ordre de se retirer d'auprès du prince des Asturies, auquel
sa place de grand inquisiteur ne lui laissait pas le loisir de donner
tous les soins nécessaires. Moins surpris que touché, il répondit avec
soumission. Il demanda en même temps la permission d'écrire au pape pour
se démettre aussi de sa charge de grand inquisiteur, qu'il obtint
aussitôt. Après quoi il offrit de se retirer dans la ville qu'il
plairait au roi de lui prescrire, où il y aurait tribunal d'inquisition,
jusqu'à ce que la réponse du pape lui permit de sortir d'Espagne. Au
milieu d'une disgrâce si marquée, il n'était pas si détaché qu'il ne
continuât d'assister au conseil, où il n'a voit plus depuis longtemps
que le vain nom de premier ministre. Cela ne dura que quelques jours\,;
il reçut un nouveau billet de Grimaldo qui, par ordre du roi, lui
ordonnait de s'abstenir de se trouver au conseil. En même temps le duc
de Popoli fut nommé gouverneur du prince des Asturies.

Popoli était un seigneur napolitain, frère du feu cardinal Cantelmi,
archevêque de Naples. J'ai parlé de lui lorsqu'il passa à Versailles, et
que le roi lui promit l'ordre du Saint-Esprit qu'il lui envoya depuis,
et lorsqu'il fut fait par le roi d'Espagne, à très bon marché, capitaine
général et général de l'armée de Catalogne, qu'il laissa au maréchal de
Berwick qui fit le siège de Barcelone. Il se déshonora partout sur le
courage, sur l'avarice, sur l'honneur, sur tous chapitres, ce qui ne
l'empêcha pas d'être grand d'Espagne, chevalier de la Toison, grand
maître de l'artillerie, capitaine des gardes du corps de la compagnie
italienne, enfin gouverneur du prince, quoiqu'il eût empoisonné sa
femme, héritière de la branche aînée de leur maison, dont par là il
avait eu tous les biens, belle, aimable, jeune, qui était fort bien avec
la reine dont elle était dame du palais, et qui ne donnait point de
prise sur sa conduite. Personne ne doutait de ce crime, et lorsque j'ai
été en Espagne, j'en ai ouï parler à la reine comme d'une chose
certaine, dont elle avait horreur. Je crois pourtant qu'il ne le commit
que depuis qu'il fut mis auprès du prince des Asturies, ou fort peu
avant, et que lors la chose n'était pas si avérée. D'ailleurs Popoli
avait grand air et grande mine, la taille et le visage mâle et agréable
des héros, beaucoup d'esprit, d'art, de manège\,; suprêmement faux et
dangereux, avec tant le langage, les grâces, les façons, les manières du
maréchal de Villeroy, à un point qui surprenait toujours. Quoique
Italien, il n'aimait point Albéroni\,; il fraya toujours avec la cabale
espagnole, dont il ne se cachait pas.

L'alliance défensive traitée entre l'Espagne et l'Angleterre s'était
refroidie par la signature de celle de cette dernière couronne avec
l'empereur. L'Espagne criait contre la mauvaise foi des Anglais, et ne
doutait pas que le traité qu'ils venaient de conclure ne fût contraire à
ses intérêts, et aux plus essentiels articles de la paix d'Utrecht. Les
Anglais se plaignaient avec hauteur des vexations que leurs marchands
souffraient sans cesse de l'Espagne\,; ce qui désolait tout commerce.
Ces plaintes mutuelles retombaient sur Albéroni, depuis longtemps chargé
seul de cette négociation\,; mais lui se crut assez habile pour profiter
de cette situation, prit un air de franchise et de disgrâce avec le
secrétaire que l'Angleterre tenait pour tout ministre à Madrid. Il lui
dit que les mauvais serviteurs du roi d'Espagne l'avaient tellement
décrié dans son esprit faible, défiant, incertain, irrésolu, comme gagné
par les Anglais, qu'il n'osait plus ouvrir la bouche de rien qui les
regardât, et gémissait devant ce secrétaire sur le préjudice que ces
pernicieux discours causaient aux intérêts du roi d'Espagne. Le but de
cette feinte était de se rendre cher aux Anglais, en les persuadant
qu'il s'exposait pour eux à déplaire au roi d'Espagne\,; gagner du temps
et attendre les événements\,; observer la conduite de la Hollande\,;
profiter du désir de cette république d'établir son commerce avec
l'Espagne\,; enfin traiter avec elle seule, ou avec l'Angleterre seule,
ou avec toutes les deux, suivant qu'il trouverait jour et convenance. Il
fut une nuit trouver Riperda chez lui, par ordre de la reine, pour le
presser d'entrer en traité. Sur quoi cet ambassadeur de Hollande
pressait ses maîtres de ne pas manquer une occasion si favorable,
{[}et{]} les assura qu'ils obtiendraient toutes conditions les plus
favorables qui les pourraient conduire à chasser d'Espagne les Français
sans retour.

Bubb, secrétaire d'Angleterre à Madrid, était de son côté fort en peine
des fâcheuses impressions que le traité de l'empereur avec le roi de la
Grande-Bretagne avait fait sur l'esprit du roi d'Espagne, lorsqu'il
reçut ordre de rendre compte au roi d'Espagne, par Albéroni, de tous les
points de ce traité, de lui en communiquer même la copie, et pour comble
de bonne foi de leur part, de lui communiquer aussi les offres que la
France leur faisait pour un traité de ligue défensive avec eux, même le
projet de la France, et la réponse que le roi d'Angleterre y avait
faite. Stanhope, qui voulait se réserver le premier mérite d'une telle
confiance, adressa à Bubb, par le même courrier, une lettre de sa main
pour Albéroni pleine de toutes les expressions qui pouvaient le flatter
davantage, et de toutes celles qu'il crut les plus propres à flatter le
roi d'Espagne. Sa malignité contre la France n'y oublia pas qu'elle y
sollicitait avec empressement la confirmation du traité d'Utrecht, le
seul qui pût faire peine personnellement au roi d'Espagne\,; et relevait
l'attention obligeante du roi son maître à éluder la demande de M. le
duc d'Orléans, et l'industrie à tourner la réponse d'une manière qui fût
agréable au roi d'Espagne. Stanhope qui, comme on l'a vu, voulait se
défaire de Monteléon, qu'il trouvait trop éclairé et trop habile,
profita de l'occasion contre un homme qu'il savait n'être ni créature
d'Albéroni, ni fort lié avec lui, et qui avait toujours fort
publiquement témoigné qu'il était persuadé que l'intérêt de l'Espagne
était d'être toujours unie avec la France. Ainsi Stanhope l'attaqua sans
ménagement par la même lettre, et y exagéra son étonnement de voir un
ambassadeur d'Espagne solliciter, de concert avec la France, la
confirmation du traité d'Utrecht, pendant que le roi d'Angleterre
évitait d'en parler, uniquement par l'attention qu'il avait aux intérêts
personnels du roi d'Espagne. Quelque satisfaction qu'Albéroni eût de
cette dépêche, il fut encore plus sensible à l'ordre que Bubb reçut en
même temps d'accuser le cardinal del Giudice d'avoir favorisé les
intérêts du Prétendant, et de demander formellement au roi d'Espagne
d'éloigner ce cardinal et ses adhérents, et de choisir des ministres
habiles et intègres.

Malgré tant de satisfaction, Albéroni joua la comédie\,: il contrefit
l'homme éreinté sur les Anglais par ses ennemis auprès du roi d'Espagne,
auquel il n'osait plus en parler, et quand il crut avoir assez joué, il
promit, comme par effort pour le bien, de se hasarder encore une fois
là-dessus auprès de son maître, et de donner promptement sa réponse. Il
la fit bientôt en effet il dit à Bubb que l'engagement pris entre
l'empereur et le roi d'Angleterre de se garantir mutuellement, non
seulement les États dont ils se trouvaient en possession actuelle, mais
encore ceux qu'ils pourraient acquérir dans la suite, avait fait faire
de sérieuses réflexions au roi d'Espagne, qui trouvait cet article
directement contre ses intérêts. Bubb ne put bien excuser cet endroit du
traité, mais il avait affaire à un homme qui voulait être persuadé en
faveur des Anglais. Il demanda donc à Bubb si ce traité portait
exclusion de toute autre alliance. Bubb répondit que non, et cita pour
preuve le traité actuellement sur le tapis entre la France et
l'Angleterre. Il se trouvait en même temps embarrassé de n'avoir point
d'instruction ni de pouvoir pour traiter avec l'Espagne. Albéroni le
tira de peine en lui disant que Stanhope lui offrait par sa lettre de
traiter, et qu'il l'avait offert verbalement à Monteléon. C'était le
matin qu'ils conféraient\,; le soir du même jour Giudice eut ordre de se
retirer absolument d'auprès du prince des Asturies\,; et le premier
ministre, satisfait du dernier coup porté à ce cardinal par les Anglais,
avertit Bubb que le roi d'Espagne était disposé à signer une alliance
défensive avec le roi de la Grande-Bretagne. Quelque désir qu'en eût ce
secrétaire, il se trouvait arrêté faute d'instruction et de pouvoir\,;
mais Albéroni, plus pressé que lui encore, répondit sur sa question de
la nature du traité pour en écrire\,: \emph{telle alliance défensive
qu'il plaira au roi d'Angleterre}. Enfin il lui dit qu'il écrirait
lui-même à Stanhope, et promit à Bubb qu'eux deux seuls en Espagne
auraient la connaissance de cette négociation, et que Monteléon n'en
serait point instruit. Il ajouta que ce serait au roi d'Angleterre à
choisir ceux de ses ministres qu'il voudrait admettre dans la confidence
de ce secret. Albéroni compta bien intéresser par là ce secrétaire. Tout
ministre employé dans une cour met sa gloire à y faire des traités, et
son dégoût à se voir enlever une négociation qu'il a entamée. Celui-ci
écrivit tout de son mieux pour qu'on lui envoyât instruction et
pouvoirs, et n'oublia rien de ce qu'il put représenter de flatteur pour
le roi d'Angleterre, tant sur les avantages du commerce que sur la
médiation qui lui pouvait résulter un jour entre l'empereur et l'Espagne
sur les affaires d'Italie, et se faire considérer par ces deux
puissances. Il pressa l'envoi de ce qu'il demandait au nom du ministre
seul confident de Leurs Majestés Catholiques, et envoya la lettre
d'Albéroni avec cette dépêche par le même courrier extraordinaire qui
lui avait apporté celles dont on vient de parler.

Dans cette situation agréable, Bubb ne laissait pas d'être mal à son
aise. Il se défiait des Espagnols et des Français, beaucoup plus encore
des Hollandais. Ceux-ci se faisaient un mérite de leur refus d'entrer
dans le traité de l'empereur et de l'Angleterre, et publiaient qu'ils
n'y entreraient jamais, et rien ne flattait plus le roi d'Espagne, qui
regardait ce traité comme un obstacle à ses vues de recouvrer un jour ce
qu'il avait perdu en Italie. Bubb sentait aussi tout le poids de
l'affaire du commerce dont il était chargé\,; que le traité entre
l'empereur et l'Angleterre rendait plus difficile. Il était fatigué des
plaintes continuelles des marchands anglais et de la lenteur et de
l'indécision de la cour de Madrid. Il n'attendait aucun succès de la
proposition qu'Albéroni lui avait faite de faire examiner et décider les
plaintes des marchands par des commissaires nommés de part et d'autre,
et il se laissait entendre qu'il fallait profiter pour finir ces
affaires de la conjoncture présente de traiter une alliance avec
l'Espagne, ou renoncer à tout commerce\,; fixer un temps à l'Espagne de
faire justice aux Anglais, et après l'expiration de ce terme déclarer
tout commerce interdit. Les négociants veulent toujours que leur intérêt
particulier soit la règle de l'État, et ne connaissent de bien public
que leur gain particulier.

Bubb craignait là-dessus la compagnie de la mer du Sud établie à
Londres, et {[}qu'elle{]} n'eût le crédit de lui attirer des ordres qui
troublassent sa négociation. Elle prétendait que la mesure d'Angleterre,
qui lui était plus avantageuse que celle d'Espagne, servît de règle à la
cargaison de leurs vaisseaux, et l'ordre commun entre toutes les nations
est que la mesure de la charge d'un vaisseau soit toujours celle du lieu
où il aborde. Cette prétention était insupportable\,; Bubb la jugeait
telle, et l'artifice en sautait aux yeux\,; ainsi il souhaitait avec
impatience que tous les points sur le traité de l'\emph{asiento}, qui
étaient encore en disputé, fussent incessamment réglés et signés. Sa
crainte fut vaine. Albéroni avait encore plus d'envie d'avancer que
lui-même. Il ne fit pas la plus légère attention à cette clause, et il
assura Bubb que le roi d'Espagne avait donné ses ordres pour la
signature du traité, qui seraient incessamment exécutés, et qui le
furent en effet. Albéroni était trop content de la disposition des
Anglais et du plaisir qu'ils lui avaient fait de s'intéresser à le
défaire du cardinal del Giudice, pour leur donner aucun prétexte de
changer. Il écrivit donc à Stanhope, dans les termes les plus farts,
pour lui témoigner la reconnaissance que le roi d'Espagne conserverait
toujours de la confiance avec laquelle le roi d'Angleterre lui avait
fait communiquer les propositions et les négociations de la France, et
la tendre amitié que Sa Majesté Catholique aurait toujours
personnellement pour Sa Majesté Britannique. Il blâma Monteléon,
condamna l'alliance qu'il avait proposée, comme n'étant qu'une simple
ratification du traité d'Utrecht, faite de concert avec la France, à qui
cet ambassadeur d'Espagne était tout dévoué, crime irrémissible dans
l'esprit de Stanhope, à qui il laissa la décision de tout.

Le fourbe se vantait à ses amis qu'il ne voulait qu'amuser les Anglais,
et se donner le temps de voir la résolution que prendraient les
Hollandais sur les instances qui leur étaient faites d'entrer dans le
traité signé entre l'empereur et l'Angleterre. Il prétendait savoir
qu'ils en étaient si mécontents qu'ils espéraient que le parlement
d'Angleterre ferait quelque jour un crime au roi Georges d'y avoir
préféré ses intérêts personnels d'usurpation sur la Suède aux intérêts
de la nation anglaise. Comme il ne s'occupait du dehors que pour sa
fortune, il l'était encore plus du dedans. Il craignait tout des
Parmesans, pour qui la reine avait de l'affection, et que quelqu'un
d'eux n'enlevât sa faveur auprès d'une princesse légère et facile à se
laisser conduire. Il empêcha, par le duc de Parme, qu'elle fît venir en
Espagne le mari de sa nourrice et leur fils capucin, et s'assura par ce
souverain qu'il n'en viendrait aucun autre qui pût lui faire ombrage
auprès d'elle. Les vapeurs du roi donnaient de la crainte aux médecins.
Ils en avaient aussi sur la santé du prince des Asturies\,; ainsi la
reine régnait en plein et en assurance, et Albéroni se sentait plus
puissant que jamais.

Ce fut dans ce point que Louville arriva à Madrid, et vint descendre et
loger chez le duc de Saint-Aignan, qui fut dans une grande surprise, et
qui n'en avait pas eu le moindre avis. Un courrier fortuit, qui
rencontra Louville à quelque distance de Madrid, le dit à Albéroni. On
peut juger, aux soupçons et à la jalousie dont il était tourmenté,
quelle fut pour lui cette alarme. Il n'ignorait pas quel était Louville,
le crédit qu'il avait eu auprès du roi d'Espagne, la violence que
M\textsuperscript{me} des Ursins et la feue reine lui avaient faite pour
le lui arracher\,; aussi la frayeur qu'il conçut de cette arrivée
inattendue fut-elle si pressante qu'il ne garda nulle mesure pour s'en
délivrer. Il dépêcha sur-le-champ un ordre par un courrier à la
rencontre de Louville, pour lui défendre d'approcher plus près de
Madrid. Le courrier le manqua\,; mais un quart d'heure après qu'il eut
mis pied à terre, il reçut un billet de Grimaldo, portant un ordre du
roi d'Espagne de partir à l'heure même. Louville répondit qu'il était
chargé d'une lettre de créance du roi, et d'une autre de M. le duc
d'Orléans pour le roi d'Espagne, et d'une commission pour Sa Majesté
Catholique, qui ne lui permettait pas de partir sans l'avoir exécutée.
M. de Saint-Aignan manda la même chose à Grimaldo. Sur cette réponse, un
courrier fut dépêché à l'heure même au prince de Cellamare, avec ordre
de demander le rappel de Louville, et de déclarer que le roi d'Espagne
avait sa personne si désagréable qu'il ne voulait ni le voir, ni laisser
traiter avec lui aucun de ses ministres. La fatigue du voyage, suivie
d'une telle réception, causa dans la nuit une attaque de néphrétique à
Louville, qui en avait quelquefois, de sorte qu'il se fit préparer un
bain, dans lequel il se mit sur la fin de la matinée.

Albéroni vint lui-même le voir chez le duc de Saint-Aignan, pour lui
persuader de s'en aller sur-le-champ. L'état où on lui dit qu'il était
ne put l'arrêter\,; il le vit malgré lui dans son bain. Rien de plus
civil que les paroles, ni de plus sec, de plus négatif, de plus absolu
que leur sens. Albéroni plaignit son mal et la peine de son voyage,
aurait souhaité de l'avoir su pour le lui avoir épargné, et désiré
pouvoir surmonter la répugnance du roi d'Espagne à le voir, du moins à
lui permettre de se reposer quelques jours à Madrid\,; qu'il n'avait pu
rien gagner sur son esprit, ni s'empêcher d'obéir au très exprès
commandement qu'il en avait reçu de venir lui-même lui porter ses ordres
de partir sur-le-champ, et de les voir exécuter. Louville lui parut dans
un état qui portait avec soi l'impossibilité de partir. Il en admit donc
l'excuse, mais en l'avertissant qu'elle ne pouvait durer qu'autant que
le mal, et que l'accès passé elle ne pourrait plus être admise. Louville
insista sur ses lettres de créance qui lui donnaient caractère public
pour exécuter une commission importante de la part du roi, neveu du roi
d'Espagne, telle que Sa Majesté Catholique ne pouvait refuser de
l'entendre directement de sa bouche, et qu'il aurait lieu de regretter
de n'avoir pas écoutée. La dispute fut vive et longue malgré l'état de
Louville, qui ne put rien gagner. Il ne laissa pas de demeurer cinq au
six jours chez le duc de Saint-Aignan, et de le faire agir comme
ambassadeur pour lui obtenir audience, quoique M. de Saint-Aignan, ami
de Louville, ne laissât pas de se sentir du secret qu'il lui fit
toujours, selon ses ordres, de l'objet de sa mission.

Louville n'osait aller chez personne, de peur de se commettre\,;
personne aussi n'osa le venir chercher. Il se hasarda pourtant, par
curiosité, d'aller voir passer le roi d'Espagne dans une rue, et pour
tenter si, en le voyant, il ne serait pas tenté de l'entendre, en cas,
comme il était très possible, qu'on lui eût caché son arrivée. Mais
Albéroni avait prévu à tout. Louville vit en effet passer le roi, mais
il lui fut impossible de faire que le roi l'aperçût. Grimaldo vint enfin
signifier à Louville un ordre absolu de partir, et avertir le duc de
Saint-Aignan que le roi d'Espagne était si en colère de l'opiniâtreté de
ce délai, qu'il ne pouvait lui répondre de ce qui arriverait si le
séjour de Louville était poussé plus loin, et qu'on ne se trouvât obligé
à manquer aux égards qui étaient dus à tout ministre représentant, et
plus qu'à tous à un ambassadeur de France. Tous deux virent bien que
l'audience à espérer était une chose entièrement impossible\,; que, par
conséquent un plus long séjour de Louville n'était bon qu'à se commettre
à une violence qui, par son éclat, brouillerait les deux couronnes\,:
ainsi au bout de sept au huit jours Louville partit, et s'en revint
comme il était allé. Albéroni commença à respirer de la frayeur extrême
qu'il avait eue. Il s'en consola par un essai de sa puissance qui le mit
à couvert de plus craindre que personne approchât du roi d'Espagne sans
son attache, ni qu'aucune affaire se pût traiter sans lui. Il en coûta
Gibraltar à l'Espagne, qu'elle n'a pu recouvrer depuis. Telle est
l'utilité des premiers ministres.

Celui-ci répandit en Espagne et en France que le roi d'Espagne avait
pris une aversion mortelle contre Louville, depuis qu'il l'avait chassé
d'Espagne pour ses insolences et ses entreprises\,; qu'il ne le voulait
jamais voir, et se tenait offensé qu'il eût osé passer les Pyrénées\,;
qu'il n'avait ni commission ni proposition à faire\,; qu'il avait trompé
le régent en lui faisant accroire que, s'il pouvait trouver un prétexte
de reparaître devant le roi d'Espagne, ce prince en serait ravi par son
ancienne affection pour lui, et que, connaissant ce prince autant qu'il
le connaissait, il rentrerait bientôt dans son premier crédit, et ferait
faire à l'Espagne tout ce que la France voudrait\,; qu'en un mot il
n'était venu que pour essayer à tirer quelque chose de ce qui lui était
dû des pensions qu'il s'était fait donner en quittant le roi d'Espagne,
mais qu'il n'avait pas pris le chemin d'en être sitôt payé. Il fallait
être aussi effronté que l'était Albéroni pour répandre ces impostures.
On n'avait pas oublié en Espagne comment M\textsuperscript{me} des
Ursins avait fait renvoyer Louville\,; combien le roi d'Espagne y avait
résisté\,; qu'elle n'avait pu venir à bout que par la France, et par ses
intrigues avec M\textsuperscript{me} de Maintenon contre le cardinal et
l'abbé d'Estrées et lui\,; et que le roi, affligé au dernier point,
cédant aux ordres donnés de France à Louville, lui avait en partant
doublé et assigné ses pensions qui lui avaient été longtemps payées, et
donné de plus une somme d'argent et le gouvernement de Courtray, qu'il
n'a perdu que par les malheurs de la guerre qui suivirent la perte de la
bataille de Ramillies. À l'égard de la commission, la nier était une
impudence extrême, d'un homme aussi connu que Louville, qui vient
descendre chez l'ambassadeur de France, qui dit avoir des lettres de
créance du roi et du régent, et une commission importante dont il ne
peut traiter que directement et seul avec le roi d'Espagne, et pour
l'audience duquel l'ambassadeur de France s'emploie au nom du roi. Rien
de si aisé que de couvrir Louville de confusion, s'il avait allégué
faux, en lui faisant montrer ses lettres de créance\,; s'il n'en eût
point eu, il serait demeuré court, et alors n'ayant point de caractère,
Albéroni aurait été libre du châtiment. Que si, avec des lettres de
créance, il n'eût eu qu'un compliment à faire pour s'introduire et
solliciter son payement, Albéroni l'aurait déshonoré bien aisément de
n'avoir point de commission, après avoir tant assuré qu'il était chargé
d'une fort importante. Mais la toute-puissance dit et fait impunément
tout ce qu'il lui plaît.

Louville de retour, il fallut renvoyer au roi d'Angleterre tout ce que
Louville avait porté en Espagne pour Gibraltar\,; et cette affaire
demeura comme non avenue, sinon qu'elle piqua fort Albéroni contre le
régent d'avoir voulu faire passer une commission secrète au roi
d'Espagne à son insu, et par un homme capable de le supplanter, et le
régent contre Albéroni qui avait fait avorter le projet avec tant
d'éclat, et lui avait osé faire sentir quelle était sa puissance, qui
tous deux ne l'oublièrent jamais, mais le régent par la nécessité des
affaires, et sans altération de sa débonnaireté. Albéroni, qui n'était
pas de ce tempérament, et qui, autrefois petit domestique du duc de
Vendôme, n'avait pas été content du duc de Noailles pendant qu'il était
en Espagne, prit contre lui une dose de haine de plus, parce qu'il sut
que renvoi de Louville avait été concerté entre le régent et lui seul,
et reçut comme une nouvelle injure une lettre d'amitié que le duc de
Noailles lui rivait envoyée par Louville\footnote{Voy. les Mémoires de
  Louville, publiés sous le titre de \emph{Mémoires secrets sur
  l'établissement de la maison de Bourbon ce Espagne} (Paris, 1818, 2
  vol.~in-8).}.

\hypertarget{chapitre-iv.}{%
\chapter{CHAPITRE IV.}\label{chapitre-iv.}}

1716

~

{\textsc{Traité de l'asiento signé à Madrid avec l'Angleterre.}}
{\textsc{- Monteléon dupe de Stanhope, jouet d'Albéroni.}} {\textsc{- Le
roi d'Angleterre à Hanovre.}} {\textsc{- L'abbé Dubois va chercher
Stanhope passant à la Haye, revient sans y avoir rien fait, repart
aussitôt pour Hanovre.}} {\textsc{- Jugement des Impériaux sur la
fascination du régent pour l'Angleterre.}} {\textsc{- Chétive conduite
du roi de Prusse.}} {\textsc{- Il attire chez lui des ouvriers
français.}} {\textsc{- Aldovrandi, d'abord très mal reçu à Rome, gagne
la confiance du pape.}} {\textsc{- Nuage léger entre lui et Albéroni,
lequel éclate contre Giudice, dont il ouvre les lettres, et en irrite le
roi d'Espagne contre ce cardinal.}} {\textsc{- Étranges bruits publiés
en Espagne contre la reine.}} {\textsc{- Albéroni les fait retomber sur
Giudice.}} {\textsc{- La peur en prend à Cellamare, son neveu, qui
abandonne son oncle.}} {\textsc{- Albéroni invente et publie une fausse
lettre flatteuse du régent à lui, et se pare de ce mensonge.}}
{\textsc{- Inquiétudes et jalousie d'Albéroni sur les François qui sont
en Espagne.}} {\textsc{- Il amuse son ami Monti, l'empêche de quitter
Paris pour Madrid, lui prescrit ce qu'il lui doit écrire sur la reine,
pour le lui montrer et s'en avantager.}} {\textsc{- Son noir manège
contre le roi d'Espagne.}} {\textsc{- Son extrême dissimulation.}}
{\textsc{- Il veut rétablir la marine d'Espagne.}} {\textsc{- Ses
manèges.}} {\textsc{- Belle leçon sur Rome pour les bons et doctes
serviteurs des rois.}} {\textsc{- Attention de l'Espagne pour
l'Angleterre sur le départ de la flotte pour les Indes, et des
Hollandais pour l'Espagne sur leur traité à faire avec l'Angleterre et
la France.}} {\textsc{- Difficultés du dernier renvoyées aux ministres
en Angleterre.}} {\textsc{- Scélératesses de Stairs.}} {\textsc{-
Perfidie de Walpole.}} {\textsc{- Frayeurs et mesures d'Albéroni contre
la venue des Parmesans.}} {\textsc{- Il profite de celles du pape sur
les Turcs, et redouble de manèges pour son chapeau, de promesses et de
menaces.}} {\textsc{- Giudice publie des choses épouvantables
d'Albéroni, bien défendu par Aubenton et Aldovrandi.}} {\textsc{-
Molinez fait grand inquisiteur d'Espagne.}} {\textsc{- Quel était le duc
de Parme à l'égard d'Albéroni.}} {\textsc{- Idées bien confuses de ce
prince.}} {\textsc{- Le pape s'engage enfin à donner un chapeau à
Albéroni.}} {\textsc{- Impossibilité présente peu durable.}} {\textsc{-
Avis d'Aldovrandi et Albéroni.}} {\textsc{- Aventure des sbires qui
suspend d'abord, puis confirme l'engagement en faveur d'Albéroni.}}
{\textsc{- Art et bassesse d'Acquaviva.}} {\textsc{- Raison de tant de
détails sur Albéroni.}} {\textsc{- Acquaviva, par ordre d'Espagne,
transfuge à la constitution.}} {\textsc{- Promesses, menaces, manèges
d'Albéroni et d'Aubenton pour presser la promotion d'Albéroni.}}
{\textsc{- Invectives atroces de Giudice et d'Albéroni l'un contre
l'autre.}} {\textsc{- Fanfaronnades d'Albéroni, et sa frayeur de
l'arrivée à Madrid du mari de la nourrice de la reine et leur fils
capucin.}} {\textsc{- Quels ces trois personnages.}} {\textsc{- Albéroni
craint mortellement la venue d'un autre Parmesan\,; écrit aigrement au
duc de Parme.}}

~

Rendu à lui-même par le départ de Louville, Albéroni n'eut rien de plus
à coeur que de terminer au gré des Anglais toutes les difficultés qui
restaient sur \emph{l'asiento}. Le traité fut signé à Madrid le 27
juillet, mais comme l'affaire durait depuis longtemps, il fut daté du 26
mai, et les ratifications du 12 juin qui furent aussi tôt réciproquement
fournies. Monteléon ignorait parfaitement tout ce qui se passait entre
l'Angleterre et l'Espagne. Il en déplorait la lenteur, et de se voir
réduit à poursuivre de misérables bagatelles lorsqu'il aurait pu traiter
utilement. Il voyait que le traité proposé par la France à l'Angleterre
n'avançait point, il se persuadait que l'intelligence entre l'empereur
et le roi de la Grande-Bretagne n'était pas si grande depuis
l'opposition que la compagnie du Levant à Londres avait mise à un
emprunt que l'empereur y voulut faire de deux cent mille livres sterling
sur la Silésie, et que le traité fait entre eux ne contenait rien de
préjudiciable à l'Espagne. Le roi d'Angleterre avait passé en Allemagne
en juillet. Il avait laissé le prince de Galles régent sous le titre de
gardien du royaume, et ce prince, changeant de matières à l'égard de la
nation, cherchait à lui plaire, mais sans cacher son désir de se venger
de Cadogan, et de Bothmar, ministre unique pour Hanovre, à qui il
attribuait les mauvais traitements que le duc d'Argyle, son favori,
avait reçus de roi son père. Le prince traitait Monteléon avec
distinction et familiarité\,; et cela persuadait cet ambassadeur qu'il
était toujours sur le même pied en Angleterre, quoiqu'il ne reçût que
rudesses, et pis encore de Methwin \footnote{Il s'agit probablement de
  Paul Methuen qui avait négocié, en 1703, entre l'Angleterre et le
  Portugal le traité qui a donné à l'Angleterre une si grande influence
  dans le Portugal et les colonies portugaises.}, qui tenait la place de
Stanhope pendant son absence à la suite du roi d'Angleterre à Hanovre.
Ainsi Monteléon, avec tout son esprit et ses lumières, était la dupe de
Stanhope qui le craignait, et le jouet d'Albéroni qui ne l'aimait point.

Châteauneuf, que nous avons vu ambassadeur en Portugal, à
Constantinople, et sans caractère chargé d'affaires en Espagne, et avec
réputation, était devenu conseiller d'État, et était lors ambassadeur à
la Haye. Il avait eu plusieurs conférences inutiles sur le traité avec
Walpole, envoyé d'Angleterre, qui agissait de concert avec le
pensionnaire, et Duywenworde disait qu'il n'aurait pouvoir de conclure
et de signer que lorsque le Prétendant aurait passé les Alpes. Stanhope
et Bernstorff, passant à la Haye pour aller à Hanovre, avait dit que la
France avait plus besoin de l'alliance proposée que l'Angleterre\,; et
ils avaient assuré les ministres de l'empereur qu'ils ne se
relâcheraient point de leurs demandes, et ne feraient rien de contraire
aux intérêts de l'empereur. Ils avaient les uns et les autres des
conférences avec les députés des États généraux aux affaires secrètes,
et les pressaient d'entrer dans l'alliance signée entre ces deux
puissances\,; mais la république, qui en craignait un engagement et un
renouvellement de guerre, éludait toujours. L'abbé Dubois, qui n'avait
fondé toutes ses vues et toutes ses espérances de fortune que sur
l'Angleterre, par le chausse-pied de son ancienne connaissance avec
Stanhope qu'il traitait de liaison et d'amitié pour se faire valoir, et
qui pour cela avait aveuglé M. lé duc d'Orléans sur l'Angleterre, comme
il a été expliqué en plus d'un endroit, saisit la conjoncture pour
persuader son maître que deux heures de conversation avec son ancien ami
avanceraient plus le traité que toutes les dépêches et que toutes les
conférences qui se tenaient à la Haye. Il s'y fit donc envoyer
secrètement pour aller parler à Stanhope à son passage. Le peu de
conférences qu'il eut avec lui n'aboutit à rien. Il revint tout de suite
bien résolu de ne quitter pas prise. Il prétexta qu'il avait trouvé son
ami si pressé de partir, et si détourné en même temps à la Haye, qu'ils
n'avaient eu loisir de rien\,; mais que Stanhope le souhaitait à
Hanovre, où à tête reposée ils pourraient travailler à l'aise et en
repas, et parvenir à quelque chose de bon.

Il n'en fallut pas davantage dans l'empressement où sa cabale avait mis
le régent pour ce traité. Il crut l'abbé Dubois de tout ce qu'il voulut
lui dire, et à peine arrivé le fit repartir pour Hanovre. Les ministres
impériaux, exempts des vues personnelles de Dubois et de la fascination
de son maître, et qui voyaient de près et nettement les choses telles
qu'elles étaient, admiraient l'empressement de la France à traiter avec
l'Angleterre. Ils disaient que la France se trouvait dans l'état le plus
heureux et le plus indépendant qu'elle n'avait qu'à jouir de la paix,
gagner du temps, voir le succès de la guerre de Hongrie, le cours des
affaires domestiques de l'Angleterre, laquelle avait beaucoup plus à
souhaiter que la France de conclure un traité avec elle. Tel était le
jugement sain de ministres qui voyaient clair, quoique si jaloux de la
France. En même temps, il n'était faux avis et impostures les plus
circonstanciées, pour les faire mieux passer, que Stairs n'écrivît sans
cesse aux ministres d'Angleterre, piqué de ce que la négociation lui
avait été enlevée par ces mêmes ministres qui connaissaient son mauvais
esprit et son venin contre la France, quoique ses protecteurs. Toutefois
il faut dire que le triste état du Prétendant promettait une prompte fin
de la fermentation de son parti, en Angleterre, que la victoire complète
que le prince Eugène avait remportée sur les Turcs à l'ouverture de la
campagne faisait regarder cette guerre comme devant être de peu de
durée\,; que l'empire accoutumé au joug de la maison d'Autriche, y était
plus soumis que jamais\,; et que la France avait à prendre garde de voir
renaître la guerre par les intérêts de l'empereur sur l'Italie\,; et
ceux de l'Angleterre sur le commerce, ennemie née de la France, lorsque
ces monarques se trouveraient libres de toute crainte chez eux.

Le roi de Prusse, attentif à s'agrandir, mais léger, inconstant et
timide, n'avait osé remuer sur Juliers à la mort de l'électeur palatin.
Il disait qu'il n'y troublerait point la branche de Neubourg tant
qu'elle subsisterait\,; mais il fit sonder le régent sur ce qu'il ferait
en cas qu'elle vint à s'éteindre, et s'il souffrirait que l'empereur en
ce cas, suivant la résolution qu'il assurait en être prise, s'emparât de
ce duché. En mémé temps il faisait faire à Vienne les plus fortes
protestations d'attachement aux intérêts de l'empereur, et y niait
formellement qu'il eût aucune négociation avec la France. Cette conduite
lui semblait d'un grand politique. Il se brouillait et se raccommodait
souvent avec ses alliés, avec le czar, avec le roi d'Angleterre son
beau-père, et fut longtemps à se déterminer s'il l'irait voir à Hanovre.
Il regardait la France comme prête à souffrir de grandes divisions par
celles des princes du sang et bâtards, des pairs et du parlement,
surtout par l'affaire de la constitution. Cette idée l'enhardit à
s'attirer encore un plus grand nombre de Français pour augmenter ses
manufactures. Il donna donc ses ordres pour persuader à plusieurs
ouvriers et autres de passer en Brandebourg, soit pour cause de religion
ou pour d'autres\,; et il crut y réussir aisément dans un temps où les
étrangers et les Français même s'accordaient à dépeindre la France comme
accablée de misère et sur le point d'une division générale.

Aldovrandi, d'abord mal reçu à Rome et fort blâmé, sut bientôt par son
adresse et par ses amis, obtenir du pape d'être écouté, lequel avait
déclaré qu'il ne lui donnerait point d'audience. Il en eut une fort
longue, dans laquelle il sut si bien manier l'esprit du pape qu'il se le
rendit tout à fait favorable, et qu'il le vit depuis souvent et
longtemps en particulier\,; mais il fut trompé dans l'espérance qu'il
avait conçue d'être incessamment renvoyé en Espagne. Il en avait apporté
deux lettres au pape de la main du roi et de celle de la reine, fort
pressantes pour le chapeau d'Albéroni. Les prétextes de faire attendre
longtemps ceux de l'espérance de qui Rome attend des services ne
manquent pas à cette cour. Aldovrandi, pressé de retourner jouir des
grands émoluments de la nonciature d'Espagne qui n'avait pu jusqu'alors
être rouverte depuis les différends entre les deux cours, et qui n'en
espérait la fin que de la promotion d'Albéroni, et qui par sa nonciature
aurait avancé la sienne, s'employait de toutes ses forces à le servir.
Le duc de Parme, sur je ne sais quel fondement, se défiait de sa bonne
foi là-dessus, et avait donné la même défiance à Albéroni. Celui-ci, qui
mettait toujours la reine d'Espagne en avant au lieu de lui-même, se
plaignit amèrement de l'ingratitude d'Aldovrandi pour cette princesse,
mais il n'osa éclater de peur de pis. Il s'apaisa bientôt, et vit enfin
que ses plaintes étaient très mal fondées.

Il éclata de nouveau contre le cardinal del Giudice, et n'épargna aucun
terme injurieux pour exagérer son ingratitude envers la reine, sans
laquelle il ne serait jamais rentré en faveur en Espagne à son retour de
France, ni sorti de l'abîme où il était tombé. Il lui reprochait la
licence avec laquelle il tombait sur le gouvernement\,; il publiait
qu'il était si bien connu en France qu'on y prévoyait généralement sa
disgrâce. Il ouvrait les lettres de la poste de Madrid, et on crut qu'il
le faisait de sa propre autorité, à l'insu du roi d'Espagne. Il y trouva
une lettre de l'ambassadeur de Sicile au roi son maître qui, lui rendant
compte d'une longue conférence qu'il avait eue avec Giudice, {[}disait
que{]} ce cardinal, après beaucoup de protestations d'attachement,
l'avait averti de ne faire aucun fond sur la cour de Madrid tant que le
crédit d'Albéroni subsisterait, parce que le duc de Parme dont il était
ministre ne songeait qu'à gagner et conserver les bonnes grâces de
l'empereur, et par conséquent ne consentirait jamais que l'Espagne fît
aucun pas pour les princes d'Italie. Albéroni porta cette lettre au roi
d'Espagne, qu'il eut la satisfaction de mettre fort en colère contre
Giudice. Tant d'autorité n'empêchait {[}pas{]} ses alarmes sur les
Français qui étaient à Madrid, bien plus fortes sur des Parmesans
abjects que de fois à autre la reine voulait faire venir. Il n'osait lui
montrer aucune opposition là-dessus, mais il redoublait ses mesures
auprès du duc de Parme pour rompre ces voyages par lui. La sauté du roi
d'Espagne menaçait, son estomac était, en grand désordre. Albéroni
l'engagea à consulter un médecin sarde qui convint avec le premier
médecin des remèdes qu'il fallait employer, en présence de la reine et
d'Albéroni seuls. Ce mystère, joint aux propos scandaleux de Burlet sur
la santé du prince des Asturies, en fit tenir des plus étranges, non
seulement aux gens du commun, mais aux plus élevés, jusqu'à publier que
la reine travaillait à porter son fils aîné don Carlos sur le trône.
Giudice, outré de sa disgrâce, dont il se prenait uniquement à Albéroni,
ne l'épargna pas en cette occasion, ni Albéroni le cardinal en mauvais
offices et en accusations d'accréditer la licence et les mensonges des
mauvais bruits. Cellamare, fils du frère du cardinal del Giudice, alarmé
de tant d'éclats, eut peur pour lui-même. Il ne songea qu'à se conserver
les bannes grâces de la reine et celles d'Albéroni. Il les leur demanda
avec tant d'empressement qu'Albéroni s'en fil un titre pour prouver
l'ingratitude du cardinal, blâmée jusque par son neveu, qui avait
toujours passé pour un homme fort sage et fort éclairé.

Albéroni n'eut pas honte de répandre un mensonge insigne. La
toute-puissance ne craint guère les démentis\,: il publia que M. le duc
d'Orléans, en rappelant Louville, lui avait expressément marqué qu'il ne
l'aurait pas envoyé s'il l'eût cru désagréable au roi d'Espagne, et
qu'incessamment il enverrait un autre homme chargé de communiquer des
choses qui ne se pouvaient confier au papier. Un pareil envoi ne lui
aurait été guère plus agréable. Il ne voulait voir de la part de la
France qui que ce soit capable d'éclairer ses actions, d'en rendre
compte au régent, d'ouvrir les yeux au roi d'Espagne. Tout Français lui
était suspect. Il aurait voulu les chasser tous d'Espagne, surtout ceux
qui étaient chargés de quelques commissions particulières pour la marine
ou pour d'autres affaires. Il les traitait de dévoués aux cabales, et
disait qu'ils prêtaient leurs maisons pour les rassembler. Sa jalousie
et son extrême défiance ne s'assuraient pas même de ses plus intimes
amis. Monti était de ce nombre et avait eu toute sa confiance avant sa
fortune. Il servait en France et il était quelquefois chargé par lui de
commissions particulières pour le régent. Monti crut avancer sa fortune
s'il pouvait aller en Espagne et profiter de son crédit. Il fut
entretenu quelque temps dans cette espérance\,; Albéroni lui mandait que
personne ne servirait mieux les deux cours que lui\,; mais cet amusement
même l'importunait, et il fit entendre à son ami qu'il n'y fallait plus
penser. Il ne voulait point de témoins de sa conduite\,; Monti lui était
commode en France pour l'en informer. Il lui prescrivait les thèmes de
ses lettres pour louer la reine de sa fermeté, et d'en parler comme
d'une héroïne qui, par son courage, établissait son autorité par toute
l'Europe. Il montrait ces lettres à la reine pour la piquer d'honneur,
et faire retomber sur elle tout ce qu'il faisait contre Giudice, dont il
se plaignait d'une manière atroce.

Le traitement fait à Louville était un affront à la France et personnel
au régent, et le triomphe de l'insolence et de l'autorité d'Albéroni.
L'équanimité avec laquelle le régent le souffrit ne put apaiser la haine
que l'Italien avait conçue d'une tentative qu'il se persuada faite
uniquement contre lui. Il prit occasion du traité qui se négociait entre
la France et l'Angleterre, pour inspirer au roi d'Espagne les sentiments
les plus sinistres de M. le duc d'Orléans, et pour les lui faire revenir
par ceux de sa dépendance qui l'approchaient. Il assurait que l'unique
but du régent était de s'assurer de la couronne en cas de malheur en
France\,; que tout lui paraissait plausible et bon pour y parvenir\,;
qu'il se liguerait même avec le Turc s'il le jugeait utile à ce dessein,
ou à empêcher le roi d'Espagne de faire valoir les justes droits de sa
naissance. Il n'osait pourtant convenir que le roi d'Espagne les voulût
soutenir, mais il avouait quelquefois à ses confidents que la plus fine
dissimulation était nécessaire sur un point si délicat, dont il fallait
écarter aux Espagnols toute idée, qui, conçue par eux, pouvait causer
des mouvements dangereux, et se conduire comme si Leurs Majestés
Catholiques ne voulaient jamais sortir de Madrid, attendre les
événements, et compter que la décision de cette grande question
dépendrait de l'Angleterre et de la Hollande. Persuadé en attendant, et
cela avec raison, que l'Espagne devait se rendre puissante par mer, il
faisait de grands projets de marine. Rien ne lui semblait difficile,
pourvu qu'il en fût chargé\,; il ne songeait, qu'à se rendre
nécessaire\,; il y réussissait pleinement, auprès de la reine, par
conséquent auprès du roi. Il se vantait que les impressions qu'on avait
voulu lui donner à son égard n'avaient fait que mieux faire connaître
son zèle et ses services\,; qu'il avait tout crédit sur la reine\,;
qu'il se moquait de ceux qui prétendaient que Macañas entretenait un
commerce secret avec le roi d'Espagne. C'est qu'il savait par la reine,
pour qui le roi n'avait point de secret, qu'Aubenton avait pensé être
perdu pour lui avoir seulement nommé le nom de Macañas, sans autre
intention que de dire qu'il en avait reçu une lettre par laquelle ce
martyr des droits des rois d'Espagne, contre les entreprises de Rome, se
recommandait à ses bans offices. Belle leçon pour les magistrats en
place et en devoir de soutenir les droits de leurs rois contre les
usurpations continuelles des papes\,! Je dis des rois, car la France a
eu aussi ses Macañas, et employés par le feu roi et ses ministres, qui
n'ont pas en un meilleur sort, sans compter le grand nombre qu'il y en a
eu depuis le célèbre Gerson. Albéroni prétendait avoir sauvé le
confesseur, parce qu'il se le croyait attaché, et se donnait pour avoir
résolu d'exterminer ses ennemis.

Au commencement de septembre, le roi d'Espagne, fit avertir le roi
d'Angleterre de sa résolution de faire partir, l'année suivante 1717,
une flotté pour la Nouvelle-Espagne et lui promit de l'avertir plus
particulièrement du mois qu'elle mettrait à la voile. Ainsi rien ne
manquait aux attentions de l'Espagne pour l'Angleterre, et à sa
ponctuelle observation de leurs traités. Les Hollandais, qui de leur
côté ménageaient l'Espagne, lui firent savoir qu'ils étaient disposés à
signer une ligue défensive avec la France et l'Angleterre. Leur dessein
était de témoigner par cet avis leur respect et leur confiance au roi
d'Espagne, et de l'inviter à entrer dans ce traité. Il répondit qu'il ne
s'en éloignait pas, mais qu'il fallait, avant de s'expliquer, qu'il fût
informé des conditions de cette alliance. L'abbé Dubois, qui regardait
la conclusion du traité avec l'Angleterre comme le premier grand pas à
la fortune, qui par degrés le mènerait à tous les autres, l'avait pressé
de toutes ses farces et de toute son industrie. Les deux principales
difficultés étaient le canal de Mardick et le séjour du prétendant à
Avignon. Le roi d'Angleterre ni Stanhope n'osèrent traiter à fond, à
Hanovre, deux points qui intéressaient la nation anglaise, et il fallut
envoyer d'Iberville à Londres pour y régler principalement celui de
Mardick avec les ministres anglais. Ceux-ci étaient persuadés que la
victoire du prince Eugène était un nouvel aiguillon à la France de
presser la conclusion du traité. Quelque bonne foi que M. le duc
d'Orléans fît paraître dans toute la négociation, la malignité de Stairs
n'en put convenir\,; l'imposture de cet honnête ambassadeur alla jusqu'à
avertir les ministres d'Angleterre que le régent était d'intelligence
avec les jacobites qui méditaient quelque entreprise\,; que le baron de
Goertz, ministre du roi de Suède, nouvellement arrivé à la Haye, n'avait
été à Paris que pour la concerter\,; que Dillon, lieutenant général au
service de France, qu'il avait déjà mandé être chargé en France des
affaires du prétendant, serait chargé de l'exécution\,; et l'impudence
était poussée jusqu'à donner ces avis, non comme de simples bruits, mais
comme des certitudes. Walpole, envoyé d'Angleterre en Hollande, chargé
de négocier pour faire entrer les États généraux dans ce traité, n'était
pas mieux intentionné que Stairs. Il avait ordre d'agir là-dessus de
concert avec l'ambassadeur de France, et faisait, à son insu, tout ce
qui lui était possible pour le traverser. C'est à quoi les ministres
impériaux travaillaient à la Haye de toute leur application. Ceux de
Suède s'en plaignaient fort, persuadés qu'ils étaient qu'ils seraient
abandonnés par la France, qui garantirait Brême et Verden au roi
d'Angleterre. Stairs, enfin, ne pouvant plus donner de soupçons sur M.
le duc d'Orléans, excitait les ministres d'Angleterre de tenir ferme à
toutes leurs demandes, parce qu'il savait que ce prince accorderait tout
plutôt que de ne pas conclure. Monteléon gardait le silence, quoiqu'il
pût aussi apporter quelques obstacles\,; il n'avait plus les mêmes
accès. Methwin lui paraissait mal disposé pour l'Espagne. Il le
remettait sur toute affaire au retour du roi d'Angleterre sans nulle
nécessité.

Albéroni, qui bravait la haine publique en Espagne, ne put se résoudre à
obéir à la duchesse de Parme qui lui ordonnait de demander à la reine sa
fille une pension ou quelque subsistance pour un homme du commun, pour
qui elle avait eu de la bonté à Parme, et qu'elle avait voulu faire
venir en Espagne plus d'une fois. Il craignit le danger de le rappeler
dans sa mémoire. Toute son attention était à conserver tout son crédit
sans partage et sans lutte, au moins jusqu'à ce qu'il fût parvenu au
chapeau\,; et pour le hâter, à donner au pape une haute idée de son
pouvoir, bien persuadé que les grâces de Rome ne sont consacrées qu'à
ses besoins et aux services qu'il lui est important de tirer. Le pape
était faible\,; il craignait les Turcs. Il désirait ardemment de hâter
les secours maritimes d'Espagne. Albéroni en profita. Il fit représenter
au pape qu'il ne devait pas perdre de temps à se déterminer\,; qu'en
différant, le printemps arriverait avant qu'il y eût rien de réglé pour
des succès qui pourraient immortaliser son pontificat\,; il lui fit
sonner bien haut que tout en Espagne était uniquement entre les mains du
roi et de la reine\,; qu'ils étaient affranchis de l'autorité que les
tribunaux et les conseils avaient prises\,; que d'eux seuls dépendaient
les ordres et les exécutions. Cela voulait dire de lui uniquement, et
que si le pape voulait être servi et content, il fallait qu'Albéroni le
fût aussi, et que le seul moyen que le pape fût satisfait était
d'avancer la promotion d'Albéroni. Aubenton, totalement dévoué au pape,
n'était attaché à Albéroni que par la crainte. Quelque confiance que le
roi d'Espagne eût en son confesseur, il n'aurait pas eu la force de le
soutenir contre la reine, si, conseillée par Albéroni, elle eût
entrepris de le faire chasser. La princesse des Ursins lui en avait
donné une leçon, qu'il n'avait pas oubliée, et Albéroni avait aussi
besoin de lui, parce que le pape, qui comptait entièrement sur lui,
ajoutait foi à ce qu'il écrivait\,; et ce qu'il mandait à Rome était du
style le plus propre {[}à{]} avancer la promotion d'un homme si zélé
pour l'Église et si capable de servir puissamment le saint-siège dans
les conjonctures difficiles où il se trouvait.

Aldovrandi, intéressé pour soi-même dans l'avancement de la promotion
d'Albéroni, pour retourner jouir de sa nonciature d'Espagne, et abréger
son chemin à la pourpre, faisait valoir au pape le caractère d'Albéroni
et son pouvoir peint d'une main que Sa Sainteté croyait si fidèle. Une
nouvelle qui courut alors par les gazettes jusqu'à Rome, et qui fit du
bruit, troubla le triumvirat. C'était la prétendue brouillerie
d'Albéroni et d'Aubenton, et qu'Albéroni allait être chassé. Quoiqu'il
n'y eût aucune apparence de vérité dans ce conte, l'impression qu'il fit
à Rome devint très importante pour Albéroni, qui se flattait tellement
de sa prochaine promotion alors, qu'il en recevait des compliments avec
une joie, en même temps avec un ridicule dont ses ennemis surent
profiter. Il s'appliqua, lui et ses deux amis, à faire tomber ce bruit,
et en démontrer à Rome le mensonge. Giudice, de son côté, que nulle
considération ne pouvait plus retenir, parce qu'il n'avait plus rien à
espérer ni à craindre, n'oubliait rien pour traverser la promotion
d'Albéroni. Il protestait qu'elle était injurieuse à la pourpre, au
pape, à l'Église\,; il demandait que le pape pour son propre honneur,
consultât les évêques et les religieux d'Espagne, sur la vie, les
moeurs, la conduite d'Albéroni, sûr que, sur leur témoignage, il
rejetterait pour toujours la pensée de promouvoir un sujet de tous
points si indigne. Outre la religion et mille noirceurs sur lesquelles
il l'attaquait, il prétendait qu'il trahissait le roi d'Espagne, et
qu'ayant été autrefois l'espion du prince Eugène en Italie, il
entretenait encore le même commerce avec lui, duquel il était largement
payé. Aubenton redoublait d'efforts à proportion, répondait de tout en
Espagne, au gré du pape, s'il voulait hâter la promotion d'Albéroni, et
mandait à Aldovrandi qu'il se souvînt qu'il était chargé de l'affaire de
Dieu, soit qu'il prétendît diviniser celle du premier ministre, ou qu'il
y eût quelque autre mystère entre eux.

Giudice s'était démis de la charge de grand inquisiteur d'Espagne.
Albéroni la fit donner à Molinez, mains pour récompenser sa fidélité et
ses travaux, que pour laisser champ libre à Acquaviva à prendre le soin
des affaires d'Espagne à Rome parce qu'il comptait sur ce cardinal qui
avait toute la confiance de la reine. On s'était d'autant plus pressé
d'y pourvoir qu'on craignait que Giudice ne rétractât sa démission du
moment qu'il serait hors de l'Espagne. Le duc de Parme en avait
averti\,; quoiqu'il n'aimât ni n'estimât Albéroni, il s'intéressait au
maintien de l'autorité d'un homme qui était son sujet et son ministre en
Espagne. Il avait par lui une part indirecte au gouvernement de cette
monarchie, à laquelle par conséquent il s'intéressait. Son grand objet
était de l'engager à des tentatives pour recouvrer quelque partie de ce
qu'elle avait perdu en Italie, dont le temps lui paraissait favorable
pour y réussir par l'occupation de l'empereur en Hongrie, et la haine
des princes d'Italie. Il sentait bien aussi que l'Espagne était trop
faible pour l'entreprendre sans secours, et qu'elle n'en pouvait espérer
que de la France\,; qu'il fallait donc ménager le régent pour l'engager
à ce secours, mais en même temps ne pas abandonner les vues de retour,
en cas de malheur en France. Des projets si contraires n'étaient pas
aisés à concilier. Tous deux étaient persuadés que les Français, fâchés
de voir l'Espagne entre les mains d'un Italien, ne songeaient qu'à le
faire chasser, et que Louville n'avait été envoyé que pour cela à
Madrid, quoique sous d'autres prétextes. Albéroni, qui connaissait les
dispositions du gouvernement de France à son égard, avait pris son parti
là-dessus, et n'en pressait que plus vivement sa promotion pour
s'acquérir un état solide, et se maquer après des ennemis de sa fortune.

Aldovrandi, qui des affres des prisons du château Saint-Ange, dont il
avait frisé la corde à Rome, était parvenu à faire goûter au pape les
raisons de son voyage, et à entrer après dans sa confiance, s'était
habilement servi de la connaissance qu'il avait de son esprit, pour le
conduire par degrés à la promotion d'Albéroni, et à rendre vaines les
machines del Giudice et de ses autres ennemis. Il en obtint l'assurance,
mais il manda à Albéroni qu'il n'y devait pas compter tant qu'il n'y
aurait comme alors qu'un seul chapeau vacant\,; que l'attente ne serait
pas longue par l'âge et les infirmités de plusieurs cardinaux\,; que le
pape craignait trop l'empereur pour lui donner ce sujet de plainte,
surtout d'empêcher que le roi d'Espagne ne donnât sa nomination à aucun
Espagnol, et ne fit instance au pape de la remplir\,; qu'il fallait
éviter la promotion des couronnes, et faire qu'il parût que la sienne
vint uniquement du pur mouvement du pape, pour cela presser l'arrivée du
secours maritime pour le secours des États d'Italie contre les Turcs, et
faciliter l'accommodement entre les cours de Rome et de Madrid, enfin
garder sur toutes ces choses le plus profond secret. Ce qu'il ne cessait
point de lui répéter, c'était de cultiver la bonne intelligence avec
Aubenton, estimé au dernier point du pape et des cardinaux Imperiali,
Sacripanti, Albani, les trois non nationaux, les plus déclarés contre la
France. Il y pouvait ajouter Fabroni avec qui ce jésuite avait fait seul
la constitution \emph{Unigenitus} avec l'art, la dextérité, le secret\,;
et la violence sur le pape et tout Rome qui ont été racontés en leur
lieu. Aldovrandi relevait l'admiration du pape pour la reine, dont il
espérait tout pour le prompt secours maritime, qu'il était de la
prudence d'Albéroni de maintenir\,; le pressait de faire hiverner la
flatte en Italie, et déplorait la situation du pape qui ne lui
permettait pas de faire ce qu'il voulait. Toute affaire d'Espagne était
subordonnée, ou passait en faveur de cette promotion, qui était la
surnageante et la plus capitale. Enfin Acquaviva et Aldovrandi
représentèrent si fortement au pape qu'il n'obtiendrait rien d'Espagne
en aucun genre que moyennant cette promotion, que Sa Sainteté qui
s'était contentée de prendre là-dessus quelque engagement avec
Aldovrandi en air de confidence, en prit un effectif avec Acquaviva, à
qui il dit dans une audience qu'il pouvait écrire positivement à Madrid
qu'elle était déterminée à faire pour Albéroni ce que la reine lui
demandait, et qu'il n'était plus question que de la manière de
l'exécuter.

La difficulté, on l'a déjà dit, c'est qu'il n'y avait qu'un chapeau
vacant que le pape destinait à un sujet protégé par l'empereur. On
croyait qu'il regardait Borromée dont la mère avait épousé Ch. Albani,
neveu du pape, qui prétendait par là compenser la promotion de Bissy,
faite pour la France. Il fallait de plus satisfaire la France en même
temps que l'Espagne en élevant de son pur mouvement deux sujets à la
pourpre, nationaux ou agréables aux couronnes, et ces ménagements
demandaient la vacance de trois chapeaux. On consolait le premier
ministre par la considération de sept cardinaux de plus de quatre-vingts
ans, et d'onze de plus de soixante-dix, sans ce qui pouvait arriver à de
plus jeunes. On l'assurait qu'il y avait tout à espérer pour lui de la
chute des feuilles. On l'avertissait surtout de faire accorder au pape
la condition réciproque, qui était un engagement du roi d'Espagne de
différer sa nomination de couronne, et d'être longtemps sans en parler
après la promotion d'Albéroni.

Une aventure très imprévue et fort subite pensa déconcerter des mesures
si bien prises. Molinez, doyen de la rote, dont il était auditeur pour
l'Espagne et chargé des affaires de cette couronne à Rome, logeait,
depuis longtemps qu'il y était seul ministre de cette couronne, dans le
palais qui lui appartenait et qui était dans la place qui en avait pris
le nom de place d'Espagne. Il s'y était fortifié d'un nombre de braves à
la solde d'Espagne contre les violences des Impériaux qui menaçaient de
s'emparer par force de ce palais, comme appartenant à l'empereur.
Molinez déchargé des affaires d'Espagne qui avaient été confiées au
cardinal Acquaviva, accoutumé à demeurer dans son propre palais, était
resté dans celui d'Espagne avec ses braves. Arriva la victoire du prince
Eugène qui transporta les Impériaux et le peuple de Rome\,; ils
promenèrent par les rues divers signes de victoires, entre antres un
char à la manière de ceux des anciens triomphes. Cette machine,
accompagnée des Impériaux, de beaucoup de peuple et des sbires, passa
dans la place et devant le palais d'Espagne. Soit que Molinez eût peur
qu'à la faveur de cette allégresse et de cette foule, on entreprît de
s'emparer du palais d'Espagne, ou qu'il prît seulement ce passage devant
sa porte pour une insulte, il fit charger et dissiper tout cet
accompagnement. Le pape qui se faisait gloire de retrancher aux
ambassadeurs les franchises qui avaient fait tant de bruit autrefois,
entra dans une telle colère qu'il envoya sur-le-champ Aldovrandi au
cardinal Acquaviva lui dire de suspendre sa dépêche à Madrid, et de n'y
rien mander de l'assurance qu'il lui avait donnée peu de jours
auparavant.

Acquaviva sans s'étonner manda au pape par le même prélat que sa dépêche
était écrite, qu'il l'enverrait sans y rien changer, parce qu'il savait
que le pape serait content. Il pria Aldovrandi de savoir du pape quelle
satisfaction il prétendait. La négociation finit presque aussitôt
qu'elle commença. Le pape demanda que l'espèce de milice qui gardait le
palais d'Espagne fût congédiée, et que les sbires pussent passer
librement dans la place d'Espagne\,; et Acquaviva, de son côté, demanda
que le pape fît respecter le palais d'Espagne comme les autres palais de
Rome, et qu'il fît passer les sbires dans les quartiers des autres
ministres étrangers, de même que dans celui d'Espagne. Ces quatre
conditions respectives furent accordées, et le pape confirma l'assurance
qu'il avait donnée pour Albéroni Acquaviva fit valoir en Espagne le
service qu'il avait rendu à Albéroni, et il avait vendu cher ce qui dans
le fond n'était rien, par ce qu'il saurait des intentions du roi
d'Espagne sur les franchises. Ce cardinal faisait pour soi en même temps
que pour le premier ministre. Les Espagnols qui étaient à Rome
murmuraient de sa facilité pour plaire au papa, aux dépens des affaires
du roi d'Espagne. Don Juan Diaz, agent d'Espagne à Rome, était celui qui
en parlait le plus haut. Acquaviva saisit ce moment pour demander qu'il
fût rappelé, et que la reine lui écrivît en approbation de sa conduite
de manière qu'il pût montrer sa lettre au pape. Tout son objet,
disait-il, était de servir Albéroni auprès du pape, pour quoi il fallait
que lui-même fût soutenu. Il disait qu'Aldovrandi méritait là-dessus
toute la protection du roi et de la reine, et qu'étant dans la première
estime et confiance du pape, il aurait seul son secret pour négocier sur
les différends d'entre les deux cours, et il insistait pour aplanir les
difficultés qui retardaient son retour et l'exercice de sa nonciature en
Espagne\,; ainsi il le servait dans cette cour de tout son pouvoir,
comme il vantait au pape l'empressement d'Albéroni à lui procurer à
temps les secours maritimes qu'il désirait avec impatience.

Si je m'arrête avec tant de détail à tous ces manèges et ces intrigues,
c'est qu'ils me semblent curieux et instructifs par eux-mêmes. Ils
montrent au naturel quel est un premier ministre tout-puissant, un roi
qui s'en laisse enfermer et gouverner, ce que peut le but d'un chapeau,
quelle est la confiance due à un confesseur jésuite, et la part que le
prince doit laisser prendre à son épouse, surtout en secondes noces, en
ses affaires. D'ailleurs les personnages de ce triumvirat ont fait tant
de bruit dans le monde, et tant de personnages divers, que ce qui les
regarde ne peut être indiffèrent à l'histoire. Pour Acquaviva, je n'en
parle que par la nécessité de la liaison avec les trois principaux, dont
deux sont devenus cardinaux, et le troisième mourait d'envie de l'être,
et l'a souvent bien espéré. Ces récits découvrent encore ce que c'est
que d'admettre des prêtres dans les affaires et dans les conseils.
Acquaviva fut averti par d'Aubenton qu'il se perdrait en Espagne s'il
continuait à penser et à agir comme il faisait sur les affaires de
France à l'égard de la constitution \emph{Unigenitus}. Il reçut en même
temps un ordre du roi d'Espagne de se conformer là-dessus à tout ce qui
pouvait plaire au pape. Il n'en fallut pas davantage à Acquaviva pour
changer de camp contre ses propres lumières en matière de doctrine et
pour rompre tout commerce avec le cardinal de Noailles. Telle est la
morale et la foi de nos prélats d'aujourd'hui et de ceux qui veulent
l'être. Je ne le dis pas sans {[}le{]} savoir et sans l'avoir vu et revu
bien des fois.

Albéroni fidèle à ses vues et à ses maximes, et bien instruit de celles
de Rome, ne s'appliquait qu'à bien persuader le pape qu'il était le seul
ministre du roi d'Espagne, le seul à qui tout son pouvoir fût confié
sans réserve, le seul à qui on pût s'adresser pour en recevoir des
grâces. Ces principes bien établis et souvent réitérés, il vantait ses
intentions et son zèle, mais il protestait que le tout serait inutile,
si le pape ne prenait de promptes résolutions\,; il promettait s'il
était assisté, c'était à dire élevé à la pourpre, que le pape aurait
avant la fin de mars à ses ordres une forte escadre bien équipée dans un
port de l'État de Gênes, mais qu'il exigeait aussi l'entière confiance
du pape, et qu'il regarderait comme offenses toutes démarches
indirectes, toutes instances faites par d'autres voies que par lui\,; et
pour colorer sa jalousie, il attribuait ces démarches indirectes à
l'ignorance de la forme et du système présent du gouvernement d'Espagne.
Aubenton par ses lettres renchérissait encore plus sur le grand et
unique pouvoir résidant uniquement dans le premier ministre. Il assurait
le pape que le secours que Sa Sainteté désirait, dépendait absolument de
lui, que le projet qu'il avait fait pour l'envoyer serait
infailliblement exécuté s'il en usait bien à son égard, c'est-à-dire
s'il lui envoyait la barrette. Mais aussi qu'elle ne devait espérer ni
secours contre les Turcs, ni accommodement des différends entre les deux
cours, si elle ne donnait à la reine d'Espagne la satisfaction qu'elle
demandait avec tant de désir et d'ardeur. Il faisait entendre clairement
à ses amis de Rome que c'était par ordre qu'il écrivait si positivement,
et il prétendait en même temps donner par là une preuve de son intime
union avec Albéroni, et démentir sur cela les bruits et les gazettes.
Albéroni avait bien des ennemis à Rome, et beaucoup de cardinaux
indignés de la prostitution de leur pourpre à un sujet tel que lui.
Giudice, qui publiait qu'il s'y en irait bientôt, y remuait contre lui
toutes sortes de machines, et ne gardait aucunes mesures sur sa personne
dans ses discours ni dans ses lettres. Albéroni ripostait avec le même
emportement, et ne cessait de l'accuser de la plus noire ingratitude
envers la reine, d'assurer nettement que la cause de cette princesse et
la sienne était la même, et que la conduite de Giudice était si décriée
que Cellamare lui-même n'hésitait pas là-dessus. Il avait envoyé à Rome
les copies des lettres que Cellamare lui avait écrites sur la disgrâce
de son oncle, et la bassesse de Cellamare avait été au point d'avoir
mandé à plusieurs personnes à Rome, que dans le naufrage de sa maison il
avait tâché de sauver sa petite barque en prenant le bon parti.

Giudice parlait et écrivait d'Albéroni comme du dernier des hommes. Il
se plaignait aussi d'Aldovrandi, comme ayant parlé contre lui à Rome
pour plaire à Albéroni. Ils se reprochaient réciproquement ingratitudes
et perfidies, et avaient tous raison à cet égard. Le premier ministre
chargeait Giudice des fâcheux bruits répandus à Madrid contre la reine,
et nouvellement d'avoir publié qu'elle avait fait venir à Madrid
l'argent venu par les derniers galions, pour en envoyer une grande
partie à Parme. Quelque semblant qu'Albéroni fît d'être fermement
certain que tout l'enfer déchaîné contre lui ne lui pourrait nuire, et
de rehausser cette confiance d'un air de philosophie qui lui faisait
dire qu'il ne demeurait chargé de tant d'envie et du poids des affaires
que par attachement pour le roi et la reine et pour le bien de l'État,
il craignait mortellement, tout ce qui pouvait avoir accès auprès de la
reine. Elle avait enfin fait venir à Madrid le mari de sa nourrice et
leur fils capucin. La nourrice était fine, adroite, et ne manquait ni de
sens ni de hardiesse. Son mari était un stupide paysan, leur fils un
fort sot moine, mais pétri d'ambition, qui ne comptait pas sur moins que
gouverner l'Espagne. La reine, qui avait souvent demandé au duc de Parme
un musicien nommé Sabadini qu'elle avait fort connu, en avait écrit avec
tant de volonté, que le duc de Parme lui promit de le faire partir dès
que le prince électeur de Bavière serait parti de Plaisance. Albéroni
craignait horriblement la présence de Sabadini, dont il avait plusieurs
fois rompu le voyage par le duc de Parme. Il lui écrivit donc aigrement
sur sa faiblesse, et l'envoi du capucin et de son père, et mit tout en
œuvre auprès de lui pour arrêter en Italie Sabadini, duquel il prenait
de bien plus vives alarmes.

\hypertarget{chapitre-v.}{%
\chapter{CHAPITRE V.}\label{chapitre-v.}}

1716

~

{\textsc{{[}Albéroni{]} compte sur l'appui de l'Angleterre\,; reçoit
avis de Stanhope d'envoyer quelqu'un de confiance veiller à Hanovre à ce
qu'il s'y traitait avec l'abbé Dubois.}} {\textsc{- Pensées des
étrangers sur la négociation d'Hanovre.}} {\textsc{- Les Impériaux la
traversent de toute leur adresse, et la Suède s'en alarme.}} {\textsc{-
Affaires de Suède.}} {\textsc{- Pernicieuse haine d'Albéroni pour le
régent.}} {\textsc{- Esprit de retour en France, surtout de la reine
d'Espagne.}} {\textsc{- Sages réflexions d'Albéroni sur le choix, le cas
arrivant.}} {\textsc{- Quel était M. le duc d'Orléans sur la succession
à la couronne.}} {\textsc{- Affaire du nommé Pomereu.}} {\textsc{-
M\textsuperscript{me} de Cheverny gouvernante des filles de M. le duc
d'Orléans.}} {\textsc{- Livry obtient pour son fils la survivance de sa
charge de premier maître d'hôtel du roi.}} {\textsc{- Effiat quitte le
conseil des finances et entre dans celui de régence.}} {\textsc{-
Honneurs du Louvre accordés à Dangeau et à la comtesse de Mailly par
leurs charges perdues.}} {\textsc{- Origine de cette grâce à leurs
charges.}} {\textsc{- Ce que c'est que les honneurs du Louvre.}}
{\textsc{- Style de la république de Venise écrivant au Dauphin\,; d'où
venu.}} {\textsc{- Entreprise de la nomination du prédicateur de l'Avent
devant le roi.}} {\textsc{- M. de Fréjus officie devant le roi sans en
dire un seul mot au cardinal de Noailles.}} {\textsc{- Abbé de Breteuil
en tabouret, rochet et camail, près du prie-Dieu du roi, comme maître de
la chapelle, condamné de cette entreprise comme n'étant pas évêque.}}
{\textsc{- Quel fut le P. de La Ferté, jésuite.}} {\textsc{- L'abbé
Fleury, confesseur du roi.}} {\textsc{- Mort de la duchesse de Richelieu
et de M\textsuperscript{me} d'Arnemonville.}} {\textsc{- Mort et
caractère du maréchal de Châteaurenaud.}} {\textsc{- Belle anecdote sur
le maréchal de Coetlogon.}} {\textsc{- Mort de la duchesse d'Orval.}}
{\textsc{- Mort de d'Aguesseau, conseiller d'État\,; son éloge.}}
{\textsc{- Saint-Contest fait conseiller d'État, en quitte le conseil de
guerre.}} {\textsc{- L'empereur prend Temeswar\,; perd son fils
unique.}} {\textsc{- La duchesse de Saint-Aignan va trouver son mari en
Espagne avec trente mille livres de gratification.}} {\textsc{- Mort,
caractère et famille de M. d'Étampes.}} {\textsc{- Mort de la comtesse
de Roucy.}} {\textsc{- Mort de M\textsuperscript{me} Fouquet\,; sa
famille.}} {\textsc{- Force grâces au maréchal de Montesquiou, au grand
prévôt, aux ducs de Guiche, de Villeroy, de Tresmes, et au comte de
Hanau.}} {\textsc{- Le duc de La Force vice-président du conseil des
finances.}} {\textsc{- Augmentation de la paye de l'infanterie.}}
{\textsc{- Caractère de Broglio, fils et frère aîné des deux maréchaux
de ce nom.}} {\textsc{- Le duc de Valentinois reçu au parlement, où les
princes du sang ni bâtards n'assistent point.}} {\textsc{- Mariage du
fils unique d'Estaing avec la fille unique de M\textsuperscript{me} de
Fontaine-Martel, et la survivance du gouvernement de Douai.}} {\textsc{-
Bonneval obtient son abolition en épousant une fille de Biron.}}
{\textsc{- Dispute entre les grands officiers de service et le maréchal
de Villeroy, qui, comme gouverneur du roi, prétend faire leur service et
le perd.}} {\textsc{- Grande aigreur entre les princes du sang et
bâtards sur les mémoires publiés par les derniers.}} {\textsc{-
Étonnante apathie de M. le duc d'Orléans.}} {\textsc{- Ma façon d'être
avec le duc de Maine et le comte de Toulouse.}}

~

La grande ressource d'Albéroni, à son avis, était l'appui qu'il se
promettait de l'Angleterre et de son commerce secret et direct avec
Stanhope. Ce ministre l'avait averti d'envoyer à la Haye quelqu'un de
confiance pour veiller aux intérêts du roi d'Espagne, dans une crise où
il s'agissait d'un nouveau système pour l'Europe. On prétend qu'Albéroni
fit part de l'avis au duc de Parme. Il ne se fiait à aucun Espagnol, et
fit nommer Beretti Landi à l'ambassade de la Haye\,; mais comme en ce
même moment Claudio Ré, que le duc de Parme tenait à Londres en qualité
de secrétaire, reçut ordre de ce prince de se rendre à Hanovre, on se
persuada que c'était pour y être chargé de la confiance d'Albéroni, sous
le prétexte de solliciter le roi d'Angleterre d'obtenir de l'empereur
d'admettre à son audience l'envoyé de Parme, et de le détourner de
presser le mariage de la princesse de Modène avec le prince Ant. de
Parme, que le duc son frère disait n'avoir pas moyen de l'apanager pour
faire cette alliance. Le dessein d'Albéroni, en se rendant maître du
négociateur pour l'Espagne, était de se réserver l'honneur de traiter et
de finir à Madrid l'essentiel de la négociation.

Tout le monde avait les yeux ouverts sur l'alliance qui se traitait
entre la France et l'Angleterre. Les étrangers la regardaient comme un
sujet de division entre le roi d'Espagne et le régent\,; ils publiaient
qu'il y en avait beaucoup déjà entre eux. L'empereur la craignait dans
la prévoyance que, lorsque les Anglais et les Hollandais seraient sûrs
de la France, leur attachement à ses intérêts diminuerait beaucoup.
Ainsi ses ministres la traversaient de tout leur possible. Il y avait à
Paris un baron d'Hohendorff fort attaché au prince Eugène, dont il avait
été aide de camp pendant la dernière guerre. Il se prétendait autorisé
de lettres de créance de l'empereur qu'il avait même montrées du temps
que Penterrieder était à Paris comme secrétaire de l'empereur, et
véritablement chargé de ses affaires. Cet Hohendorff avait même alors
proposé au régent une alliance avec l'empereur, qui n'avait pas eu de
suite. Cet homme ne cessait d'échauffer la vivacité de Stairs,
d'ailleurs si contraire au traité, parce qu'il avait été tiré de ses
mains pour être porté à la Haye puis à Hanovre entre Stanhope et l'abbé
Dubois, et parce qu'il baissait la France. Hohendorff lui disait
continuellement que le régent tromperait les Anglais, et que le
Prétendant ne sortirait point d'Avignon. On excitait d'un autre côté la
Suède, à qui on persuadait faussement que la France sacrifierait ses
intérêts au roi d'Angleterre, et lui garantirait la possession de Brème
et de Verden qu'il lui avait usurpés, tellement que l'ambassadeur de
Suède, qui, de tout temps, était attaché à la France, en prit des
impressions qui lui firent tenir des discours peu mesurés. Les affaires
du roi son maître prenaient une face plus riante. Ses ennemis avaient
assemblé de grandes forces pour faire une descente dans la province de
Schonen, et envahir après la Suède\,: le czar était à Copenhague en
dessein de passer la mer, et de commander cette expédition. Il s'y
brouilla avec le roi de Danemark, au point que l'entreprise fut différée
au printemps, les troupes renvoyées et les dépenses inutiles, qui
avaient été fort à charge au Danemark\,; le roi de Suède n'en put
profiter. Il avait des troupes, mais ni argent ni marine\,: il voulut
acheter quelques vaisseaux en France et en Hollande, où était pour lors
le baron de Goertz qui était chargé de ses finances, et qu'il y avait
envoyé. Il lui dépêcha donc un officier, et un autre au baron Spaar, son
ambassadeur en France pour cet achat. Il envoya par cette voie ordre à
Spaar de cultiver les bannes dispositions de la France, de lui persuader
qu'il voulait la paix, et de presser le payement des subsides qu'elle
lui donnait. Il n'osait même avec ses ministres s'expliquer qu'en termes
généraux sur ses desseins secrets, tant les bruits dont on vient de
parler lui faisaient craindre un trop entier engagement de la France
avec l'Angleterre.

Quelque désir qu'eût l'Espagne de prendre avec cette dernière couronne
des liaisons particulières, Albéroni ne voulait faire avec elle de
traité que totalement séparé et détaché de celui de la France. Les vues
sur l'avenir, et sur lesquelles il évitait soigneusement de s'expliquer,
ne convenaient point avec une alliance commune. Persuadé que le régent
ne lui pardonnerait pas, il ne cessait d'assurer le roi et la reine
d'Espagne qu'ils ne devaient jamais compter sur la bonne foi ni sur les
paroles de ce prince. Il n'ignorait pas que le génie et les désirs de
cette princesse étaient entièrement tournés vers le trône de France en
cas de malheur. Elle sentait l'importance de cacher ce sentiment pour ne
pas s'exposer à perdre le certain pour l'incertain, et {[}craignait{]}
ce que penseraient les Espagnols, et ce qu'ils diraient, si, après ce
qu'ils avaient fait et souffert depuis quinze ans pour soutenir leur roi
sur le trône, il les exposait par son abandon à recevoir un nouveau roi
de la main des Anglais et des Hollandais. Albéroni lui disait que ces
deux puissances disposeraient absolument des couronnes de France et
d'Espagne, et que c'était pour cela que M. le duc d'Orléans n'oubliait
rien pour les gagner. Albéroni néanmoins réfléchissait quelquefois sur
le danger qu'il y aurait pour le roi et pour la reine à changer de
couronne, encore plus pour lui-même. Il se représentait les Français
turbulents volages, hardis\,; il était agité de la multitude des princes
du sang capables avec le temps d'inquiéter le souverain, et qui
deviendraient comme des chevaux indomptés et sans bride ni frein, si la
minorité durait\,: le parlement de Paris lui paraissait devenu, comme
autrefois, le correctif et le fléau de l'autorité royale. Il concluait
de ces réflexions que, si la monarchie d'Espagne pouvait se rétablir, le
roi d'Espagne aurait fort à balancer sur le choix d'un royaume qu'il
acquerrait et qu'il gouvernerait très difficilement, ou d'un autre dont
il était en possession, qu'il pouvait gouverner despotiquement et comme
en dormant. En effet, il n'y a point de pays où la soumission soit plus
entière qu'en Espagne, ni où la volonté et l'autorité du roi sait plus
affranchies de toutes farines, ni plus à couvert de toute résistance.

Tandis qu'on était si intérieurement occupé en Espagne des futurs
contingents, je puis dire avec la plus exacte vérité que c'est la chose
dont M. le duc d'Orléans le fut toujours le moins. Il est des
vraisemblances qui n'ont aucune vérité, et des vérités qui n'ont point
de vraisemblances. Celle-ci est de ce nombre au premier degré, et je ne
crois pas que, depuis qu'il est dans l'univers des monarchies
héréditaires, aucun héritier collatéral immédiat s'en soit moins soucié,
y ait moins pensé, qui ait plus sincèrement désiré que la succession ne
s'ouvrit point\,; dirai-je tout, et le croira-t-on\,? qui ait été moins
touché, plus embarrassé, plus importuné de porter la couronne. Jamais en
aucun temps rien même d'indirect là-dessus\,; jamais quoi que ce soit
sur cette matière dans aucun des conseils\,; et si quelquefois
l'indispensable connexité des affaires étrangères l'ont amené dans le
cabinet du régent entre deux au trois de ses plus confidents, elle ne
s'y traitait précisément que par nécessité, simplement, courtement, même
avec une sorte de contrainte sans parenthèses, sans rien d'inutile,
comme on aurait raisonné sur la succession d'Angleterre ou de
l'empereur. Les plus familiers connaissaient si bien M. le duc d'Orléans
sur ce sujet, qu'il n'est arrivé à pas un d'eux de laisser échapper
devant lui aucune sorte de flatterie là-dessus. Je suis peut-être celui
avec qui cela a le plus été traité tête à tête avec lui à propos de sa
conduite, des affaires étrangères, dont il me disait tout ce qui ne
passait pas au conseil, à propos encore des finances et de la
constitution. À la vérité il ne voulait pas perdre son droit. Je l'y
fortifiais même\,; mais il n'en était touché que du côté de son honneur
et de sa sûreté, desquels il ne se pouvait agir que le malheur ne fût
arrivé, considérations qui au contraire le lui faisaient craindre. Alors
nous nous en parlions comme de toute autre sorte d'affaire importante.
Il ne se cachait pas de moi ainsi tête à tête\,; et je le connaissais
trop pour qu'il y eût réussi. Jamais je ne l'ai surpris an aucun
chatouillement là-dessus, aucun air de joie, aucune échappée flatteuse,
jamais {[}à{]} en prolonger le raisonnement. Je n'outrerai rien quand je
dirai que cela allait à l'insipidité et à une sorte d'apathie, que je
sens qui m'aurait impatienté. Si le fils de Mgr le duc de Bourgogne
m'eût été moins tendrement et précieusement cher, et qu'il se fût agi de
succéder à un autre.

On a vu plusieurs fois dans ces Mémoires que le feu roi avait fait du
lieutenant de police de Paris, une espèce de ministre secret et
confident, une sorte d'inquisiteur dont les successeurs de La Reynie,
par qui commencèrent ces fonctions importantes, mais obscures,
étendirent beaucoup le champ pour se donner plus de relations avec le
roi, et cheminer mieux vers l'importance, l'autorité, la fortune. Le
régent, moins autorisé que le feu roi, et qui avait plus de raisons que
lui d'être informé et d'arrêter les intrigues, trouva dans cette place
Argenson, qu'on a vu qui avait su se faire valoir à lui de l'affaire du
cordelier, amené par M. de Chalais, et en avait, je crois, à bon marché,
acquis les bonnes grâces. Argenson, qui avait beaucoup d'esprit, et qui
avait désiré cette place comme l'entrée, la base et le chemin de sa
fortune, l'exerçait très supérieurement, et le régent se servit de son
ministère avec beaucoup de liberté. Le parlement, qui n'était attentif
qu'à faire valoir partout son autorité, pour le moins comme en
compétence avec celle du régent, souffrait avec impatience ce qu'il
appelait les entreprises de la cour. Il voulait se dédommager du silence
qu'il avait été forcé de garder là-dessus sous le dernier règne, et
reprendre aux dépens du régent tout ce qu'il avait perdu sur les
fonctions de la police, dont il est le supérieur. Le lieutenant de
police lui en est comptable, jusque-là qu'il en reçoit les ordres, même
les réprimandes à l'audience publique, debout et découvert à la barre du
parlement, de la bouche du premier président ou de celui qui préside,
qui ne l'appelle ni maître ni monsieur, mais nûment par son nom, quoique
le lieutenant de police se soit trouvé les recevoir étant alors
conseiller d'État. Le parlement voulut donc humilier d'Argenson qu'il
haïssait du temps du feu roi, donner au régent une dure et honteuse
férule, préparer pis à son lieutenant de police, faire parade et preuve
de son pouvoir, en effrayer le public, et s'arroger celui de borner
celui du régent.

Argenson s'était souvent servi sous l'autre règne, et quelquefois
depuis, d'un drôle intelligent et adroit, qui était fort à sa main, et
qui se nommait Pomereu, pour des découvertes, pour faire arrêter des
gens, et quelquefois les garder chez lui quelque temps. Le parlement
crut avec raison qu'en faisant arrêter cet homme sous d'autres
prétextes, il trouverait le bout d'un fil qui le conduirait en bien des
tortuosités curieuses et secrètes qui donneraient beau jeu à son
dessein, et le parerait en même temps lui-même de la protection de la
sûreté publique, contre la tyrannie des enlèvements obscurs et des
chartres privées. Il se servit pour cela de la chambre de justice pour y
paraître moins, mais composée de ses membres, qui souffla si bien les
procédures de peur d'être arrêtée en chemin, que le premier soupçon
qu'on en put avoir fut d'apprendre que Pomereu était par arrêt de cette
chambre dans les prisons de la Conciergerie, qui sont celles du
parlement. Argenson, qui en eut l'avis tout aussitôt, alla au moment
même trouver le régent, qui à l'instant fit expédier une lettre de
cachet, avec laquelle il envoya main-forte pour tirer Pomereu de prison,
si le geôlier faisait la moindre difficulté de le remettre aux porteurs
de la lettre de cachet, lequel n'en osa faire aucune. L'exécution fut si
prompte que cet homme ne fut pas une fleure dans la prison, et que ceux
qui l'y avaient mis n'eurent pas le temps d'ouvrir un coffre de papiers,
qui avait été transporté avec lui à la Conciergerie, et qu'on eut grand
soin d'emporter en l'en tirant. En même temps on écarta, et on mit à
couvert tout ce qui pouvait avoir trait à cet homme, et aux choses où il
avait été employé. On peut juger du dépit du parlement de se voir si
hautement et si subitement enlever une proie dont il comptait faire un
si grand usage\,; il n'oublia donc rien pour émouvoir le public par ses
plaintes et par ses cris contre un tel attentat à la justice. La chambre
de justice députa au régent qui se moqua d'elle, en permettant gravement
aux députés de faire reprendre leur prisonnier, mais sans leur dire un
seul mot sur sa sortie de prison. Il était dans Paris en lieu où on ne
craignait personne. La chambre de justice sentit la dérision et cessa de
travailler. Elle crut embarrasser le régent, mais c'eût été à leurs
propres dépens. Cela ne dura qu'un jour au deux. Le duc de Noailles alla
leur parler\,; ils comprirent qu'il n'en serait autre chose\,; que s'ils
s'opiniâtraient on se passerait d'eux, et qu'on aurait d'autres moyens
d'exécuter ce qu'on {[}avait{]} entrepris contre les gens d'affaires.
Ils se remirent à travailler, et le parlement en fut pour sa levée de
bouclier, et n'avoir montré que sa mauvaise volonté et en même temps son
impuissance.

M. le duc d'Orléans nomma gouvernante de mesdemoiselles ses filles
M\textsuperscript{me} de Cheverny dont le mari était déjà gouverneur de
M. le duc de Chartres. Ils en étaient l'un et l'autre fort capables, et
la naissance et les emplois précédents de Cheverny honorèrent fort ces
places qu'ils voulurent bien accepter.

Livry, premier maître d'hôtel du roi, obtint pour son fils la survivance
de sa charge, et de conserver un brevet de retenue de quatre cent
cinquante mille livres qu'il avait dessus.

Effiat, ravi d'abord d'être quelque chose, trouva enfin son mérite peu
distingué par la vice-présidence du conseil des finances. Il n'y voulut
plus demeurer, mais entrer dans celui de régence à la dernière place. M.
le duc d'Orléans eut la pitoyable facilité de le lui accorder, à la
grande satisfaction de ses bons amis le duc du Maine, le maréchal de
Villeroy et le chancelier. Personne ne s'en douta que lorsque cela fut
fait.

Ce prince, dont la facilité se pouvait appeler un dévoiement, accorda
les honneurs du Louvre leur vie durant à Dangeau et à la comtesse de
Mailly, qu'ils avaient perdus avec leurs charges de chevalier d'honneur
et de dame d'atours par la mort de la dernière Dauphine. Le feu roi les
leur avait donnés avec ces charges, n'y ayant lors ni reine ni Dauphine.
C'en fut le premier exemple, qu'ils durent à M\textsuperscript{me} de
Maintenon. Il n'y avait jamais eu que chez la reine où ces charges
donnassent ces honneurs, et encore fort nouvellement\,; et je doute même
que cela ait été du temps de la reine mère, avant le mariage du roi son
fils, tout au plus avant sa régence. Pour chez les Dauphines, il n'y en
avait point eu depuis la mort de François Ier jusqu'au mariage de
Monseigneur\,; car la trop fameuse Marie Stuart, qui la fut un moment,
garda et communiqua à François Ier, son mari, Dauphin, le nom et le rang
de reine et de roi d'Écosse en l'épousant\,; d'où vient, pour le dire en
passant, que la république de Venise a conservé de là l'usage, en
écrivant à nos Dauphins, de les traiter à la royale, et de suscrire leur
lettre \emph{au roi dauphin}.

On a vu en son lieu, ici, à propos de M\textsuperscript{me} de
Maintenon, qu'au mariage de Monseigneur elle voulut avoir une dame
d'honneur de sa confiance\,; que pour cela on fit passer la duchesse de
Richelieu, dame d'honneur de la reine, à M\textsuperscript{me} la
Dauphine\,; que pour payer sa complaisance on fit présent au duc de
Richelieu de la charge de chevalier d'honneur, avec permission dès lors
de la vendre tout ce qu'il en pourrait trouver\,; que
M\textsuperscript{me} de Maintenon voulut un titre pour se recrépir, et
qui l'approchât de la Dauphine sans la contraindre pour le service\,;
que pour cela il y eut pour le premier exemple deux dames d'atours\,: la
maréchale de Rochefort pour l'être en effet, et M\textsuperscript{me} de
Maintenon pour en avoir le nom. Ainsi le chevalier d'honneur et la
première dame d'atours se trouvant avoir par eux-mêmes les honneurs du
Louvre, M\textsuperscript{me} de Maintenon, à titre de seconde dame
d'atours, les prit modestement, sous prétexte de l'éloignement des cours
où tous les carrosses entrent de l'appartement qu'elle occupait dès
lors, et qu'elle n'a jamais changé, sur le palier du grand degré
vis-à-vis celui du roi. Ces honneurs du Louvre ne sont rien autre chose
que le privilège d'entrer dans son carrosse, ou en chaise avec des
porteurs de sa livrée, dans la cour réservée où il n'entre que les
carrosses et les porteurs en livrée des gens titrés. M. de Richelieu
vendit bientôt après sa charge de chevalier d'honneur cinq cent mille
livres à Dangeau. La charge était bien supérieure à celle de dame
d'atours. M\textsuperscript{me} de Maintenon, toujours modeste, se piqua
d'honneur sur les honneurs du Louvre qu'elle avait, et les fit donner à
Dangeau. Au mariage de Mgr le duc de Bourgogne, M\textsuperscript{me} de
Dangeau était déjà une des favorites de M\textsuperscript{me} de
Maintenon, qui la fit première dame du palais, rendre à son mari pour
rien la charge de chevalier d'honneur qu'il avait perdue à la mort de
M\textsuperscript{me} la Dauphine, et donner celle de dame d'atours à la
comtesse de Mailly, fille de son cousin germain, qu'elle avait élevée
chez elle comme sa nièce, et gardée jusqu'au mariage de M. le duc de
Chartres, qu'elle la fit dame d'atours, pour le premier exemple d'une
petite-fille de France, comme on l'a vu en son lieu. En même temps
qu'elle fit rendre à Dangeau les honneurs du Louvre, sur son exemple à
elle, elle les fit donner à la comtesse de Mailly. C'était une grâce de
peu d'usage pour ces deux personnes. Dangeau était dans une grande
vieillesse et hors de gamme par le total changement de la cour, ne
sortait presque plus de chez lui, ni sa femme non plus, très pieuse et
très retirée\,; et la comtesse de Mailly tombée tout à fait dans
l'obscurité, et passant sa vie au fond de la Picardie, d'où elle ne
revint que pour être dame d'atours de la reine, par l'intrigue de ses
enfants sans qu'elle y eût même pensé. Mais c'était pourtant une grâce
qu'ils ne méritaient pas de M. le duc d'Orléans. Tous deux lui étaient
fort apposés. Dangeau, avec toute sa fadeur et sa politique, ne peut se
contenir là-dessus dans l'espèce de gazette qu'il a laissée, dont on
parlera ailleurs. Il n'avait jamais été de rien\,; mais son commerce et
sa société à la cour du feu roi n'était qu'avec tout ce qui était le
plus contraire à M. le duc d'Orléans. C'était plaire alors, et le bon
air. Son attachement servile à M\textsuperscript{me} de Maintenon, et à
tout ce qu'elle aimait, celui de M\textsuperscript{me} de Mailly à cette
tante, leur avaient fait épouser ses passions, desquelles après ils ne
purent se défaire.

La fête de la Toussaint fit du bruit et des querelles. Le roi entend ce
jour-là une grand'messe pontificale, vêpres et le sermon l'après-dînée.
Celui qui le fait prêche l'Avent devant le roi, et c'est le grand
aumônier qui nomme de droit les prédicateurs de la chapelle. Le cardinal
de Rohan, qui n'ignorait ni ne pouvait ignorer l'interdiction des
jésuites, en voulut nommer un, mais dont le nom pût soutenir
l'entreprise. Il choisit le P. de La Ferté, frère du feu duc de La
Ferté, dont la veuve était soeur de la duchesse de Ventadour\,; et le P.
de La Ferté accepta sur la parole du cardinal de Rohan, sans voir ni
faire rien dire au cardinal de Noailles. Ce cardinal apprit cette
nouvelle aux derniers jours d'octobre, qui jusqu'alors avait été tenue
fort secrète. Il n'eut pas peine à comprendre que cette affectation de
nommer un jésuite ne pouvait avoir d'objet qu'une insulte, tant à sa
personne qu'à sa qualité de diocésain. Rien n'était plus aisé que de la
rendre inutile. Il avait interdit les jésuites\,; il n'y avait qu'à
faire signifier au P. de La Ferté une interdiction personnelle de la
messe, du confessionnal et de la chaire. Il usait de son droit qui ne
pouvait lui être contesté, comme le cardinal de Rohan avait usé du sien,
mais avec entreprise contre l'interdiction générale de
l'ordinaire\footnote{Ce mot est en abrégé dans le manuscrit, et les
  anciens éditeurs ont lu l'\emph{ordre\,;} il s'agit ici de
  l'interdiction prononcée par l'évêque diocésain, qu'on appelait
  l'\emph{ordinaire}.}, au lieu qu'il n'y aurait eu rien à reprendre
dans cette démarche très régulière du cardinal de Noailles. Sa douceur
si souvent déplacée, et mal employée, ne voulut pas faire cette manière
d'éclat qui n'eût été que la suite forcée de celui qui était déjà fait,
et il prit le mauvais parti de nommer un prédicateur pour la chapelle,
au lieu du P. de La Ferté, dont il n'avait pas le droit. Le cardinal de
Rohan, ravi de lui voir prendre le change, et de n'avoir qu'à soutenir
son droit, le maintint de façon qu'il fallut porter la chose devant M.
le duc d'Orléans.

Le crédit, où le duc de Noailles était pour lors, l'eût emporté d'un
mot, s'il avait voulu le dire\,; mais dès la mort du roi tout était
tourné en lui au personnel, mieux caché auparavant. Il n'avait jamais
perdu son grand objet de vue\,: il voulait être premier ministre. Son
crédit, la part que le régent lui donnait de tout, et les commissions
qu'il s'en attirait pour tout, lui en augmentaient les espérances\,; il
en voulait ranger les obstacles de tous les côtés. Il frayait déjà avec
les cardinaux de Rohan et Bissy, et avec les jésuites\,; il n'avait donc
garde de les choquer pour un oncle dont il n'avait plus besoin, et dont
la cause lui pouvait faire embarras, tandis qu'en ne disant mot, et lui
laissant démêler cette affaire particulière sans s'en mêler, il se
faisait un mérite envers ceux qu'il cultivait, qui pouvait tourner en
preuve qu'ils n'avaient rien à craindre de lui sur celle de la
Constitution, par conséquent leur ôter l'envie de le traverser et de le
barrer dans le chemin au premier ministère. À son défaut M. de Châlons,
son autre oncle, intimement uni avec le cardinal son frère, mais qui, en
affaires du monde, n'était pas grand clerc, alla nasiller coup sur coup
au régent, qui emporté par ses plus vrais ennemis, M\textsuperscript{me}
de Ventadour, le maréchal de Villeroy, Effiat, Besons, son P. du
Trévoux, celui-ci sot et point méchant, et qu'il ménageait et traitait
tous comme ses amis intimes, décida pour le P. de La Ferté, et le fit
prêcher au scandale de tout le monde non confit en cabale de
Constitution\,; car ceux même qui de bonne foi et sans vue de fortune
étaient pour la Constitution détestèrent cette entreprise.

M. de Fréjus commença, à la même fête, tout petit garçon qu'il était
encore, à montrer les cornes au cardinal de Noailles, et à vérifier la
prophétie que le feu roi lui avait faite, lorsqu'à force de reins il lui
arracha l'évêché de Fréjus pour l'abbé Fleury\,: \emph{qu'il se
repentirait de l'avoir fait évêque}. Le roi l'entendait de ses moeurs et
de sa conduite\,; et véritablement alors, qui aurait pu l'entendre
autrement\,? M. de Fréjus dit pontificalement la grand'messe devant le
roi sans en demander permission ni en faire la moindre civilité, suivant
le droit et la coutume jusque-là non interrompue, au cardinal de
Noailles, qui le sentit et le méprisa. L'après-dînée, à vêpres, la
duchesse de La Ferté quêta à l'issue du sermon de son beau-frère. Ce fut
une autre nouveauté de voir quêter une vieille femme\,; mais elle voulut
par là courtiser la soeur, et le triomphe du cardinal de Rohan sur
toutes règles de discipline. Cette même messe fit une autre querelle.
L'abbé de Breteuil, mort depuis évêque de Rennes, y parut sur un
tabouret, en rochet et camail noir, joignant le prie-Dieu du roi à
gauche en avant, comme maître de la chapelle, {[}charge{]} qu'il avait
achetée du cardinal de Polignac. Les aumôniers du roi, qui sont là
debout en rochet avec le manteau noir par-dessus, se plaignirent de
cette comparution de l'abbé de Breteuil, et traitèrent son tabouret et
son camail d'entreprise, parce qu'il n'était pas évêque. Les plaintes en
furent portées à M. le duc d'Orléans qui, perquisition faite, condamna
l'abbé de Breteuil. Le cardinal de Rohan ne laissa pas de se trouver
embarrassé de soutenir pendant tout l'Avent son entreprise, quoiqu'il en
eût eu l'avantage. Il crut qu'après l'avoir remportée, le plus sage
était le parti de la modération, mais sans y paraître à découvert. Huit
jours après la Toussaint, le P. de La Ferté alla dire à M. le duc
d'Orléans qu'il le suppliait de le dispenser de prêcher l'Avent devant
le roi, parce qu'il ne voulait point être un sujet de discorde entre le
cardinal de Noailles et le cardinal de Rohan. M. le duc d'Orléans le
prit au mot avidement, et lui dit qu'il l'en louait fort, et qu'il le
soulageait beaucoup. Ce P. de La Ferté avait été séduit au collège, et
s'était fait jésuite malgré le maréchal son père, qui fit tout ce qu'il
put pour l'en empêcher\,; et qui n'en parlait qu'avec emportement. Il
était grand, très bien fait, très bel homme, ressemblait fort an duc de
La Ferté son frère dont il avait toutes les manières, et n'était point
du tout fait pour être jésuite. Il était éloquent et savait assez,
beaucoup d'esprit et d'agrément\,; le jugement n'y répondait pas. Il
prêchait bien sans être des premiers prédicateurs. On traîna un jour le
duc de La Ferté à son sermon, dont après on lui demanda son avis\,:
\emph{L'acteur}, dit-il, \emph{m'a paru assez bon, mais la pièce assez
mauvaise}. Le P. de La Ferté ne s'était pas toujours bien accordé avec
les jésuites\,; il ne fut pas, je crois, sans repentir de s'être laissé
enrôler par eux. Sans ses voeux, il aurait été duc et pair à la mort de
son frère, qui ne laissa point d'enfants. À la fin les jésuites et lui,
lassés de lui et lui d'eux, le malmenèrent, puis le confinèrent à la
Flèche où il vécut peu et tristement, et y mourut encore assez peu âgé.
Le cardinal de Noailles interdit les trois maisons des jésuites de
Paris, et ôta les pouvoirs au peu à qui il les avait laissés.

En ce même temps l'abbé Fleury, qui avait été sous-précepteur des trois
princes fils de Monseigneur jusqu'à la fin de leur éducation, fut nommé
confesseur du roi. Le maréchal de Villeroy ni M. de Fréjus n'y voulaient
point de jésuite. L'emploi précédent, sans avoir eu part à la disgrâce
de M. de Cambrai, l'y porta. Il avait vécu à la cour dans une grande
retraite et dans une grande piété toute sa vie, fort caché depuis que
son emploi avait cessé. Il n'avait pris aucune part à l'affaire de la
Constitution, parce qu'il ne songea jamais à être évêque, et que,
n'étant point en place qui l'y obligeât, il aima mieux demeurer en paix
à ses études. L'exacte et savante \emph{Histoire ecclésiastique} qu'on a
de lui, et ses excellentes et savantes préfaces en forme de discours
au-devant de chacun des livres qui composent ce grand ouvrage, rendront
à jamais témoignage de son savoir et de son amour pour la vérité. Il eut
peine à consentir à son choix\,; il {[}ne{]} s'y détermina que par l'âge
du roi, où il n'y avait rien à craindre, et par le sien, qui lui
donnerait bientôt prétexte de se retirer, comme il fit en effet avant
qu'il pût avoir lieu de craindre son ministère, pendant lequel il ne
parut que pour la pure nécessité.

M\textsuperscript{me} d'Armenonville mourut de la petite vérole, qui fit
sur jeunes et vieux bien du ravage toute cette année. Peu de jours après
la duchesse de Richelieu en mourut aussi sans enfants. Elle était fille
unique du marquis de Noailles, frère du cardinal et de la duchesse de
Richelieu, troisième femme du père de son mari. C'était une très jeune
femme, mais de vertu, d'esprit et de beaucoup de mérite, que le bel air
de son mari n'avait pas rendue heureuse.

Le maréchal de Châteaurenaud mourut à plus de quatre-vingts ans. C'était
un fort homme d'honneur\,; très brave, très bon homme, et très grand et
heureux homme de mer, où il avait eu de belles actions, que le malheur
même de Vigo ne put ternir. Avec tout cela, il se peut dire qu'il
n'avait pas le sens commun. Son fils unique avait épousé une dernière
soeur du duc de Noailles, par où il avait eu la survivance de la grande
lieutenance générale de Bretagne qu'avait son père. Trois jours avant sa
mort, le duc de Noailles avait furtivement obtenu et fait expédier
sur-le-champ un brevet de retenue de cent vingt mille livres pour sa
soeur, sur la charge de vice-amiral, qui jamais n'avait été vendue, et
qui fut présenté à Coetlogon, premier lieutenant général qui la demanda,
qui ne s'attendait à rien moins qu'à cette apparition, et qui n'en
voulut pas payer un denier. C'était, aussi bien que Châteaurenaud, un
des plus braves hommes et des meilleurs hommes de mer qu'il y eût. Sa
douceur, sa justice, sa probité et sa vertu ne furent pas moindres. Il
avait acquis l'affection et l'estime de toute la marine, et plusieurs
actions brillantes lui avaient fait beaucoup de réputation chez les
étrangers. Il avait du sens avec un esprit médiocre, mais fort suivi et
appliqué. On fut honteux à la fin de cette espièglerie de brevet de
retenue, pour n'en dire pis, et sans lui plus rien demander on lui donna
la vice-amirauté. Le duc de Noailles rapporta le brevet de retenue à M.
le duc d'Orléans, qui le jeta au feu, et fit donner les cent vingt mille
livres aux dépens du roi, que le duc de Noailles fit payer à sa soeur en
grand ministre qui ne négligeait rien. Je dépasserai tout de suite le
temps de ces Mémoires sur Coetlogon, en faveur de sa vertu et de la
singularité du fait.

M. le Duc, devenu premier ministre sous les volontés de
M\textsuperscript{me} de Prie, sa funeste maîtresse, et tous les deux
sous la fatale tutelle des frères Pâris, fit, au premier jour de l'an
1724, une promotion de maréchaux de France et une de chevaliers de
l'ordre, toutes deux fort ridicules. Il donna l'ordre à Coetlogon, aussi
mal à propos qu'il ne le fit point maréchal de France, au scandale de la
marine, de toute la France et de tous les étrangers qui le connaissaient
de réputation. Coetlogon en fut vivement touché\,; mais, consolé par le
cri public, il n'en fit aucune plainte, et s'enveloppa dans sa vertu et
dans sa modestie. Quelques années après, étant fort vieux, il se retira
dans une des maisons de retraite du noviciat des jésuites, où il ne
pensa plus qu'à son salut par toutes sortes de bonnes oeuvres. Alors
d'Antin et le comte de Toulouse, qui avait épousé la veuve de son fils,
soeur du duc de Noailles, laquelle en avait eu deux fils, songèrent à
faire donner au cadet de ces deux petits-fils de d'Antin tout jeune, la
vice-amirauté de Coetlogon, pour, avec l'appui du comte de Toulouse,
amiral, son beau-père, voler de là rapidement au bâton de maréchal de
France. Ils le proposèrent à Coetlogon, ils lui offrirent tout l'argent
qu'il en voudrait tirer\,; enfin ils lui montrèrent le bâton de maréchal
de France, qu'il avait si bien mérité. Coetlogon demeura inflexible, dit
qu'il ne vendrait point ce qu'il n'avait pas voulu acheter, protesta
qu'il ne ferait point ce tort au corps de la marine de priver de leur
fortune ceux que leurs services et leur ancienneté devaient faire
arriver après lui. On sut cette généreuse réponse, moins par lui que par
les gens qui lui avaient été détachés, et par les plaintes du peu de
succès. Le public y applaudit et la marine en fut comblée. Peu après il
tomba malade de la maladie dont il mourut.

Son neveu, car il n'avait point été marié, touché de la privation pour
sa famille de l'illustration que son oncle avait si bien méritée, fit
tant que le comte de Toulouse obtint du cardinal Fleury, premier
ministre alors, le bâton de maréchal de France pour Coetlogon qui se
mourait, qui ne savait rien de ce que faisait son neveu, et qui n'en
pouvait plus jouir. Son confesseur lui annonça cet honneur. Il répandit
qu'autrefois il y aurait été fort sensible\,; mais qu'il lui était
entièrement indifférent dans ces moments, où, il voyait plus que jamais
le néant du monde qu'il fallait quitter, et le pria de ne lui parler
plus que de Dieu, dont il ne fit plus que s'occuper uniquement. Il
mourut quatre jours après sans avoir pensé un instant à son bâton. Cette
promotion singulière rappela celle de M. de Castelnau, et la fourberie
du cardinal Mazarin que le cardinal Fleury s'applaudit d'avoir si bien
imitée.

La duchesse d'Orval mourut à quatre-vingt-dix ans. Elle était
belle-fille du célèbre Maximilien de Béthune, premier duc de Sully, et
belle-soeur du fameux duc de Rohan. M. d'Orval fut chevalier de l'ordre
en 1633, et duc à brevet en 1652. Il avait été, dès 1627, premier écuyer
de la reine Anne d'Autriche\,; et il était veuf de la fille du maréchal
duc de La Force, duquel mariage le duc de Sully d'aujourd'hui est
arrière-petit-fils. La duchesse d'Orval était Harville, soeur de
Palaiseau.

D'Aguesseau, conseiller d'État et du conseil royal des finances du feu
roi, et de celui des finances d'alors, mourut en même temps à
quatre-vingt-deux ans\,; père du procureur général, qui tôt après fut
fait chancelier. C'était un petit homme de basse mine, qui, avec
beaucoup d'esprit et de lumières, avait toute sa vie été un modèle, mais
aimable, de vertu, de piété, d'intégrité, d'exactitude dans toutes tes
grandes commissions de son état par où il avait passé, de douceur et de
modestie, qui allait jusqu'à l'humilité, et représentant au naturel ces
vénérables et savants magistrats de l'ancienne roche\footnote{Voy. la
  vie de ce magistrat écrite par son fils. Elle fait partie des
  \emph{Oeuvres du chancelier d'Aguesseau}. Il serait à souhaiter qu'on
  la publiât à part pour donner une idée de cette ancienne magistrature
  si bien louée par Saint-Simon.} qui sont disparus avec lui, soit dans
ses meubles et son petit équipage, soit dans sa table et son maintien.
Sa femme était de la même trempe, avec beaucoup d'esprit. Il n'avait
aucune pédanterie\,; la bonté et la justice semblaient sortir de son
front. Il avait laissé en Languedoc, où il avait été intendant, les
regrets publics et la vénération de tout le mande. Son esprit était si
juste et si précis que les lettres qu'il écrivait des lieux de ses
différents emplois disaient tout sans qu'on ait jamais pu faire
d'extrait de pas une. Je fis tout ce que je pus pour obtenir sa place de
conseiller d'État pour Le Guerchois, son gendre, intendant de
Franche-Comté, mon ami particulier, depuis bien des années que lui et sa
famille m'avaient si bien servi à Rouen dans le procès qu'on a vu en son
lieu que j y gagnai contre le duc de Brissac et la duchesse d'Aumont. Je
n'en pus venir à bout, parce qu'en même temps Bâville, ce funeste roi de
Languedoc plutôt qu'intendant, demanda à se démettre de sa place de
conseiller d'État en faveur de Courson, son fils. M. le duc d'Orléans,
qui vit la conséquence de l'exemple, et ne voulant pas le refuser, la
donna à Saint-Contest, et celle que je demandais à Courson\,; mais je
n'eus pas longtemps à attendre. En même temps les conseillers d'État
obligèrent Saint-Contest à quitter le conseil de guerre, pour n'y pas
céder aux gens de qualité qui en étaient. On a vu en son temps la
naissance de cette rare prétention lorsque La Houssaye, conseiller
d'État et intendant d'Alsace, fut nommé en troisième pour le congrès de
Bade, où il ne voulut pas céder au comte du Luc. On a vu en son lieu que
le feu roi s'en moqua\,; mais il le souffrit, et nomma Saint-Contest,
maître des requêtes alors et intendant de Metz, pour aller à Bade.

L'empereur fit, par le prince Eugène la, conquête de Temeswar, en
Hongrie, et perdit son fils unique âgé de sept mois.

La duchesse de Saint-Aignan alla trouver son mari en Espagne, pour
lequel j'obtins une gratification qu'elle emporta. Elle fut de trente
mille livres.

M. d'Étampes mourut dans un âge avancé. Il était riche, honnête homme et
fort brave. Il avait été chevalier d'honneur de Madame, puis capitaine
des gardes de Monsieur, qui le fit chevalier de l'ordre en 1688 de la
façon qu'on l'a raconté en son temps. Il était petit-fils du maréchal
d'Étampes, et par ses grand'mères des maréchaux de Fervaques et Praslin.
Son père était premier écuyer de Monsieur, frère de Louis XIII\,; et sa
mère était fille de Puysieux, secrétaire d'État, et de sa seconde femme,
Ch. d'Étampes-Valencey, dont un frère s'avisa, pour le premier de sa
race, de se faire de robe, et fut conseiller d'État, qu'elle n'appelait
jamais que mon frère le bâtard, parce que son frère aîné était chevalier
du Saint-Esprit, grand maréchal des logis et gouverneur de Montpellier
et de Calais, un autre archevêque de Reims, un autre cardinal, et sa
soeur mariée au maréchal de La Châtre. Cette M\textsuperscript{me} de
Puysieux avait un grand crédit sur la reine mère, et dans le monde une
considération singulière. Elle maria son fils à la soeur du duc de La
Rochefoucauld, favori de Louis XIV, et le ruina en dépenses
extravagantes, entre autres à manger pour cent mille écus de collets de
points de Gênes, qui étaient fort à la mode alors. Puysieux, mort
chevalier de l'ordre, son frère l'évêque de Soissons, et Sillery père de
Puysieux d'aujourd'hui, étaient ses petits-fils.

En même temps mourut la comtesse de Roucy, sans nous donner signe de vie
ni de repentir. J'ai été trop de ses amis, et j'en ai été trop mal payé
depuis, pour vouloir rien dire d'elle, d'autant que j'ai suffisamment
exposé ma conduite et la sienne, et celle de son mari, dans l'éclat
qu'ils jugèrent à propos de faire pour essayer vainement d'obtenir une
charge de capitaine des gardes du corps.

Peu après mourut à Paris M\textsuperscript{me} Fouquet dans une grande
piété, dans une grande retraite et dans un exercice continuel de bonnes
oeuvres toute sa vie. Elle était veuve de Nicolas Fouquet, célèbre par
ses malheurs, qui, après avoir été huit ans surintendant des finances,
paya les millions que le cardinal Mazarin avait pris la jalousie de MM.
Le Tellier et Colbert\,; un peu trop de galanterie\footnote{Voy. Note I
  à la fin du volume.} et de splendeur, et trente-quatre ans de
prison\footnote{Fouquet fut arrêté à Nantes en 1661 (septembre) et
  mourut à Pignerol en 1680, comme le dit Saint-Simon quelques lignes
  plus loin\,; il n'a donc pas été trente-quatre ans prisonnier. Les
  précédents éditeurs avaient substitué \emph{vingt-quatre ans} à
  trente-quatre ans. Cette correction, en altérant le texte, n'avait pas
  le mérite de l'exactitude chronologique, puisque la captivité de
  Fouquet n'a duré que dix-neuf ans.} à Pignerol, parce qu'on ne put lui
faire pis malgré tout le crédit des ministres et l'autorité du roi, dont
ils abusèrent jusqu'à avoir mis tout en oeuvre pour le faire périr. Il
mourut à Pignerol en 1680, à soixante-cinq ans, tout occupé depuis
longues années de son salut. Lui et cette dernière femme, grand'mère de
Belle-Île, seraient maintenant bien étonnés de la monstrueuse et
complète fortune qu'il a su faire, et par quels degrés il y est parvenu.
Cette M\textsuperscript{me} Fouquet était soeur de Castille, père du
père de M\textsuperscript{me} de Guise. Il s'appelait Montjeu, était
trésorier de l'épargne, et sa mère était fille du célèbre président
Jeannin. Il avait acheté en 1657 du président de Novion, qui fut depuis
premier président et ôté de place pour ses friponneries, la charge de
greffier de l'ordre. On l'arrêta en même temps que M. Fouquet, et on lui
ôta ses deux charges et le cordon bleu. Sa résistance à donner sa
démission de celle de greffier de l'ordre la fit donner par commission à
Châteauneuf, secrétaire d'État, qui l'eut longtemps de la sorte, jusqu'à
ce que le titulaire, lassé de tant d'années d'exil, donna enfin sa
démission. Je raconte en deux mots ces vieilleries parce qu'elles sont
pour la plupart oubliées, et que, par la postérité qui en reste, elles
méritent qu'on s'en souvienne quelquefois.

M. le duc d'Orléans qui, sans distinction pour le moins, lâchait tout à
amis et plus encore à ennemis, que cela ne lui réconciliait pas le moins
du monde, donna au maréchal de Montesquiou, tout à M. du Maine, le
commandement de Bretagne, et la commission d'en tenir les états qu'avait
le maréchal de Châteaurenaud\,; cent mille écus de brevet de retenue au
grand prévôt sur sa charge fort inutilement\,; au duc de Villeroy,
capitaine des gardes du corps, et au duc de Guiche, colonel du régiment
des gardes, la survivance de leurs charges pour leurs fils aînés tout
jeunes, et celle encore de leurs gouvernements. Le duc de Tresmes eut
aussi pour son fils aîné la survivance de sa charge de premier
gentilhomme de la chambre.

Il fit au comte de Hanau une grâce également étrange et préjudiciable à
l'État. Ce comte, le premier de l'empire, et qui vivait delà le Rhin
avec une cour de souverain, dont il avait les États et les richesses,
avait, pour un grand revenu et un vaste domaine de morceaux différents,
des fiefs situés dans le pays Messin, qui étaient tous masculins, et
tombaient, faute d'hoirs mâles, à la nomination du roi les uns, et les
autres à celle de l'évêque de Metz\,; mais qui retombaient à celle du
roi, par les difficultés qui avaient arrêté jusqu'alors la foi et
hommage des évêques de Metz qui ne l'avaient pas rendue. Le comte de
Hanau n'avait point de garçons, mais une seule fille, à qui il voulut
donner ses fiefs en la mariant à un prince de Hesse-Darmstadt. C'est à
quoi M. le duc d'Orléans consentit le plus légèrement du monde, et lui
fit promptement expédier tout ce qui était nécessaire pour la solidité.
Il est vrai qu'il n'y avait point d'ouverture de fief, puisque le comte
d'Hanau était plein de vie, mais il n'y avait qu'à attendre sans faire
cette très inutile grâce anticipée à un seigneur allemand pour marier sa
fille à un autre Allemand, tous deux sujets de l'empire, tous deux delà
Rhin, tous deux qui ne pouvaient jamais servir ou nuire, et laisser au
roi à faire, à la mort du comte d'Hanau, de riches présents domaniaux
qui se présentent si rarement à faire, pour récompenser des seigneurs
français dont tant se ruinent à son service, et se défaire de ces
princes allemands avec qui {[}il faut{]} compter pour de grandes terres
au milieu, pour ainsi dire, du royaume, qui y font des amis et des
espions.

Le duc de La Force, qui grillait d'être de quelque chose, et qui en
était bien capable, intrigua si bien qu'il eut la place de
vice-président du conseil des finances qu'avait quittée le marquis
d'Effiat, dont les appointements étaient de vingt mille livres de rente.
Je lui représentai qu'il ne lui convenait pas de se parer de la robe
sale d'Effiat, d'être en troisième avec le maréchal de Villeroy et le
duc de Noailles, et parmi un tas de gens de robe qui y faisaient tout,
et qui ne le reconnaîtraient en rien, parce que Rouillé y était maître
absolu sous le duc de Noailles, que la matière de ce conseil était sale
de sa nature, odieuse presque en tout, dont les règles du dérèglement,
les formes, le jargon étaient fort dégoûtants. J'ajoutai qu'il n'y
serait de rien, par conséquent méprisé, ou que s'il voulait se mêler de
quelque chose, il se soulèverait toute cette robe qui se croirait
dérobée par un intrus, et qui vivrait avec lui en conséquence, et
donnerait une jalousie au duc de Noailles et un dépit de se voir
éclairé, dont sûrement il le ferait rudement repentir dès qu'il le
pourrait, parmi son sucre, son miel et ses caresses. J'ajoutai que de
l'humeur dont le parlement se montrait sur tout, de la misère publique,
du délabrement des finances, de la facilité du régent et {[}de{]} sa
timidité trop reconnue, il en pourrait résulter dès embarras fâcheux à
qui se serait mêlé des finances, et à lui plus qu'à pas un par la rage
du parlement à notre égard\,; enfin que le temps des opérations de la
chambre de justice, qu'il verrait suivies d'une grande déprédation des
taxes par la facilité du régent, était encore grande raison de le
déprendre du goût de cette place. Je ne me contentai pas de lui faire
faire ces réflexions pour une fois. Je les réitérai plusieurs sans y
gagner quoi que ce soit. L'affaire était presque faite, quand il m'en
parla\,; à ce que je vis après, il s'était apparemment douté que je ne
l'approuverais pas\,: aussi n'y voulus-je prendre aucune part, et elle
s'acheva comme elle avait été conduite. Quand M. le duc d'Orléans me
l'apprit, à qui je n'en avais pas ouvert la bouche, je ne pus m'empêcher
de montrer en gros mon sentiment. Quoiqu'il me parût en être bien aise,
il finit par trouver que j'avais raison\,; mais à chose faite je me
contentai de l'écorce, et ne voulus pas descendre au détail comme
j'avais fait avec le duc de La Force. Il se trouva très malheureusement
dans la suite que je n'avais que trop bien rencontré.

Broglio, gendre du chancelier Voysin, qui du temps de sa toute-puissance
dans les derniers temps du feu roi lui avait fait donner un gouvernement
et une inspection d'infanterie, était fils et frère aîné des maréchaux
de Broglio, dont il fut toute sa vie le fléau. C'était un homme de
lecture, de beaucoup d'esprit, très méchant, très avare, très noir,
d'aucune sorte de mesure, pleinement et publiquement déshonoré sur le
courage et sur toute sorte de chapitres\,; avec cela effronté, hardi,
audacieux, et plein d'artifices, d'intrigues et de manèges, jusque-là
que son beau-père le craignait, lui qui se faisait redouter de tout le
mande. Il se piquait avec cela de la plus haute impiété et de la plus
raffinée débauche, pourvu qu'il ne lui en coûtât rien, quoique fort
riche. Je n'ai guère vu face d'homme mieux présenter celle d'un réprouvé
que la sienne\,; cela frappait. Un gendre de Voysin ne devait pas être
un titre pour entrer dans la familiarité de M. le duc d'Orléans, qui
peut-être de tout le règne du feu roi ne lui avait jamais parlé. Je ne
sais qui le lui produisit, car sa petite cour obscure, qu'il appelait
ses roués et que le monde ne connaissait point sous d'autre nom, me fut
toujours parfaitement étrangère. Mais Broglio s'y initia si bien qu'il
fut de tous les soupers, et que de là il se mit à parler troupes en
d'autres temps au régent, sous prétexte de la connaissance que leur
usage et son inspection lui en avait donnée. Il s'ouvrit ainsi
quelquefois le cabinet où on lui voyait porter un portefeuille. De ce
travail, qui dura quelque temps deux et trois fois la semaine, sortit
une augmentation de paye de six deniers par soldat, avec un profit
dessus pour chaque capitaine d'infanterie, qui coûtèrent au roi pour
toujours sept cent mille livres par an. Il capta pour cela quelques gens
du conseil de guerre qui n'osèrent s'y opposer, dans la certitude que
Broglio n'eût rien oublié pour s'en faire un mérite dans les troupes à
leurs dépens, mais dont presque tout ce conseil et le public entier cria
beaucoup, dans un temps de paix et de désordre des finances qui ne
pouvaient suffire aux plus pressants besoins.

Broglio comptait bien se continuer du travail, et devenir par là un
personnage, et il avait persuadé le régent que les troupes l'allaient
porter sur les pavais. Tous deux se trompèrent lourdement M. le duc
d'Orléans, par une augmentation fort pesante aux finances, qui ne se
pouvait plus rétracter, qui ne tint lieu de rien, et dont le {[}gros{]}
des troupes ne s'aperçut seulement pas\,; Broglio en ce qu'il ne mit
plus le pied dans le cabinet pour aucun travail, et qu'il demeura dans
l'opprobre qu'il méritait à tant de titres. Il fut enfin noyé tout à
fait sous le ministère du cardinal Fleury, contre qui, en faisant sa
tournée, il s'échappa en propos les plus licencieux. Le cardinal, qui en
fut informé aussitôt, lui envoya ordre de revenir sur-le-champ, et, en
punition de son insolence, lui ôta sa direction sans récompense, car il
était devenu directeur de l'infanterie dont les appointements sont de
vingt mille livres. Il demeura donc chez lui fort obscur à Paris, et
fort délaissé. Quelque temps après il maria son fils à la fille de
Bezwald\footnote{On écrivait ordinairement ce nom \emph{Bezenval} ou
  \emph{Besenval} et on prononçait \emph{Besval}. Le baron de Besenval,
  fils de celui dont il est ici question, a laissé de curieux Mémoires.
  Voy. la notice de M. Sainte-Beuve sur ce personnage.}, colonel du
régiment des gardes suisses, et longtemps employé avec capacité en
Pologne et dans le Nord, et voulut la clause expresse que son fils ne
sertirait point, et que lui ni sa femme ne verraient jamais le roi, la
reine ni la cour. Je pense que voilà le premier exemple d'une si
audacieuse folie. Elle a été pleinement accomplie, et son fils a
toujours vécu inconnu, et dans la dernière obscurité.

Le duc de Valentinois fut enfin reçu le 14 décembre au parlement. Les
princes du sang ni bâtards ne s'y trouvèrent point\,; M. le duc
d'Orléans le leur avait fait promettre pour éviter tout inconvénient
entre eux. Il donna à d'Antin la survivance de sa charge des bâtiments
pour son second fils, que depuis son mariage an appelait le marquis de
Bellegarde.

M. d'Estaing maria son fils à la fille unique de M\textsuperscript{me}
de Fontaine-Martel, qui était une riche et noble héritière, ce qui fut
un mariage très assorti. M. le duc d'Orléans, qui, pour les raisons si
honnêtes qu'on a vues ailleurs, aimait M\textsuperscript{me} de
Fontaine-Martel et tout ce qui portait le nom de M. d'Arcy, son
beau-frère, et qui affectionnait particulièrement M. d'Estaing, qui
avait fort servi sous lui, et qui était un très galant homme, leur donna
sous la cheminée la survivance du gouvernement de Douai, qui est très
gros et qu'avait M. d'Estaing.

Biron, aujourd'hui si comblé d'honneurs et de richesses, et son fils
aussi de son côté, était fort pauvre alors, et chargé d'une grande
famille. Je l'avais fait entrer, comme on l'a vu, dans le conseil de
guerre. La nécessité pousse quelquefois à d'étranges choses\,: il
s'était enrôlé parmi les roués, et soupait presque tous les soirs chez
M. le duc d'Orléans avec eux, où pour plaire il en disait des
meilleures. Par ce moyen, il obtint une des plus étranges grâces que M.
le duc d'Orléans pût accorder et du plus pernicieux exemple. On a vu en
son lieu la désertion de Bonneval aux ennemis de la tête de son régiment
en Italie, et l'infâme cause de cette désertion. Il était homme de
qualité, de beaucoup d'esprit, avec du débit éloquent, de la grâce, de
la capacité à la guerre, fort débauché, fort mécréant, et le pillage
n'est pas chose qui effarouche les Allemands. Avec ces talents il était
devenu favori du prince Eugène, logé chez lui à Vienne, défrayé, et en
faisant les honneurs, et lieutenant général dans les troupes de
l'empereur. Soit esprit de retour, soit désir de se nettoyer d'une
fâcheuse tare, soit dessein d'espionnage et de se donner moyen de se
faire valoir chez l'empereur, il désira des lettres d'abolition, et
d'oser revenir se remontrer dans sa patrie. Biron en profita pour lui
faire épouser une de ses filles pour rien, lui pour son dessein du
crédit de Biron. L'abolition fut promise, le mariage conclu, et Bonneval
avec un congé pour trois mois de l'empereur vint consommer ces deux
affaires. Le régent néanmoins voulut faire approuver l'abolition au
conseil de régence. Je n'en pus avoir la complaisance. J'opinai contre,
et appuyai longtemps sur les raisons de n'en jamais accorder pour pareil
crime. Je ne fus pas le seul, mais peu s'y opposèrent, et en peu de
mots. Ainsi Bonneval vit le roi, le régent et tout le monde. Biron me
l'amena chez moi. Je n'ai point vu d'homme moins embarrassé. M. de
Lauzun fit la noce chez lui. Dix ou douze jours après, Bonneval s'en
retourna à Vienne, et n'a pas vu sa femme depuis, qui demeura toujours
chez son père. La catastrophe unique de Bonneval n'est ignorée de
personne. Il y aura peut-être occasion dans la suite d'en parler.

Le maréchal de Villeroy, à l'ombre de M\textsuperscript{me} de Ventadour
sa bonne amie, de l'enfance du roi, et du peu d'assiduité et de soin que
ce petit âge demandait des grands officiers de son service, s'était peu
à peu insinué à faire toutes leurs fonctions. Il était d'âge à se
souvenir de ce qui s'était passé en pareil cas entre son père,
gouverneur du feu roi, et les grands officiers de son service. Il
prétendait le leur ôter, et le faire tant que le roi aurait un
gouverneur, quoique condamné par l'exemple de son père\,; mais c'était
le temps des prétentions et des entreprises de toutes les espèces, et
celui des \emph{mezzo-termine} si chéris de la faiblesse ou de la
politique de M. le duc d'Orléans, qui ôtaient toujours quelque chose à
qui avait droit et raison pour le donner à qui ne l'avait pas, et
perpétuaient les divisions et les querelles. Les grand chambellan et
premiers gentilshommes de la chambre, grand maître et les deux maîtres
de la garde-robe présentèrent donc là-dessus un mémoire à M. le duc
d'Orléans, qui se trouva bien empêché d'avoir affaire des deux côtés à
si forte partie, dont la plus nombreuse, bien sûre de son droit, ne
voulut tâter d'aucun tempérament, et qui étaient pour abandonner leurs
fonctions avec un grand éclat, mais garder soigneusement leurs charges.
Le maréchal n'eut à leur opposer que ses grands airs, son importance,
son entreprise, dont un homme comme lui ne pouvait pas avoir le démenti.
À la fin pourtant il l'eut, et tout du long, et sans réserve\,; et les
grands officiers maintenus dans toutes leurs fonctions, même jusqu'à lui
ôter leur service s'ils arrivaient après qu'il l'aurait commencé. Il fut
outré, mais il fallut obéir à raison, droit et jugement, et n'en parler
pas davantage.

L'année finit dans une grande aigreur et fort marquée entre les princes
du sang et légitimés. Les deux mémoires que Davisard, avocat général du
parlement de Toulouse, avait faits pour les derniers étaient peu
mesurés. Il se crut au temps du feu roi. Il travailla à la manière dont
le P. Daniel avait fabriqué son \emph{Histoire de France}, dont on a
parlé en son lieu. Il en parut deux mémoires coup sur coup. L'égalité
était peu ménagée. C'était réponse au premier mémoire des princes du
sang, qui, en attendant leur réplique, à laquelle on travaillait, se
contraignirent peu en discours. M. le duc d'Orléans y fut mêlé de part
et d'autre, pour s'autoriser de lui, parce qu'il avait vu les mémoires
avant le public, et il en fut fort embarrassé. Ce prince était peut-être
le seul homme de tous les pays, et de tous les âges, qui, en si place,
le pût être de pareille affaire. Il avait largement éprouvé qu'il
n'avait pas un plus cruel ennemi que le duc du Maine, qui, pour usurper
l'autorité que lui donnait la nature, n'avait rien oublié pour le
perdre, et pour le déshonorer par ce qu'il y a de plus horrible, de plus
touchant, de plus odieux\,; qui lui avait disputé cette autorité en
pleine séance au parlement\,; et qui, tout particulier qu'il était
redevenu, établi comme il se le trouvait, dressait manifestement autel
contre autel contre lui. L'apothéose à laquelle il s'était élevé avait
révolté le ciel et la terre\,; ses artifices et les menées de
M\textsuperscript{me} sa femme n'en avaient pu encore adoucir l'horreur.

Ce procès du bâtard contre le légitime, cette parité d'état et d'issue
d'un double adultère public, ou d'une épouse reine, cette identité si
entière entre des enfants sortis du sacrement et du crime, révoltait
encore la nature, et n'intéressait pas moins le fils et la postérité de
M. le duc d'Orléans que la branche de Bourbon. Ainsi justice, vérité,
raison, religion, nature, intérêt de naissance, intérêt de pouvoir,
intérêt d'honneur, intérêt de sûreté (déshonorerai-je tant de saintes
raisons par un motif bien moins pur, mais si cher et si vif dans tous
les hommes\,?) intérêt si puissant de vengeance, tout concourait dans M.
le duc d'Orléans d'être ravi de se voir enfin en état de briser un
colosse sous lequel il avait été si près d'être écrasé, et de pouvoir le
mettre si facilement et si sûrement en miettes, avec la bénédiction de
Dieu et l'acclamation de tous les ordres du royaume et de tout le monde
en particulier, excepté une poignée d'affranchis ou de valets. Qui en sa
place n'eût pas acheté bien cher le bonheur d'une telle position\,? Elle
ne fit pas la plus légère sensation sur M. le duc d'Orléans\,; et pour
comble de la plus incroyable apathie, un détachement de soi-même si
prodigieux, et dont l'occasion aurait fait trembler les plus grands
saints sur eux-mêmes, ne lui fut d'aucun mérite, ni pour ce monde,
envers lequel il s'aveugla et se méprit si lourdement, ni pour l'autre
vers lequel il ne fit pas la plus légère réflexion. Hélas\,! la main de
Dieu était sur lui et sur le royaume\,; et il était dans cette affaire
la proie et le jouet d'Effiat, et des autres gens de cette espèce que le
duc du Maine avait auprès de lui, dont il ne se déliait pas, tandis
qu'il y était en garde contre ses plus éprouvés serviteurs.

Comme sur le parlement, j'avais pris le parti de ne lui jamais ouvrir la
bouche sur les bâtards. L'intérêt de rang, et ce qui s'était passé entre
M. du Maine et moi à la fin de l'affaire du bonnet sous le feu roi, me
rendait suspect, et après tout ce que nous nous étions dit dans d'autres
temps l'un à l'autre, sur tout ce qui regardait les bâtards, et en
particulier M. du Maine à son égard, il était honteux et empêtré avec
moi, et je n'avais plus rien à lui dire. Les princes du sang avaient été
fort aises de notre requête contre les bâtards qui n'avaient osé s'en
fâcher, mais qui l'étaient beaucoup. Je n'avais pas pris la peine d'en
rien dire au duc du Maine après qu'elle fut présentée, quoique revenus
ensemble comme an l'a vu sur un pied d'honnêteté. Pour le comte de
Toulouse, auprès de qui j'étais toujours nécessairement au conseil, au
premier qui se tint depuis la requête présentée je lui en fis civilité,
et je le priai de se souvenir que ce n'était, même fort tard, que ce que
j'avais toujours dit que nous ferions, à M\textsuperscript{me} la
duchesse d'Orléans et à M. {[}le duc{]} et M\textsuperscript{me} la
duchesse du Maine, du vivant du roi et depuis sa mort. Cela fut
honnêtement reçu, et les manières entre lui et moi n'en furent pas
depuis le moins du monde altérées\,; M. du Maine non plus\,; mais je
profitais et devant et après la requête de ce que je n'étais jamais de
son côté pour ne m'en point approcher. Lui quelquefois venait avant
qu'on se mît en place m'attaquer de politesse, et même encore depuis la
requête, mais sans nous en parler. Chez eux je n'y allais jamais. Je le
trouvais assez rarement chez M\textsuperscript{me} la duchesse
d'Orléans, et la conversation nous allait familièrement sans parler de
rien de conséquence. J'y trouvais fort sauvent M. le comte de Toulouse.
Avec lui nous parlions de tout, excepté de nos affaires avec eux, et des
leurs avec les princes du sang, mais jamais qu'entre
M\textsuperscript{me} la duchesse d'Orléans, moi en tiers, rarement mais
quelquefois la duchesse Sforze, qui ne nous fermait pas la bouche.
C'était de bonne heure les après-dînées où M\textsuperscript{me} la
duchesse d'Orléans n'était visible qu'à nous. Il faut maintenant parler
de ce qui se passa dans les derniers mais de cette année sur les
affaires étrangères.

\hypertarget{chapitre-vi.}{%
\chapter{CHAPITRE VI.}\label{chapitre-vi.}}

1716

~

{\textsc{Albéroni continue ses manèges de menaces et de promesses au
pape pour hâter son drapeau\,; y fait une offre monstrueuse.}}
{\textsc{- Sa conduite avec Aubenton.}} {\textsc{- Souplesse du
jésuite.}} {\textsc{- Réflexion sur les entreprises de Rome.}}
{\textsc{- Albéroni se soumet Aubenton avec éclat, qui baise le fouet
dont il le frappe, et fait valoir à Rome son pouvoir et ses menaces.}}
{\textsc{- Gesvres, archevêque de Bourges, trompé par le pape, qui est
moqué et de plus en plus menacé et pressé par Albéroni, qui fait écrire
vivement par la reine d'Espagne jusqu'à se prostituer.}} {\textsc{-
Triste situation de l'Espagne.}} {\textsc{- Abattement politique du P.
Daubenton, qui sacrifie à Albéroni une lettre du régent au roi
d'Espagne.}} {\textsc{- Audacieux et pernicieux usage qu'en fait
Albéroni.}} {\textsc{- Il fait au régent une insolence énorme.}}
{\textsc{- Réflexion.}} {\textsc{- Albéroni, dans l'incertitude et
l'embarras des alliances du régent, consulte Cellamare.}} {\textsc{-
Efforts des Impériaux contre le traité désiré par le régent.}}
{\textsc{- Conduite des Hollandais avec l'Espagne.}} {\textsc{-
Conférence importante avec Beretti.}} {\textsc{- Caractère de cet
ambassadeur d'Espagne.}} {\textsc{- Sentiment de Cadogan, ambassadeur
d'Angleterre à la Haye, sur l'empereur.}} {\textsc{- Étrange réponse
d'un roi d'Espagne au régent dictée par Albéroni, qui triomphe par des
mensonges.}} {\textsc{- Albéroni profite de la peur des Turcs et de
l'embarras du pape sur sa constitution Unigenitus, pour presser sa
promotion par menaces et par promesses.}} {\textsc{- Offres du pape sur
le clergé des Indes et d'Espagne.}} {\textsc{- Monstrueux abus de la
franchise des ecclésiastiques en Espagne.}} {\textsc{- Réflexion.}}
{\textsc{- Le pape ébranlé sur la promotion d'Albéroni par les cris des
Espagnols, raffermi par Aubenton.}} {\textsc{- Confiance du pape en ce
jésuite.}} {\textsc{- Basse politique de Cellamare et de ses frères à
Rome.}} {\textsc{- Cardinal de La Trémoille dupé sur la promotion
d'Albéroni, pour laquelle la reine d'Espagne écrit de nouveau.}}
{\textsc{- Sentiment d'Albéroni sur les alliances traitées par le
régent.}} {\textsc{- Il consulte Cellamare.}} {\textsc{- Réponse de cet
ambassadeur.}} {\textsc{- Manèges des Impériaux contre les alliances que
traitait le régent.}} {\textsc{- Altercations entre eux et les
Hollandais sur leur traité de la Barrière, qui ouvrent les yeux à ces
derniers et avancent la conclusion des alliances.}} {\textsc{- Beretti
abusé.}} {\textsc{- L'Espagne veut traiter avec les Hollandais.}}
{\textsc{- Froideur du Pensionnaire, qui élude.}}

~

Albéroni n'avait proprement qu'une unique affaire, c'était celle de son
chapeau, à laquelle toutes celles d'Espagne, dont il était entièrement
le maître, étaient subordonnées, et ne se traitaient que suivant la
convenance de l'unique. Ainsi, répondant aux avis qu'on a vu
qu'Aldovrandi lui avait donnés en lui mandant l'engagement que le pape
avait enfin pris de lui donner un chapeau, il lui manda que, sans
l'accomplissement de cette condition, la reine d'Espagne ne consentirait
jamais à aucune de toutes les choses que le pape pourrait désirer, comme
aussi, en recevant la grâce désirée, il promettait en récompense que le
pape ne serait ni pressé ni inquiété, de la part de l'Espagne sur la
promotion des couronnes, la sienne à lui étant faite. Il alla plus loin,
et ce plus loin fait frémir dans la réflexion de ce que peut un
ecclésiastique premier ministre, et jusqu'à quel excès monstrueux la
passion d'un chapeau le transporte\,: il offrit à ce prix une
renonciation perpétuelle du roi d'Espagne au droit de nomination de
couronne. En même temps il affectait d'aimer et de louer Aubenton, parce
qu'il le savait bien avant dans l'estime et dans l'affection du pape.
Ces sentiments toutefois dépendaient du besoin qu'il pouvait avoir du
confesseur, et de sa soumission, entière pour lui, nonobstant le crédit
et la confiance que sa place lui donnait auprès du roi d'Espagne. Le
jésuite en fit bientôt l'expérience. Il reçut une lettre du cardinal
Paulucci, qui le pressait de faire en sorte qu'en attendant
l'accommodement des deux cours, le roi d'Espagne eût la complaisance de
laisser jouir le pape de la dépouille des évêques qui viendraient à
mourir. C'était un des points de contestation entre les deux cours, et
contre lequel le conseil d'Espagne se serait fort élevé, surtout ainsi
par provision. À ce trait, pour le dire en passant, on reconnaît bien le
chancre rongeur de Rome sur les États qui s'en laissent subjuguer. Le
tribunal de la nonciature d'une part ôte aux évêques tout le
contentieux, et toute leur autorité sur leur clergé, et sur les
dispenses des laïques\,; et d'autre part celui de l'inquisition leur
enlève tout ce qui regarde la doctrine et les moeurs, et les soumet
eux-mêmes à sa juridiction, en sorte qu'il ne leur reste que les
fonctions manuelles\,; et quant à l'argent, quel droit a le pape sur la
dépouille des évêques morts, et de frustrer leurs héritiers et leurs
créanciers\,! Ce texte engagerait à un long discours qui n'est pas de
notre narration, mais qu'on ne peut s'empêcher de faire remarquer à
propos de la folle prostitution de la France à l'égard de Rome, depuis
la plaie que la Constitution \emph{Unigenitus}, et les noires cabales
qui l'ont enfantée et soutenue, y a portée dans le sein de l'Église et
de l'État.

Aubenton, qui voyait sans cesse le roi d'Espagne en particulier, lequel
souvent lui parlait d'affaires, s'avisa de lui montrer cette lettre de
Paulucci sans en avoir fait part à Albéroni. Celui-ci ne fut pas
longtemps à le savoir. Bien moins touché pour l'intérêt du roi d'Espagne
de cette sauvage proposition, que piqué de ce qu'Aubenton avait osé en
parler au roi d'Espagne à son insu, il fit donner au confesseur une
défense sévère et précise de se plus mêler d'aucune affaire de Rome, et
fit savoir à Rome, par le duc de Parme, que la reine avait été très
piquée de voir que le pape se rétractait sur plusieurs conditions
concertées à Madrid avec Aldovrandi, et que, si les différends ne
s'accommodaient promptement, le nonce ne serait point reçu à la cour
d'Espagne, laquelle n'enverrait au pape aucune sorte de secours contre
les Turcs. Aubenton, sentant à qui il avait affaire, enraya tout court.
Il manda même à Rome que sans Albéroni il ne pouvait rien, et que le
moyen sûr de le perdre, et en même temps les affaires, était d'en tenter
par lui sans le premier ministre. Aussi lui fut-ce une leçon, dont il
sut profiter, pour ne hasarder plus de parler au roi de quoi que ce fût
que de concert avec un premier ministre si jaloux et si maître. Tous
deux avaient intérêt de protéger Aldovrandi à Rome pour profiter de son
crédit. Ils le firent très fortement au nom du roi et de la reine par
Acquaviva. Le pape lui réitéra sa promesse pour dès qu'il pourrait
disposer de trois chapeaux.

Acquaviva savait que l'un des trois était destiné à l'archevêque de
Bourges, et que le pape l'en avait fait assurer, qui ne le fut pourtant
qu'en 1719, avec les couronnes, et un an après Albéroni. Avec ces bonnes
nouvelles, Acquaviva exhortait Albéroni à presser l'envoi du secours
promis pour avancer son chapeau sitôt que les trois vacances le
pourraient permettre. Ce ne fut pas l'avis d'Albéroni, piqué de la
remise de sa promotion à l'attente de la vacance de trois chapeaux.
L'escadre espagnole était à Messine, le pape demandait instamment
qu'elle hivernât dans quelque port de la côte de Gênes, pour l'avoir
plus tôt au printemps\,; tout à coup elle fit voile pour Cadix. En même
temps Albéroni accabla le pape de protestations de n'avoir jamais
d'autres volontés que les siennes, et d'assurances que les vaisseaux
pour hiverner à Cadix n'en seraient pas moins promptement au printemps
dans les mers d'Italie\,; en même temps il dépeignait la reine d'Espagne
comme n'étant pas si docile, avec toutes les couleurs les plus propres
pour faire tout espérer de son attachement naturel au saint-siège, de
son affection pour la personne du pape, de la bonté de son coeur très
reconnaissant, et tout craindre de son pouvoir absolu en Espagne, si
elle se voyait amusée et moquée, sur quoi il n'y avait point de retour à
espérer. Ce portrait était vif quoique long\,; il était fait pour être
vu du pape\,; et il n'y avait rien d'oublié sur l'entière possession où
Albéroni était de la confiance de la reine. Il obtint une lettre de sa
main au cardinal Acquaviva, par laquelle elle lui ordonnait de presser
le pape de sa part de le promouvoir incessamment. Cette lettre faisait
valoir ses mérites envers le saint-siège, et assurait que les
résolutions importantes qui restaient encore à prendre pour la
perfection de l'ouvrage commencé dépendaient de cette promotion. La
reine s'abaissait à dire qu'indépendamment de ce qu'elle était, et de
l'intérêt qu'avait le pape de lui accorder ce qu'elle lui demandait avec
tant d'instance, elle croyait pouvoir se flatter qu'en considération
d'une dame il sortirait des règles générales. Enfin elle promettait au
pape et à sa maison une reconnaissance éternelle, et que le roi
d'Espagne, content de la promotion d'Albéroni, garderait le silence sur
celle des couronnes. En envoyant cette lettre qui devait être montrée au
pape, le premier ministre, honteux de son impatience, faisait entendre
de grandes idées qu'il était chargé d'exécuter, dont la reine, prévoyant
les suites, ne voulait pas l'y exposer sans armes dans un pays où
l'agitation était grande\,; mais ces idées, il se gardait bien d'en
laisser rien entendre, sous prétexte que la matière était trop grave
pour le papier.

Tout était dans le dernier désordre en Espagne, tout le monde criait\,;
personne ne pouvait remédier à rien. Au fond tout tremblait devant un
homme dont on jugeait aisément que l'arrogance et la conduite feraient
enfin sa perte, mais qui en attendant était maître absolu des affaires,
des grâces, des châtiments, et de toute espèce, et qui n'épargnait qui
que ce fût. Toutes les avenues d'approcher du roi étaient absolument
fermées. Aubenton seul était excepté\,; mais il sentait si bien que sa
place était en la main d'Albéroni qu'il n'écoutait personne qui lui
voulût parler d'affaires, qu'il renvoyait tout à Albéroni\,; et comme il
était de leur intérêt que personne ne pût aborder le roi qu'avec leur
attache, le confesseur avait promis au premier ministre de l'avertir de
tout ce qu'il découvrirait. M. le duc d'Orléans, fort mécontent de la
manière dont Louville avait été chassé plutôt que renvoyé d'Espagne,
sans avoir pu obtenir audience, ni même attendre d'être rappelé, en
écrivit au roi d'Espagne\,; et comme il se plaignait d'Albéroni, il ne
voulut pas que sa lettre passât par lui, et la fit envoyer par le P. du
Trévoux au P. Daubenton pour la remettre immédiatement au roi d'Espagne.
Dès que le confesseur l'eut reçue il l'alla dire au premier ministre
pour en avertir la reine. On peut juger de l'effet.

Albéroni s'emporta jusqu'aux derniers excès. Il cria à l'ingratitude
parce qu'il avait fait rendre une barque française prise à Fontarabie,
et fait payer malgré le conseil de finance quelque partie des sommes
dues aux troupes françaises qui avaient servi l'Espagne en la dernière
guerre. Non content de ces clameurs, il écrivit une lettre à Monti
remplie de plaintes amères sur celles que M. le duc d'Orléans avait
portées au roi d'Espagne par la voie du confesseur, avec ordre de la
montrer à ce prince, et dans laquelle il eut l'audace de marquer que ce
jésuite aurait été perdu sans la sage conduite qu'il avait eue
d'informer la reine de ce dont il était chargé. Les protestations
d'attachement à Son Altesse Royale y étaient légères. Il le dépeignait
comme uniquement attentif aux événements qu'il envisageait, et ce
qu'Albéroni ne voulait pas dire comme de soi parce qu'il était trop
fort, il le prêtait vrai ou faux aux ministres d'Angleterre et de
Hollande qui étaient à Madrid, et qui disaient qu'en leur pays tout le
monde était persuadé que M. le duc d'Orléans ne songeait qu'à s'assurer
de la couronne, et que, lorsque toutes les mesures seraient bien prises,
la personne du roi ne l'embarrasserait pas. Avant d'aller plus loin dans
la lettre, qui n'admirera l'horreur de ce propos, et l'impudence sans
mesure de ne l'écrire que pour le faire voir à M. le duc d'Orléans\,?
Remettons-nous en cet endroit les énormes discours semés, et de temps en
temps renouvelés, avec tant d'art et de noirceur sur la mort de nos
princes, leur germe, leurs sources, leurs appuis, leurs usages, et
l'étonnante situation d'Effiat entre M. le duc d'Orléans et le duc du
Maine, et d'Effiat chargé par M. le duc d'Orléans d'entretenir, comme on
l'a vu, un commerce de lettres avec Albéroni qu'il connaissait fort du
temps qu'il était au duc de Vendôme, auquel Effiat était sourdement lié
par le duc du Maine. Ajoutons que ce n'est pas de ce canal naturel dont
Albéroni se sert pour faire montrer sa monstrueuse lettre à M. le duc
d'Orléans, mais d'un étranger isolé qui ne tenait à personne. Je m'en
tiens à ces courtes remarques, et je continue le récit de cette lettre.
Il la concluait par déplorer le malheur de M. le duc d'Orléans, et gémir
sur l'opinion qu'il prétendait que le public avait prise de lui. Que
dire d'une pareille insulte d'un abbé Albéroni au régent de France,
entée sur une autre, et du premier ordre, faite au roi de France et au
régent, l'une et l'autre uniquement produites par l'intérêt particulier
et la jalousie d'autorité du petit Albéroni\,?

Au milieu de cette incroyable audace, il se trouvait également
embarrassé des alliances que formait la France et des moyens de les
traverser. Tantôt il pensait que l'Espagne devait se contenter
d'observer ce qui se passerait, tantôt il blâmait cette tranquillité, et
voulait, disait-il, contre-miner les batteries du régent. Quelquefois il
le condamnait de faiblesse de mendier de nouveaux traités et de
nouvelles alliances avec les puissances étrangères\,; et dans ces
incertitudes il demandait conseil au prince de Cellamare, auquel il
promettait le plus profond secret, comme ne doutant pas qu'étant dès
avant la mort du feu roi ambassadeur d'Espagne à Paris, il ne fût bien
instruit des dispositions du royaume, sur lesquelles il fonderait ses
avis.

L'empereur était fort fâché de ces nouvelles liaisons que la France
était sur le point de former. Ses ministres dans les Pays-Bas ne le
dissimulaient point. Le même Prié, qu'on a vu en son lieu si audacieux à
Rome vis-à-vis du pape et du maréchal de Tessé, allait commander aux
Pays-Bas autrichiens, dont le prince Eugène avait le titre de gouverneur
général\,; passant à la Haye pour se rendre à Bruxelles, il fit tous ses
efforts pour empêcher la conclusion du traité. Les Hollandais en même
temps n'oubliaient rien pour flatter le roi d'Espagne par Riperda, leur
ambassadeur à Madrid, et par leurs protestations à Beretti, ambassadeur
d'Espagne arrivé à la Haye vers le milieu d'octobre, de ne conclure rien
au préjudice du roi son maître avec Prié, qui était alors à la Haye.
Beretti leur dit qu'il ne doutait pas que Prié ne leur proposât de
garantir à l'empereur non seulement les États dont il était en
possession, mais aussi ses prétentions sur ceux qu'il n'avait pas, et
leur représenta combien cette garantie offenserait le roi d'Espagne\,; à
quai ils répondirent que l'Angleterre, à qui ce prince accordait de si
grands avantages, était entrée en cet engagement sans que le roi
d'Espagne eût témoigné en être blessé, et qu'ils ne voyaient pas qu'il
eût plus de sujet de se plaindre d'eux s'ils suivaient l'exemple de
l'Angleterre. Beretti leur distingua la différence de position, en ce
que l'empereur ne pouvait, sans troupes et sans vaisseaux pour les
transporter, forcer l'Angleterre à lui tenir une garantie que
vraisemblablement elle ne promettait que pour l'honneur du traité\,; au
lieu que les Provinces-Unies, entourées de troupes impériales, seraient
bien forcées de recevoir la loi lorsqu'elles se trouveraient obligées
par leurs garanties à fournir leurs secours. Ce ministre ajouta que, si
la Hollande ne faisait que suivre l'exemple de l'Angleterre, l'Espagne
n'avait pas besoin de tenir un ambassadeur près d'eux, que celui qui
résidait à Londres devait suffire.

Beretti était homme d'esprit, mais grand parleur, plein de bonne opinion
de lui-même, attentif à se faire valoir des moindres choses, à faire
croire en Espagne que personne ne réussissait plus heureusement que lui
en affaires, qu'on traitait plus volontiers avec lui qu'avec nul autre
par la réputation de sa probité, surtout d'en persuader Albéroni, auquel
il mandait que le Pensionnaire n'avait ni estime ni confiance pour
Riperda, ce qui était vrai, mais dans la crainte que le premier ministre
ne voulût traiter avec cet ambassadeur à Madrid, et par conséquent lui
enlever la négociation. Il mandait que Cadogan, ministre d'Angleterre à
la Haye, blâmait les desseins chimériques de l'empereur, les tenait
contraires aux intérêts de cette couronne, dont les conseils, s'ils
étaient écoutés à Vienne, y porteraient à faire une prompte paix avec le
roi d'Espagne. Le mécontentement et l'agitation de l'Angleterre
persuadait à Cadogan qu'on y manquait moins de volonté que de chef et de
moyens pour faire une révolution\,; que la paix assurée avec la France
éteignait toutes ces espérances et tout péril de rébellion, ce qui
pouvait changer par les démarches que l'empereur, une fois délivré de la
guerre du Turc, pourrait faire à l'égard de ses prétentions et porter de
nouveau la guerre dans les États du roi d'Espagne. Il paraissait aussi
que, à mesure que le traité avançait avec la France, le ministère
anglais changeait de sentiments et de maximes sur les affaires générales
de l'Europe.

Cellamare remit dans ce temps-ci au régent la réponse du roi d'Espagne à
sa lettre, qu'il avait voulu faire passer par Aubenton, dont on vient de
parler il n'y a pas longtemps. Albéroni, qui l'avait dictée, faisait
dire au roi d'Espagne que tout ce qui avait été exécuté à l'égard de
Louville s'était fait par ses ordres\,; et que, pour ce qui était
d'entretenir un commerce secret de lettres avec lui par la voie de son
confesseur, il désirait que les lettres qu'il vaudrait désormais lui
écrire fussent remises à son ambassadeur à Paris. Cette réponse fut un
nouveau triomphe pour Albéroni. Il avait de plus profité de la lettre de
M. le duc d'Orléans pour vanter sa probité incorruptible que la France
n'avait pu corrompre\,; qu'elle lui avait fait proposer de demander le
payement de la pension de six mille livres, que le feu roi lui avait
autrefois donnée\,; c'est-à-dire que M. de Vendôme lui avait obtenue,
dont on murmura bien alors, et les arrérages qui en étaient dus,
payement qu'il était bien sûr d'obtenir\,; que n'ayant pas voulu y
entendre, on lui avait vilainement jeté l'un et l'autre à la tête\,;
qu'après cette tentative on avait envoyé Louville à Madrid, avec ordre
exprès (quel hardi mensonge\,!) de ne rien faire que par sa direction,
et avec une lettre du régent pour lui\,; que sous ces fleurs était caché
le dessein de remettre auprès du roi d'Espagne un homme insolent,
capable de reprendre l'ancien ascendant qu'il avait eu sur l'esprit du
roi d'Espagne, et de le tenir en tutelle, après avoir détruit celui, qui
était lui-même, que la cour de France regardait comme le plus grand
correctif des cabales. Il se plaignait après de M. le duc d'Orléans, et
plus encore du duc de Noailles à qui il attribuait tout ce projet, et
qu'il disait avoir suffisamment connu dans des conjonctures critiques\,;
mais ce ne pouvait être que du temps qu'il était bas valet de M. de
Vendôme. Enfin, il prétendait que les Français étaient au désespoir de
voir que le roi d'Espagne voulait être le maître de sa maison, c'était à
dire franchement Albéroni.

La licence avec laquelle les Anglais et les Hollandais coupaient des
bois de Campêche dans les forêts du roi d'Espagne aux Indes, et
rapportaient en Europe, lui donna des sujets d'en faire des plaintes, et
fit découvrir beaucoup de grandes malversations des Espagnols mêmes, qui
donnèrent lieu au premier ministre d'ouvrir toutes les lettres du
nouveau monde pour en être mieux instruit. Il prétendit qu'il y en avait
quantité qui touchaient à la religion. Il ne manqua pas d'en faire sa
cour au pape, et de se parer de son zèle à y remédier. En même temps il
fit agir ses agents ordinaires près de lui, Aubenton par écrit,
Acquaviva et Aldovrandi de vive voix, avec le même manège de promesses
et de menaces qui ont déjà été vues, et alors d'autant plus de saison
que le pape était averti que les Turcs, quoique maltraités en Hongrie,
travaillaient puissamment un grand armement pour les mers d'Italie, dont
il avait conçu une grande frayeur de laquelle Albéroni espérait tout
pour avancer sa promotion. Le premier ministre se servit aussi du
témoignage d'Aubenton pour assurer le pape de l'attachement du roi
d'Espagne à la saine doctrine, et de sa soumission parfaite à son
autorité. Ce mérite retombait en plein sur Albéroni, et faisait d'autant
plus d'impression qu'il ajoutait foi entière à ce jésuite, surtout
encore sur cette matière, et qu'il croyait, à cette occasion, avoir
besoin d'appui contre la France. Tout cela fit que le pape ne voulut
écouter rien contre Albéroni, ni contre Aubenton, même éloigna les
accusations qui lui venaient en foule contre eux, persuadé qu'il ne
fallait pas mécontenter des gens dont il avait besoin dans la
conjoncture où il se trouvait alors.

Acquaviva en profitait pour presser le pape, tant sur la promotion
d'Albéroni, que pour accorder au roi d'Espagne les moyens de hâter le
secours qu'il lui destinait. Le pape se rendit plus traitable sur ce
dernier article. Il résolut d'accorder un million d'écus sur le clergé
des Indes, pour tenir lieu de l'imposition appelée \emph{sussidio y
escusado}, dont le roi d'Espagne voulait le rétablissement à perpétuité,
et ce million n'était payable qu'une fois\,; ainsi l'offre ne répondant
pas à la demande, Acquaviva ne voulut pas s'en contenter, et le pape y
ajouta un million cinq cent mille livres à lever sur le clergé
d'Espagne. Il restait une troisième affaire bien plus importante à
régler\,: l'abus des franchises du clergé est porté en Espagne, et dans
les pays subjugués par la tyrannie romaine et l'aveuglement grossier,
{[}à un tel point{]} que tout ecclésiastique est exempt\,; jusque dans
son patrimoine, de quelque sorte d'imposition que ce puisse être. Mais
ce n'est pas tout, c'est qu'à un abus si énorme se joignait, comme de
droit, la plus parfaite friponnerie et le mensonge le plus avéré\,; tout
le bien d'une famille se mettait sur la tête d'un ecclésiastique qui lui
donnait sous main de bonnes sûretés\,; à ce moyen elle jouissait de son
bien à l'ombre ecclésiastique, et n'en payait pas un sou d'aucune
imposition. Ajoutez cela à la nécessité de recourir au pape pour obtenir
des secours d'un clergé qui regorge des biens du siècle, et au pouvoir
du tribunal de l'inquisition et de celui de la nonciature, qui anéantit
totalement les évêques, et on verra, et encore en petit, jusqu'où va la
domination romaine, quand on a la faiblesse et l'aveuglement de s'en
laisser dompter.

On espérait donc voir bientôt une fin à ce différend, mais on craignait
fort les traverses des Espagnols, surtout de l'arrivée à Rome du
cardinal del Giudice, et ce Diaz, agent d'Espagne à Rome, qui criait de
toute sa force contre la promotion d'Albéroni. Les Espagnols ne
pouvaient supporter de voir toutes les affaires de la monarchie entre
les mains des Italiens, soit dans son centre, soit à Rome et ailleurs\,;
et leurs cris, fondés sur l'indignité du personnage, l'honneur de la
pourpre, le respect de l'Église, la réputation du pape, portés jusqu'à
lui par les ennemis d'Acquaviva et d'Aldovrandi, ne laissaient pas de
l'ébranler beaucoup. Mais bientôt après les lettres d'Aubenton
réparaient tout. Le pape si défiant ne se pouvait défier de l'ambition
ni de l'esclavage de ce jésuite, dans la pleine conviction où il lui
avait plu de s'établir du dévouement sans réserve d'Aubenton à sa
personne et à son autorité, dont aucun autre attachement, ni sa place
même, ne pouvait affaiblir la plénitude, et c'était de ces témoignages
dont Aldovrandi faisait boucher pour raffermir le pape sur cette
promotion, et sur l'accommodement des différends avec l'Espagne. Ce
prélat craignit de la part des neveux de Giudice qui étaient à Rome, et
voulut agir auprès d'eux, mais il n'y trouva nul obstacle à vaincre.
Cellamare leur aîné, sage et habite, mais bas courtisan, craignant pour
sa fortune, leur avait écrit de façon qu'il n'y eût rien à appréhender
de leur part. Aldovrandi était en peine aussi que la France ne mît des
obstacles, mais il fut rassuré par le cardinal de La Trémoille, qui lui
promit de contribuer plutôt que de traverser parce que le pape ne
pouvait refuser de donner un chapeau à la France, lorsqu'il en
accorderait un au premier ministre d'Espagne, ce que l'événement ne
vérifia pas. Ainsi, tout s'aplanissant devant lui, le pape dans le
besoin qu'il croyait en avoir, lui faisait faire souvent des compliments
et des assurances d'une estime et d'une confiance qu'il n'avait pas, et
d'une reconnaissance de son zèle et de ses services aussi fictive.
Aldovrandi demanda une nouvelle lettre de la main de la reine pour
presser de nouveau cette promotion, et voulut qu'elle contînt des
menaces contre quiconque la voudrait traverser. Albéroni soutenait ces
menées par ses promesses en maître absolu qu'il était, et par ses
préparatifs. Il disposait de l'argent venu par les galions, il
abandonnait le projet des travaux des ports de Cadix et du Ferrol, et il
assurait qu'il paraîtrait une flotte au mois de mars dans les mers
d'Italie, telle qu'il ne s'en était point vu depuis Philippe II, si le
pape prenait le parti d'exécuter de sa part ce qu'il fallait pour cela,
c'est-à-dire de lui envoyer la barrette.

Il ne s'expliquait point sur la ligue qui se négociait entre la France,
l'Angleterre et la Hollande\,; il ne jugeait pas que le roi d'Espagne
fût encore en état de prendre aucun parti, et qu'il ne fallait laisser
pénétrer rien de ce qu'il pouvait penser. Il se contentait de raisonner
sur tant ce qui se passait pour arriver à cette triple alliance, de
conclure que l'Europe ne pouvait subsister dans l'état où elle était, et
de vouloir persuader que la situation du roi d'Espagne était meilleure
que celle de toutes les autres puissances. Néanmoins il consulta
Cellamare sur la conduite qu'il estimait que le roi d'Espagne dût tenir
dans la situation présente. Cet ambassadeur lui répondit que son
sentiment était que le roi d'Espagne devait vendre cher ce qu'il ne
voudrait pas garder, supposé qu'il prît la résolution de l'abandonner
(c'est-à-dire ses droits sur la couronne de France), ou de surmonter à
quelque prix que ce fût les difficultés capables d'éloigner
l'acquisition d'un bien qu'il désirait. Il ajoutait que, suivant le
cours ordinaire du monde, beaucoup de gens désapprouvaient la ligue avec
l'Angleterre dans le pays où il était, pendant que d'autres
l'approuvaient. Le roi d'Angleterre eut beau assurer l'empereur qu'il
n'y avait aucun article dans ce traité qui fût préjudiciable aux
intérêts de la maison d'Autriche, il ne put calmer ses soupçons. Ses
ministres redoublèrent d'activité pour le traverser à mesure qu'ils
croyaient s'avancer, et le suspendirent quelque temps par les
difficultés qu'ils eurent le crédit de faire former par quelques villes
de Hollande, que les ambassadeurs de France, sincèrement secondés par
celui d'Angleterre, eurent beaucoup de peine à surmonter.

La vivacité des Anglais en cette occasion déplut fort aux Impériaux. Ils
étaient irrités contre les Hollandais par les différends sur le traité
de la Barrière, où il survenait toujours quelque nouvelle difficulté.
Entre autres l'empereur se prétendait dégagé du payement de un million
cinq cent mille livres pour l'entretien des garnisons hollandaises dans
les places des Pays-Bas, parce qu'il disait que cette condition n'était
établie que sur la supposition que le revenu de ces provinces était de
deux millions d'écus, et qu'il n'allait pas à huit cent mille par an.
Ces altercations ne nuisirent point au traité, non plus que les manèges
et les instances de Prié qui, partant de la Haye pour Bruxelles fort
mécontent de son peu de succès, laissa échapper quelques menaces qui
firent sentir aux Hollandais le besoin qu'ils avaient de se faire des
amis et des protecteurs contre les entreprises et les chicanes de
l'empereur, maître de les inquiéter par ces mêmes états qu'ils avaient
eu tant de soin de lui procurer à la paix d'Utrecht.

Beretti mandait en Espagne que la crainte de l'empereur, dont les
Hollandais s'étaient environnés, les rendait Français. Il citait le
comte de Welderen et d'autres principaux des États généraux pour avoir
dit qu'ils avaient été les dupes de l'empereur et des Anglais qui
avaient augmenté\,: l'un ses États, les autres leur commerce, aux dépens
de leur république. Il louait le régent d'avoir si bien pris son temps
pour le traité qu'il croyait, avec bien d'autres, avoir coûté un million
à la France\,; et qui dans la vérité n'avait pas coûté un écu. Il
maintenait que ce traité n'empêcherait pas la Hollande d'en faire un
plus particulier avec l'Espagne parce que cela convenait à leur
intérêt\,; qu'ainsi le traité ne coûterait rien au roi d'Espagne parce
qu'il était recherché des Hollandais, qui pour rien ne lui voulaient
déplaire, au lieu qu'ils étaient recherchés par les Français. Quoique
trompé sur l'argent du traité, et sur ce que les Hollandais ne le
concluraient point s'ils remarquaient que cette alliance fût trop
suspecte à l'Espagne, il était dans le vrai sur l'apposition constante
que la Hollande apportait à l'union des deux monarchies sur la même
tête, et il était persuadé que c'est ce qui l'avait déterminée à traiter
avec le régent. Il était peiné de n'être pas assez instruit des
intentions de l'Espagne. Il craignait que les ambassadeurs de France ne
le fissent tomber dans quelque piège\,; et il croyait remarquer que leur
conduite avec lui était tendue à le tromper, du moins à l'empêcher de
jeter quelque obstacle à la négociation qu'ils désiraient ardemment de
conclure. Il les examinait de près, et il remarqua qu'ils n'avaient
point de portrait du roi chez eux, et qu'ils ne nommaient jamais son
nom. Il se trouva bientôt fort loin de ses espérances et de celles qu'il
avait si positivement données.

Albéroni lui ordonna de déclarer au Pensionnaire que le roi d'Espagne
était prêt à traiter avec la république, et de demander que les pouvoirs
en fussent envoyés à Riperda, parce que c'était à Madrid que le roi
d'Espagne voulait traiter. Beretti se voyant enlever la négociation vit
les personnages principaux de la république et leurs intentions avec
d'autres yeux. Heinsius lui répandit, avec une froide joie des bonnes
intentions du roi d'Espagne, que ses maîtres étant actuellement occupés
à traiter avec la France, il fallait achever cet ouvrage, et laisser au
temps à mûrir les affaires pour mettre plus sûrement la main à l'oeuvre
suivant que les conjonctures y seraient propres. Beretti lui voulut
faire craindre les desseins de l'empereur. Le Pensionnaire ne disconvint
pas que la conduite de Prié à la Haye n'eût ouvert les yeux, et changé
dans plusieurs l'inclination autrichienne, mais il évita toujours
d'approfondir la matière, d'où Beretti conclut qu'Heinsius voulait faire
le traité avec l'Espagne, non à Madrid, mais à la Haye.

\hypertarget{chapitre-vii.}{%
\chapter{CHAPITRE VII.}\label{chapitre-vii.}}

1716

~

{\textsc{Le traité entre la France et l'Angleterre signé à la Haye, qui
effarouche les ministres de la Suède.}} {\textsc{- Intrigue des
ambassadeurs de Suède en Angleterre, en France et à la Haye, entre eux,
pour une révolution en Angleterre en faveur du Prétendant.}} {\textsc{-
Lettre importante d'Erskin au duc de Marr sur le projet inconnu du czar,
mais par lui conçu.}} {\textsc{- Médecins britanniques souvent cadets
des premières maisons.}} {\textsc{- Adresse de Spaar à pomper Canillac
et à en profiter.}} {\textsc{- Goertz seul se refroidit.}} {\textsc{-
Précaution du roi d'Angleterre peu instruit.}} {\textsc{- Il fait
travailler à la réforme de ses troupes, et diffère de toucher aux
intérêts des fonds publics.}} {\textsc{- Artifices du ministère
d'Angleterre secondés par ceux de Stairs.}} {\textsc{- Fidélité de
Goertz fort suspecte.}} {\textsc{- Le roi d'Angleterre refuse sa fille
au prince de Piémont par ménagement pour l'empereur.}} {\textsc{-
Scélératesse de Bentivoglio contre la France.}} {\textsc{- Nouveaux
artifices pour presser la promotion d'Albéroni.}} {\textsc{- Acquaviva
fait suspendre la promotion de Borromée au moment qu'elle s'allait
faire, et tire une nouvelle promesse pour Albéroni dès qu'il y aurait
trois chapeaux vacants.}} {\textsc{- Défiances réciproques du pape et
d'Albéroni, qui arrêtent tout pour quelque temps.}} {\textsc{- Le duc de
Parme élude de faire passer à la reine d'Espagne les plaintes du régent
sur Albéroni\,; consulte ce dernier sur ce qu'il pense du régent.}}
{\textsc{- Sentiment du duc de Parme sur le choix à faire par le roi
d'Espagne, en cas de malheur en France.}} {\textsc{- Insolentes
récriminations d'Albéroni, qui est abhorré en Espagne, qui veut se
fortifier par des troupes étrangères.}} {\textsc{- Crainte et nouvel
éclat d'Albéroni contre Giudice.}} {\textsc{- Imprudence de ce
cardinal.}} {\textsc{- Avidité du pape.}} {\textsc{- Impudence et
hypocrites artifices d'Albéroni et ses menaces.}} {\textsc{- Réflexion
sur le cardinalat.}} {\textsc{- Albéroni veut sacrifier Monteléon à
Stanhope, et laisser Beretti dans les ténèbres et l'embarras\,; veut
traiter avec la Hollande à Madrid\,; fait divers projets sur le commerce
et sur les Indes\,; se met à travailler à la marine et aux ports de
Cadix et du Ferrol.}} {\textsc{- Abus réformés dans les finances, dont
Albéroni tire avantage pour hâter sa promotion, et redouble de manèges,
de promesses, de menaces, d'impostures et de toutes sortes d'artifices
pour y forcer le pape\,; {[}il est{]} bien secondé par Aubenton.}}
{\textsc{- Son adresse.}} {\textsc{- La reine d'Espagne altière, et le
fait sentir au duc et à la duchesse de Parme.}} {\textsc{- Peines de
Beretti.}} {\textsc{- Heinsius veut traiter avec l'empereur avant de
traiter avec l'Espagne.}} {\textsc{- Conditions proposées par la
Hollande à l'empereur, qui s'opiniâtre au silence.}} {\textsc{- Manèges
des Impériaux et de Bentivoglio pour empêcher le traité entre la France,
l'Angleterre et la Hollande.}}

~

Cependant\footnote{Voyez la note II en fin de volume.} le traité entre
la France et l'Angleterre fut signé à la Haye à la fin de novembre, mais
secrètement, à condition qu'il n'en serait rien dit de part ni d'autre
pendant un mois, terme jugé suffisant pour laisser le temps aux
Hollandais de prendre une dernière résolution sur la conclusion de cette
alliance. Elle déplut particulièrement aux Suédois, qui par là se
crurent abandonnés de la France. Le comte de Gyllembourg était
ambassadeur de cette couronne en Angleterre. Le baron de Spaar avait le
même caractère en France\,; et le baron de Goertz, ministre d'État et
chef des finances de Suède, était de sa part à la Haye. Dès qu'ils
virent avancer le traité entre la France et l'Angleterre, ils crurent
que la principale ressource du roi de Suède était d'exciter des troubles
en Angleterre. Il y avait longtemps que Gyllembourg le proposait, et
qu'il assurait que les difficultés n'en étaient pas si grandes qu'on se
le figurait.

Spaar et Goertz se virent sur la frontière\,; le dernier vint faire un
tour à Paris. Ils convinrent tous deux qu'il fallait profiter de la
disposition générale de l'Écosse en faveur du Prétendant, et d'une
grande partie de celles de l'Angleterre. Goertz retourné à la Haye fut
de nouveau pressé par Gyllembourg, qui lui manda que les jacobites
demandaient dix mille hommes, et qu'il croyait que l'argent ne
manquerait pas. Goertz ignorait les intérêts du roi de Suède là-dessus.
On prétend que Spaar et lui étaient convenus de différer à lui rendre
compte de ce projet jusqu'à ce qu'eux-mêmes y aperçussent plus de
solidité. Ils ne pouvaient hasarder de l'en instruire par lettres, qui
n'arrivaient jusqu'au roi de Suède qu'avec beaucoup de difficulté et de
danger d'être interceptées. Il fallait donc trouver un homme sûr et
capable de l'informer de tout le détail du projet pour en rapporter ses
ordres. Spaar jeta les yeux sur Lenck à qui, de préférence à son propre
neveu, il avait fait donner le régiment d'infanterie qu'il avait au
service de France, quand il y fut fait officier général. Il fallait un
prétexte pour ce voyage. Le régent était en peine de savoir les
intentions du roi de Suède sur la paix du Nord. Spaar lui proposa
d'envoyer Lenck en Suède homme sûr et fidèle, et très capable d'obliger
le roi de Suède à répondre précisément sur les points dont le régent
voulait être éclairci. La conjoncture pressait son départ. Les offres
d'argent étaient considérables. Spaar apprit d'un des principaux
jacobites qu'ils avaient fait passer trente mille pièces de huit en
Hollande, c'était à la mi-octobre, et qu'il y en arriverait autant
incessamment\,; qu'ils offraient ces sommes au roi de Suède en attendant
mieux, en peine seulement sur la manière de les lui faire accepter, et
des moyens ensuite de {[}les{]} faire passer entre ses mains. Spaar leva
ces difficultés, déjà prévues entre lui et Goertz, et proposa, comme ils
en étaient convenus, de faire écrire une lettre à Goertz par le duc
d'Ormond au par le comte de Marr, contenant cette offre, et faire en
même temps passer en Hollande les autres trente mille pièces de huit
qu'ils disaient être prêtes. Le dessein des deux ministres de Suède
était d'en acheter quelques vaisseaux en France, et de lever quelques
matelots pour les équiper. Le roi de Suède leur en avait demandé mille
ou quinze cents, mais sans songer à l'entreprise d'Angleterre, dont il
n'était pas informé. Ses ministres, persuadés de l'importance de
l'expédition, y employèrent le banquier Hoggers, dont ils connaissaient
la vivacité. Il s'était fait un prétexte d'armer quelques vaisseaux, par
un traité avec le conseil de marine, pour apporter des mâts de Norvège
dans les magasins du roi. Il avait donc à Brest trois navires du roi
qu'il prétendait armer en guerre, et un quatrième de cinquante-huit
pièces de canon qu'il avait fait passer au Havre, où apparemment les
trois autres le devaient aller joindre\,; et ces quatre vaisseaux
devaient être commandés par un officier du roi de Suède que Goertz
devait envoyer à Paris. La lettre du duc d'Ormond vint à Spaar pour
Goertz, dont le premier crut que l'autre se contenterait, quoique les
termes ne fussent si fort les mêmes que ceux qui avaient été demandés\,;
et en même temps les assurances que les soixante mille pièces de huit
seraient dans la fin de décembre remises à Paris, à la Haye ou à
Amsterdam.

Le mécontentement conçu par le czar de ses alliés, et l'abandon en
conséquence de la descente au pays de Schonen, fut un autre fondement
d'espérance pour Spaar. Le czar avait auprès de lui un médecin écossais
qui était en même temps son confident et son ministre. Il faut savoir
que dans toute la Grande-Bretagne la profession de médecin n'est
au-dessous de personne, et qu'elle est souvent exercée par des cadets
des premières maisons. Celui-ci était cousin germain du comte de Marr,
et comme lui portait le nom d'Erskin. Il écrivit à son cousin, que le
roi Jacques III venait de faire duc, que le projet de Schonen échoué, et
le czar, brouillé avec ses alliés, ne voulait plus rien entreprendre
contre le roi de Suède\,; qu'il désirait sincèrement faire la paix avec
lui\,; qu'il haïssait mortellement le roi Georges, avec qui il n'aurait
jamais de liaison\,; qu'il connaissait la justice de la cause du roi
Jacques\,; qu'il s'estimerait glorieux, après la paix faite avec le roi
de Suède, de s'unir avec lui pour tirer de l'oppression et rétablir sur
le trône de ses pères le légitime roi de la Grande-Bretagne\,; qu'il
était donc entièrement disposé à finir la guerre, et à prendre des
mesures convenables à ses intérêts et à ceux de la Suède\,; qu'il n'en
devait pas faire les premiers pas, puisqu'il avait l'avantage de son
côté, mais qu'il était facile de terminer cet accommodement par un ami
commun et sincère, avant même que qui que ce soit eût loisir de le
soupçonner\,; qu'il n'y avait point de temps à perdre, ni laisser aux
alliés du Nord le loisir de se raccommoder\,; qu'ayant un grand nombre
de troupes, il était obligé de prendre incessamment un parti, mais aussi
que cette circonstance rendait la paix plus avantageuse au roi de Suède.
Spaar fut informé de ces particularités par le duc de Marr, qui lui
proposa en même temps d'envoyer à Erskin un homme affidé pour ménager
l'accommodement. Spaar répondit qu'il confierait seulement l'un et
l'autre à Goertz, pour avoir son sentiment sur l'usage qu'on pouvait
faire des dispositions du czar et sur l'envol proposé.

Cet ambassadeur voulut s'éclaircir des véritables sentiments de la
France à l'égard de la Suède, et pour tâcher de les pénétrer alla voir
Canillac. Il commença par le désabuser du bruit qui avait couru que la
Suède eût accepté la médiation de l'empereur à l'exclusion de celle de
la France, puis tomba sur la pressante nécessité dont il était d'envoyer
promptement un homme de confiance au roi de Suède, avec de l'argent et
des offres de service. Canillac en convint, conseilla à Spaar d'en
parler au régent, promit de l'appuyer. Spaar, encouragé par ce début,
dit qu'il lui revenait de toutes parts que le czar désirait de faire la
paix avec la Suède\,; que rien n'était plus important que de profiter de
la dissension des alliés du Nord, et que de prévenir la réunion que
d'autres pourraient procurer entre eux\,; qu'il croyait donc qu'il
serait à propos que le régent fît passer sans délai un homme de
confiance auprès du czar, pour lui offrir ses offices et sa médiation.
Canillac convint encore de l'importance de la chose, mais ajouta qu'il
ne savait comment M. le duc d'Orléans pourrait, sans se commettre,
envoyer ainsi vers un prince avec qui la France n'avait jamais eu aucun
commerce. L'ambassadeur répliqua que la liaison qui était entre la
France et la Suède autorisait et rendait même très naturelles toutes les
démarches que le régent ferait. Il ajouta diverses représentations qui
ne persuadèrent pas. Canillac demeura dans son sentiment qu'il était
indispensable d'envoyer incessamment quelqu'un au roi de Suède, et qu'il
ne voyait pas comment le régent pouvait envoyer vers le czar. Spaar,
jugeant par là du peu d'empressement d'agir auprès du czar en faveur du
roi de Suède, conclut à redoubler de soins pour profiter de la discorde
de la ligue du Nord\,; qu'il était inutile de rien attendre de la
France, mais qu'il fallait conserver les dehors avec elle, comme le roi
de Suède le lui ordonnait. Il espéra même que le régent, dépêchant Lenck
au roi de Suède, lui donnerait une lettre de créance pour ce prince,
lequel par ce moyen pourrait faire des affres au czar, comme proposées
par la médiation et de la part de la France\,; que si elles étaient
agréées l'utilité en serait pour la Suède\,; si refusées, le désagrément
serait pour la France. Spaar était persuadé que nul sacrifice ne devait
coûter pour obtenir la paix avec le czar, dont un des principaux
avantages serait l'expédition d'Angleterre\,; que cette paix devait la
précéder, et de laquelle le succès serait assuré s'il devenait passible
d'engager le czar à fournir la moitié des vaisseaux et des troupes.
Cette espérance le refroidit sur l'armement d'Hoggers. Il faisait
réflexion que, si jamais le régent découvrait que les vaisseaux vendus
par le conseil de marine dussent servir à une pareille expédition, il
les ferait arrêter immédiatement après que l'armement serait achevé\,;
et qu'en ce cas, outre le malheur d'être découverts, il en coûterait
encore au roi de Suède cinq cent mille livres en faux frais. Il ne
voyait pas le même inconvénient à faire partir les matelots que le roi
son maître demandait, et il se proposait de les envoyer en Suède dès
qu'il aurait touché le premier argent des sommes promises.

Le zèle des ministres de Suède pour le Prétendant n'avait d'objet que
l'intérêt du roi leur maître, par l'utilité qu'il pourrait retirer des
mouvements de la Grande-Bretagne. Il fut donc embarrassé de la question,
que lui fit faire le Prétendant, s'il lui serait permis de passer et de
séjourner aux Deux-Ponts. Spaar considéra cette permission comme une
déclaration inutile, et de plus très nuisible aux intérêts de celui qui
la demandait. Il prévoyait que le roi de Suède n'y consentirait jamais.
Il le représenta en vain à celui qui lui parlait\,; et sur ses instances
réitérées, il promit d'en écrire à Goertz. Tous deux étaient pressés par
Gyllembourg de déterminer le roi de Suède à l'entreprise. Il leur
représentait que les choses étaient parvenues au point qu'il fallait
renoncer à Brème ou aux Hanovriens\,; que le succès en Écosse n'était
pas difficile\,; que dix mille hommes suffiraient tant le mécontentement
était général\,; qu'on ne demandait qu'un corps de troupes réglées,
auquel les gens du pays se joindraient\,; que s'il était transporté en
mars dans la saison des vents d'ouest, et dans le temps qu'on y
songerait le moins, la révolte serait générale\,; qu'il faudrait encore
porter des armes pour quinze au vingt mille hommes, ne pas s'embarrasser
de chevaux, dont an trouverait suffisamment dans le pays, surtout mettre
peu d'Anglais dans la confidence. Avec ces précautions Gyllembourg
prétendait qu'on pouvait s'assurer du succès dans un pays abondant, si
disposé à la révolution que de dix personnes on pouvait sûrement en
compter neuf de rebelles. On promettait de lui faire toucher soixante
mille livres sterling quand il ferait voir un pouvoir du roi de Suède,
et que ce prince assurerait les bien intentionnés de les assister. Ils
avaient cependant peine à lui remettre un plan de leur entreprise. Ils
craignaient d'en écrire le détail, de multiplier le secret, et de
s'exposer s'il était découvert aux mêmes peines que tant d'autres
avaient subies depuis un an. Néanmoins ils lui promirent de lui confier
ce plan avant peu de jours, et l'un de ceux qui traitaient avec lui
l'assura qu'ils n'avaient rien à craindre de la part du régent.

Malgré ces dispositions Goertz hésitait de s'embarquer avec les
jacobites, et quoiqu'il eût témoigné d'abord de l'empressement pour le
projet comme le seul moyen de délivrer le roi de Suède de l'embarras de
la ligue de ses ennemis, il avait apparemment changé de vues. Il ne
répondit pas seulement à la proposition qui lui avait été faite d'agir
par la voie d'Erskin\,; il prétendit avoir assez d'autres canaux dont il
se pourrait servir utilement. Il promit cependant à Spaar de lui envoyer
par Hoggers pour cent mille écus de lettres de change, immédiatement
après qu'il aurait reçu les éclaircissements qu'il avait demandés. Sa
froideur ne ralentit point les jacobites. Ils firent assurer Spaar
qu'ils avaient déjà remis des sommes assez considérables à Paris, qu'ils
en remettraient encore de plus fortes, et ils n'oublièrent rien pour se
bien assurer la Suède.

Le roi Georges et les siens, instruits en général des espérances que les
jacobites fondaient sur les secours de la Suède, n'en étaient guère en
peine. Néanmoins, au hasard de choquer les Anglais en allant contre
leurs formes, le roi Georges expédia de Hanovre un ordre à Norris,
amiral de l'escadre anglaise dans la mer Baltique, de laisser à
Copenhague six vaisseaux de guerre, sous prétexte d'assurer le commerce
des Anglais contre les insultes des Suédois dans le nord. L'alliance
entre la France et l'Angleterre était encore secrète, mais personne n'en
doutait. Le ministère anglais, quoique à regret, ne voulut pas attendre
d'avoir la main forcée sur la réforme des troupes par le parlement,
lorsqu'il apprendrait la signature du traité, et ils commencèrent à y
travailler. Par la même raison ils voulaient réduire à cinq pour cent
les intérêts qui se payaient sur les fonds publics, dont les fonds
excédaient quarante millions sterling. Néanmoins ils eurent peine à se
déterminer sur un point si capital, et malgré la certitude du traité
fait avec la France, ils affectèrent de craindre le Prétendant.

Le roi de Suède était le seul dont ils pouvaient faire envisager les
desseins\,; et Stairs, toujours à leur main pour le trouble, leur avait
mandé que ce prince s'était engagé par un traité à secourir le
Prétendant. Mais les affaires de la Suède n'étaient pas en état
d'effrayer les Anglais. Il fallait leur montrer quelque autre puissance.
Ainsi Stairs, à qui ces nouvelles ne coûtaient rien à inventer, répondit
que l'empereur, très irrité du traité, écouterait les propositions du
Prétendant pour se venger du roi d'Angleterre. Le roi de Prusse se
plaignait du roi Georges son beau-père, qui méprisait sa légèreté.
Gyllembourg pressait toujours Spaar et Goertz d'informer de leurs
résolutions le roi leur maître. Mais Goertz le secondait mal. Sa
fidélité était suspecte, et la manière dont il avait déjà servi d'autres
puissances favorisait les soupçons. L'Angleterre, malgré ses agitations
domestiques, était considérée comme ayant beaucoup de part aux affaires
générales de l'Europe. Le roi de Sicile si attentif à ses intérêts
recherchait son amitié et son alliance. Il envoya le baron de
Schulembourg qui servait dans ses troupes, et neveu de celui qui venait
de défendre Corfou dont les Turcs avaient {[}levé{]} le siège, trouver
le roi d'Angleterre à Hanovre sitôt qu'il y fut arrivé. On sut, après
quelque temps de secret, que c'était pour traiter le mariage d'une fille
de ce prince avec le prince de Piémont, mais que le roi d'Angleterre,
qui ménageait infiniment l'empereur, n'avait pas voulu écouter une
proposition qu'il savait lui devoir être fort désagréable. Le roi de
Sicile vivait dans une grande inquiétude des dispositions de l'empereur
à son égard. L'Italie était remplie d'Allemands qui pouvaient l'attaquer
à tous moments. La paix de Hongrie pouvait changer la face des affaires,
il se trouvait sans alliés, et quoique la France fût garante de la paix
d'Utrecht, il n'en espérait point de secours, parce qu'il croyait le
régent, son beau-frère, trop sage pour faire la guerre uniquement pour
autrui.

Bentivoglio qui, pour avancer sa promotion et l'autorité romaine, ne
cessait d'exciter Rome aux plus violents partis, et de tâcher lui-même à
mettre la France en feu par ses intrigues continuelles, chercha
d'ailleurs à lui susciter des ennemis. Il vit chez lui Hohendorff. Ils
s'expliquèrent confidemment sur le traité de {[}la{]} France avec
l'Angleterre, qui était lors sur le point d'être signé. Hohendorff
voulut douter que le pape consentît à la retraite du Prétendant
d'Avignon, qui par sa demeure en cette ville romprait le traité, dont ce
malheureux prince serait mal conseillé de faciliter la conclusion. Il
ajouta qu'il ne pouvait croire que la France, pour l'en faire sortir,
usât de violence contre le pape. Le nonce répandit, à ce qu'on prétend,
qu'il était facile à la France de faire partir le Prétendant sans user
de violence, en le menaçant de ne lui plus payer de pensions. Hohendorff
aurait dû alors offrir que l'empereur y suppléât\,; mais il se contenta
de conclure que ce prince était perdu s'il passait en Italie. Le nonce
en demeura persuadé. Il écrivit au pape que l'Église était intéressée à
rompre une ligue que les ennemis du saint-siège et de la religion
regardaient comme le plus solide fondement de leurs espérances. Ce
n'était pas la première fois qu'il avait prêté auprès du pape les plus
malignes intentions au régent sur l'alliance qu'il voulait faire avec
les hérétiques, et sur la douceur qu'il témoignait aux huguenots dans le
royaume. Ils se revirent une seconde fois. Hohendorff dit au nonce qu'il
allait dépêcher un courrier à l'empereur, pour lui conseiller de
contre-miner, par d'autres ligues, celle que la France venait enfin de
signer, que la plus naturelle serait avec le pape pour la sûreté
réciproque de leurs États, laquelle étant promptement déclarée, ferait
penser la France à deux fois à ne pas donner à l'empereur un sujet de
rupture, en attaquant Avignon\,; qu'il y avait du temps pour négocier,
puisque les ouvrages du canal de Mardick ne devaient être détruits que
dans le mais de mai\,; enfin il s'avança d'assurer, sans consulter la
volonté ni les finances de son maître, qu'il fournirait de l'argent au
Prétendant s'il était nécessaire, et pressa le nonce d'engager le pape
de faire parler de cette affaire à l'empereur duquel elle serait bien
reçue.

Le nonce, craignant les reprochés de Rome de s'être trop avancé,
prétendit s'être excusé de faire cet office, mais il y rendit compte de
la proposition, l'accompagnant de toutes les raisons qui pouvaient
engager le pape à la regarder comme avantageuse à la religion. Il
continuait, comme il avait déjà fait sauvent, à représenter au pape la
ligue de la France avec les protestants comme l'ouvrage des ministres
jansénistes, dans la vue d'établir en France le jansénisme, dont
l'unique remède était de leur apposer une ligue entre le pape et le
premier prince de la chrétienté, de mettre un frein aux entreprises des
ennemis de la religion, et de rendre le gouvernement de France plus
traitable quand il verrait ce qu'il aurait à craindre. Ce furieux nonce,
si digne du temps des Guise, tâcha, mais inutilement, de persuader à la
reine douairière d'Angleterre de préférer pour son fils ces espérances
frivoles à la promesse que faisait le régent de lui continuer les mêmes
pensions que le feu roi lui avait toujours données, s'il consentait
volontairement à se retirer d'Avignon en Italie. La reine, sans
s'expliquer, pria le nonce d'insinuer au pape d'écrire de sa main à
l'empereur en faveur de son fils, et de donner là-dessus des ordres
pressants à son nonce à Vienne.

Le pape, persuadé de la gloire qu'un accommodement avantageux de ses
différends avec l'Espagne donnerait à son pontificat, n'était pas mains
touché de l'utilité qu'il croyait trouver dans sa bonne intelligence
avec le toi d'Espagne, pour établir en France les maximes et l'autorité
de la cour de Rome. Aubenton, fabricateur de la constitution
\emph{Unigenitus}, et son homme de toute confiance, ne cessait de
l'assurer du respect, de l'attachement, de la soumission pour lui et
pour le saint-siège du roi d'Espagne, dont il gouvernait la conscience,
de son honneur pour les jansénistes, et de tout ce qu'il se passait en
France là-dessus. En même temps ce jésuite, lié avec Albéroni, qu'il
savait maître de le chasser et de le conserver dans sa place,
représentait continuellement au pape la nécessité d'élever promptement à
la pourpre un homme qui disposait seul et absolument du roi et de la
reine d'Espagne. Acquaviva et Aldovrandi agissaient avec la même
vivacité.

Vers la fin de novembre, ce cardinal reçut une lettre de la main de la
reine d'Espagne, pleine d'ardeur pour cette promotion. Il la fit voir au
pape, et le pressa si vivement, que Sa Sainteté n'eut de ressource pour
s'en débarrasser que de lui demander un peu de temps. Cela leur fit
juger qu'il ne résisterait pas longtemps. Tout de suite ils proposèrent
à Albéroni, pour hâter et faciliter tout, et pour plaire aussi à
Alexandre Albani, second neveu du pape, qui mourait d'envie d'être
envoyé en Espagne, par jalousie de son frère aîné, qui avait eu pareille
commission pour Vienne, de le demander pour aller terminer tous les
différends des deux cours. Ils désiraient donc que le roi d'Espagne
écrivît à Acquaviva pour le demander au pape\,; que cette lettre fût
apportée par un courrier exprès, accompagnée de celle d'Albéroni et
d'Aubenton, pour D. Alexandre, et ils représentaient qu'il était celui
des deux neveux que le pape aimait le mieux, qu'ils acquerraient à
l'Espagne par ce moyen, comme Vienne s'était attaché son frère aîné.
Aldovrandi, qui ne s'oubliait pas, désira que ses deux amis lui fissent
quelque mérite auprès d'Alexandre, et souhaitait pour son avancement
faire avec lui le voyage d'Espagne. Ils jugeaient ces mesures
nécessaires pour se mettre en garde contre beaucoup d'ennemis puissants
qu'Aldovrandi avait à Rome, dont Giudice se montrait le plus passionné.
Acquaviva, qui le craignait, assurait qu'il traitait secrètement avec la
princesse des Ursins, ce qui ne pouvait avoir d'objet que pour perdre la
reine, et y employer peut-être le nom du prince des Asturies, sur la
tendresse duquel Giudice comptait beaucoup. Il ajoutait qu'il fallait
bien prendre garde à ceux qui approchaient de ce jeune prince, surtout
des inférieurs, et se défier des artifices de Giudice, qui faisait
toutes sortes de bassesses pour se raccommoder avec le cardinal de La
Trémoille, et se laver auprès de lui d'avoir eu part à la disgrâce de sa
soeur.

Le pape, fortement pressé, avait positivement promis un chapeau pour
Albéroni, dès qu'il y en aurait trois vacants. Acquaviva n'osa en être
content, et pressa de plus en plus. Le pape, qui sentait l'embarras où
la promotion d'Albéroni seul le jetterait à l'égard de la France et de
l'empereur qu'il craignait bien davantage, répliqua que si les Allemands
étaient mécontents, ils se porteraient aux dernières violences.
Acquaviva, ne pouvant se servir de la peur en cette occasion, qui était
le grand ressort pour conduire le pape, l'employa pour empêcher la
promotion de Borromée, maître de chambre du pape et beau-frère de sa
nièce, au moment qu'il allait entrer au consistoire pour la faire. Le
pape se défendit sur ce que le chapeau vacant le devait dédommager de
celui de Bissy, accordé au feu roi, du consentement de l'empereur et du
roi d'Espagne. À la fin pourtant il se rendit et promit de suspendre la
promotion de Borromée, et de nouveau encore de faire Albéroni dès qu'il
y aurait trois chapeaux.

La conjoncture était favorable à Albéroni. Les préparatifs maritimes des
Turcs étaient grands, la frayeur du pape proportionnée, qui n'attendait
de secours que de l'Espagne. Il tâchait de le gagner par de belles
paroles et des remerciements prodigués sur le secours de l'été
précédent. Cette fumée ne faisait aucune impression sur un Italien,
savant dans les artifices de sa nation. Pour se procurer le secours que
le pape désirait, il en fallait donner les moyens, que le pape avait
lui-même offerts au roi d'Espagne sur le clergé d'Espagne et des Indes.
Acquaviva en sollicitait l'expédition\,; mais l'irrésolution du pape
éternisait les affaires, celles même qui dépendaient de lui et qu'il
souhaitait le plus. Albéroni se plaignait d'un retardement dont il
sentait personnellement le préjudice. Il assurait que le secours serait
tout prêt si le pape voulait finir les affaires d'Espagne\,; mais que ne
les finissant pas, l'armement devenait impossible\,; il s'étendait
surtout ce qu'il avait à souffrir de la part du roi et de la reine, qui
le regardaient comme un agent de Rome, qui lui en reprochaient les
lenteurs avec tant de sévérité, qu'il prévoyait qu'ils lui défendraient
bientôt de s'en plus mêler, comme ils avaient fait au P. Daubenton\,; et
là-dessus représentations et menaces, tous les ordinaires avec toutes
les souplesses du confesseur pour les faire valoir. Ils avaient affaire
à une cour où l'artifice est aisément démêlé. Le pape, mal prévenu pour
Albéroni, se défia que son chapeau étant accordé, il serait fertile en
expédients pour éluder les promesses faites en vue de l'obtenir, et
résolut de ne le donner que lorsque les affaires d'Espagne seraient
entièrement terminées. Albéroni, qui pensait le même du pape, déclarait
qu'elles le seraient à son entière satisfaction dans le moment même
qu'il recevrait la nouvelle de sa promotion, et n'avait garde de les
finir auparavant, dans la défiance d'en être la dupe. Ce manège de
réciproque défiance dura ainsi assez longtemps entre eux.

Le régent se plaignait fort d'Albéroni\,; il avait même laissé entendre
plusieurs fois au duc de Parme qu'il ne serait pas fâché qu'il fît
là-dessus quelques démarches auprès de la reine\,; mais un duc de Parme
se tenait heureux et honoré qu'un de ses ministres gouvernât l'Espagne
ainsi il s'était réduit à avertir Albéroni de bien servir l'Espagne sans
donner à la France des sujets de se plaindre de lui. Les instances du
régent redoublèrent\,: elles firent dire au duc de Parme qu'elles
approchaient de la violence, mais sans rien obtenir de lui qui ne
voulait point de changement dans le gouvernement d'Espagne. Il eut
seulement plus de curiosité de savoir par Albéroni même ce qu'il pensait
et pouvait pénétrer de plus particulier sur la personne, les vues, et ce
qu'il appelait les manèges de M. le duc d'Orléans\,; mais, persuadé au
reste que, quoi que ce prince pût penser et faire, le véritable intérêt
du roi d'Espagne était de demeurer sur son même trône\,; qu'il y aurait
trop d'imprudence de quitter le certain pour l'incertain, et que dans
les événements qui pouvaient arriver, il risquerait de perdre et la
France et l'Espagne, s'il voulait faire valoir les droits de sa
naissance. Albéroni lui répondit que, sûr de sa propre conscience et
probité, il ne pouvait attribuer qu'à ses ennemis les plaintes que
faisait le régent de sa conduite\,; qu'il avait toujours tâché de
mériter ses bonnes grâces, et de maintenir la bonne intelligence entre
les deux couronnes\,; il en alléguait les deux misérables preuves qu'on
a vus plus haut\,; qu'il ne pouvait donc attribuer le mécontentement de
ce prince qu'à ce qui s'était passé à l'égard de Louville mais qu'il se
plaignait lui-même de ce que le régent s'était laissé séduire par des
gens malintentionnés, au point d'avoir écrit des plaintes contre lui au
roi d'Espagne.

Cet homme de bien et de si bonne conscience savait qu'on l'accusait en
France d'une intelligence trop particulière avec les Anglais, et de les
avoir trop favorisés dans leurs dernières conventions avec l'Espagne.
Rien ne lui pouvait de plaire davantage que cette accusation où
l'avarice et l'infidélité, tout au moins la plus grossière ignorance ou
mal habileté étaient palpables. Il tâchait donc de récriminer\,: il
disait que ce n'était pas à la France à trouver à redire que l'Espagne,
pour conserver la paix, fît beaucoup mains que ceux qui sacrifiaient le
canal de Mardick pour être bien avec l'Angleterre, duquel les ouvrages
sont si importants, que le ministre d'Angleterre à Madrid avait dit tout
haut dans l'antichambre du roi d'Espagne, que la France aurait dû faire
la guerre pour le soutenir, et non pas une ligue pour le détruire. Ainsi
l'aigreur augmentait tous les jours, et Albéroni, parmi de fréquentes
protestations du contraire, aliénait de tout son pouvoir l'esprit du roi
d'Espagne contre le régent\,: les discours les plus odieux et les
raisonnements les plus étranges se publiaient sur M. le duc d'Orléans à
Madrid publiquement, et le premier ministre leur donnait cours et poids.
Il semblait qu'il eût dessein de se fortifier par des troupes étrangères
il fit demander au roi d'Angleterre la permission de lever jusqu'à trais
mille hommes dans la Grande-Bretagne, Irlandais ou autres, avec promesse
que ceux qui se trouveraient protestants ne seraient point inquiétés sur
leur religion. Il était si abhorré en Espagne, que la mort de l'archiduc
fit en même temps la joie du palais et la douleur de Madrid et de toute
l'Espagne, excédée du gouvernement du seul Albéroni. Moins il y avait de
princes de la maison d'Autriche, moins le roi d'Espagne se croyait
d'ennemis, et moins les Espagnols comptaient avoir de libérateurs et de
vengeurs.

Albéroni craignait encore plus ses ennemis personnels que ceux qui ne
l'étaient que pour le bien de l'État. Il était donc fort en peine de ce
que ferait Giudice contre lui, quand il serait arrivé à Rome. Ce
cardinal, qui depuis sa disgrâce ne se possédait plus, s'était échappé
dans une harangue qu'il avait faite à l'inquisition sur les intentions
de la reine, et sur la captivité où elle retenait le prince des
Asturies, dont en même temps il fit l'éloge. Albéroni ne manqua pas
d'exagérer à Rome l'ingratitude du cardinal, et tous les bienfaits qu'il
avait lui et les siens reçus de la reine. Il l'accusa de s'être opposé
le plus fortement à recevoir Aldovrandi à Madrid, qui n'y aurait jamais
été reçu sans la reine, laquelle seule avait empêché l'éloignement de
devenir plus grand entre les deux cours, comme Giudice le désirait\,; et
pour ne rien oublier de ce qui pouvait établir sur ses ruines le crédit
de la reine à Rome, c'est-à-dire le sien, il l'annonça comme un homme
qui ferait l'hypocrite à Rome, qui ne paraîtrait occupé que de
l'éternité, qui déplorerait les plaies que la religion souffrait en
Espagne de sa disgrâce et de son absence, et qui publierait toutes
sortes de faussetés et d'artifices qu'il serait facile au cardinal
Acquaviva de dévoiler. Mais lorsque l'accommodement entre les deux
cours, et la satisfaction personnelle du premier ministre à laquelle
tout le reste tenait, semblait s'approcher de plus en plus, l'impatience
du pape de se saisir en Espagne d'usurpations utiles, pensa tout
renverser. Il voulait s'approprier la dépouille des évêques, qui était
un des points des différends entre les deux cours. On a vu qu'il l'avait
fait demander comme par provision par le P. Daubenton, en attendant que
cet article fût réglé\,; on a vu aussi le mauvais succès de cette inique
demande.

Le pape ne s'en rebuta pas\,: n'y pouvant plus employer Aubenton, il
envoya un ordre direct à Giradilli, auditeur qu'Aldovrandi avait laissé
à Madrid, de faire pressamment la même demande, qui obéit par des
instances si fortes et si réitérées, qu'il fut au moment d'être chassé
de Madrid, dont Albéroni ne s'excusa que sur ce que cet homme était
connu depuis longtemps pour être agent du cardinal Acquaviva. Le premier
ministre jeta les hauts cris sur l'ingratitude de Rome pour la reine qui
avait tout fait pour cette cour. Il entra sur cela en de grands détails
et en de grands raisonnements, couverts du prétexte du zèle pour la
gloire et le service du pape et de la religion, qui en souffraient
beaucoup. Il protestait, on même temps, que ce n'était que par une vue
si pure qu'il déplorait les retardements que cette cour apportait à la
grâce que la reine demandait avec tant d'instance et depuis si
longtemps, sa promotion, qui perdrait son nom et son mérite pour devenir
justice, si elle n'était accordée que lors de celle des couronnes. Il
prévoyait, avec une grande douleur, que la reine, voyant le pape
inflexible sur un point qui touchait son honneur, se porterait aux
dernières extrémités si cette satisfaction qu'elle attendait, et le roi
aussi, avec la dernière impatience, se différait plus longtemps. Cet
homme détaché ne donnait ces avis que par zèle pour le saint-siège\,;
sans retour sur soi-même, en homme fidèlement attaché au pape, occupé de
contribuer à sa gloire et à son repos\,; qu'un particulier comme lui
était trop content des assurances du pape\,; que deux ou trois mois de
plus ou de moins ne lui étaient rien\,; qu'il désirerait faire de plus
grands sacrifices\,; mais qu'il n'osait parler, parce que le roi et la
reine lui reprocheraient qu'il ne songeait qu'à ses intérêts
particuliers, et comptait peu leur honneur offensé. Il ajoutait que,
quelque puissante que fût la raison de l'honneur et de la réputation de
têtes couronnées, l'impatience de la reine était fondée sur des raisons
particulières et secrètes, qui n'étaient pas moins pressantes que celles
du point d'honneur. Il les expliquait à ses amis à Rome il leur disait
que la reine envisageant le présent et l'avenir, que d'un côté elle
voyait la nécessité de donner un nouvel ordre au gouvernement de la
monarchie, et de supprimer ces conseils qui ne se croyaient pas
inférieurs à l'ancien aréopage, et en droit de donner des lois à leurs
souverains\,; d'un antre côté, elle considérait la santé menaçante du
roi d'Espagne par sa maigreur, ses vapeurs, sa mélancolie\,; par
conséquent le besoin qu'elle avait d'un ministre fidèle à qui elle pût
tout confier, lequel pour pouvoir lui donner ses conseils sans crainte,
avait besoin nécessairement d'un bouclier tel que la pourpre romaine,
pour le mettre à couvert de ceux qu'il ne pourrait éviter d'offenser.
Mais lorsqu'il écrivait de la sorte, il avait réduit tous les conseils à
néant, dont il avait pris, lui tout seul, les fonctions, les places, le
pouvoir. Il n'avait pas craint de le mander à tous les ministres que
l'Espagne tenait an dehors avec défense de rendre aucun compte à qui que
ce soit qu'à lui seul des affaires dont ils étaient chargés, et de ne
recevoir ordre de personne que de lui, ainsi qu'il se pratiquait dans
tout l'intérieur de la monarchie.

Il voyait aussi les choses de trop près pour pouvoir se flatter que la
reine venant à perdre le roi, ce qui n'avait alors qu'une apparence fort
éloignée, les Espagnols qui abhorraient sa personne et le gouvernement
étranger, qui n'aimaient guère mieux une reine italienne qui n'était pas
la mère de l'héritier présomptif et nécessaire\,; qui n'avait eu aucun
ménagement pour eux, et assez peu pour ce prince qui leur était si cher,
se laissassent subjuguer une seconde fois par une reine et un ministre
étrangers, qui n'auraient plus le nom du roi pour couverture, pour
prétexte et pour bouclier. Il n'y avait pas si longtemps que la minorité
de Charles II était passée pour avoir oublié que les seigneurs, ayant
don Juan à leur tête, firent chasser les favoris et les ministres
confidents de la reine mère et régente, fille et soeur d'empereurs, par
conséquent elle-même de la maison d'Autriche, le P. Nithard à Rome,
Vasconcellos aux Philippines, et lui ôtèrent toute son autorité. Mais
tout était bon à Albéroni pour leurrer le pape et ramener au point où il
voulait le réduire, qui était de le déclarer cardinal sans plus de
délai. Reste à voir ce que c'est qu'une dignité étrangère qui met à
l'abri de tout, par conséquent qui permet et qui enhardit à entreprendre
tout. C'était aussi l'usage qu'Albéroni se proposait bien de faire de
cette dignité après laquelle il soupirait avec tant d'emportement,
s'embarrassant très peu d'ailleurs des succès de tant de négociations,
dont les événements à venir étaient si importants à l'Espagne, et
faisaient le principal et peut-être le seul objet du roi et de la reine
d'Espagne.

Pour plaire à Stanhope il voulait accorder le congé à Monteléon qui le
demandait, fatigué de n'être instruit de rien, du changement à son égard
des ministres restés à Londres depuis le départ pour Hanovre, et d'être
mal payé de ses appointements. Quoiqu'il aimât mieux Beretti son
compatriote, il le laissait sans aucune instruction à la Haye sur ce que
la France y traitait. L'abbé Dubois, qui, après avoir arrêté l'alliance
à Hanovre, était venu à la Haye pour la conclure et la signer, et pour
aider à Châteauneuf à y faire entrer les États généraux, assurait
Beretti qu'il n'y avait rien dans ce traité que de conforme aux intérêts
du roi d'Espagne\,; lui et Châteauneuf l'avertissaient que la Hollande
avait résolu de faire avec l'empereur une alliance particulière\,; qu'il
était à craindre que son exemple n'y entraînât les autres provinces de
cette république\,; qu'ils devaient tous trois travailler de concert à
la traverser\,; qu'il était nécessaire qu'il parlât fortement là-dessus
aux bourgmestres d'Amsterdam et de Rotterdam. Beretti, qui était très
défiant, et qui était livré à lui-même parce qu'il ne recevait aucune
instruction d'Albéroni, comme on l'a remarqué, se figura que le but des
ambassadeurs de France était de confirmer de plus en plus la validité
des renonciations, d'employer toutes sortes de matériaux pour en
consolider l'édifice, engager le roi d'Espagne dans l'alliance qu'ils
étaient sur le point de signer avec l'Angleterre et la Hollande, et à
donner lui-même par là une nouvelle approbation et une nouvelle force au
traité d'Utrecht.

Dans une conjoncture qui lui semblait si délicate, Beretti déplaisait
d'autant plus à Albéroni, qu'il lui demandait des ordres précis que ce
confident de la reine ne lui voulait pas donner. Il lui reprochait son
inquiétude et sa curiosité. Il l'avertissait de se régler sur
l'indifférence que le roi et la reine d'Espagne témoignaient sur les
alliances négociées par la France, de ne pas chercher à pénétrer au delà
des instructions qu'on lui voulait bien donner, de se souvenir que
c'était à Madrid qu'ils voulaient traiter si la Hollande voulait faire
avec l'Espagne une alliance d'autant plus avantageuse que le roi avait
pris la résolution d'admettre désormais tous les étrangers au commerce
des Indes, de ne faire aucunes représailles sur les marchandises
embarquées en temps de paix, moyennant de leur part l'engagement
réciproque de n'attaquer aucun vaisseau revenant des Indes, et si ce
projet s'exécutait, donner à tout commerçant étranger voix dans la junte
générale que le roi établirait à Cadix pour le commerce. Le projet était
de supprimer en même temps la contractation de Séville et d'abolir
l'indult\footnote{Le mot \emph{indult} a ici un sens particulier et
  désigne le droit que le roi d'Espagne prélevait sur les galions qui
  apportaient les produits de l'Amérique espagnole. La
  \emph{contractation de Séville} était la chambre de commerce de cette
  ville.}, qu'on imposait depuis longtemps sur les vaisseaux qui
revenaient des Indes, au lieu duquel on établirait un tarif certain sur
les retours des flottes. Le dessein était aussi d'armer huit vaisseaux
pour lesquels on attendait les agrès de Hollande pour la fin de l'année,
qui devaient partir en avril, de faire apporter tout le tabac à Cadix,
vendu désormais sur le seul compte du roi, dont on faisait espérer un
profit du double, dont on verrait l'effet en 1718, et qu'en attendant on
offrait déjà pour l'année 1717 une augmentation de trois cent mille
écus. Albéroni se flattait de rendre le commerce d'Espagne plus
florissant que jamais par sa prévoyance, et par la plénitude d'autorité
qui lui serait confiée, et il commença à la fin de cette année 1716 à
faire travailler aux ports de Cadix et du Ferrol en Galice dont la
situation est admirable, sur lequel on avait de grandes vues, et le lieu
principal où on se proposait de bâtir des vaisseaux.

Un autre projet proposé par le prince de Santo-Buono-Carraccioli,
vice-roi du Pérou, homme de beaucoup d'esprit et de mérite, fut de
démembrer de son commandement les provinces de Santa-Fé, Carthagène,
Panama, Quito, la Nouvelle-Grenade, pour en faire le département d'un
troisième vice-roi, résidant à Santa-Fé, et cela fut approuvé du roi
d'Espagne. Le marquis de Valero, vice-roi du Mexique, donnait aussi de
grandes espérances\,; il voulait être regardé comme attaché à la reine.
C'était de ce nom qu'Albéroni appelait ses amis, et ce fut de ceux-là
dont il tâcha de remplir les places subalternes lorsqu'il changea tous
ces postes au commencement de 1717. Les abus étaient grands et les
prétextes ne manquaient pas de faire les retranchements qu'il méditait.
Plusieurs conseillers du conseil des Indes trouvés en grandes fraudes,
furent chassés, et plusieurs juntes de finances supprimées. Albéroni
comptait que de ces dépenses épargnées, le roi d'Espagne tirerait plus
de deux cent cinquante mille écus par an. Bien des gens se trouvaient
intéressés dans ce bouleversement\,; ainsi Albéroni tirant un mérite de
sa hardiesse à l'entreprendre, se fondait en nouvelles raisons, toutes
modestement résultantes du seul intérêt du service du roi, de le
garantir de la vengeance de tant de gens si irrités, et ce moyen était
unique, c'est-à-dire d'être promptement revêtu de la pourpre.

De là nouveaux ressorts et nouveaux manèges employés à Rome pour vaincre
la lenteur du pape, qui de son côté voulait des modifications à son gré
sur ce qui avait préliminairement été convenu sur les différends des
deux cours avec Aldovrandi à Madrid, et remettre cette affaire à Rome à
une congrégation. Le premier ministre et le confesseur, qui seuls s'en
étaient mêlés, menacèrent à leur tour d'une junte sur ces affaires qui
ferait voir au pape la différence de sa hauteur et de son opiniâtreté
d'avec la conduite de deux hommes dévoués au saint-siège, et qui pour
cela même, encourraient toute la haine de cette junte et de l'Espagne
entière. Albéroni, que rien ne pouvait détourner de son unique affaire,
avait sain de faire dire au pape qu'il ne craignait aucune opposition à
son chapeau de la part de la France\,; et comme les mensonges les plus
grossiers ne coûtaient rien là-dessus ni à lui ni au P. Daubenton, il se
vanta au pape de toute l'estime du régent, dont il le faisait assurer
sauvent, et même lui avait fait mander par le P. du Trévoux que Son
Altesse Royale désirait entretenir directement avec lui une secrète
correspondance de lettres.

La confiance du pape et de la cour de Rome en Daubenton, sûre de son
abandon à son autorité, à ses maximes par les effets, ne put être
obscurcie par les efforts de Giudice, qui ne craignait pas d'assurer le
pape que ce fourbe le trompait, et qu'il était capable de sacrifier son
baptême à la conservation de sa place. Ce jésuite ne laissait pas
d'avoir moyen de faire passer à Rome ses sentiments particuliers, et par
là ne craignait point qu'il lui fût rien imputé de ce que Rome trouvait
contre ses maximes dans ce que le roi d'Espagne le chargeait d'y écrire.
Ainsi le pape insistant sur l'entière exemption de toute imposition de
tous les biens patrimoniaux des ecclésiastiques d'Espagne, Aubenton lui
fit savoir nettement que cet article ne s'obtiendrait jamais, non pas
même avec aucun équivalent, parce que l'intention du roi d'Espagne
n'était pas d'augmenter par là ses revenus, mais de soulager ses sujets
à supporter les taxes qui grossissaient, et qui retombaient sur eux, à
mesure que les ecclésiastiques, exempts d'en payer aucune\,; acquéraient
des biens laïques. Aubenton revenait après à dissuader le pape de mettre
aucune de ces choses convenues à Madrid avec Aldovrandi en congrégation,
et à le menacer de les voir renvoyer à une junte en Espagne, dont il
verrait le terrible effet. Il ajoutait que le retour d'Aldovrandi en
Espagne était nécessaire, mais avec la grâce si instamment demandée, le
chapeau d'Albéroni, si le pape voulait obtenir toute sorte de
satisfaction qui ne lui serait donnée qu'à ce prix\,; que la reine,
irritée de tant de délais, était capable de se porter à toutes sortes
d'extrémités\,; que le ressentiment de se croire amusée et méprisée
allait en elle jusqu'à la fureur, sans qu'Albéroni, qui la voudrait
calmer au prix de son sang, osât plus lui ouvrir la bouche, surtout
depuis qu'ayant osé lui faire un jour quelque représentation, elle
l'avait fait taire et lui avait dit qu'elle voyait bien que six mois et
un an de retardement ne lui faisait rien, mais qu'un moment de
retardement faisait beaucoup à sa dignité et blessait son honneur.
C'était par de tels artifices qu'Albéroni comptait persuader le pape de
sa tranquillité sur le moment de sa promotion\,; qu'il ne la désirait
prompte que pour l'intérêt du pape, et que tout sujet qu'il enverrait à
Madrid serait sûr d'y réussir, s'il y trouvait contente du pape la reine
qui pouvait tout.

Il est vrai qu'elle était altière et qu'elle s'offensait fort aisément.
Elle le fit vivement sentir à la duchesse de Parme sa mère, qui de son
côté ne l'était pas moins. Il ne s'agissait néanmoins que de bagatelles,
mais la parfaite intelligence ne revint plus. Le duc de Parme, son oncle
et son beau-père, en sentit un autre trait pour ne l'avoir pas avertie à
temps du sujet de l'envoi du secrétaire Ré de Londres à Hanovre. Il se
trouva plus flexible que la duchesse sa femme\,; il s'excusa et dissipa
cette aigreur.

Albéroni, qui avait un commerce direct de lettres avec Stanhope, voulait
traiter avec l'Angleterre et la Hollande, laisser à Beretti le soin de
débrouiller le plus difficile avec les États généraux, et se réserver la
gloire d'achever à Madrid le traité avec Riperda. Beretti sentait le
poids de ce qu'on exigeait de lui, et en représentait toutes
difficultés. Il savait par le Pensionnaire même qu'il croyait de
l'intérêt de ses maîtres de traiter avec l'empereur avant de traiter
avec l'Espagne, et Beretti le soupçonnait de ne vouloir remettre la
négociation à Madrid, que pour la retarder, et parce qu'il serait plus
maître de donner ses ordres à Riperda, que d'une négociation qui se
traiterait à la Haye\,; mais l'empereur ne répondait point à
l'empressement de ce même Heinsius, et ne faisait aucune réponse aux
propositions que les États généraux lui avaient faites. La première
était de modérer le nombre de troupes qu'ils devaient fournir pour la
défense des Pays-Bas catholiques s'ils étaient attaqués\,; ils étaient
engagés par le traité de {[}la{]} Barrière à fournir en ce cas huit
mille hommes de pied et quatre mille chevaux. Ils voulaient plus de
proportion entre ces assistances et leurs forces, et des secours
conformes aux conjonctures sans spécification. En second lieu ils
demandaient qu'il plût à l'empereur de spécifier les princes qu'il
prétendait comprendre dans l'alliance\,; et en troisième lieu
l'observation exacte de la neutralité d'Italie. Enfin ils refusaient de
s'engager dans ce qui pourrait arriver au delà des Alpes et dans la
guerre contre les Turcs. Nonobstant le silence de l'empereur sur ces
propositions, ses ministres étaient fort inquiets, de l'alliance prête à
conclure entre la France, l'Angleterre et la Hollande, et ils
n'oubliaient rien à la Haye ni même à Paris pour la traverser.
Hohendorff continuait à voir Bentivoglio, et quoique encore sans ordre
de Vienne, il pressait ce nonce d'insinuer au Prétendant de ne point
sortir d'Avignon, dans l'opinion que cela dérangerait ce qui avait été
concerté et causerait une rupture. Le nonce l'espérait de même, et
goûtait avec plaisir tous les avis qu'on lui donnait des difficultés qui
s'opposaient à la signature du traité, et sa rupture comme un moyen
infaillible de ranger le régent au bon plaisir du pape sur l'affaire de
la constitution.

\hypertarget{chapitre-viii.}{%
\chapter{CHAPITRE VIII.}\label{chapitre-viii.}}

1717

~

{\textsc{1717.}} {\textsc{- Singularités à l'occasion du collier de
l'ordre envoyé au prince des Asturies, et par occasion du duc de
Popoli.}} {\textsc{- Caylus obtient la Toison.}} {\textsc{- Mort de
M\textsuperscript{me} de Langeais.}} {\textsc{- Mort de
M\textsuperscript{lle} de Beuvron.}} {\textsc{- Je prédis en plein
conseil de régence que la constitution deviendra règle et article de
foi.}} {\textsc{- Colloque curieux là même entre M. de Troyes et moi.}}
{\textsc{- Le procureur général d'Aguesseau lit au cardinal de Noailles
et à moi un mémoire transcendant sur la constitution.}} {\textsc{- Abbé
de Castries, archevêque de Tours, puis d'Albi, entre au conseil de
conscience.}} {\textsc{- Son caractère.}} {\textsc{- Abbaye d'Andecy
donnée à une de mes belles-soeurs.}} {\textsc{- Belle prétention des
maîtres des requêtes sur toutes les intendances.}} {\textsc{- Mort et
caractère de l'abbé de Saillant.}} {\textsc{- Je fais donner son abbaye,
à Senlis, à l'abbé de Fourilles.}} {\textsc{- Mort de
M\textsuperscript{me} d'Arco.}} {\textsc{- Paris-égout des voluptés de
toute l'Europe.}} {\textsc{- Mort du chancelier Voysin.}} {\textsc{-
Prompte adresse du duc de Noailles.}} {\textsc{- D'Aguesseau, procureur
général, chancelier.}} {\textsc{- Singularité de son frère.}} {\textsc{-
Ma conduite avec le régent et avec le nouveau chancelier.}} {\textsc{-
Joly de Fleury, procureur général.}} {\textsc{- Le duc de Noailles,
administrateur de Saint-Cyr avec Ormesson sous lui.}} {\textsc{- Famille
et caractère du chancelier d'Aguesseau.}} {\textsc{- Réponse étrange du
chancelier à une sage question du duc de Grammont l'aîné.}}

~

L'année 1717 commença par une bagatelle fort singulière\,: Le feu roi
avait voulu traiter en fils de France les enfants du roi d'Espagne qui,
par leur naissance, n'en étaient que petits-fils\,; et les renonciations
intervenues pour la paix d'Utrecht n'avaient rien changé à cet usage
dont les alliés ne s'aperçurent pas, et dont les princes, que les
renonciations du roi d'Espagne regardaient, ne prirent pas la peine de
s'apercevoir non plus. Suivant cette règle, tous les fils du roi
d'Espagne portèrent, comme fils de France, le cordon bleu en naissant,
et depuis la mort du roi, le roi d'Espagne, qui avait toujours les
pensées de retour bien avant imprimées, fut très soigneux de maintenir
cet usage d'autant plus que la France y entrait par l'envoi de
l'huissier de l'ordre, qui à chaque naissance d'infant partait aussitôt
pour lui porter le cordon bleu. Cette première cérémonie se fait sans
chapitre et sans nomination\,: le prince n'est chevalier que lorsqu'il
reçoit le collier. Le roi n'était point encore chevalier ni le prince
des Asturies. Le roi, son père, dès que ce prince approcha de dix ans,
demanda pour lui le collier avec instance\,; il n'y eut pas moyen de le
faire attendre jusqu'au lendemain du sacre du roi qu'il reçut lui-même
le collier. Le régent manda donc tous les chevaliers de l'ordre dans le
cabinet où se tendit le conseil de régence aux Tuileries. Le roi, au
sortir de sa messe, vint s'asseoir dans son fauteuil du conseil au bout
de la table, et ne se couvrit point. M. le duc d'Orléans se tint debout
et découvert à sa droite, et tous les chevaliers de même sans ordre le
long de la table des deux côtés\,; les officiers commandeurs au bas bout
de la table, vis-à-vis du roi. M. le duc d'Orléans proposa d'envoyer
deux colliers au roi d'Espagne avec une commission pour les conférer,
l'un au prince des Asturies, l'autre à son gouverneur le duc de Popoli à
qui le feu roi avait promis l'ordre et le permis de le porter en
attendant qu'il eût le collier.

Cela fut appuyé de l'exemple d'Henri IV qui n'étant pas encore sacré ni
chevalier de l'ordre, et qui même ne le portait pas parce qu'il était
encore huguenot, donna une commission au maréchal de Biron, chevalier de
l'ordre, et le premier de son parti, pour recevoir et donner le collier
de l'ordre à son fils qui fut depuis amiral, maréchal et duc et pair de
France, et décapité à Paris, dernier juillet 1602, et donner en même
temps le cordon bleu à Renaud de Beaulne archevêque de Bourges, depuis
de Sens, à qui six mois auparavant le roi avait donné la charge de grand
aumônier de France, qu'il avait ôtée avec le cordon bleu qui y est
attaché à Jacques Amyot relégué dans son diocèse d'Auxerre\,; et qui
s'était montré grand ligueur. Ainsi le cardinal de Bouillon n'a pas été
le premier à qui cette charge et le cordon bleu qui y est joint aient
été ôtés. Ce fut en faveur du même Amyot, qui était fils d'un artisan et
que son esprit, son savoir et son éloquence avait fait précepteur des
enfants d'Henri II, qu'Henri III, en créant l'ordre du Saint-Esprit,
attacha à la charge de grand aumônier de France qu'Amyot avait lors
celle de grand aumônier de l'ordre, sans preuves, parce qu'il n'en
pouvait faire, ce qui a toujours subsisté depuis. Le maréchal de Biron,
en vertu de la commission d'Henri IV, fit cette cérémonie dans l'église
collégiale de Mantes, le dernier décembre 1591. Henri IV fit dans
l'église abbatiale de Saint-Denis son abjuration publique, le dimanche
25 juillet 1593, entre les mains du même Renaud de Beaulne, archevêque
de Bourges, qui dit tout de suite la messe pontificalement et le
communia\,; il fut sacré le premier dimanche de carême, 27 février 1594,
et reçut le lendemain le collier de l'ordre du Saint-Esprit, et Clément
IX, Aldobrandin, le voyant maître de Paris et de tout le royaume, lui
donna l'absolution, le 17 septembre 1595.

Le régent ne voulut pas tenir cette assemblée sans le roi, et y voulut
suivre la moderne manière que le feu roi avait introduite dans les
chapitres, où en faveur de ses ministres officiers de l'ordre, qui, à
l'exception du seul chancelier de l'ordre, y sont debout et découverts,
tandis que tous les chevaliers sont assis en rang et couverts, n'en
tenait plus que debout et découvert lui-même. Ainsi le roi fut
découvert, et il ne fût assis qu'à cause de son âge\,; non qu'il puisse
y avoir de proportion entre le roi et ses sujets, mais parce que, depuis
que l'ordre a été institué, les rois ne se sont jamais assis ni couverts
aux chapitres, qu'ils n'y aient fait en même temps asseoir et couvrir
tous les chevaliers\,; c'est aussi ce qui se pratiqua de tout temps
jusqu'à cette heure dans tous les chapitres de l'ordre de la Jarretière
et de celui de la Toison d'or. Ce dernier ordre fut donné en ce temps-ci
par le roi d'Espagne à Caylus que nous avons vu être allé servir en
Espagne après son combat avec le fils aîné du comte d'Auvergne.

M\textsuperscript{me} de Langeais mourut le premier jour de cette année
à Luxembourg à Paris, où elle avait un appartement. Elle était soeur du
feu maréchal de Navailles et avait quatre-vingt-neuf ans. Son mari
s'appelait Cordouan. Le huguenotisme avait fait ce mariage. Elle avait
été longtemps en Hollande\,; elle revint se convertir et eut six mille
livres de pension.

Le maréchal d'Harcourt perdit M\textsuperscript{lle} de Beuvron, sa
soeur, fille d'esprit, de mérite et de conduite, qui avait de la
considération, et qui s'était retirée depuis assez longtemps dans un
couvent en Normandie.

Quoique l'affaire de la constitution n'entre point dans ces Mémoires par
les raisons que j'en ai alléguées, il s'y trouve certains faits qui me
sont particuliers, ou qui ne sont connus, qui y doivent trouver place
comme il est déjà arrivé quelquefois, parce que j'ai lieu de douter
qu'ils la trouvent dans l'histoire de cette fameuse affaire, dont les
auteurs les auront pu aisément ignorer. Quoiqu'elle se traitât dans le
cabinet du régent avec Effiat, le premier président, les gens du roi,
divers prélats, l'abbé Dubois, le maréchal d'Huxelles, il ne laissait
pas d'en revenir quelquefois au conseil de régence dans quelques
occasions. M. de Troyes s'y signalait toujours en faveur de la
constitution, et des prétentions de Rome, en pénitence apparemment d'y
avoir été toute sa vie fort opposé. Il rendait compte de tout au nonce
Bentivoglio. Je ne sais à son âge quel pouvait être son but. Un des
premiers jours de ce mois-ci de janvier, il fut question de la
constitution au conseil de régence. Je ne m'étendrai pas sur quoi, parce
que je n'ai pas dessein de m'arrêter à cette matière. Je voyais un grand
emportement pour exiger une soumission aveugle sans explication et sans
réplique, et que ce parti d'une obéissance sans mesure allait toujours
croissant.

Je ne fus pas de l'avis de M. de Troyes\,; il s'anima\,; nous disputâmes
tous deux\,; il s'abandonna tellement à ses idées que je lui répondis
brusquement que dans peu la constitution ferait une belle fortune, parce
que je voyais que de proche en proche elle parviendrait bientôt à
devenir dogme et article de foi\,: là-dessus voilà M. de Troyes à
s'exclamer à la calomnie, et que je passais toujours le but\,; de là à
s'étendre pour montrer que la constitution ne pouvait jamais devenir ni
dogme, ni règle, ni article de foi\,; qu'à Rome cela n'était entré dans
la tête de personne, et que le cardinal Tolomeï qui avait été toute sa
vie jésuite, et de jésuite avait été fait cardinal, s'était moqué avec
dérision quand on lui avait touché cette corde. Quand il eut bien crié,
je regardai tout le conseil, et je dis\,: «\,Messieurs, trouvez bon que
je vous prenne tous ensemble et chacun en particulier à témoin de tout
ce que je viens de prédire sur la fortune de la constitution, de tout ce
que M. de Troyes a répondu, combien il s'est étendu à prouver qu'il est
impossible par sa nature qu'elle puisse jamais être proposée en article,
dogme, ou règle de foi, et qu'on s'en moque à Rome, et de me permettre
de vous faire souvenir de ce qui se passe ici aujourd'hui quand la
constitution aura fait enfin cette fortune comme je vous répète que cela
ne tardera point à arriver.\,» M. de Troyes cria de nouveau à
l'absurdité\,: pour n'en pas faire à deux fois, au bout de six mois, et
même moins, je fus prophète.

Le dogme, la règle de foi pointèrent. Les grands athlètes de la
constitution l'établirent dans leurs discours et dans leurs écrits, et
en peu de temps la prétention en fut portée jusqu'où on la voit
parvenue. Dès que cette opinion commença à se montrer à découvert avec
autorité, je ne manquai pas de faire souvenir en plein conseil de
régence de ma prophétie, et des exclamations de M. de Troyes\,; puis, me
tournant vers lui, je lui dis avec un souris amer\,: «\, Vous m'en
croirez, monsieur, une autre fois\,! Oh bien, ajoutai-je, nous en
verrons bien d'autres.\,» Personne ne dit mot, ni le régent non plus. Je
ne vis jamais homme si piqué ni si embarrassé que M. de Troyes, qui
rougit furieusement, et qui la tête basse ne répondit pas un seul mot.
Ces deux scènes firent chacune quelque bruit en leur temps\,; elles ne
tenaient en rien au secret du conseil, je ne me contraignis pas de les
rendre, ni plusieurs du conseil de régence non plus. M. le duc d'Orléans
ne le trouva point mauvais\,: il fit semblant, ou crut en effet que
j'allais trop loin comme M. de Troyes, et fut ou fit le semblant d'être
fort surpris quand ma prophétie se vérifia. M. le cardinal de Noailles
avait des audiences de M. le duc d'Orléans assez fréquentes\,; les
prétentions de l'abbé Dubois ne l'avaient pas encore culbuté\,: la
petite vérole dont Paris était plein se mit dans l'archevêché, et
l'obligea d'en sortir, parce que M. le duc d'Orléans qui voyait le roi
presque tous les jours ne voulait aucun commerce avec le moindre soupçon
de mauvais air. La duchesse de Richelieu, veuve en premières noces de M.
de Noailles, frère du cardinal, était demeurée en liaison intime avec
lui, et fort bien avec tous les Noailles\,: elle avait bâti une fort
belle maison au bout du faubourg Saint-Germain, qui est aujourd'hui
revenue par ricochet aux Noailles\,: elle y offrit retraite au cardinal
qui l'accepta.

Étant chez elle il me proposa un rendez-vous dans son cabinet avec le
procureur général qui avait envie, et lui aussi, que j'entendisse la
lecture d'un mémoire qu'il venait d'achever sur l'affaire de la
constitution, et qui n'était pas à portée de m'en parler lui-même\,;
parce que les affaires du roi m'avaient refroidi avec lui. J'eus en
effet quelque peine à consentir. Enfin je me laissai aller au cardinal,
et le rendez-vous fut pris chez la duchesse de Richelieu où il logeait,
pour le surlendemain trois heures après midi. Je m'y rendis, la porte
fut bien fermée. Nous étions tous trois seuls, et la lecture dura deux
heures. L'objet du mémoire était de montrer qu'il n'y avait aucun moyen
de recevoir une bulle qui était aussi contraire que l'était la
constitution \emph{Unigenitus} à toutes les lois de l'Église, et aux
maximes et usages du royaume, fondées sur les libertés de l'Église
gallicane, qui elles-mêmes ne sont que l'observation des canons et des
règles établies de tout temps dans l'Église universelle, et qui n'ont
été maintenues dans leur intégrité que dans l'Église de France contre
les entreprises de la cour de Rome. Outre l'érudition qui sans
affectation était répandue dans tout le mémoire, et la beauté de la
diction sans recherche d'éloquence, il était admirable par le tissu
d'une chaîne de preuves dont les chaînons semblaient naître
naturellement les uns des autres, qui portaient les preuves de tout le
contenu du mémoire dans un ordre qui en faisait la clarté, et dans un
degré qui en formait une évidence à laquelle il était impossible de se
refuser. Il était d'ailleurs contenu dans toutes les bornes que la
primauté de Rome sur toutes les églises pouvait justement exiger, et
dans le respect dû à la dignité et à la personne du pape. La conclusion
était de lui renvoyer sa bulle après avoir jusqu'alors tenté et cherché
inutilement quelque moyen de la pouvoir recevoir, uniquement guidés dans
tout le travail qui s'était fait là-dessus à marquer la bonne volonté,
le désir et le respect pour le saint-siège et pour le pape. Je fus
charmé de cette pièce, et je montrai au procureur général dans toute
l'étendue de l'impression qu'elle m'avait faite\,: Le cardinal de
Noailles n'en fut pas moins satisfait. Nous raisonnâmes ensuite avant de
nous séparer. Mais le malheur était que la religion et la vérité
n'étaient pas le gouvernail de cette malheureuse affaire, comme ni l'une
ni l'autre n'en avaient été la source du côté de Rome et de ceux qui
s'étaient employés à la demander, à la fabriquer, à la soutenir, et à la
conduire pour leur ambition au point où nous la voyons, aux dépens de la
religion, de la vérité, de la justice, de l'Église et de l'État, de tant
de savantes écoles, et de tant d'illustres corps d'ecclésiastiques et de
réguliers, enfin d'un peuple immense de saints et de savants
particuliers.

L'abbé de Castries, premier aumônier de M\textsuperscript{me} la
duchesse de Berry, et fort bien avec elle et avec M\textsuperscript{me}
la duchesse d'Orléans, qui aimait fort son frère et sa belle-soeur, qui
étaient, comme on l'a vu plus d'une fois, à elle, fut nommé à
l'archevêché de Tours. J'y contribuai aussi avec force, et je ne
comprends pas pourquoi il en fut besoin au secours de ces deux
princesses. Il était bien fait et avait un esprit extrêmement aimable,
sage et doux, et fort sûr dans le commerce. Lui et son frère chez qui il
demeurait avaient beaucoup d'amis, et il était désiré dans les
meilleures compagnies. Cela choqua tellement le feu roi depuis qu'on
l'eut infatué de noms inconnus, et de crasse de séminaires pour être
maîtres des nominations, et après des évêques, que l'abbé de Castries ne
put jamais le devenir. Il fut peu à Tours qui était lors fort pauvre
quoique un grand siège. Il fut sacré par le cardinal de Noailles avec
qui il était fort bien, et aussitôt après il entra au conseil de
conscience où des deux places destinées à des évêques il n'y en avait
qu'une de remplie par le frère du maréchal de Besons, lors archevêque de
Bordeaux. Les chefs de la constitution crièrent beaucoup du consécrateur
et de la place. Leurs aboiements n'empêchèrent pas qu'Albi ayant vaqué
peu de temps après, ce riche archevêché lui fût donné, en sorte qu'il
n'alla jamais à Tours. Longues années depuis il a eu l'ordre du
Saint-Esprit, et vit encore fort vieux et adoré dans son diocèse, où il
a toujours très assidûment résidé, tout occupé des devoirs de son
ministère. Je fis donner en même temps la petite abbaye d'Andecy à une
soeur de M\textsuperscript{me} de Saint-Simon, religieuse de Conflans
près Paris, fort sainte fille, mais qui n'était pas faite pour en
gouverner une plus grande. Lorsque j'allai le lui apprendre, elle
s'évanouit, puis refusa, et ce ne fut qu'à peine qu'on la lui fit
accepter. Elle en tomba fort malade et la fut longtemps. Peu de
religieuses deviennent abbesses de la sorte.

Boucher, fils d'un secrétaire du chancelier Boucherat, qui s'y était
fort enrichi, était beau-frère de M. Le Blanc, dont la diverse fortune a
depuis fait tant de bruit dans le monde. Ils avaient épousé les deux
soeurs\,; Le Blanc pointait fort auprès de M. le duc d'Orléans. Il en
obtint l'intendance d'Auvergne pour son beau-frère, qui était président
en la cour des aides. Rien de si plaisant que le scandale que les
maîtres des requêtes en prirent, et que l'éclat qu'ils osèrent en faire.
C'était le temps de tout prétendre et de tout oser. Aussi firent-ils les
hauts cris d'une place qui leur était dérobée, comme si, pour être
intendant, il fallait être maître des requêtes, et qu'on n'en eût jamais
fait que de leur corps. Ils députèrent au chancelier pour écouter et
porter leurs plaintes au régent. Tous deux se moquèrent d'eux et tout le
monde aussi.

L'abbé de Saillant mourut médiocrement vieux. Il était frère de
Saillant, lieutenant général, lieutenant-colonel du régiment des gardes,
et commandant à Metz et dans les trois évêchés. C'eût été un honnête
homme s'il avait eu des moeurs. La débauche, l'agrément de l'esprit et
la sûreté du commerce lui avaient acquis des amis considérables, le
maréchal de Luxembourg entre autres intimement, qui à force de bras lui
avait procuré quelques abbayes. Il en avait une assez bonne dans Senlis.
Je logeais alors dans une maison des jacobins, rue Saint-Dominique, dont
la vue était sur leur jardin, où j'avais une porte. Le devant de la
maison voisine était occupé par Fourilles, capitaine aux gardes, qui
était aveugle, et s'était retiré avec un cordon rouge. Je le voyais tous
les jours se promener deux et trois heures dans ce jardin des jacobins,
conduit par son fils, qui était abbé sans ordres ni bénéfices, et qui
lui lisait pendant toute la promenade. Tous deux avaient l'esprit orné,
et le père en avait beaucoup. Cette assiduité me toucha. Je m'informai
doucement du jeune homme, car il n'avait pas vingt ans. Il m'en revint
du bien, et qu'il ne quittait pas son père, à qui il lisait presque
toute la journée. Je ne les connaissais point ni personne de leurs
amis\,; jamais ils n'étaient venus chez moi, pas un de la famille,
jamais je n'avais parlé à aucun. Je me mis dans la tête de faire donner
cette abbaye de Senlis à un si honnête fils, j'en fis l'histoire à M. le
duc d'Orléans, et je l'obtins. Jamais gens plus étonnés qu'ils le furent
quand je le leur allai dire. Je me fis un vrai plaisir d'avoir fait
récompenser cette piété, et j'eus lieu dans la suite d'en être encore
plus content par l'honnête et sage conduite de l'abbé, et par leur
reconnaissance.

M\textsuperscript{me} d'Arco mourut à Paris, où elle donnait à jouer
tant qu'elle pouvait. Elle s'appelait étant fille M\textsuperscript{lle}
Popuel, était fort belle, et avait été longtemps maîtresse déclarée, en
Flandre, de l'électeur de Bavière, dont elle avait eu le chevalier de
Bavière. Son mari était frère du maréchal d'Arco, qui commandait en chef
les troupes de Bavière, et dont il a été fait ici mention quelquefois
dans les guerres précédentes.

Le goût, l'exemple et la faveur du feu roi avait fait de Paris l'égout
des voluptés de toute l'Europe, et le continua longtemps après lui.
Outre les maîtresses du feu roi, ses bâtards, ceux de Charles IX, car
j'en ai vu une veuve et sa belle-fille, ceux d'Henri IV, ceux de M. le
duc d'Orléans, à qui sa régence a fait une immense fortune, les deux
branches des deux frères Bourbons, Malause et Busset, les Vertus bâtards
du dernier duc de Bretagne, les bâtardes des trois derniers Condé, et
jusqu'aux Rothelin, bâtards de bâtards, c'est-à-dire d'un cadet de
Longueville, desquels bâtards d'Orléans le dernier est mort de mon
temps, et M\textsuperscript{me} de Nemours sa soeur bien plus tard
encore\,; Rothelin, dis-je, qui dans ces derniers temps ont osé se
croire quelque chose, et l'ont presque persuadé par l'audace d'une
couronne de prince du sang qu'ils ont arborée depuis qu'elles sont
toutes tombées dans le plus surprenant pillage\,; outre ce peuple de
bâtards français, Paris a ramassé les maîtresses des rois d'Angleterre
et de Sardaigne, et deux de l'électeur de Bavière, et les nombreux
bâtards d'Angleterre, de Bavière de Savoie, de Danemark, de Saxe, et
jusqu'à ceux de Lorraine, qui tous y ont fait de riches, de grandes et
de rapides fortunes, y ont entassé des ordres, des grades plus que
prématurés, une infinité de grâces et de distinctions de toutes sortes,
plusieurs des honneurs et des rangs les plus distingués, dont pas un
d'eux n'eût été seulement regardé dans aucun autre pays de l'Europe\,;
enfin jusqu'aux plus infâmes fruits des plus monstrueux incestes et les
plus publics, d'un petit duc de, Montbéliard, déclarés solennellement
tels par le conseil aulique de Vienne, rejetés comme tels par tout
l'empire et de toute la maison de Wurtemberg, lesquels toutefois ont eu
l'audace d'y vouloir faire les princes, et y ont trouvé l'appui d'autres
prétendus princes, qui avec l'usurpation du rang, et une naissance
légitime et française, ne sont pas plus princes qu'eux de tant d'écumes
que la France seule s'est trouvée capable de recevoir, et entre toutes
les nations de l'Europe, d'honorer et d'illustrer par-dessus sa première
noblesse qui a eu la folie d'y concourir et d'y applaudir la première,
il faut pourtant avouer qu'un bâtard d'Angleterre et un autre de Saxe
ont rendu de grands services à l'État en commandant glorieusement les
armées.

La veille de la Chandeleur nous soupions plusieurs en liberté chez
Louville. Un moment après qu'on eut servi le fruit, on vint parler à
l'oreille de Saint-Contest, conseiller d'État, qui sortit de table
aussitôt. Son absence fut courte\,; mais il revint si occupé, en nous
promettant de nous apprendre de quoi, que nous ne songeâmes plus qu'à
sortir de table. Quand nous fûmes rentrés autour du feu, il nous dit la
nouvelle. C'est que le chancelier Voysin, soupant chez lui avec sa
famille, se portant bien, avait été tout d'un coup frappé d'une
apoplexie, et était tombé à l'instant comme mort sur
M\textsuperscript{me} de Lamoignon, Voysin comme lui, et qu'en un mot il
n'en avait pas pour deux heures. En effet, il ne vécut guère au delà, et
la connaissance ne lui revint plus. J'ai assez fait connaître ce
personnage pour n'avoir rien à y ajouter. La femme de Saint-Contest
était Le Maistre, de cette ancienne et illustre magistrature de Paris,
et soeur de la mère d'Ormesson et de la femme du procureur général sur
lequel Saint-Contest porta aussitôt ses désirs. Après ce récit, il nous
quitta pour aller l'avertir. Il trouva toute la maison couchée et
endormie\,; en sorte qu'il y retourna le lendemain de bonne heure, et
tira le procureur général de son lit. Celui-ci compta si peu que cette
grande place pût le regarder, qu'il ne s'en donna pas le moindre
mouvement\,; il s'habilla tranquillement, et s'en alla avec sa femme à
sa grand'messe de paroisse à Saint-André des Arcs.

Le duc de Noailles, averti le soir ou dans la nuit, ne négligea pas une
si grande occasion de s'avancer vers la place de premier ministre, qui
ne cessa jamais de faire l'objet le plus cher de tous ses voeux. De tout
temps il était ami du procureur général. Le mérite solide du père, la
réputation brillante du fils, n'avaient pu échapper aux Noailles qui les
avaient tous fort cultivés. Le duc de Noailles ne pouvait avoir un
chancelier plus à son point. Il se persuada de plus qu'il gouvernerait
cet esprit doux, incertain, qui se trouverait comme un aveugle au milieu
du bruit et des cabales, et qui se sentirait heureux qu'un guide tel que
le duc de Noailles voulût le conduire. Plein de cette idée qui ne le
trompa point, il alla trouver M. le duc d'Orléans comme il sortait de
son lit, et venait se mettre sur sa chaise percée, l'estomac fort
indigeste, et sa tête fort étourdie du sommeil et du souper de la
veille, comme il était tous les matins en se levant, et du temps encore
après. Le duc de Noailles fit sortir le peu de valets qui se trouvèrent
là, apprit à M. le duc d'Orléans la mort du chancelier, et dans
l'instant bombarda la charge pour d'Aguesseau. Tout de suite il le manda
au Palais-Royal, où il se tint jusqu'à son arrivée pour plus grande
précaution. Dans cet intervalle, Larochepot, Vaubourg et Trudaine,
conseillers d'État, le premier gendre, les deux autres beaux-frères de
Voysin, vinrent rapporter les sceaux au régent, qui mit la cassette sur
sa table et les congédia avec un compliment. Le messager qui avait été
dépêché à d'Aguesseau ne le trouvant point chez lui, le fut chercher à
sa paroisse. Il vint sur-le-champ au Palais-Royal comme M. le duc
d'Orléans venait d'achever de s'habiller, qui avait demandé son
carrosse. D'Aguesseau trouva le duc de Noailles avec M. le duc d'Orléans
dans son cabinet, qui, avec les compliments flatteurs dont on accompagne
toujours de pareilles grâces, lui déclara celle qu'il lui faisait. Fort
peu après, il sortit de son cabinet, et prenant d'Aguesseau par le bras,
il dit à la compagnie qu'ils voyaient en lui un nouveau et très digne
chancelier, et tout de suite, faisant porter la cassette des sceaux
devant lui, il alla monter en carrosse avec la cassette et le
chancelier. Il le mena aux Tuileries, en fit l'éloge au roi, puis lui
présenta la cassette des sceaux sur laquelle le roi mit la main pour la
remettre à d'Aguesseau, tandis que M. le duc d'Orléans la tenait.

D'Aguesseau l'ayant reçue de la sorte fut modeste à l'affluence des
compliments\,; il s'y déroba le plus tôt qu'il put, et s'en alla chez
lui avec la précieuse cassette, où tout était plein de parents et d'amis
en émoi du message de M. le duc d'Orléans, qui, dans l'occurrence de la
vacance, avait fait grand bruit à Saint-André des Arcs et dans tous les
quartiers voisins. D'Aguesseau, dans sa surprise, ne vit qu'un étang, et
ne se remit que dans son carrosse en allant chez lui seul avec les
sceaux. Après les premières bordées qu'il fallut essuyer en y arrivant,
il monta chez son frère, espèce de philosophe voluptueux, de beaucoup
d'esprit et de savoir, mais tout des plus singuliers. Il le trouva
fumant devant son feu en robe de chambre. «\,Mon frère, lui dit-il en
entrant, je viens vous dire que je suis chancelier.\,» L'autre se
tournant\,: «\,Chancelier, dit-il\,; qu'avez-vous fait de l'autre\,? ---
Il est mort subitement cette nuit. --- Oh bien\,! mon frère, j'en suis
bien aise\,; j'aime mieux que vous le soyez que moi.\,» C'est tout le
compliment qu'il en eut. Le duc de Noailles en reçut de beaucoup de
gens. Il était visible qu'il avait fait le chancelier, et il était bien
aise que personne n'en doutât. J'appris cette nouvelle de bonne heure
dans la matinée.

J'allai l'après-dînée au Palais-Royal\,; M. le duc d'Orléans n'était pas
remonté de chez M\textsuperscript{me} la duchesse d'Orléans\,; j'y
descendis par les cabinets. Je le trouvai au chevet de son lit où elle
était pour quelque migraine. Il me parla tout aussitôt de la nouvelle du
jour. Comme la chose était faite, je suivis ma maxime de n'y rien
opposer. Je lui dis qu'il ne pouvait choisir pour cette grande place de
magistrat plus savant, plus lumineux, plus intègre, ni dont l'élévation
dût être plus approuvée. J'ajoutai seulement que son âge fâcherait
beaucoup de gens qui par le leur n'auraient plus d'espérance, et que je
souhaitais que d'Aguesseau oubliât qu'il avait passé sa vie jusqu'alors
dans le parlement, et tout ce dont il s'y était imbu, pour ne se
souvenir que des devoirs de son office et de sa reconnaissance.
L'engouement où la flatterie des applaudissements à ce choix l'avaient
mis l'empêcha de sentir le poids de cette parole dont il eut lieu de se
souvenir depuis. Dans cet enthousiasme il me demanda avec une sorte
d'inquiétude comment j'étais avec lui. J'avais dès le matin pris mon
parti dans la seule vue du bien des affaires. Je répondis qu'il pouvait
se souvenir qu'avant la mort du roi, je lui avais proposé, et souvent
pressé de chasser Voysin quand il serait le maître, et de donner les
sceaux au bonhomme d'Aguesseau\,; que le plaidoyer de son fils dans
notre procès de préséance contre M. de Luxembourg lui avait acquis mon
coeur et mon estime\,; que sans commerce par la différence de notre
genre de vie, et celle de nôtre demeure, ces mêmes sentiments étaient
demeurés en moi\,; qu'il était vrai qu'ils s'étaient changés en froideur
très marquée depuis l'affaire du bonnet, et ce qui s'était passé à
l'égard du parlement\,; mais que dans l'espérance que d'Aguesseau
deviendrait en tout chancelier de France, et qu'il se dépouillerait de
ses premiers préjugés, je vivrais avec lui sur ce pied-là pour le bien
des affaires, et que, dès ce même jour\,; j'irais lui faire mes
compliments. Je l'exécutai en effet\,; dont M. le duc d'Orléans me parut
fort soulagé et fort aise, et le nouveau chancelier infiniment touché.
Sa charge de procureur général fut en même temps donnée à Joly de
Fleury, premier avocat général, et le duc de Noailles, qui ne négligeait
pas les moindres choses, se fit donner l'administration des biens de la
maison de Saint-Cyr comme une chose de convenance qu'avait le chancelier
Voysin, et prit pour s'en mêler directement sous lui d'Ormesson, maître
des requêtes alors, frère de la nouvelle chancelière.

Un chancelier doit être un personnage, et dans une régence il ne se peut
qu'il n'en soit un. Celui-là l'a été si longtemps, puisqu'il vit encore,
et a été si battu de la fortune dans cette grande place qui semblerait
en être le port et l'asile, que tant de raisons m'engagent à passer sur
la règle que je me suis faite de ne m'étendre point sur ceux qui sont
encore au monde dans le temps que j'écris.

Il naquit le 26 novembre 1668\,; avocat général, 12 janvier 1691, à
vingt-deux ans et demi\,; procureur général, 19 novembre 1700 à
trente-deux ans\,; chancelier et garde des sceaux de France, 2 février
1717, à quarante-six ans. Le père de son père était maître des comptes,
il est bon de n'aller pas plus loin. Ce maître des comptes maria
pourtant sa fille au père de MM. d'Armentières et de Conflans, tous deux
gendres de M\textsuperscript{me} de Jussac dont j'ai parlé ailleurs et
du bailli de Conflans, avec la petite terre de Puyseux qu'ils en ont
encore, et les soeurs du chancelier ont été mariées, longtemps avant
qu'il le fût, la cadette à M. Le Guerchois, mort conseiller d'État sans
enfants, l'autre à M. de Tavannes, père et mère de M. de Tavannes,
lieutenant général et commandant en Bourgogne et chevalier de l'ordre,
et de l'archevêque de Rouen, grand aumônier de la reine, ci-devant
évêque-comte de Châlons, dont par brevet il a conservé le rang.

D'Aguesseau, de taille médiocre, fut gros, avec un visage fort plein et
agréable, jusqu'à ses dernières disgrâces, et toujours avec une
physionomie sage et spirituelle, un oeil pourtant bien plus petit que
l'autre. Il est remarquable qu'il n'a jamais eu voix délibérative avant
d'être chancelier, et qu'on se piquait volontiers au parlement de ne pas
suivre ses conclusions, par une jalousie de l'éclat de la réputation
qu'il avait acquise, qui prévalait à l'estime et à l'amitié. Beaucoup
d'esprit, d'application, de pénétration, de savoir en tout genre, de
gravité et de magistrature, d'équité, de piété et d'innocence de moeurs,
firent le fonds de son caractère., On peut dire que c'était un bel
esprit et un homme incorruptible, si on en excepte l'affaire des
Bouillon, qui a été racontée\,; avec cela doux, bon, humain, d'un accès
facile et agréable, et dans le particulier de la gaieté et de la
plaisanterie salée, mais sans jamais blesser personne\,; extrêmement
sobre, poli sans orgueil, et noble sans la moindre avarice,
naturellement paresseux, dont il lui était resté de la lenteur. Qui ne
croirait qu'un magistrat orné de tant de vertus et de talents, dont la
mémoire, la vaste lecture, l'éloquence à parler et à écrire, la justesse
jusque dans les moindres expressions des conversations les plus
communes, avec les grâces de la facilité, n'eût été le plus grand
chancelier qu'on eût vu depuis plusieurs siècles\,? Il est vrai qu'il
aurait été un premier président sublime, il ne l'est pas moins que,
devenu chancelier, il fit regretter jusqu'aux d'Aligre et aux Boucherat.
Ce paradoxe est difficile à comprendre, il se voit pourtant à l'oeil
depuis trente ans qu'il est chancelier, et avec tant d'évidence que je
pourrais m'en tenir là\,; mais un fait si étrange mérite d'être
développé. Un si heureux assemblage était gâté par divers endroits qui
étaient demeurés cachés dans sa première vie, et qui éclatèrent tout à
la fois sitôt qu'il fut parvenu à la seconde. La longue et unique
nourriture qu'il avait prise dans le sein du parlement l'avait pétri de
ses maximes et de toutes ses prétentions, jusqu'à le regarder avec plus
d'amour, de respect et de vénération que les Anglais n'en ont pour leurs
parlements\,; qui n'ont de commun que le nom avec les nôtres\,; et je ne
dirai pas trop quand j'avancerai qu'il ne regardait pas autrement tout
ce qui émanait de cette compagnie, qu'un fidèle bien instruit de sa
religion regarde les décisions sur la foi des conciles oecuméniques. De
cette sorte de culte naissaient trois extrêmes défauts qui se
rencontraient très fréquemment\,: le premier, qui était toujours, pour
le parlement, quoi qu'il pût entreprendre contre l'autorité royale, ou
d'ailleurs au delà de la sienne, tandis que son office, qui le rendait
le supérieur et le modérateur des parlements et la bouche du roi à leur
égard, l'obligeait à les contenir quand il passait leurs bornes, surtout
à leur imposer avec fermeté, quand ils attentaient à l'autorité du roi.
Son équité et ses lumières lui montraient bien l'égarement du parlement
à chaque fois qu'il s'y jetait, mais de le réprimer était plus fort que
lui. Sa mollesse, secondée de cette sorte de culte dont il l'honorait,
était peinée, affligée de le voir en faute\,; mais de laisser voir qu'il
y fût tombé était un crime à ses yeux, dont il gémissait de voir
souiller les autres, et dont il ne pouvait se souiller lui-même. Il
mettait donc tous ses talents à pallier, à couvrir, à excuser, à donner
des interprétations captieuses à éblouir sur les fautes du parlement, à
négocier avec lui d'une part, avec le régent d'autre, à profiter de sa
timidité, de sa facilité, de sa légèreté pour tout émousser, tout
énerver en lui, en sorte qu'au lieu d'avoir en ce premier magistrat un
ferme soutien de l'autorité royale, et un vrai juge des justices, on en
tirait à peine quelque bégaiement forcé qui affaiblissait encore le peu
à quoi il avait pu se résoudre à peine, et qui donnait courage, force et
hauteur au parlement\,; et si quelquefois il s'est expliqué avec lui en
d'autres termes, ce n'était qu'après un long combat, et toujours bien
plus faiblement qu'il n'était convenu de le faire.

Un second inconvénient était l'extension de ce culte particulier du
parlement à tout ce qui portait robe, je dis jusqu'à des officiers de
bailliages royaux. Tout homme portant robe devait selon lui imposer le
dernier respect, quoi qu'il fît\,; on ne pouvait s'en plaindre qu'avec
la dernière circonspection. Les plaintes n'étaient pas écoutées sans de
longues preuves juridiquement ordonnées\,; avec cela même elles étaient
rejetées avec grand dommage pour le plaignant, si grand qu'il fût, si
elles n'étaient appuyées de la dernière évidence\,; alors cela lui
paraissait bien fâcheux. Il se tournait tout entier à sauver l'honneur
de la robe, comme si la robe en général était déshonorée parce qu'un
fripon en était revêtu pour son argent. Il proposait des compositions,
des accommodements, et si les plaignants étaient d'une certaine espèce,
des désistements pour s'en rapporter à lui\,; enfin il avait recours à
des longueurs ruineuses qui pouvaient équivaler à des dénis de justice,
et toujours l'homme de robe en sortait au meilleur marché, et surtout le
plus blanc qu'il pouvait, et le plus légèrement tancé. Dans cet esprit,
il ne comprenait pas comment on pouvait se porter à casser un arrêt du
parlement. Il employait pour l'éviter tous les mêmes manèges, et ce
n'était qu'après la plus belle défense qu'il souffrait que l'affaire fût
portée au bureau des cassations. Ce bureau, composé par lui comme tous
les autres du conseil, n'ignorait pas son extrême répugnance. On peut
croire qu'il savait la ménager, et qu'il fallait des raisons bien
claires pour les engager à porter la cassation au conseil, qui à son
tour n'avait pas moins de ménagement que le bureau. Si malgré tout cela
l'évidence l'entraînait, le chancelier, qui ne pouvait se résoudre à
prononcer le blasphème de casser, inventa le premier une autre formule,
et prononçait que \emph{l'arrêt serait comme non avenu}, encore
n'était-ce pas sans quelque péroraison de défense, ou de gémissement\,;
or, on voit que cela attaque clairement la justice distributive.

Un autre mal sorti de la même source, c'était un attachement aux formes,
et jusqu'aux plus petites, si littérale, si précise, si servile que
toute autre considération, même de la plus évidente justice,
disparaissait à ses yeux devant la plus petite formalité. Il y était
tellement attaché, comme à l'âme et à la perpétuité des procès qui sont
la source de l'autorité et des biens de la robe, qu'il ne tint pas à lui
qu'il ne les introduisit au conseil des dépêches, où jamais on n'en
avait ouï parler, bien loin de s'y arrêter. L'absurdité était manifeste.
Ce conseil n'est établi que pour juger des différends qui ne peuvent
rouler sur des formes, ou des procès qu'il plaît au roi d'évoquer à sa
personne, et qu'il juge lui tout seul, parce que là ceux qui en sont
n'ont que voix consultative. Il faudrait donc que le roi fût instruit de
la forme comme un procureur, ou qu'il jugeât à l'aveugle sur celle des
gens qui la sauraient. Or ces gens-là l'ignorent comme nous l'ignorions
tous, ou l'ont oubliée comme les secrétaires d'État qui y rapportent, ou
du moins qui y opinent quand il y entre un autre rapporteur, et qui
n'ont ni le temps ni la volonté de les rapprendre. Le chancelier fit en
deux ou trois occasions la tentative d'alléguer les formes au conseil
des dépêches\,; quoique bien avec lui, je l'interrompis autant de fois,
je combattis sa tentative, et à chaque fois elle demeura inutile avec un
grand regret de sa part qu'il montra fort franchement.

Le long usage du parquet lui avait gâté l'esprit. Il était étendu et
lumineux, et orné d'une grande lecture et d'un profond savoir. L'état du
parquet est de ramasser, d'examiner, de peser et de comparer les raisons
des deux et des différentes parties, car il y en a souvent plusieurs au
même procès, et d'étaler cette espèce de bilan, pour m'exprimer ainsi,
avec toutes les grâces et les fleurs de l'éloquence devant les juges,
avec tant d'art et d'exactitude qu'il ne soit rien oublié d'aucune part,
et qu'aucun des nombreux auditeurs ne puisse augurer de quel avis
l'avocat général sera avant qu'il ait commencé à conclure. Quoique le
procureur général, qui ne donne ses conclusions que par écrit, ne soit
pas exposé au même étalage, il est obligé au même examen, à la même
comparaison, au même bilan, dans son cabinet, avant de se déterminer à
conclure. Cette continuelle habitude pendant vingt-quatre années à un
esprit scrupuleux en équité et en formes, fécond en vues, savant en
droit, en arrêts, en différentes coutumes, l'avait formé à une
incertitude dont il ne pouvait sortir, et qui, lorsqu'il n'était point
nécessairement pressé par quelque limite fixe, prolongeait les affaires
à l'infini. Il en souffrait le premier\,; c'était pour lui un
accouchement que se déterminer\,; mais malheur à qui était dans le cas
de l'attendre. S'il était pressé, par exemple, par un conseil de régence
où une affaire se devait juger à jour pris, il flottait errant jusqu'au
moment d'opiner, étant de la meilleure foi jusque-là tantôt d'un avis,
tantôt de ravis contraire, et opinait après, quand son tour arrivait,
comme il lui venait en cet instant. J'en rapporterai en son lieu un
exemple singulier entré mille autres.

Sa lenteur et son irrésolution s'accordaient merveilleusement à ne rien
finir. Un autre défaut y contribuait encore, c'est qu'il était le père
des difficultés. Tant de choses diverses se présentaient à son esprit,
qu'elles l'arrêtaient. Je l'ai dit du duc de Chevreuse, je le répète ici
de ce chancelier\,; il coupait un cheveu en quatre. Aussi étaient-ils
fort amis. Ce n'était pas qu'il n'eût l'esprit fort juste, mais la
moindre difficulté l'embarrassait, et il en cherchait partout avec le
même soin que d'autres en mettent à les lever. Ses meilleurs amis, les
affaires qu'il affectionnait, n'en étaient pas plus exempts que les
autres, et ce goût des difficultés devint une plaie pour tout ce qui
avait à passer par ses mains. La vieille duchesse d'Estrées-Vaubrun, qui
brillait d'esprit et qui était intimement de ses amies, fut un jour
pressée de lui parler pour quelqu'un. Elle s'en défendait par la
connaissance qu'elle avait de ce terrain si raboteux. «\,Mais, madame,
lui dit ce client, il est votre ami intime. --- Il est vrai,
répondit-elle\,; il faut donc vous dire quel est M. le chancelier\,:
c'est un ami travesti en ennemi.\,» La définition était fort juste. À
tant de défauts essentiels, qui pourtant ne venaient pour la plupart que
de trop de lumières et de vues, de trop d'habitude du parquet, de la
nourriture qu'il avait uniquement prise dans le parlement, et qui bien
{[}loin{]} d'attaquer l'honneur et la probité n'étaient grossis que par
la délicatesse de conscience, il s'en joignit d'autres qui ne venaient
que de sa lenteur naturelle et de trop d'attachement à bien faire il ne
pouvait finir à tourner une déclaration, un règlement, une lettre
d'affaires tant soit peu importante. Il les limait et les retouchait
sans cesse. Il était esclave de la plus exacte pureté de diction, et ne
s'apercevait pas que cette servitude le rendait très souvent obscur, et
quelquefois inintelligible. Son goût pour les sciences couronnait tous
ces inconvénients. Il aimait les langues, surtout les savantes, et il se
plaisait infiniment à toutes les parties de la physique et de la
mathématique. Il ne laissait pas encore d'être métaphysicien. Il avait
pour toutes ces sciences beaucoup d'ouverture et de talent\,; il aimait
à les creuser, et à faire chez lui à huis clos des exercices sur ces
différentes sciences avec ses enfants et quelques savants obscurs. Ils y
prenaient des points de recherches pour l'exercice suivant, et cette
sorte d'étude lui faisait perdre un temps infini, et désespérait ceux
qui avaient affaire à lui, qui allaient dix fois chez lui sans pouvoir
le joindre à travers les fonctions de son office et les amusements de
son goût. C'était précisément pour les sciences qu'il était né. Il est
vrai qu'il eût été un excellent premier président, mais à quoi il eût
été le plus propre, c'eût été d'être uniquement à la tête de toute la
littérature, des Académies, de l'observatoire, du Collège royal, de la
librairie, et c'est où il aurait excellé. Sa lenteur sans incommoder
personne, et ses faciles difficultés n'auraient servi qu'à éclaircir les
matières, et son incertitude, indépendante alors de la conscience, n'eût
tendu qu'à la même fin. Il n'aurait eu affaire qu'à des gens de lettres
et point au monde, qu'il ne connut jamais, et dont, à la politesse près,
il n'avait nul usage. Il serait demeuré éloigné du gouvernement et des
matières d'État, où il fut toujours étranger jusqu'à surprendre par une
ineptie si peu compatible avec tant d'esprit et de lumières.

En voilà beaucoup, mais encore un coup de pinceau. Le duc de Grammont
l'aîné, qui avait beaucoup d'esprit, m'a conté que se trouvant un matin
dans le cabinet du roi à Versailles, tandis que le roi était à la messe,
et tête-à-tête avec le chancelier, {[}il{]} lui demanda dans la
conversation si depuis qu'il était chancelier, avec le grand usage qu'il
avait des chicanes et de la longueur des procès, il n'avait jamais pensé
à faire un règlement là-dessus qui les abrégeât et en arrêtât les
friponneries. Le chancelier lui répondit qu'il y avait si bien pensé
qu'il avait commencé à en jeter un règlement sur le papier, mais qu'en
avançant il avait réfléchi au grand nombre d'avocats, de procureurs,
d'huissiers que ce règlement ruinerait, et que la compassion qu'il en
avait eue lui avait fait tomber la plume de la main. Par la même raison
il ne faudrait ni prévôts ni archers qui arrêtent les voleurs, et qui
les mettent en chemin certain du supplice, dont par cette raison la
compassion était encore plus grande. En deux mots, c'est que la durée et
le nombre des procès fait toute la richesse et l'autorité de la robe, et
que par conséquent il les faut laisser pulluler et s'éterniser. Voilà un
long article\,; mais je l'ai cru d'autant plus curieux qu'il fait mieux
connaître comment un homme de tant de droiture, de talents et de
réputation, est peu à peu parvenu, par être sorti de son centre, à
rendre sa droiture équivoque, ses talents pires qu'inutiles, à perdre
toute sa réputation, et à devenir le jouet de la fortune.

\hypertarget{chapitre-ix.}{%
\chapter{CHAPITRE IX.}\label{chapitre-ix.}}

1717

~

{\textsc{Infamie du maréchal d'Huxelles sur le traité avec
l'Angleterre.}} {\textsc{- Embarras et mesures du régent pour apprendre
et faire passer au conseil de régence le traité d'Angleterre.}}
{\textsc{- Singulier entretien, et convention plus singulière, entre M.
le duc d'Orléans et moi.}} {\textsc{- Le traité d'Angleterre porté et
passé au conseil de régence.}} {\textsc{- Étrange malice qu'en opinant
j'y fais au maréchal d'Huxelles.}} {\textsc{- Conseil de régence où la
triple alliance est approuvée.}} {\textsc{- Je m'y oppose en vain à la
proscription des jacobites en France.}} {\textsc{- Brevet de retenue de
quatre cent mille livres au prince de Rohan, et survivance à son fils de
sa charge des gens d'armes.}} {\textsc{- Le roi mis entre les mains des
hommes.}} {\textsc{- Présent de cent quatre-vingt mille livres de
pierreries à la duchesse de Ventadour.}} {\textsc{- Survivance du grand
fauconnier à son fils enfant.}} {\textsc{- Famille, caractère et mort de
la duchesse d'Albret.}} {\textsc{- Survivances de grand chambellan et de
premier gentilhomme de la chambre aux fils, enfants, des ducs de
Bouillon et de La Trémoille, lequel obtient un brevet de retenue de
quatre cent mille livres.}} {\textsc{- Survivance de la charge des
chevau-légers au fils, enfant, du duc de Chaulnes, et une augmentation
de brevet de retenue jusqu'à quatre cent mille livres.}} {\textsc{-
Survivance de la charge de grand louvetier au fils d'Heudicourt.}}
{\textsc{- Survivance inouïe d'aumônier du roi au neveu de l'abbé de
Maulevrier.}} {\textsc{- Étrange grâce pécuniaire au premier
président.}} {\textsc{- Quatre cent mille livres de brevet de retenue à
Maillebois sur sa charge de maître de la garde-robe.}} {\textsc{- Mort
de Callières.}} {\textsc{- Abbé Dubois secrétaire du cabinet du roi avec
la plume.}} {\textsc{- Il procure une visite de M. le duc d'Orléans au
maréchal d'Huxelles.}} {\textsc{- Abbé Dubois entre dans le conseil des
affaires étrangères par une rare mezzo-termine qui finit sa liaison avec
Canillac.}} {\textsc{- Comte de La Marck ambassadeur auprès du roi de
Suède.}} {\textsc{- J'empêche la destruction de Marly.}} {\textsc{-
J'obtiens les grandes entrées.}} {\textsc{- Elles sont après prodiguées,
puis révoquées.}} {\textsc{- Explication des entrées.}}

~

Le traité entre la France et l'Angleterre, signé, comme on l'a dit, à la
Haye, était demeuré secret dans l'espérance d'y faire accéder les
Hollandais\,; mais ce secret, qui commençait à transpirer, ne put être
réservé plus longtemps au seul cabinet du régent. Il fallut bien, avant
qu'il devînt public, en faire part au conseil de régence, et auparavant
au maréchal d'Huxelles, qui devait le signer et en envoyer la
ratification. C'était l'ouvrage de l'abbé Dubois et son premier grand
pas vers la fortune. Il avait tellement craint d'y être traversé qu'il
avait obtenu du régent de n'en faire part à personne\,; mais je n'ai
jamais douté que le duc de Noailles et Canillac, alors ses croupiers,
n'en fussent exceptés. Huxelles, jaloux au point où il l'était des
moindres choses, était outré de voir l'abbé Dubois dans toute la
confiance, et traiter à Hanovre, puis à la Haye, à son insu de tout ce
qu'il s'y passait. Au premier mot que le régent lui dit du traité il le
fut encore davantage, et n'écouta ce qu'il en apprit que pour le
contredire. Le régent, essaya de le persuader\,; il n'en reçut que des
révérences, et {[}Huxelles{]} s'en alla bouder chez lui\,; L'affaire
pressait, et l'abbé Dubois, pour sa décharge, voulait la signature du
chef du conseil des affaires étrangères, {[}à cause{]} du caractère et
du poids que bien ou mal à propos Huxelles avait su s'acquérir dans le
monde. Le régent le manda, l'exhorta, se fonda en raisonnements
politiques. Huxelles silencieux, respectueux, ne répondit que par des
révérences, et forcé enfin de s'expliquer sur sa signature, il supplia
le régent de l'excuser de signer un traité dont il n'avait jamais ouï
parler avant qu'il fût signé à la Haye, et quoi que le régent pût faire
et dire, raisons, caresses, excuses, tout fut inutile, et le maréchal
s'en retourna chez lui.

Effiat lui fut détaché, qui rapporta que, pour toute réponse, le
maréchal lui avait déclaré qu'il se laisserait plutôt couper la main que
de signer. Le régent, pressé par l'intérêt de l'abbé Dubois, et parce
que la nouvelle du traité transpirait de jour en jour, prit une
résolution fort étrange à sa faiblesse accoutumée\,: il envoya d'Antin,
qu'il instruisit du fait, dire au maréchal d'Huxelles de choisir, ou de
signer, ou de perdre sa place, dont le régent disposerait aussitôt en
faveur de quelqu'un qui ne serait pas si farouche que lui. Oh\,! la
grande puissance de l'orviétan\,! cet homme si ferme, ce grand citoyen,
ce courageux ministre qui venait de déclarer deux jours auparavant qu'on
lui couperait plutôt le bras que de signer, n'eut pas plutôt ouï la
menace, et senti qu'elle allait être suivie de l'effet, qu'il baissa la
tête sous son grand chapeau qu'il avait toujours dessus, et signa tout
court sans mot dire. Tout cela avait trop duré pour être ignoré des
principaux de la régence. Le maréchal de Villeroy m'en parla avec dépit.
Il était piqué aussi du secret qui lui avait été fait tout entier\,; et
moi, sans vouloir entrer dans le mécontentement commun avec un homme
aussi mal disposé pour M. le duc d'Orléans, je ne lui cachai point que
j'étais sur ce traité dans la même ignorance. Dubois et les siens me
craignaient sur l'Angleterre. Il avait pris ses précautions contre la
confiance que le régent avait en moi, en sorte qu'alors même, ce prince
ne m'avait point parlé du traité, et que depuis que j'avais su qu'il y
en avait un de signé, je ne lui en avais pas aussi ouvert la bouche.
L'affaire du maréchal d'Huxelles fit du bruit, et lui fit grand tort
dans le monde. Ou il ne fallait pas, aller si loin, ou il fallait avoir
la force d'aller jusqu'au bout, et ne se pas déshonorer en signant à
l'instant de la menace. Cette aventure le démasqua si bien qu'il n'en
est jamais revenu avec le monde. La signature faite, il fut question de
montrer le traité au conseil de régence, et de l'y faire approuver. Pas
un de ceux qui le composaient n'en avait su que ce qu'il en avait appris
par le monde\,; c'est-à-dire qu'il y en avait un. Cela n'était pas
flatteur\,; aussi M. le duc d'Orléans y craignit-il des oppositions et
du bruit. Il passa donc la matinée du jour qu'il devait parler du traité
l'après-dînée au conseil de régence à mander séparément l'un après
l'autre tous ceux qui le composaient, à le leur expliquer, à les
arraisonner, les caresser, s'excuser du secret, en un mot les capter et
s'en assurer.

Je fus mandé comme les autres. Je le trouvai seul dans son cabinet sur
les onze heures. Dès qu'il m'aperçut\,: «\,Au moins, me dit-il en,
souriant avec un peu d'embarras, n'allez pas tantôt nous faire une
pointe sur ce traité d'Angleterre dont on parlera au conseil.\,» Et tout
de suite il me le conta avec toutes les raisons dont il put le
fortifier. Je lui répondis que je savais depuis quelques jours, comme
bien d'autres qui l'avaient, appris par la ville, qu'il y avait un
traité signé avec l'Angleterre\,; qu'il jugeait bien que j'ignorais ce
qu'il contenait, puisqu'il ne m'en avait point parlé\,; que par
conséquent j'étais hors d'état d'approuver et de désapprouver ce qui
m'était inconnu. J'ajoutai que, pour pouvoir l'un ou l'autre avec
connaissance, il faudrait avoir examiné le traité à loisir et les
difficultés qui s'y étalent rencontrées, voir l'étendue des engagements
réciproques, les comparer, examiner encore l'effet du traité par rapport
à d'autres traités, en un mot un travail à tête reposée pour bien peser
et se déterminer dans une opinion\,; que n'ayant rien de tout cela, et
ce qu'il m'en disait ainsi en courant, et au moment qu'il allait être
porté au conseil, n'était pas une instruction dont on pût se
contenter\,; qu'ainsi je ne pouvais rien dire ni pour ni contre, et que
je me contenterais de me rapporter d'une chose qui m'était inconnue, à
son avis, de lui qui en était parfaitement instruit. Ce propos, à ce
qu'il me parut, le soulagea beaucoup. Il m'était arrivé plus d'une fois
de m'opposer fortement à ce qu'il voulait faire passer, en matière
d'État aussi bien qu'en d'autres. Un jour que j'avais disputé sur une
matière d'État qui entraînait chose qu'il voulait faire passer, et que
je l'avais emporté au contraire un matin au conseil de régence, j'allai
l'après-dînée chez lui. Dès qu'il me vit entrer (et il était seul)\,:
«\,Eh\,! avez-vous le diable au corps, me dit-il, de me faire péter en
la main une telle affaire\,? --- Monsieur, lui répondis-je, j'en suis
bien fâché, mais de toutes vos raisons pas une ne valait rien. --- Eh\,!
à qui le dites-vous\,? reprit-il\,; je le savais bien\,; mais devant
tous ces gens-là je ne pouvais pas dire les bonnes,\,» et tout de suite
me les expliqua. «\,Je suis bien fâché, lui dis-je, si j'avais su vos
raisons, je me serais contenté de vos raisonnettes. Une autre fois, ayez
la bonté de me les expliquer auparavant, parce que, quelque attaché que
je vous sois, sitôt que je suis en place assis au conseil, j'y dois ma
voix à Dieu et à l'État, à mon honneur et à ma conscience, c'est-à-dire
à ce que je crois de plus sage, de plus utile, de plus nécessaire en
matières d'État et de gouvernement, ou de plus juste en autres
matières\,; sur quoi ni respect, ni attachement, ni vue d'aucune sorte
ne doit l'emporter. Ainsi, avec tout ce que je vous dois et que je veux
vous rendre plus que personne, ne comptez point que j'opine jamais
autrement que par ce qui me paraîtra. Ainsi, lorsque vous voudrez faire
passer quelque chose de douteux ou de difficile, où vous ne voudrez pas
tout expliquer, ayez la bonté de me dire auparavant le fait et vos
véritables raisons, ou s'il y a trop de longueur et d'explication, de
m'en faire instruire\,; alors, possédant bien la matière, je serai de
ravis que vous désirerez, ou si le mien ne peut s'y ranger, je vous le
dirai franchement. Par l'arrêt même intervenu sur la régence, vous avez
pouvoir d'admettre et d'ôter qui il vous plaira au conseil de régence\,;
à plus forte raison d'en exclure pour une fois ou pour plusieurs\,;
ainsi, quand bien instruit, je ne pourrai me rendre à ce que vous
affectionnerez à faire passer, dites-moi de m'abstenir du conseil le
jour que cette affaire y sera portée, et non seulement je n'en serai
point blessé\,; mais je m'en abstiendrai sous quelque prétexte, en sorte
qu'il ne paroisse point que vous l'ayez désiré. Je ne dirai mot sur
l'affaire à qui du conseil m'en pourra parler, comme moi l'ignorant ou
n'étant pas instruit, et je vous garderai fidèlement le secret.\,» M. le
duc d'Orléans me remercia beaucoup de cette ouverture, me dit que
c'était là parler en honnête homme et en ami, et, puisque je le voulais
bien, qu'il en profiterait. On verra dans la suite qu'en effet il en
profita quelquefois\,; mais pour ce traité il ne le voulut pas faire\,;
il craignit que cela ne parût affecté, et se contenta comme il put de
l'avis que je venais de lui déclarer.

L'après-dînée nous voilà tous au conseil, et tous les yeux sur le
maréchal d'Huxelles, qui avait l'air fort embarrassé et fort honteux. Si
le duc d'Orléans ouvrit la séance par un discours sur la nécessité et
l'utilité du traité, qu'il dit à la fin au maréchal d'Huxelles de lire.
Le grand point entre plusieurs autres, était la signature sans les
Hollandais. Le maréchal lut à voix basse et assez tremblante\,; puis le
régent lui demanda son avis. «\, De l'avis du traité\,;» répondit-il
entre ses dents, en s'inclinant. Chacun dit de même. Quand ce vint à
moi, je dis que, dans l'impossibilité où je me trouvais de prendre un
avis déterminé sur une affaire de cette importance dont j'entendais
parler pour la première fois, je croyais n'avoir point de plus sage
parti à prendre que de m'en rapporter à Son Altesse Royale, et me
tournant tout court au maréchal d'Huxelles que je regardai entre deux
yeux, «\, et aux lumières, ajoutai-je, de M. le maréchal qui est à la
tête des affaires étrangères, et qui sans doute a apporté tous ses soins
et toute sa pénétration à celle-là. Je ne pus me refuser cette malice à
cet étui de sage de la Grèce et de citoyen romain. Chacun me regarda en
baissant incontinent les yeux, et plusieurs ne purent s'empêcher de
sourire, et de m'en parler au sortir du conseil.

J'ai retardé le récit de celui-ci, qui fut tenu du vivant de Voysin qui
y assista, pour n'en faire pas à deux fois de celui qu'on verra bientôt
pour consentir à la triplé alliance, c'est-à-dire lorsque la Hollande
entra enfin en tiers dans celle dont on vient de parler. Dans le
premier, on nous avait bien parlé de la condition de la sortie du
Prétendant d'Avignon pour se retirer en Italie. Cela était dur\,; mais
dès que le parti était pris de s'unir étroitement avec le roi
d'Angleterre, il était difficile qu'il n'exigeât pas cette condition
après ce qui s'était tenté en Écosse, et il ne l'était pas moins de n'y
pas consentir si on voulait établir la confiance. Mais ce qui fut dès
lors promis de plus, et qui nous fut déclaré au conseil de la triple
alliance, roula sur la proscription des ducs d'Ormond et de Marr, et de
tous ceux qui étant jacobites déclarés se tenaient en France ou y
voudraient passer. Le régent s'engageait à faire sortir les premiers de
toutes les terres de la domination de France, et de n'y en souffrir
aucun des seconds. À quelque distance que ce conseil fût tenu de celui
dont on vient de parler, il n'en était qu'une suite prévue et désirée
même dès lors. Le régent n'en prévint personne, parce qu'il n'y
craignait point d'avis contraire. J'y résistai à l'inhumanité de cette
proscription. J'alléguai des raisons d'honneur, de compassion, de
convenance sur une chose qui, ne roulant que sur quelques particuliers
dont le chef et le moteur était bien loin en Italie, ne pouvait nuire à
la tranquillité du roi d'Angleterre, ni lui causer aucune inquiétude. Je
fus suivi de plusieurs, de ceux surtout qui opinaient après moi, et il
n'y avait que le chancelier et les princes légitimés et légitimes\,;
mais plusieurs de ceux qui avaient opiné revinrent à mon avis.

Le régent, dont la parole était engagée là-dessus dès le premier traité
par l'abbé Dubois, parla après nous, loua notre sentiment, regretta de
ne pouvoir le suivre, laissa sentir un engagement pris, fit valoir la
nécessité de ne pas chicaner sur ce qui ne regardait que des
particuliers, et sur le point de terminer heureusement une bonne
affaire, de ne jeter pas inutilement des soupçons dans des esprits
ombrageux si susceptibles d'en rendre. Chacun vit bien ce qui était\,;
on baissa la tête, et la proscription passa avec le reste, dont pour
l'honneur de la couronne, et par mille considérations, j'eus grand mal
au coeur. L'abbé Dubois ne tarda pas à revenir triomphant de ses succès,
et à en venir presser les fruits personnels. Pour flatter le roi
d'Angleterre et se faire un mérite essentiel auprès de lui et de
Stanhope, il avait usé, sur la proscription des jacobites, de la même
adresse qui lui avait si bien réussi à livrer son maître à l'Angleterre.
Quelques jours après ce conseil, je ne pus m'empêcher de reprocher à ce
prince cette proscription comme une inhumanité d'une part, et une
bassesse de l'autre\,; et à lui faire une triste comparaison de
l'éclatante protection que le feu roi avait donnée aux rois légitimes
d'Angleterre jusqu'à la dernière extrémité de ses affaires, dans
laquelle même ses ennemis n'avaient pas osé lui proposer la proscription
à laquelle Son Altesse Royale s'engageait dans un temps de paix et de
tranquillité. À cela il me répondit qu'il y gagnait autant et plus que
le roi d'Angleterre, parce que la condition étant réciproque, il se
mettait par là en assurance que l'Angleterre ne fomenterait point les
cabales et les desseins qui se pouvaient former contre lui dans tous les
temps\,; qu'elle l'avertirait au contraire de tout ce qu'elle en
pourrait découvrir\,; et qu'elle ne protégerait ni ne recevrait aucuns
de ceux qui seraient contre lui. À cette réponse je me tus, parce que je
reconnus l'inutilité de pousser cette matière plus loin, où je n'eus pas
peine à reconnaître l'esprit et l'impression de l'abbé Dubois. Le
Prétendant partit en même temps d'Avignon, fort à regret, pour se
retirer en Italie.

On apprit de Vienne un événement fort bizarre. Le comte de Windisgratz,
président du conseil aulique, et le comte de Schomborn, vice-chancelier
de l'empire et coadjuteur de Bamberg, se battirent en duel. Je n'en ai
su ni la cause ni les suites\,; mais cela parut une aventure fort
étrange pour des gens de leur âge, et dans les premiers postes des
affaires de l'empire et de la cour de l'empereur. Le comte de Konigseck,
après quelque séjour à Bruxelles, arriva à Paris avec le caractère
d'ambassadeur de l'empereur.

M. le duc d'Orléans fit en ce temps-ci plusieurs grâces, de
quelques-unes desquelles il aurait pu se passer, ou {[}à{]} gens fort
inutiles, ou à d'autres qu'elles ne lui gagnèrent pas. Le maréchal de
Matignon avait acheté autrefois du comte de Grammont le gouvernement du
pays d'Aunis, qu'il avait eu à la mort de M. de Navailles, qui avait en
même temps celui de la Rochelle qu'on en sépara alors. Le maréchal de
Matignon en avait obtenu la survivance pour son fils, de M. le duc
d'Orléans. Marcognet, gouverneur de la Rochelle, mourut, qui en avait
dix-huit mille livres d'appointements. Le maréchal de Matignon prétendit
que ce gouvernement devait être rejoint au sien. M. le duc d'Orléans y
consentit, et crut en être quitte à bon marché de réduire à six mille
francs les appointements de dix-huit mille livres qu'avait Marcognet.
Bientôt après il se laissa aller à en donner aussi la survivance au même
fils du maréchal, et finalement d'augmenter le brevet de retenue dû
maréchal de cent mille francs. Il en avait eu un du feu roi de cent
trente mille livres\,; ainsi il fut en tout de deux cent trente mille
livres, qui est tout ce qu'il en avait payé au comte de Grammont.

En finissant de travailler avec le chancelier et les cardinaux de
Noailles et de Rohan, le régent dit au dernier, qui n'y songeait
seulement pas ni son frère non plus, qu'il donnait au prince de Rohan
quatre cent mille livres de brevet de retenue sur son gouvernement de
Champagne, et à son fils la survivance de sa charge de capitaine des
gens d'armes. La vérité est que les deux frères en firent des excuses au
monde, comme honteux de recevoir des grâces du régent à qui ils étaient
tout en douceur, et avaient toujours été diamétralement contraires, ne
le furent pas moins, et tournèrent doucement son bienfait en dérision.

En mettant le roi entre les mains des hommes, M. le duc d'Orléans donna
pour plus de soixante mille écus de pierreries de la succession de feu
Monseigneur à la duchesse de Ventadour, qui n'en fut pas plus touchée de
reconnaissance que les Rohan, et qui ne lui était pas moins opposée,
comme ce prince ne l'ignorait pas ni d'elle ni d'eux. Ces grâces
pouvaient aller de pair avec celles qu'il avait si étrangement
prodiguées à La Feuillade.

Il en fit une au grand fauconnier des Marais, homme obscur qu'on ne
voyait jamais ni lui ni pas un des siens\,; qui ouvrit la porte à tous
les enfants pour les survivances de leur père, en donnant celle du grand
fauconnier à son fils, qui n'avait pas sept ans, sans que personne y eût
seulement pensé pour lui. On ne croirait pas que ce fût par un
raffinement de politique. Noailles, Effiat et Canillac avaient enfilé
les moeurs faciles du régent à la servitude du parlement. L'abbé Robert
était un des plus anciens et un des plus estimés conseillers clercs de
la grand'chambre, et il était frère du défunt père de la femme de des
Marais. Le régent crut par là avoir fait un coup de partie qui lui
dévouerait l'abbé Robert et tout le parlement. Ces trois valets, qui le
trahissaient pour leur compte, le comblèrent d'applaudissements, et il
les aimait beaucoup, tellement que je le vis dans le ravissement de
cette gentillesse, sans avoir pu gagner sur moi la complaisance de
l'approuver. On ne tardera pas à voir si j'eus tort, et comment on se
trouve de jeter les marguerites devant les pourceaux.

En conséquence d'une grâce si bien appliquée, il n'en put refuser deux
pour des enfants à la duchesse d'Albret. Elle était fille du feu duc de
La Trémoille, cousin germain de Madame, qui l'avait toujours traité
comme tel avec beaucoup d'amitié, et Monsieur avec beaucoup de
considération. Sa fille avait passé sa première jeunesse avec
M\textsuperscript{me} la duchesse de Lorraine et avec M. le duc
d'Orléans, qui avaient conservé les mêmes sentiments pour elle. Elle se
mourait d'une longue et cruelle maladie, et c'était la meilleure femme
du monde, la plus naturelle, la plus gaie, la plus vraie, la plus
galante aussi, mais qu'on ne pouvait s'empêcher d'aimer. Elle demanda en
grâce à M. le duc d'Orléans de lui donner la consolation avant de mourir
de voir la survivance de grand chambellan à son fils aîné, et celle de
premier gentilhomme de la chambre de son frère à son neveu. Elle obtint
l'une et l'autre, mais je ne sais par quelle raison la dernière ne fut
déclarée qu'un peu après sa mort, qui suivit de près ces deux grâces. Le
fils de M. de La Trémoille avait neuf ans, et le père eut en même temps
quatre cent mille livres de brevet de retenue.

Après la survivance des gens d'armes, celle des chevau-légers ne pouvait
pas se différer. M. de Chaulnes et tous les siens l'avaient méritée par
le contradictoire de la conduite des Rohan à l'égard de M. le duc
d'Orléans. Ce prince la lui accorda donc pour son fils qui n'avait pas
douze ans, et une augmentation de cent quatre-vingt mille livres à son
brevet de retenue, qui devint par là de quatre cent mille livres.

Le robinet était tourné\,: Heudicourt, vieux, joueur et débauché qui
n'avait jamais eu d'autre existence que sa femme, morte il y avait
longtemps, et qui elle-même n'en avait aucune que par
M\textsuperscript{me} de Maintenon, obtint pour son fils, mauvais
ivrogne, la survivance de sa charge de grand louvetier.

Enfin l'abbé de Maulevrier, dont j'ai quelquefois parlé, imagina une
chose inouïe. On a vu qu'après avoir vieilli aumônier du feu roi, il
avait enfin été nommé à l'évêché d'Autun qu'il avait refusé par son âge.
Il était demeuré aumônier du roi. Il en demanda hardiment la survivance
pour son neveu, et il l'eut aussitôt sans la plus petite difficulté.

Le premier président, qui voulait jouer le grand seigneur par ses
manières et par sa dépense, était un panier percé, toujours affamé.
Encouragé par l'aventure de la survivance du grand fauconnier, tout
valet à tout faire qu'il fût toute sa vie du duc du Maine, au su du
public et en particulier de M. le duc d'Orléans eut l'effronterie de
faire à ce prince la proposition que voici. Le feu roi lui avait donné
un brevet de retenue de cinq cent mille livres, et comme rien n'était
cher de ce qui convenait aux intérêts du duc du Maine, ce cher fils lui
obtint peu après une pension de vingt-cinq mille livres. Ainsi le
premier président, qui par son brevet de retenue avait sa charge à lui
pour le même prix qu'elle lui avait coûté, en eut encore le revenu comme
s'il ne l'avait point payée. La facilité du régent et sa terreur du
parlement firent imaginer au premier président de demander au régent de
lui faire payer les cinq cent mille livres de son brevet de retenue, en
conservant toutefois sa pension, et il l'obtint sur-le-champ. Ainsi il
acheva d'avoir sa charge pour rien, et eut vingt-cinq mille livres de
rente pour avoir la bonté de la faire. M. et M\textsuperscript{me} du
Maine et lui en rirent bien ensemble. Le reste du monde s'indigna de
l'avidité de l'un et de l'excès de la faiblesse de l'autre. Il n'y eut
que les trois affranchis du parlement, Noailles, Canillac et d'Effiat,
qui trouvèrent cette grâce fort bien placée. Il n'y eut pas jusqu'à
Maillebois à qui M. le duc d'Orléans donna un brevet de quatre cent
mille livres sur sa charge de maître de la garde-robe.

Callières mourut, et ce fut dommage. J'ai parlé ailleurs de sa capacité
et de sa probité. Il était secrétaire du cabinet et avait la plume.
L'abbé Dubois, qui voulait dès lors aller à tout, mais qui sentait qu'il
avait besoin d'échelons, voulut cette charge avec la plume, quoique peu
convenable à un conseiller d'État d'Église. Désirer et obtenir fut pour
lui la même chose. Il songea aussi à se fourrer dans le conseil des
affaires étrangères, comme ces plantes qui s'introduisent dans les
murailles et qui enfin les renversent. Il en sentit la difficulté par la
jalousie et le dépit qu'en aurait le maréchal d'Huxelles, et par
l'embarras de ceux de ce conseil avec lui, depuis cette belle prétention
de conseillers d'État si bien soutenue. Il n'était pas encore en état de
montrer les dents. Pour faire sa cour au maréchal d'Huxelles, qui de
honte boudait et ne sortait de chez lui que pour le conseil depuis son
aventure du traité d'Angleterre, Dubois fit entendre à son maître
qu'ayant fait faire au maréchal ce qu'il voulait, il ne fallait pas
prendre garde à la mauvaise grâce ni à la bouderie\,; que c'était un
vieux seigneur qui avait encore sa considération\,; qu'il se disait
malade\,; qu'il était bon d'adoucir l'amertume d'un homme qui était à la
tête des affaires étrangères, et dont on avait besoin, parce qu'on ne
pouvait pas toujours lui cacher tout\,; et que ce serait une chose fort
approuvée dans le monde, et qui aurait sûrement un grand effet sur le
maréchal, s'il voulait bien prendre la peine de l'aller voir. Il n'en
fallut pas davantage à la facilité du régent pour l'y déterminer. Il
alla donc chez le maréchal d'Huxelles, et comme la visite n'avait pour
but que de lui passer la main sur le dos, en quoi M. le duc d'Orléans
était grand maître, il l'exécuta fort bien, et le maréchal, assez
sottement glorieux pour être fort touché de cet honneur, se reprit à
faire le gros dos. Après ce préambule l'abbé Dubois fut déclaré du
conseil des affaires étrangères.

Il alla incontinent chez tous ceux qui en étaient leur protester qu'il
n'avait aucune prétention de préséance. Pour cette fois, il disait vrai.
Il ne voulait qu'entrer en ce conseil, sans encourir leur mal grâce,
pour les rares et modernes prétentions de gens dont il ne comptait pas
de demeurer le confrère. Mais ils s'alarmèrent. Les
\emph{Mezzo-termine}, si favoris du régent, furent cherchés pour
accommoder tout le monde. Il offrit à l'abbé d'Estrées, à Cheverny et à
Canillac des brevets antidatés, qui les feraient conseillers d'État
avant l'abbé Dubois, moyennant quoi ils le précéderaient sans que les
conseillers d'État pussent s'en plaindre. Cela était formellement
contraire au règlement du conseil de 1664, qu'on a toujours suivi
depuis, qui fixe le nombre des conseillers d'État à trente\,; savoir\,:
trois d'Église, trois d'épée, et vingt-quatre de robe. Ce nombre alors
se trouvait rempli. Les conseillers d'État ne s'accommodaient point de
cette supercherie, ils voulaient une préséance nette. Ces trois
seigneurs du conseil des affaires étrangères trouvaient encore plus
mauvais de ne précéder l'abbé Dubois que par un tour d'adresse.
Néanmoins il leur en fallut à tous passer par là, et Canillac reçut le
los, qu'il avait mérité dès la mort du roi, de l'avoir emporté avec le
duc de Noailles sur moi pour la robe, comme je l'ai raconté dans son
temps, quand on fit les conseils.

Ce qu'il y eut d'admirable pendant le cours de cette belle négociation,
qui dura plusieurs jours, fut que les gens de qualité, à qui la cabale
de M. et de M\textsuperscript{me} du Maine avait eu soin avec tant
d'art, toujours entretenu, de faire prendre les ducs en grippe, se
montrèrent, en cette occasion, qui les touchait si directement, les très
humbles serviteurs de la robe, tant ils montrèrent de sens, de jugement
et de sentiment. La jalousie du grand nombre qui ne pouvait pas trouver
place dans les conseils se reput avec un plaisir malin de la
mortification des trois du conseil des affaires étrangères, sans faire
aucun retour sur eux-mêmes. Je ne dissimulerai pas que j'en pris un peu
aussi de voir cette bombe tomber à plomb sur Canillac, par la raison que
je viens d'en dire. Il en fut outré plus que pas un des deux autres, et
au point que ce fut l'époque du refroidissement entre lui et l'abbé
Dubois, qui bientôt après vola assez de ses ailes pour se passer du
concours de Canillac, à qui la jalousie, jointe à ce premier
refroidissement, en prit si forte qu'elle le conduisit à une brouillerie
ouverte avec l'abbé Dubois, qui, à la fin, comme on le verra en son
temps, lui rompit le cou et le fit chasser. C'est peut-être le seul bien
qu'il ait fait en sa vie.

Le comte de La Marck fut nommé en ce temps-ci ambassadeur auprès du roi
de Suède, et ce fut un très bon choix. C'est le même dont j'ai parlé
plus d'une fois, et qui bien longtemps après a été ambassadeur en
Espagne, et y a été fait grand d'Espagne et chevalier de la Toison d'or.
Il était chevalier du Saint-Esprit en 1724.

Je me souviens d'avoir oublié chose qui mérite qu'on s'en souvienne pour
la singularité du fait, et que je vais rétablir de peur qu'elle ne
m'échappe encore. Une après-dînée, comme nous allions nous asseoir en
place au conseil de régence, le maréchal de Villars me tira à part, et
me demanda si je savais qu'on allait détruire Marly. Je lui dis que non,
et en effet je n'en avais pas ouï parler, et j'ajoutai que je ne pouvais
le croire. «\,Vous ne l'approuvez donc pas,\,» reprit le maréchal. Je
l'assurai que j'en étais fort éloigné. Il me réitéra que la destruction
était résolue, qu'il le savait à n'en pouvoir douter et que, si je la
voulais empêcher, je n'avais pas un moment à perdre. Je répondis,
{[}lors{]} qu'on se mettait en place, que j'en parlerais incessamment à
M. le duc d'Orléans. «\,Incessamment, reprit vivement le maréchal,
parlez-lui-en dans cet instant même, car l'ordre en est peut-être déjà
donné. »

Comme tout le conseil était déjà assis en place, j'allai par derrière à
M. le duc d'Orléans, à qui je dis à l'oreille ce que je venais
d'apprendre, sans nommer de qui\,; que je le suppliais, au cas que cela
fût, de suspendre jusqu'à ce que je lui eusse parlé, et que j'irais le
trouver au Palais-Royal après le conseil. Il balbutia un peu, comme
fâché d'être découvert, et convint pourtant de m'attendre. Je le dis en
sortant au maréchal de Villars, et je m'en allai au Palais-Royal, où M.
le duc d'Orléans ne disconvint point de la chose. Je lui dis que je ne
lui demanderais point qui lui avait donné un si pernicieux conseil. Il
voulut me le prouver bon par l'épargne de l'entretien, le produit de
tant de conduites d'eau, de matériaux et d'autres choses qui se
vendraient, et le désagrément de la situation d'un lieu où le roi
n'était pas en âge d'aller de plusieurs années, et qui avait tant
d'autres belles maisons à entretenir avec une si grande dépense, dont
aucune ne pouvait être susceptible de destruction. Je lui répondis qu'on
lui avait présenté là des raisons de tuteur d'un particulier, dont la
conduite né pouvait ressembler en rien à celle d'un tuteur d'un roi de
France\,; qu'il fallait avouer la nécessité de la dépense de l'entretien
de Marly, mais convenir en même temps que sur celles du roi c'était un
point dans la carte, et s'ôter en même temps de la tête le profit des
matériaux, qui se dissiperait en dons et en pillage\,; mais que ce
n'était pas ces petits objets qu'il devait regarder, mais considérer
combien de millions avaient été jetés dans cet ancien cloaque pour en
faire un palais de fées, unique en toute l'Europe en sa forme, unique
encore par la beauté de ses fontaines, unique aussi par la réputation
que celle du feu roi lui avait donnée\,; que c'était un des objets de la
curiosité de tous les étrangers de toutes qualités qui venaient en
France\,; que cette destruction retentirait par toute l'Europe avec un
blâme que ces basses raisons de petite épargne ne changeraient pas\,;
que toute la France serait indignée de se voir enlever un ornement si
distingué\,; qu'encore que lui ni moi pussions n'être pas délicats sur
ce qui avait été le goût, et l'ouvrage favori du feu roi, il devait
éviter de choquer sa mémoire, qui par un si long règne, tant de
brillantes années, de si grands revers héroïquement soutenus, et
l'inespérable fortune d'en être si heureusement sorti, avait laissé le
monde entier dans la vénération de sa personne\,; enfin qu'il devait
compter que tous les mécontents, tous les neutres même, feraient groupe
avec l'ancienne cour pour crier au meurtre\,; que le duc du Maine,
M\textsuperscript{me} de Ventadour, le maréchal de Villeroy ne
s'épargneraient pas de lui en faire un crime auprès du roi, qu'ils
sauraient entretenir pendant la régence, et bien d'autres avec eux lui
inspirer de le relever contre lui quand elle serait finie. Je vis
clairement qu'il n'avait pas fait la plus légère réflexion à rien de
tout cela. Il convint que j'avais raison me promit qu'il ne serait point
touché à Marly, et qu'il continuerait à le faire entretenir, et me
remercia de l'avoir préservé de cette faute. Quand je m'en fus bien
assuré\,: «\,Avouez, lui dis-je, que le roi en l'autre monde serait bien
étonné s'il pouvait savoir que le duc de Noailles vous avait fait
ordonner la destruction de Marly, et que c'est moi qui vous en ai
empêché. --- Oh\,! pour celui-là, répondit-il vivement, il est vrai
qu'il ne le pourrait pas croire.\,» En effet, Marly fut conservé et
entretenu\,; et c'est le cardinal Fleury qui, par avarice de procureur
de collège\,; l'a dépouillé de sa rivière, qui en était le plus superbe
agrément.

Je me hâtai de donner cette bonne nouvelle au maréchal de Villars. Le
duc de Noailles qui, outre l'épargne de l'entretien et les matériaux
dont il serait à peu près demeuré le maître, était bien aise de faire
cette niche à d'Antin, qui avait osé défendre son conseil du dedans du
royaume de ses diverses entreprises, fut outré de se voir arraché
celle-ci. Pour n'en avoir pas le démenti complet, il obtint au moins, et
bien secrètement de peur d'y échouer encore, que, tous les meubles,
linges, etc., seraient vendus. Il persuada au régent, embarrassé avec
lui de la rétractation de la destruction de Marly, que tout cela serait
gâté et perdu quand le roi serait en âge d'aller à Marly, qu'en le
vendant, on tirerait fort gros et un soulagement présent\,; et que dans
la suite le roi le meublerait à son gré. Il y avait quelques beaux
meubles, mais comme tous les logements et tous les lits des courtisans,
officiers, grands et petits, garde-robes, etc., étaient meublés des
meubles, draps, linges, etc., du roi, c'était une immensité, dont la
vente fut médiocre par la faveur et le pillage, et dont le remplacement
a coûté depuis des millions. Je ne le sus qu'après que la vente fut
commencée, dont acheta qui voulut à très bas prix\,; ainsi je ne pus
empêcher cette très dommageable vilenie.

Parmi une telle prodigalité de grâces, je crus en pouvoir demander une,
qui durant le dernier règne avait {[}été{]} si rare et si utile, et par
conséquent si chère ce fut les grandes entrées chez le roi, et je les
obtins aussitôt. Puisque l'occasion s'en offre, il est bon d'expliquer
ce que sont les différentes sortes d'entrées, ce qu'elles étaient du
temps du feu roi, et ce qu'elles sont devenues depuis. Les plus
précieuses sont les grandes, c'est-à-dire d'entrer de droit dans tous
les lieux retirés des appartements du roi, et à toutes les heures où le
grand chambellan et les premiers gentilshommes de la chambre entrent.
J'en ai fait remarquer ailleurs l'importance sous un roi qui accordait
si malaisément des audiences, et qui étaient toujours remarquées, à qui,
avec ces entrées, on parlait tête-à-tête, toutes les fois qu'on le
voulait, sans le lui demander, et sans que cela fût su de tout le
monde\,; sans compter la familiarité que procurait avec lui la liberté
de le voir en ces heures particulières. Mais elles étaient réglées par
l'usage\,; et elles ne permettaient point d'entrer à d'autres heures
qu'en celles qui étaient destinées pour elles. Depuis que je suis arrivé
à la cour jusqu'à la mort du roi, je ne les ai vues qu'à M. de Lauzun, à
qui le roi les rendit lorsqu'il amena la reine d'Angleterre et qu'il lui
permit de revenir à la cour, et à M. de La Feuillade le père. Les
maréchaux de Boufflers et de Villars les eurent longtemps après, par les
occasions qui ont été ici marquées en leur temps. C'étaient les seuls
qui les eussent par eux-mêmes. Les charges qui les donnent sont grand
chambellan, premier gentilhomme de la chambre, grand maître de la
garde-robe, et le maître de la garde-robe en année\,; les enfants du
roi, légitimes et bâtards, et les maris et les fils de ses bâtardes.
Pour Monsieur et M. le duc d'Orléans, ils ont eu de tout temps ces
entrées, et comme les fils de France, de pouvoir entrer et voir le roi à
toute heure, mais ils n'en abusaient pas. Le duc du Maine et le comte de
Toulouse avaient le même privilège, dont ils usaient sans cesse, mais
c'était par les derrières.

Les secondes entrées, qu'on appelait simplement les entrées, étaient
purement personnelles\,; nulle charge ne les donnait, sinon celle de
maître de la garde-robe à celui des deux qui n'était point d'année. Le
maréchal de Villeroy les avait parce que son père avait été gouverneur
du roi\,; Beringhen, premier écuyer\,; le duc de Béthune, par
l'occasion, que j'en ai rapportée ailleurs. De petites charges les
donnaient aussi, qui, n'étant que pour des gens du commun, en faisaient
prendre à de plus distingués pour profiter de ces entrées, et ces
charges sont les quatre secrétaires du cabinet restées dans le commun,
et les deux lecteurs du roi. Dangeau et l'abbé son frère avaient acheté,
puis revendu quelque temps après une charge de lecteur et en avaient
conservé les entrées. Celles-là étaient appelées au lever longtemps
après, les grandes, quelque temps avant les autres, mais au coucher
elles ne sortaient qu'avec les grandes, d'ailleurs fort inférieures aux
grandes dans toute la journée, mais fort commodes aussi les soirs quand
on voulait parler au roi. On a vu dans son lieu quel parti le duc de
Béthune en tira, et que sans ce secours il n'aurait jamais été duc et
pair. M. le Prince eut ces entrées-là au mariage de M. le Duc avec
M\textsuperscript{me} la Duchesse fille du roi.

Les dernières entrées sont celles qu'on appelle de la chambre\,; toutes
les charges chez le roi les donnent. Le comte d'Auvergne les avait\,; je
n'en ai point vu d'autres\,; on ne s'avisait guère de les désirer. Elles
étaient appelées au lever un moment avant les courtisans distingués\,;
d'ailleurs nul privilège que le botter du roi. On appelait ainsi
lorsqu'il changeait d'habit en allant ou en revenant de la chasse ou de
se promener\,; et à Marly tout ce qui était du voyage y entrait sans
demander. Ailleurs, qui n'avait point d'entrées en était exclus. Le
premier gentilhomme de la chambre avait droit, et en usait toujours, d'y
faire entrer quatre ou cinq personnes au plus à la fois, à qui il le
disait, ou qui le lui faisaient demander par l'huissier, pourvu que ce
fût gens de qualité ou de quelque distinction. Enfin les entrées du
cabinet étaient le droit d'y attendre le roi, quand il y entrait après
son lever, jusqu'à ce qu'il y eût donné l'ordre pour ce qu'il voulait
faire dans la journée, et de lui faire là sa cour, et quand il revenait
de dehors, où il ne faisait qu'y passer pour aller changer d'habit\,;
hors cela ces entrées-là n'y entraient point. Les cardinaux et les
princes du sang avaient les entrées de la chambre et celles du cabinet,
et toutes les charges en chef. Je ne parle point des petites de service
nécessaire qui avaient ces différentes entrées, dont le long et ennuyeux
détail ne donnerait aucune connaissance de la cour. Outre ces entrées il
y en avait deux autres, auxquelles pas un de ceux qui par charge ou
personnellement avaient celles dont on vient de parler, n'était admis\,:
c'était les entrées de derrière, et les grandes entrées du cabinet. Je
n'ai vu personne les avoir que le duc du Maine et le comte de Toulouse,
qui avaient aussi toutes les autres, et MM. de Montchevreuil et d'O,
pour avoir été leurs gouverneurs, qui les avaient conservées\,; Mansart,
et après lui M. d'Antin, par la charge des bâtiments. Ces quatre-là
entraient quand ils voulaient dans les cabinets du roi par les
derrières, les matins, les après-dînées quand le roi ne travaillait pas,
et c'était la plus grande familiarité de toutes et la plus continuelle,
et dont ils usaient journellement\,; mais jamais en aucun lieu où le roi
habitât ils n'entraient que par les derrières, et n'avaient aucune des
autres entrées dont j'ai parlé, auparavant, sinon que ceux qui avaient
celles du cabinet les y trouvaient, parce que en entrant par derrière
ils y pouvaient être en tout temps, sans pouvoir aussi sortir que par
derrière. Avec ces entrées ils se passaient aisément de toutes les
autres. Les grandes entrées du cabinet n'avaient d'usage que depuis que
le roi sortait de souper jusqu'à ce qu'il sortit de son cabinet pour
s'aller déshabiller et se coucher. Ce particulier ne durait pas une
heure. Le roi et les princesses étaient assis, elles toutes sur des
tabourets, lui dans son fauteuil\,; Monsieur y en prenait un
familièrement aussi, parce que c'était dans le dernier particulier.
M\textsuperscript{me} la dauphine de Bavière n'y a jamais été admise, et
on a vu en son lieu que Madame ne l'y a été qu'à la mort de
M\textsuperscript{me} la dauphine de Savoie. Il n'y avait là que les
fils de France debout, même Monseigneur et les bâtards et bâtardes du
roi, et les enfants et gendres des bâtardes\,; MM. de Montchevreuil et
d'O, et des moments quelques-uns des premiers valets de chambre, et
rarement Fagon quelques instants. Chamarande avait cette entrée comme
ayant été premier valet de chambre du roi, en survivance de son père
dont il avait conservé toutes les entrées. Aussi, quoique lieutenant
général fort distingué, et fort aimé et considéré dans le monde, qu'il y
eût un temps infini que son père avait vendu sa charge dont lui n'avait
été que survivancier, et qu'il eût été premier maître d'hôtel de
M\textsuperscript{me} la dauphine de Bavière, il ne pût jamais aller à
Meudon, parce que en ces voyages ceux qui en étaient avaient l'honneur
de manger avec Monseigneur\,; mais quelquefois il était de ceux de
Marly, parce que le roi n'y mangeait qu'avec les dames. Pour revenir au
cabinet des soirs, les dames d'honneur des princesses qui étaient avec
le roi, ou la dame d'atours de celles qui en avaient, et les dames du
palais de jour de M\textsuperscript{me} la dauphine de Savoie se
tenaient dans le premier cabinet, où elles voyaient passer le roi dans
l'autre et repasser pour s'aller coucher. La porte d'un cabinet à
l'autre demeurait ouverte, et ces dames s'asseyaient entre elles comme
elles voulaient, sur des tabourets hors de l'enfilade. Il n'y avait que
les princes et les princesses qui avaient soupé avec le roi, et leurs
dames, qui entrassent par la chambre\,; tous les autres entraient par
derrière où par la porte de glaces de la galerie. À Fontainebleau
seulement, où il n'y avait qu'un grand cabinet, les dames des princesses
étaient dans la même pièce qu'elles avec le roi\,; celles qui étaient
duchesses, et la maréchale d'Estrées depuis qu'elle fut grande
d'Espagne, étaient assises en rang, joignant la dernière princesse.
Toutes les autres, et la maréchale de Rochefort aussi, dame d'honneur de
M\textsuperscript{me} la duchesse d'Orléans, étaient debout, quelquefois
assises à terre, dont elles avaient la liberté, et la maréchale comme
elles, à qui on ne donnait point là de carreau pour s'asseoir, comme les
femmes des maréchaux de France non ducs en ont chez la reine, où
pourtant, je ne sais pourquoi, elles aiment mieux demeurer debout. Ce
n'est qu'aux audiences et aux toilettes qu'elles en peuvent avoir,
jamais à la chapelle\,; au dîner et, au souper, toujours debout\,; et
elles y vont sans difficulté.

Je fus le premier qui obtins les grandes entrées. D'Antin, qui n'avait
plus l'usage, des siennes, les demanda après comme en dédommagement, et
les eut. Bientôt après, sur cet exemple et par même raison, elles furent
accordées à d'O. On les donna aussi à M. le prince de Conti, seul prince
du sang qui ne les eût pas, parce qu'il était le seul prince du sang qui
ne sortit point de M\textsuperscript{me} de Montespan. Cheverny et
Gamaches, qui les avaient chez le Dauphin père du roi, dont ils étaient
menins avant qu'il fût Dauphin, les eurent aussi\,; et peu à peu la
prostitution s'y mit, comme on vient de la voir aux survivances et aux
brevets de retenue. On verra dans la suite que l'abbé Dubois, devenu
cardinal et premier ministre, profita de cet abus pour en faire
rapporter les brevets à tous ceux qui en avaient. Il n'en excepta que le
duc de Berwick pour les grandes, et Belle-Ile pour les premières, qui ne
les avaient eues que bien depuis. Il s'était alors trop tyranniquement
rendu le maître de M. le duc d'Orléans pour que je ne les perdisse pas
avec tous les autres de ce règne-ci les entrées par derrière ont
disparu\,; et les soirées du roi, qui se passent autrement que celles du
feu roi, n'ont plus donné lieu à ces grandes entrées du cabinet des
soirs. Les autres ont subsisté dans leur forme ordinaire. Je parlerais
ici de ces justaucorps à brevet, que peu à peu M. le duc d'Orléans donna
à qui en voulut, sans s'arrêter au nombre, et les fit par là tomber tout
à fait, si je ne les avais ici expliqués ailleurs.

\hypertarget{chapitre-x.}{%
\chapter{CHAPITRE X.}\label{chapitre-x.}}

1717

~

{\textsc{Mariage de Mortagne avec M\textsuperscript{lle} de Guéméné.}}
{\textsc{- Mariage du duc d'Olonne avec la fille unique de Vertilly.}}
{\textsc{- Mariage de Seignelay avec M\textsuperscript{lle} de
Walsassine.}} {\textsc{- Princes du sang pressent vivement leur
jugement, que les bâtards tâchent de différer.}} {\textsc{- Requête des
pairs au roi à fin de réduire les bâtards à leur rang de pairs et
d'ancienneté entre eux.}} {\textsc{- Grand prieur assiste en prince du
sang aux cérémonies du jeudi et vendredi saints chez le roi.}}
{\textsc{- Plusieurs jeunes gens vont voir la guerre en Hongrie.}}
{\textsc{- M. le prince de Conti, gouverneur du Poitou, entre au conseil
de régence et en celui de la guerre.}} {\textsc{- M. le Duc prétend que,
lorsque le conseil de guerre ne se tient pas au Louvre, il se doit tenir
chez lui, non chez le maréchal de Villeroy.}} {\textsc{- Il est condamné
par le régent.}} {\textsc{- Pelletier-Sousy entre au conseil de régence
et y prend la dernière place.}} {\textsc{- M\textsuperscript{me} de
Maintenon malade fort à petit bruit.}} {\textsc{- Mort, fortune et
caractère d'Albergotti.}} {\textsc{- Sa dépouille.}} {\textsc{- Fin et
effets de la chambre de justice.}} {\textsc{- Triple alliance signée à
la Haye, qui déplaît fort à l'empereur, qui refuse d'y entrer.}}
{\textsc{- Mouvements de Beretti pour empêcher un traité entre l'Espagne
et la Hollande.}} {\textsc{- Conversation importante chez Duywenworde,
puis avec Stanhope.}} {\textsc{- Mesures de Beretti contre l'union de la
Hollande avec l'empereur, et pour celle de la république avec
l'Espagne.}} {\textsc{- Motifs du traité de l'Angleterre avec la France,
et du désir de l'empereur de la paix du Nord.}} {\textsc{- Divisions en
Angleterre et blâme du traité avec la France.}} {\textsc{- Menées et
mesures des ministres suédois et des jacobites.}} {\textsc{- Méchanceté
de Bentivoglio à l'égard de la France et du régent.}} {\textsc{-
Étranges pensées prises à Rome de la triple alliance.}} {\textsc{-
Instruction et pouvoir d'Aldovrandi retournant de Rome en Espagne.}}
{\textsc{- Manèges d'Albéroni pour avancer sa promotion.}} {\textsc{-
Son pouvoir sans bornes\,; dépit et jalousie des Espagnols.}} {\textsc{-
Misères de Giudice.}} {\textsc{- Vanteries d'Albéroni.}} {\textsc{- Il
fait de grands changements en Espagne.}} {\textsc{- Politique et mesures
entre le duc d'Albe et Albéroni.}} {\textsc{- Caractère de Landi, envoyé
de Parme à Paris.}} {\textsc{- Vives mesures d'Albéroni pour détourner
les Hollandais de traiter avec l'empereur, et les amener à traiter avec
le roi d'Espagne à Madrid.}} {\textsc{- Artificieuses impostures
d'Albéroni sur la France.}} {\textsc{- Il se rend seul maître de toutes
les affaires en Espagne.}} {\textsc{- Fortune de Grimaldo.}} {\textsc{-
Giudice s'en va enfin à Rome.}} {\textsc{- Mesures d'Albéroni avec
Rome.}} {\textsc{- Étranges impressions prises à Rome sur la triple
alliance.}} {\textsc{- Conférence d'Aldovrandi avec le duc de Parme à
Plaisance.}} {\textsc{- Hauteur, à son égard, de la reine d'Espagne.}}
{\textsc{- L'Angleterre, alarmée des bruits d'un traité négocié par le
pape entre l'empereur et l'Espagne, fait là-dessous des propositions à
Albéroni.}} {\textsc{- Sa réponse à Stanhope.}} {\textsc{- Son
dessein.}} {\textsc{- Son artifice auprès du roi d'Espagne pour se
rendre seul maître de toute négociation.}} {\textsc{- Fort propos du roi
d'Espagne à l'ambassadeur de Hollande sur les traités avec lui et
l'empereur.}}

~

Mortagne, chevalier d'honneur de Madame, dont j'ai parlé quelquefois,
avait une espèce de maison de campagne dans le fond du faubourg
Saint-Antoine, ou il demeurait le plus qu'il pouvait. M. de Guéméné, qui
n'aimait point à marier ses soeurs ni ses filles, et qui ne se
corrigeait point par l'exemple de ses soeurs qui s'étaient enfin mariées
sans lui, avait une de ses filles dans un couvent tout voisin de la
maison de Mortagne, lequel avait fait connaissance avec elle, et pris
grande pitié de ses ennuis et de la voir manquer de tout. Il y suppléa
par des présents, et l'amitié s'y mit de façon qu'ils eurent envie de
s'épouser. Les Rohan jetèrent les hauts cris, car Mortagne, qui était un
très galant homme, et qui avait servi avec distinction, s'appelait
Collin, et n'était rien du tout du pays de Liège, comme on l'a dit ici
en son lieu. Mortagne ne s'en offensa point. Il leur fit dire que ce
n'était que par compassion du misérable état de cette fille qui manquait
de tout, qui se désespérait d'ennui et de misère, et qui avait
trente-cinq ans, qu'il la voulait épouser\,; qu'il leur donnait un an
pour la pourvoir\,; mais que s'ils ne la mariaient dans l'année, il
l'épouserait aussitôt après. Ils ne la marièrent point. Ils comptèrent
empêcher que Mortagne l'épousât\,; il se moqua d'eux. La fille fit des
sommations respectueuses, et ils se marièrent publiquement dans toutes
les règles. Ils ont très bien vécu ensemble, car il était fort honnête
homme, et sa femme se crut en paradis. Il en vint une fille, que le fils
aîné de Montboissier, capitaine des mousquetaires noirs après Canillac,
son cousin, a épousée.

Le duc d'Olonne épousa aussi la fille unique de Vertilly, maréchal de
camp, qui avait été major de la gendarmerie, fort honnête homme et
officier de distinction, frère cadet d'Harlus, qui avait été deux
campagnes de suite brigadier de la brigade où était mon régiment,
desquels j'ai parlé dans les temps. Cette fille était riche. C'étaient
de bons gentilshommes de Champagne.

Seignelay, troisième fils de M. de Seignelay, ministre et secrétaire
d'État, mort dès 1690, quitta le petit collet et se maria à la fille de
Walsassine, officier général de la maison d'Autriche dans les Pays-Bas.
Il la perdit bientôt après n'en ayant qu'une fille, que Jonsac, fils
aîné de celui dont on a vu le combat avec Villette, a épousée. Seignelay
se remaria à une fille de Biron avant la fortune de ce dernier.

Tout s'aigrissait de plus en plus entre les princes du sang et les
bâtards. Les premiers voulaient un jugement, et en pressaient le régent
tous les jours\,; les bâtards ne cherchaient qu'à gagner du temps. Les
pairs, tout déplorables qu'ils fussent par leur conduite, s'étaient déjà
engagés, comme on l'a vu, à se soutenir contre les entreprises sans
nombre et sans exemple qu'ils en avaient essuyés sous le poids du
dernier règne. Je vis le régent fort peiné de l'empressement journalier
des princes du sang, et en même temps fort embarrassé à s'en défendre.
Nous ne crûmes donc pas devoir différer de présenter au roi une requête
précise, et sa copie au régent, dont le tissu était mesuré en termes,
mais très fort sur la chose, et dont voici les conclusions\,: «\,A ces
causes, sire, plaise à Votre Majesté en révoquant et annulant l'édit du
mois de juillet 1714, et la déclaration du 5 mai 1694, en tout son
contenu, ensemble l'édit du mois de mai en 1711, en ce qu'il attribue à
MM. le duc du Maine et comte de Toulouse et à leurs descendants mâles le
droit de représenter les anciens pairs aux sacres des rois, à
l'exclusion des autres pairs de France, et qui leur permet de prêter
serment au parlement à l'âge de vingt ans.\,» C'est-à-dire demander
précisément qu'ils fussent réduits en tout et partout au rang des autres
pairs de France, et parmi eux à celui de leur ancienneté d'érection et
de leur première réception au parlement. Après qu'elle eut été rédigée,
examinée et approuvée, elle fut signée dans une assemblée générale que
nous tînmes chez l'évêque duc de Laon, en l'absence de M. de Reims, qui
la signa comme d'autres absents par procuration expresse. Sitôt qu'elle
fut signée, MM. de Laon et de Châlons, avec six pairs laïques, allèrent
la présenter au roi, auprès duquel le maréchal de Villeroy les
introduisit en arrivant\,; et le roi prit civilement la requête des
mains de M. de Laon, qui en deux mots lui dit de quoi il s'agissait. Il
ne répondit rien, car il ne répondit jamais aux princes du sang ni aux
bâtards en recevant leurs requêtes. En même temps que ces huit pairs
partirent pour se rendre aux Tuileries, l'évêque duc de Langres et les
ducs de La Force, de Noailles et de Chaumes s'en allèrent au
Palais-Royal, où M. le duc d'Orléans les attendait\,; et les fit entrer
en arrivant dans son cabinet, où il les reçut avec ses grâces
accoutumées et peu concluantes. Peu de faux frères osèrent se montrer
tels en cette occasion. Le duc de Rohan, jamais d'accord avec personne
ni avec lui-même, en fut un. Les ducs d'Estrées et Mazarin étaient des
excréments de la nature humaine, à qui le reste des hommes ne daignait
parler. Estrées ne parut jamais parmi nous\,; Mazarin fut mis par les
épaules, littéralement, dehors dans une de nos assemblées chez M. de
Laon, et depuis cette ignominie sans exemple qu'il mérita tout entière,
il n'osa plus s'y présenter. D'Antin se trouvait dans une situation
unique, qui engagea à la considération de ne lui en point parler. Le
prince de Rohan devait trop aux amours de Louis XIV, et avait trop
d'intérêt au désordre, à l'usurpation, à l'interversion de tout ordre,
de toute règle, de tout droit pour pouvoir demander à faire rendre
justice et à faire compter raison et vertu. Le duc d'Aumont s'était si
pleinement déshonoré par sa conduite dans l'affaire du bonnet, et si à
découvert dans la conférence de Sceaux, comme on l'a vu dans son lieu,
que presque aucun de nous ne lui parlait, et qu'il lui coûta peu de
mettre, en ne signant point, la dernière évidence aux infamies qu'il
avait dès lors découvertes.

Je ne sais dans quel esprit M. le duc d'Orléans permit une chose fort
étrange qui, dans les vives circonstances où on en était sur les
querelles de rang et les requêtes au roi là-dessus, n'était bonne qu'à
les échauffer de plus en plus, et à tenter les princes du sang de
quelque parti violent. À la connaissance que j'avais de M. le duc
d'Orléans, de son humble et respectueuse déférence pour l'audace et les
vices effrénés du grand prieur, il ne put lui résister, et pour
s'excuser à soi-même, il voulut peut-être se faire accroire que ce trait
pourrait enrayer la presse extrême que les princes du sang lui faisaient
de juger, dans la défiance que cela leur ferait naître qu'il ne leur
serait pas favorable. Non content de laisser servir le grand prieur à la
cène, il lui permit tacitement ce que M. de Vendôme et lui n'avaient
jamais ni eu ni osé demander du temps du feu roi, qui fut d'être assis
pendant le sermon de la cène avec les princes du sang, le dernier en
même rang et honneurs qu'eux. Sur les plaintes qui en furent portées au
régent, il montra le trouver mauvais, et promit d'y donner ordre. Il
pouvait dès lors l'empêcher, puisqu'il y était. Le lendemain, vendredi
saint, le grand prieur parut à l'office du jour à la chapelle en même
place et honneurs. M. le duc d'Orléans dit après qu'il l'avait oublié,
mais il ne laissa pas d'ordonner au grand maître des cérémonies de
l'écrire sur son registre. Il protesta seulement que cela n'arriverait
plus, et se moqua ainsi des princes du sang, sans nécessité aucune que
de complaire à l'insolence d'un audacieux qui sentait bien à qui il
avait affaire. Je ne voulus pas seulement prendre la peine de lui en
parler\,: c'était l'affaire des princes du sang encore plus que la
nôtre.

La paix profonde, qui avait toutes sortes d'apparences de durer
longtemps, donna lieu à plusieurs jeunes gens qui n'avaient encore pu
voir de guerre, de demander la permission de l'aller chercher en
Hongrie. La maison de Lorraine, si foncièrement attachée à celle
d'Autriche, en donna l'exemple par le prince de Pons et le chevalier de
Lorraine, son frère, qui l'obtinrent, et partirent aussitôt. M. du Maine
crut devoir écouter le désir du prince de Dombes, qui l'obtint de même.
Alincourt, fort jeune, second fils du duc de Villeroy, y alla aussi, et
quelques autres\,; mais ce zèle des armes devint contagieux. On commença
à se persuader qu'à ces âges-là on ne pouvait se dispenser de suivre cet
exemple\,; ce qui obligea avec raison le régent à défendre que personne
lui demandât plus d'aller en Hongrie, et qu'il fit une défense générale
d'y aller. M. le prince de Conti voulut faire comme les autres. Il se
laissa apaiser par de l'argent. Il acheta de La Vieuville le médiocre
gouvernement de Poitou, que M le duc d'Orléans fit payer pour lui par le
roi, en mettre les appointements sur le pied des grands gouvernements,
et en même temps il le fit entrer au conseil de régence. Quelques jours
après, il y fit entrer Pelletier de Sousi, qui n'y venait que les jours
de finance. Quoique très ancien conseiller d'État, il prit la dernière
place après MM. de Troyes, Torcy et Effiat, qui ne l'étaient point, sans
que les conseillers d'État en murmurassent. Ce haut et bas de leur part,
je ne l'ai point compris, et sitôt après tant de bruit à l'occasion de
l'entrée de l'abbé Dubois dans le conseil des affaires étrangères. M. le
prince de Conti entra aussi au conseil de guerre, qui se tenait chez le
maréchal de Villars. M. le Duc, qui n'y fut point, le trouva mauvais, et
prétendit que, lorsqu'il ne se tenait point au Louvre, ce devait être
chez lui à l'hôtel de Condé. M. le duc d'Orléans se moqua de cette
prétention, et, pour la rendre ridicule, il alla lui-même au conseil de
guerre qui se tint chez le maréchal de Villars quelques jours après.

M\textsuperscript{me} de Maintenon, oubliée et comme morte dans sa belle
et opulente retraite de Saint-Cyr, y fut considérablement malade, sans
que cela fût presque su, ni que cela fît la moindre sensation sur ceux
qui l'apprirent.

Albergotti fut trouvé presque mort le matin par ses valets entrant dans
sa chambre, et ne vécut que peu d'heures après. Il avait des attaques
d'épilepsie qu'il cachait avec grand soin, et il s'en joignit
d'apoplexie. Il était neveu de Magalotti, Florentin comme lui, qui avait
été capitaine des gardes du cardinal Mazarin, et qui mourut lieutenant
général et gouverneur de Valenciennes, duquel j'ai parlé en son temps.
Le maréchal de Luxembourg, ami intime de Magalotti, avait fait
d'Albergotti comme de son fils, ce qui l'avait mis dans les meilleures
compagnies de la cour et de l'armée, et l'avait fort lié avec tout ce
qui l'était avec M. de Luxembourg, par conséquent avec M. le Duc et M.
le prince de Conti, et avec toute la cabale de Meudon, car il savait
s'échafauder et aller de l'un à l'autre. Pour le faire connaître en deux
mots, c'était un homme digne d'être confident et instrument de Catherine
de Médicis. C'est montrer tout à la fois quel était son esprit et ses
talents, quels aussi son coeur et son âme. Le maréchal de Luxembourg et
ses amis, et M. le prince de Conti s'en aperçurent les premiers. Il les
abandonna pour M. de Vendôme lors de son éclat avec eux. Albergotti
sentit de bonne heure qu'il pointait à tout. Ses moeurs étaient
parfaitement homogènes aux siennes. Il se dévoua à lui pour la guerre,
et par lui à M. du Maine, pour la cour. Ceux qu'il déserta le trouvèrent
si dangereux qu'ils n'osèrent se brouiller ouvertement avec lui, mais ce
fut tout. C'était un grand homme sec, à mine sombre, distraite et
dédaigneuse, fort silencieux, les oreilles fort ouvertes et les yeux
aussi. Obscur dans ses débauches, très avare et amassant beaucoup\,;
excellent officier général pour les vues et pour l'exécution, mais fort
dangereux pour un général d'armée et pour ceux qui servaient avec lui.
Sa valeur était froide et des plus éprouvées et reconnue, avec laquelle
toutefois les affronts les plus publics et les mieux assénés ne lui
coûtaient rien à rembourser et à laisser pleinement tomber en faveur de
sa fortune. On a vu en son lieu celui qu'il essuya de La Feuillade après
le malheur de Turin\,; et on en pourrait citer d'autres aussi éclatants,
sans qu'il en ait jamais fait semblant même avec eux, ni qu'il en soit
un moment sorti de son air indifférent et de son silence, à propos
duquel je dirai, comme une chose bien singulière, que
M\textsuperscript{lle} d'Espinoy m'a conté que M\textsuperscript{me} sa
mère le menant une fois de Paris à Lille, où elle allait avec ses deux
filles pour ses affaires, personne de ce qui était du voyage, ni
elles-mêmes, lui dans leur carrosse, ne lui entendirent proférer un seul
mot depuis Paris jusqu'à Lille. Il eut l'art de se mettre bien avec tous
ceux de qui il pouvait attendre, et sur un pied fort agréable avec le
roi, et le plus honnêtement qu'il pouvait avec le gros du monde,
quoiqu'il n'ignorât d'être haï, et qu'on se défiait beaucoup de lui. Il
devint ainsi lieutenant général commandant des corps séparés, chevalier
de l'ordre et gouverneur de Sarrelouis. Il avait outre cela douze mille
livres de pension. À cette conduite on peut juger qu'il ne s'était
jamais donné la peine de s'approcher de M. le duc d'Orléans. Pendant le
dernier Marly du roi nous fûmes surpris, M\textsuperscript{me} de
Saint-Simon et moi, de le voir entrer dans sa chambre. Jamais il ne nous
avait parlé. Il y revint trois ou quatre fois de suite avec un air aisé.
J'entendis bien, et elle aussi, à quoi nous devions cet honneur. Nous le
reçûmes honnêtement, mais de façon qu'il sentît que nous ne serions pas
ses dupes. Nous ne le revîmes plus depuis. Il n'était point marié, et ne
fut regretté de personne. Son neveu eut son régiment royal-italien, qui
valait beaucoup, et Madame fit donner le gouvernement de Sarrelouis au
prince de Talmont.

Enfin, quelques jours avant la semaine sainte, le chancelier alla le
matin à la chambre de justice la remercier et la finir. Elle avait duré
un an et quelques jours, et coûta onze cent mille francs. Lamoignon s'y
déshonora pleinement, et Portail y acquit tout l'honneur possible. Cette
chambre fit beaucoup de mal et ne produisit aucun bien. Le mal fut les
friponneries insignes, les recelés, les fuites, et le total discrédit
des gens d'affaires à quoi elle donna lieu\,; le peu ou point de bien
par la prodigalité des remises qui furent faites sur les taxes, et les
pernicieux manèges pour les obtenir. Je ne puis m'empêcher de répéter
que je voulais, comme on l'a vu en son lieu, qu'on fît en secret ces
taxes par estime fort au-dessous de ce à quoi elles pouvaient monter\,;
les signifier aux taxes en secret, les uns après les autres\,; les leur
faire payer à l'insu de tout le monde et à l'insu les uns des autres,
mais en tenir des registres bien sûrs et bien exacts\,; leur faire
croire que, par considération pour eux, on ne voulait pas les peiner,
encore moins les décrier, en leur faisant des taxes publiques\,; mais
qu'il fallait aussi que, en conservant leur honneur et leur crédit, le
roi fût aidé. Par cette voie, on le leur aurait laissé tout entier, puni
leurs rapines, perçu pour le roi tout ce qui aurait été payé, et ôté
toute occasion de frais et de modération de taxes, et de dons sur leur
produit, parce que les taxes mêmes auraient été ignorées, par où il se
serait trouvé qu'en taxant, sans proportion, moins qu'on ne fit et sans
frais, il en serait entré infiniment plus dans les coffres du roi qu'il
n'y en entra par la chambre de justice. Je voulais en même temps que de
ces taxes on payât de la main à la main tous les brevets de retenue
existant, quels qu'ils fussent, avec bien ferme résolution de n'en
accorder jamais\,; en payer tous les régiments et toutes les charges
militaires, et les principales charges de la cour, même les charges de
présidents à mortier, et d'avocats et procureur général du parlement de
Paris\,; rendre toutes ces charges libres n'en plus laisser vendre
aucune ni un seul régiment, et les réserver à toujours en la disposition
gratuite du roi, à mesure de leurs vacances. J'y comprenais aussi les
gouverneurs généraux et particuliers, et leurs lieutenances. Je parlais
sans intérêt, je n'avais ni charge, ni régiment, ni gouvernement de
province, ni brevet de retenue. Aussi M. le duc d'Orléans goûta-t-il
beaucoup cette proposition\,; mais le duc de Noailles, se voyant à la
tête des finances, en voulut tout le pouvoir et le profit, flatter la
robe, et, par un mélange utile à ses affaires de terreur et de
débonnaireté, devenir l'effroi, l'espérance ou l'amour de la gent
financière qui a des branches fort étendues dans tous les trois états du
royaume. Ainsi il lui fallut tout l'appareil d'une chambre de justice,
après quoi il ne fut plus question d'un emploi si utile. La facilité
inconcevable du régent avait déjà donné les survivances et les brevets
de retenue à pleines mains, sans choix ni distinction quelconque, et
voulut continuer cette aveugle prodigalité, comptant ne donner rien et
s'attacher tout le monde. Il se trouva qu'il en donna tant que personne
de cette multitude ne lui sut aucun gré d'avoir eu ce que tant d'autres
en obtenaient sans peine, et que, honteux lui-même de n'avoir rien
laissé à disposer au roi, il eut l'imprudence d'autoriser l'ingratitude,
en disant qu'il serait le premier à lui conseiller de ne laisser
subsister aucune de ces grâces. On le craignit un temps\,; mais la
rumeur devint si grande, par la multitude des intéressés, qu'on n'osa
enfin y toucher.

Enfin, après bien des négociations et des délais, les États généraux se
déterminèrent à accéder au traité fait entre la France et l'Angleterre,
et le firent signer pour eux à la Haye, le 4 janvier\,: c'est ce qu'on
nomma la triple alliance défensive. Beretti pressait toujours le
pensionnaire Heinsius d'une ligue particulière avec l'Espagne. Heinsius
le remettait jusqu'à ce qu'on vît finir de façon ou d'autre la
négociation avec la France, et Beretti attribuait ces remises à la
crainte de déplaire à l'empereur. Cependant, de concert avec le
Pensionnaire, il s'adressa au président de semaine, qui lui promit de
porter sa proposition à l'assemblée des États généraux, et lui fit
espérer qu'elle y serait bien reçue. Beretti comptait mal à propos sur
l'opposition de la France, quoiqu'il fût certain que l'intérêt et le
dessein de cette couronne fussent de faciliter l'alliance de l'Espagne
avec les Provinces-Unies, et qu'il n'y eût de puissance en Europe que
l'empereur à qui elle pût déplaire. Il ne s'en cachait pas, ni de son
chagrin de la triple alliance. L'Angleterre et la hollande le pressaient
d'y entrer. Il rejeta la proposition des Hollandais avec tant de mépris,
que Heinsius, si passionné autrichien toute sa vie, ne pût s'empêcher
d'en montrer son dépit à Beretti. À Stanion, chargé des affaires
d'Angleterre à Vienne, le prince Eugène répondit qu'il ne voyait pas
l'utilité dont il serait à l'empereur d'entrer dans un traité qui ne
tendait qu'à confirmer Philippe V sur le trône d'Espagne. La conséquence
en était si visible que Beretti changea d'avis, et se persuada enfin que
la France désirait que le roi d'Espagne entrât au plus tôt en alliance
avec l'Angleterre et la Hollande, non dans la vue des intérêts de
l'Espagne, mais de ceux de M. le duc d'Orléans.

Beretti, faute d'instructions de Madrid, n'avait osé donner au président
de semaine un mémoire, selon la coutume, et s'était contenté de lui
parler. Nonobstant ce défaut de forme, sa proposition avait été envoyée
aux Provinces, et Beretti cherchait à découvrir les sentiments des
personnages principaux. Un jour qu'il alla voir le baron de Duywenworde,
il y rencontra le comte de Sunderland, qui venait d'Hanovre, où le roi
d'Angleterre était encore. Beretti n'osait parler devant ce tiers.
Duywenworde le tira bientôt de peine. Il dit à Sunderland que le roi
d'Espagne proposait une ligue à sa république\,; qu'il ne doutait pas
que ce ne fût conjointement avec l'Angleterre, par la liaison qui devait
toujours unir ces deux puissances\,; et il déclara qu'à cette condition
il y concourrait de tout son pouvoir. Beretti répondit que si l'alliance
était faite avec ces deux puissances, elle en serait d'autant plus
agréable au roi son maître. On s'expliqua de part et d'autre sur l'objet
qu'elle devait avoir. Sunderland et Duywenworde dirent tous deux que le
traité avec la France en devait être le modèle, et la tranquillité de
l'Europe le but. Ils ajoutèrent, sans que Beretti s'y attendît, que la
garantie s'étendrait seulement sur les États que l'empereur possédait
actuellement\,; que leurs maîtres avaient pris une ferme résolution de
ne pas souffrir que ce prince, déjà trop puissant, s'étendît davantage,
qu'il serait temps qu'il abandonnât ses chimères, et qu'il fît la paix
avec le roi d'Espagne\,; que le bruit courait qu'elle se négociait par
l'entremise du pape. Là-dessus Sunderland décria fort la faiblesse de
cette entremise, l'attachement des parents du pape pour l'empereur\,; et
soutint que, quand même le pape aurait agi en médiateur équitable,
l'empereur serait toujours maître de lui manquer de parole, et qu'il
n'en serait pas de même à l'égard de l'Angleterre et de la Hollande,
dont la médiation serait beaucoup plus sûre et plus juste\,; que leur
intention était de mettre l'Europe en repos\,; et que le roi d'Espagne
en ferait l'épreuve, s'il voulait se fier à ces deux puissances.

Stanhope, venant d'Hanovre à la Haye, précéda de peu de jours le passage
du roi d'Angleterre\,; il tint à Beretti le même propos. Il s'étendit
sur la nécessité de l'union de l'Espagne avec l'Angleterre, sur les
malheurs de la dernière guerre qui avait désolé l'Espagne, dans laquelle
il s'était trouvé\,; sur l'ancienne maxime des Espagnols de paix avec
l'Angleterre\,; sur les sentiments du roi d'Angleterre, qui répondaient
à ceux du roi d'Espagne\,; enfin jusqu'à trouver dans ces deux princes
une conformité de caractère, et il parla comme Sunderland sur la
prétendue négociation du pape. Il promit que, si le roi d'Espagne avait
confiance en lui, il travaillerait de manière qu'il en serait
satisfait\,; que l'Angleterre forcerait l'empereur à convenir de ce qui
serait juste, ensuite à tenir les conventions faites\,; que la
succession de Parme et de Plaisance serait assurée à la reine d'Espagne
et à don Carlos à l'infini\,; que les droits du roi d'Espagne sur Sienne
seraient maintenus\,; qu'elle empêcherait la maison d'Autriche de
s'emparer de la Toscane. Enfin Stanhope promit tout ce qui pouvait
plaire le plus au roi et à la reine d'Espagne, ou Beretti embellit et
augmenta le compte qu'il en rendit. Beretti soupçonna que les
ambassadeurs de France, qui étaient à la Haye, n'eussent part à la façon
dont Stanhope s'était expliqué sur la succession de Parme qui touchait
si personnellement et si sensiblement la reine d'Espagne, pour
l'engager, par cet intérêt, à faire entrer le roi son mari dans la
triple alliance, par conséquent à confirmer encore plus, en faveur des
renonciations, les dispositions faites par le traité d'Utrecht. Il crut
voir, par des traits échappés dans la conversation à Stanhope, que
l'union entre la France et l'Angleterre n'était pas aussi sincère ni
aussi étroite de la part des Anglais que le monde se la figurait. Il
était confirmé dans cette pensée sur ce que Stanhope s'était
particulièrement attaché à lui montrer qu'il faisait une extrême
différence, pour la solidité des alliances, entre celle de la France et
celle que l'Angleterre contracterait avec l'Espagne\,; et que, pour lui
faire sentir l'importance de cette confidence, il lui avait demandé un
secret sans réserve à l'égard de tout Français, Hollandais et Anglais,
et il lui offrait d'entretenir avec lui une correspondance régulière
après son retour en Angleterre, d'où il le remit à lui répondre sur la
permission qu'il demanda pour le roi d'Espagne de lever trois mille
Irlandais.

Beretti, avec ces notions et ces mesures prises, se mit à travailler du
côté d'Amsterdam à empêcher les États généraux de presser l'empereur
d'entrer dans la ligue. Il les savait disposés à lui garantir les droits
et les États qu'il possédait en Italie, ce qui était fort contraire aux
intérêts du roi d'Espagne. Il sut qu'Amsterdam voulait éloigner cette
garantie\,; c'en était assez pour éloigner l'empereur d'entrer dans le
traité, et il était de l'intérêt du roi d'Espagne de profiter de cette
conjoncture pour presser la république de se déterminer sur la
proposition qu'il lui avait faite, qui d'ailleurs était mécontente de
l'infidélité des Impériaux sur l'exécution du traité de la Barrière.
Mais il lui fallut essuyer les longueurs ordinaires du gouvernement de
ce pays. L'Angleterre était toujours menacée de forts mouvements.

Le nombre des jacobites y était toujours grand, nonobstant l'abattement
de ce parti\,; c'est ce qui pressa Georges de se rendre à Londres, sans
s'arrêter en Hollande, et ce qui lui lit conclure son traité avec la
France, bien persuadé que sa tranquillité au dedans dépendait de cette
couronne, et de la retraite du Prétendant au delà des Alpes.
Penterrieder avait été dépêché de Vienne à Hanovre pour le traverser. Il
n'en était plus temps à son arrivée. Il fallut se contenter de
l'assurance positive qu'il ne contenait aucun article contraire aux
intérêts de la maison d'Autriche, et d'écouter l'applaudissement que se
donnait le roi d'Angleterre des avantages, tant personnels que
nationaux, qu'il en tirait. Penterrieder avait ordre aussi de travailler
à la paix du Nord. L'empereur s'intéressait à sa conclusion pour tirer
facilement des troupes qui étaient employées à cette guerre, pour en
grossir les siennes en Hongrie, où il n'était plus question que d'ouvrir
la campagne de bonne heure.

Le roi d'Angleterre protesta de son désir, en représentant les
difficultés infinies qui naissaient des intérêts et des jalousies des
confédérés, et sur ce qu'il ignorait encore ce que les ministres de
Suède lui préparaient en Angleterre. La division y était grande, non
seulement entre les deux partis toujours opposés, mais dans le dominant,
mais entre les ministres, mais dans la famille royale. Le gros blâmait
le traité avec la France, qui désunissait l'Angleterre, contre son
véritable intérêt, d'avec l'empereur. Il le trouvait inutile, parce
{[}que{]}, ne leur pouvant être bon que par des conditions avantageuses
pour le commerce, il n'y en était pas dit un mot. La considération du
repos de leur royaume ne les touchait point. Ils disaient que
l'Angleterre ne pouvait demeurer unie qu'autant qu'on lui présenterait
un objet qui lui fît craindre la désunion\,; que le prétendant était cet
objet qui, disparaissant, dissiperait les craintes, dont la fin
donnerait lieu aux passions particulières de faire plus de mal que les
guerres du dehors. Ainsi ils trouvaient mauvais qu'il y eût une
stipulation de secours de la France si l'Angleterre en avait besoin,
parce que, si c'était en troupes, la nation n'en voulait point chez elle
d'étrangères\,; si en argent, le royaume n'en manque pas, et il lui
était honteux d'en recevoir d'un autre. C'est qu'encore que le parti
dominant, qui était les whigs, eût toujours été déclaré pour la maison
d'Autriche, il s'était laissé gagner par le roi Georges et par ses
ministres allemands uniquement occupés de la grandeur de la maison
d'Hanovre en Allemagne\,: changement d'autant plus étonnant que le
ministère whig souhaitait peu auparavant que le roi d'Espagne voulût
revenir contre ses renonciations, et que l'esprit du parti fût encore le
même. Ses adversaires, ravis de les voir divisés, demeuraient
spectateurs tranquilles des scènes qui se préparaient à l'ouverture, et
pendant les séances du parlement, et dressaient cependant leurs
batteries pour déconcerter celles de la cour qui voulait conserver ses
troupes dans la paix la plus profonde, que les torys voulaient faire
réformer comme contraires à la liberté de l'Angleterre et fort à charge
par la dépense. Ces dispositions achevaient de persuader Georges de
l'utilité de son traité avec la France, et de la nécessité de cultiver
et de fortifier tant qu'il pourrait cette alliance. Stairs eut ordre de
dire que son maître la regardait comme un prélude à des affaires bien
plus importantes et bien plus étendues. Stairs eut ordre aussi
d'observer infiniment les démarches du baron de Goertz, qui était alors
à Paris, que le roi d'Angleterre regardait comme un de ses plus grands
ennemis, dont il commençait à découvrir les intrigues et celles des
autres ministres de Suède.

Gyllembourg, envoyé de Suède en Angleterre, qui voyait de près le
mécontentement et les mouvements qui y étaient, persuadé qu'il était de
l'intérêt de son maître de profiter de ces divisions, suivit avec
chaleur les projets qu'il avait formés pour exciter des troubles en
Angleterre, et procurer par là une diversion, la plus favorable que le
roi de Suède pût espérer. Il négociait donc en même temps deux affaires,
dont la première, qu'il ne cachait point, pouvait contribuer au succès
de l'autre, qui devait être secrète. La première était un traité qu'il
voulait faire avec des négociants anglais, pour leur faire porter des
blés en Suède et y prendre du fer en échange. Il communiquait cette
affaire à Goertz, et tout ce qu'il faisait aussi pour la seconde, qui
étaient les mesures qu'il prenait avec les jacobites\,; mais il
craignait, pour le secret d'une affaire si importante, la pénétration de
la Hollande, où on savait jusqu'aux moindres démarches des ministres
étrangers. Il était averti par ses amis des mesures qu'il fallait
prendre et du temps à transporter des troupes suédoises et de
l'artillerie sur les côtes d'Écosse ou d'Angleterre. Ils demandaient dix
vaisseaux de guerre pour escorter les bâtiments de transport. Il était
impossible de tenter d'en acheter en Angleterre sans s'exposer à être
découvert\,; et pour les bâtiments de transport, le danger n'en était
pas moindre, si on en tirait un trop grand nombre d'Angleterre en
Hollande. L'expédient pour ces derniers fut d'avertir que le roi de
Suède ferait vendre dans un certain temps les prises faites par ses
sujets dans la mer Baltique, d'engager sous ce prétexte plusieurs
négociants de se rendre à Gottembourg, qui y ferait ces emplettes en
même temps que leur échange de blé pour du fer. Quelques officiers de
marine, qui entraient dans le projet, croyaient, par les raisons de leur
métier, que le mois de janvier serait le plus favorable pour ce
transport, et supputaient qu'un bâtiment de trois cents tonneaux pouvait
porter trois cents hommes, et les chevaux à proportion\,; mais ils
représentaient la nécessité d'appeler en Suède quelques officiers
Anglais qui connussent les côtes, pour conduire l'expédition. On était
alors au mois de janvier. On a vu que le roi, étant à Hanovre, avait
ordonné à l'escadre Anglaise qui était à Copenhague d'y demeurer.
L'amirauté d'Angleterre, piquée que cela eût été fait sans elle, avait
fait des représentations sur ce séjour, comme contraire au bien de la
nation, et avait en même temps fait disposer des lieux pour y faire
hiverner vingt-cinq des plus grands navires d'Angleterre\,; par
conséquent nulle apparence que de quelques mois cette couronne eût aucun
navire en mer.

La difficulté de l'argent était la principale. Mais celui qui dirigeait
le projet de la part des Anglais, étant revenu à Londres vers le 15
janvier, dit à Gyllembourg que, sur un ordre du comte de Marr, il avait
fait délivrer en France à la reine douairière d'Angleterre vingt mille
pièces pour les Suédois, qu'il avait fait demander au même comte en quel
endroit il ferait payer le reste de la somme que les amis étaient fort
inquiets du bruit qui courait de la mésintelligence entre le baron Spaar
et Goertz, et qu'ils avaient appris avec plaisir que Gyllembourg devait
passer en Hollande pour conférer avec Goertz. Le compte de ce qui avait
été payé montait lors à vingt-cinq mille pièces. Gyllembourg en demanda
dix mille avant son départ, et une lettre du frère du médecin du czar,
pour s'en servir en cas de besoin. On lui promit une bonne somme
lorsqu'il passerait en Hollande\,; mais Gyllembourg et ceux de
l'entreprise étaient également inquiets de l'ordre reçu de remettre
l'argent à la reine d'Angleterre en France, au lieu de le remettre à
Gyllembourg\,; suivant le premier plan, et de tirer une quittance signée
de lui. Ils craignaient surtout la France, et l'étroite intelligence qui
était entre le roi d'Angleterre et le régent, qui lui donnerait non
seulement tous les secours promis dans les cas stipulés, mais tous les
avis de tout ce qu'il pourrait découvrir pour sa conservation sur le
trône.

Bentivoglio, toujours porté au pis sur le régent, et à tout brouiller en
France, prétendait que la fin secrète du traité avec l'Angleterre était
de former et fortifier en Allemagne le parti protestant contre le parti
catholique, et qu'il ne s'agissait pas seulement de détruire en
Angleterre la religion catholique, qu'on devait regarder désormais comme
bannie de ce royaume, mais d'enlever à la maison d'Autriche la couronne
impériale, et de la mettre sur la tête d'un protestant. Il menaçait déjà
Rome de suivre le sort des catholiques de l'empire, et de devenir la
proie des protestants. Après avoir ainsi intimidé le papa, il
l'exhortait à s'unir plus étroitement que jamais à l'empereur dont
l'intérêt devenait celui de la religion, et, pour avoir lui-même part à
ce grand ouvrage, il entretenait souvent le baron d'Hohendorff, fourbe
plus habile que le nonce, et qui lui faisait accroire que, touché de ses
lumières, de son zèle et de ses projets, il envoyait exactement à Vienne
tous les papiers qu'il lui communiquait. Cette ressource d'union à
l'empereur était encore la seule que Bentivoglio faisait envisager à
Rome pour soutenir en France l'autorité apostolique, et pour engager le
pape aux violences, dont par lui-même Sa Sainteté était éloignée. Il
l'assurait que les liaisons seules qu'il pourrait prétendre
{[}étaient{]} avec les princes catholiques, dans une conjoncture où tous
les remèdes palliatifs qu'on n'avait cessé d'employer malgré ses
instances, s'étaient tous tournés en poison contre la saine doctrine et
l'autorité de la cour de Rome\footnote{Cette phrase a été changée et en
  partie supprimée dans les éditions précédentes.}. Ceux qui la
gouvernaient étaient persuadés que sa seule ressource pour sauver son
pouvoir, et suivant son langage la religion en France, était une liaison
parfaite entre le pape et le roi d'Espagne, et le seul moyen d'y
conserver la saine doctrine, et la loi de nature. Aubenton était
exactement instruit de ces sentiments, sur le fidèle et entier
dévouement duquel le pape comptait entièrement. Ce jésuite et Albéroni
étaient en même temps avertis par Rome que la triple alliance qui venait
d'être signée ne tendait qu'au préjudice du roi d'Espagne, et à
maintenir la couronne de France dans la ligne d'Orléans\,; et
l'engagement réciproque de maintenir aussi la couronne d'Angleterre dans
la ligne protestante était traité d'infâme, dont la conclusion était que
le roi d'Espagne agirait prudemment de prendre des liaisons avec les
Allemands. Telles étaient les dispositions de Rome quand Aldovrandi en
partit pour retourner en Espagne. Il eut ordre de passer à Plaisance
pour y faciliter le succès de sa négociation par les avis et le crédit
du duc de Parme.

L'instruction d'Aldovrandi était fort singulière il emportait des brefs
qui accordaient au roi d'Espagne une imposition annuelle de deux cent
mille écus sur les biens ecclésiastiques d'Espagne et des Indes, avec
pouvoir d'augmentation suivant le besoin, à proportion de ce que ces
mêmes biens payaient déjà pour le tribut appelé \emph{sussidio y
excusado}. Les ecclésiastiques d'Espagne s'y opposaient au point de
tenir à Rome pour cela un chanoine de Tolède appelé Melchior Guttierez,
qui pesait fort au cardinal Acquaviva. Le grand objet du pape était
d'obtenir l'ouverture de la nonciature à Madrid, depuis si longtemps
fermée, et de faire admettre Aldovrandi en qualité de nonce. Il lui
enjoignit donc de garder précieusement les brefs d'imposition sur les
biens ecclésiastiques, et de ne les délivrer qu'après son admission à
l'audience en qualité de nonce, et lui permit en même temps de les
délivrer avant de prendre le caractère de nonce, si on insistait
là-dessus. Acquaviva, qui le découvrit, on avertit le roi d'Espagne, et
dans la connaissance qu'il avait du peu de stabilité des résolutions du
pape, conseilla de commencer par se faire remettre ces brefs. La
promotion d'Albéroni en était un autre article que les défiances
mutuelles rendaient difficile. Le pape, de peur qu'on ne se moquât de
lui après la promotion faite, n'y voulait procéder qu'après
l'accommodement conclu. Albéroni, qui avait la même opinion du pape, ne
voulait rien finir avant d'être fait cardinal. Pour sortir de cet
embarras, Aldovrandi fut chargé de déclarer que, lorsque, le pape
saurait, par un courrier qu'il dépêcherait en arrivant, que les ordres
dont il était porteur étaient du goût du roi d'Espagne, il ferait
aussitôt la promotion d'Albéroni, avant même d'en savoir davantage, ni
l'effet de la parole que le roi d'Espagne aurait donnée. Aldovrandi,
quelque bien qu'il fût avec Albéroni et Aubenton, y désira des
précautions contre ses ennemis\,; Acquaviva, qui avait le même intérêt,
y manda d'être en garde contre tout ce qui viendrait des Français, sur
le compte de ce nonce, qu'ils haïssaient comme trop attaché, à leur gré,
au parti du roi d'Espagne, à l'égard des événements qui pouvaient
arriver en France, avec force broderies, pour appuyer cet avis.

Albéroni avait déclaré que non seulement le neveu du pape, mais que qui
que ce fût qu'il voulût envoyer à Madrid, y pouvait être sûr d'une
réception agréable, et du succès des ordres dont il serait chargé, si sa
promotion était faite\,; mais que, s'il arrivait les mains vides, il
n'aurait qu'à s'en retourner aussitôt, et qu'Aldovrandi même n'y serait
pas souffert, quand bien il se réduirait à demeurer comme un simple
particulier sans aucun caractère. Il disait et il écrivait qu'il n'y
avait pas moyen d'adoucir une reine irritée par tant de délais
trompeurs, qu'il rappelait tous\,; il insistait, comme sur un mépris et
un manque de parole insupportables, sur la promotion du seul Borromée,
que le pape voulait faire, et qui était dévoué et dépendant de la maison
d'Autriche\,; qu'il donnerait la moitié de son sang, et qu'il n'eût
jamais été parlé de sa promotion, tant il prévoyait de malheurs de cette
source\,; qu'Aubenton était exclu d'ouvrir la bouche sur quoi que ce fût
qui regardât Rome\,; qu'il prévoyait qu'il recevrait incessamment la
même défense. Il se prévalait ainsi de la timidité du pape pour en
arracher par effroi ce qu'il désirait avec tant d'ardeur, et protestait
en même temps de sa reconnaissance, de sa résignation parfaite aux
volontés du pape, on y mêlant toujours la crainte des ressentiments
d'une princesse vive, dont il tournait toujours les éloges à faire
valoir la confiance dont elle l'honorait, et son crédit supérieur à
toutes les attaques. Sa faveur, en effet, était au plus haut point. Il
avait dissipé, anéanti, absorbé tous les conseils\,; lui seul donnait
tous les ordres, et c'était à lui seul que ceux qui servaient au dedans
et au dehors les demandaient, et les recevaient. La jalousie était
extrême de la part des Espagnols qui, grands et petits, se voyaient
exclus de tout, et voyaient tous les emplois entre les mains
d'étrangers, qui ne tenaient en rien à l'Espagne, et qui n'étaient
attachés qu'à la reine et à Albéroni, pour leur fortune et leur
conservation.

Giudice ne pouvait se résoudre à quitter la partie, et quoique accablé
des plus grands dégoûts, il ne pouvait renoncer à l'espérance de se
rétablir auprès du roi d'Espagne\,; il se voulait persuader et encore
plus au pape, qu'il sacrifiait les peines de sa demeure à Madrid à Sa
Sainteté et à sa religion, et lui mandait sans ménagements de termes
tout ce qu'il pouvait de pis contre Albéroni, Aubenton et Aldovrandi
qu'il lui reprochait de croire plutôt que de consulter le clergé
séculier et régulier d'Espagne sur ce qu'il pensait d'eux, lequel était
pourtant le véritable appui de l'autorité pontificale dans la monarchie.
À la fin ne pouvant plus tenir avec quelque honneur, il résolut de
partir, et prit en partant des mesures pour se procurer la faveur du roi
de Sicile, et une conférence avec lui en passant.

Albéroni se moquait de lui publiquement. Il vantait la forme nouvelle du
gouvernement, et les merveilles qu'il avait déjà opérées dans les
finances et dans la marine. Campo Florido, que si longtemps après nous
avons vu ici ambassadeur d'Espagne et chevalier du Saint-Esprit, fut
fait président des finances\,; don André de Paëz, président du conseil
des Indes, qui fut fort diminué, et dont encore tous les créoles furent
chassés. Le comte de Frigilliana, grand d'Espagne, père d'Aguilar,
desquels j'ai parlé plus d'une fois, fut démis de la présidence du
conseil d'État, mais on en laissa les appointements à sa vieillesse. Le
conseil des Indes, sans la signature duquel colle du roi ne servait à
rien aux Indes, reçut défense de plus rien signer, et celle du roi seul
y fut substituée. Le conseil de guerre, dont la présidence fut laissée
au marquis de Bedmar, grand d'Espagne et chevalier du Saint-Esprit, de
qui j'ai aussi parlé, sans autorité, et le conseil réduit à quatre
membres de robe qui ne s'y pouvaient mêler que des choses judiciaires.
S'il s'agissait de faire le procès à des officiers généraux, ils furent
réservés au roi d'Espagne ou aux officiers généraux qu'il y commettrait.
Les appointements des grands emplois furent fort réduits. Par exemple
ceux du président du conseil de Castille ou du gouverneur, qui étaient
de vingt-deux mille écus, furent fixés à quinze mille. Les secrétaires
du \emph{despacho}\footnote{On a déjà vu plus haut que ce mot désignait
  la secrétairerie d'État.} furent réduits de dix-huit mille à douze
mille écus, et eux exclus de toutes places de conseillers dans les
conseils\,; le nombre des commis fort réduit, et eux uniquement fixés à
leur emploi dans leur bureau. Il joignit en une les doux places de
secrétaire de la police et des finances, fit d'autres changements dans
les subalternes et abolit l'abus introduit par le conseil de Castille
dans les provinces et dans les villes qui lui payaient quatre pour cent
de toutes les sommes qu'elles étaient obligées d'emprunter jusqu'au
remboursement de ces sommes.

Albéroni faisait beaucoup valoir la sagesse et l'utilité de tout ce
qu'il faisait dans l'administration du gouvernement. Il n'en laissait
rien ignorer au duc de Parme, même fort peu des affaires. Quoiqu'il se
sentit plus en état de protéger son ancien maître qu'en besoin d'en être
protégé, son nom et cotte liaison ne lui étaient pas inutiles auprès de
la reine d'Espagne. Pour les affaires de Rome, il ne lui en cachait
aucune. Les deux points que cette cour désirait le plus d'obtenir de
l'Espagne étaient que l'escadre promise contre les Turcs se rendit dans
le 15 avril, au plus tard, dans les mers de Corfou, et qu'Aldovrandi en
arrivant en Espagne y rouvrit la nonciature avec toutes les prérogatives
de ses prédécesseurs. Le duc de Parme, intéressé particulièrement à lui
plaire, pressait Albéroni de tout faciliter sur ces deux articles, et
pour lui marquer l'intérêt qu'il prenait en lui, il lui donnait en ami
des conseils pour éviter de nouvelles plaintes du régent. Sa pensée
était qu'il y avait des gens auprès de ce prince qui pour leur intérêt
particulier cherchaient à le brouiller avec l'Espagne. Enfin, pour aider
de tout son pouvoir Albéroni à Paris, il en rappela son envoyé Pichotti
qui s'était déchaîné contre ce premier ministre, et y envoya l'abbé
Landi qui était si bien dans son esprit qu'il aurait été précepteur du
prince des Asturies sans les réflexions personnelles que la reine fit
sur ce choix.

Landi était doux et insinuant. Il avait de l'esprit et des lettres. Il
était mesuré et de bonne compagnie, mais il avait été bibliothécaire du
cardinal Imperiali, qui était une école à devenir aussi passionné
autrichien que mauvais français. Albéroni encore alors ministre public
du duc de Parme à Madrid, quoique premier ministre d'Espagne, était le
confident secret de la reine à l'égard de sa maison, comme sur le
gouvernement de l'État, et des chagrins réciproques. Elle et la duchesse
sa mère étaient aisées à s'offenser, et le duc de Parme, plus liant et
plus doux, était souvent embarrassé entre l'une et l'autre, pour des
bagatelles domestiques, dont Albéroni l'aidait à se tirer. Tous deux
avaient intérêt à vivre ensemble dans une étroite amitié, et Albéroni
avait soin de lui rendre compte des affaires dont il était occupé, et
souvent encore des projets qu'il formait.

Un de ceux qu'il avait le plus à coeur était d'empêcher les Hollandais
de faire avec l'empereur une alliance défensive, et de les amener à en
conclure une avec le roi d'Espagne, que, pour sa vanité, il voulait
traiter lui-même à Madrid. Il se réjouissait d'espérer que la triple
alliance brouillerait l'Europe, principalement si elle était suivie
d'une ligue avec l'empereur. Il ordonnait à Beretti de déclarer
nettement que l'Espagne prendrait ses mesures, si les Provinces-Unies
traitaient effectivement avec l'empereur. Quelque médiocre cas qu'il fît
de Riperda\,; il le ménageait par l'intérêt commun d'attirer la
négociation à Madrid, lequel de son côté exagérait les plaintes de
l'Espagne, comme si elle eût cru le traité avec l'empereur entamé, et il
se répandit avec ses maîtres en reproches, en avis et en menaces sur
leur conduite avec l'Espagne, qui, comptant sur leur amitié, n'avait
pris des mesures avec aucune puissance, et avait envoyé quatre vaisseaux
à la mer du Sud pour en chasser les Français. Beretti eut ordre en même
temps de protester contre l'alliance que les États généraux feraient
avec l'empereur, et de prendre d'eux son audience de congé dans le
moment que la négociation serait commencée. Albéroni y mêlait ses
plaintes particulières\,; il disait que le roi d'Espagne aurait raison
de lui reprocher la partialité qu'il avait toujours témoignée pour la
Hollande, et les conseils qu'il lui avait toujours donnés de préférer
son alliance à toute autre. Il ajoutait que leur conduite allait
confirmer des bruits fâcheux répandus contre les principaux du
gouvernement, accusés de s'être laissé gagner par trois millions
distribués entre eux par la France, pour traiter avec elle, comme elle
avait fait pour acheter la paix d'Utrecht. Il demandait pourquoi des
ministres infidèles n'étaient pas punis, et c'était pour éviter un tel
inconvénient que le roi d'Espagne voulait traiter à Madrid, comme
quelques particuliers de Hollande, dans la vue de se procurer les mêmes
avantages, voulaient traiter à la Haye\,; que toute idée de négociation
s'évanouirait si la république traitait avec l'empereur.

Beretti eut ordre de s'expliquer dans les termes les plus forts, et de
bien faire entendre que le silence que le roi d'Espagne avait gardé sur
la triple alliance, c'était qu'il n'avait aucun sujet de s'opposer à des
traités entre des puissances amies\,; mais que de leur en voir faire un
avec le seul ennemi qu'il eût, ce traité ne pouvait avoir d'objet que le
préjudice et le dommage de la couronne d'Espagne. Il était pourtant vrai
que cette prétendue tranquillité d'Albéroni sur la triple alliance
n'était que feinte. Il disait que les vues et les agitations du régent
étaient trop publiques pour être ignorées\,; qu'en son particulier, il
n'avait qu'à se louer des nouvelles assurances de l'amitié et de la
confiance la plus intime, que le régent lui avait données par le marquis
d'Effiat et par le P. du Trévoux, avec les plus fortes protestations de
la parfaite opinion de sa probité\,; mais qu'elles ne le rassuraient pas
contre les brouillons dont il était environné, quelque attention qu'il
voulût prendre pour le rendre content de sa conduite. Telles étaient les
impostures et les artificieuses vanteries d'Albéroni.

Toujours inquiet de tous les avis qui pouvaient venir au roi d'Espagne,
il fit donner un ordre positif à tous ses ministres au dehors de ne plus
écrire par la voie du conseil d'État, mais d'adresser à Grimaldo toutes
les dépêches. Encore les voulut-il sèches, et que le véritable compte
des affaires lui fût adressé par dos lettres particulières à lui-même.
Grimaldo avait été présenté au duc de Berwick, en Espagne, pour être son
secrétaire espagnol. Il ne le prit pas, parce que lui-même ne savait pas
un mot d'espagnol alors. Orry, qui savait la langue, le prit, et s'en
accommoda fort, par conséquent la princesse des Ursins. Ce fut où
Albéroni le connut du temps qu'il était en Espagne valet du duc de
Vendôme, et après qu'il l'eut perdu, résident, puis envoyé de Parme.
M\textsuperscript{me} des Ursins chassée, Grimaldo demeura obscur dans
les bureaux, d'où il fut tiré par Albéroni, à mesure qu'il crût en
puissance. Il en fit son principal secrétaire confident pour les
affaires. Ce fut lui avec qui je traitai en Espagne, et que j'y trouvai
le seul ministre avec qui le roi dépêchait. Il n'avait point pris de
corruption de ses deux maîtres. Si je parviens jusqu'au temps d'écrire
mon ambassade, j'aurai beaucoup d'occasion de parler de lui.

Enfin le cardinal del Giudice, ne pouvant plus tenir en Espagne, on
partit le 22 janvier sans avoir pu obtenir la permission de prendre
congé du roi et de la reine. Il alla par la Catalogne s'embarquer à
Marseille, pour se rendre à Rome par la Toscane.

Le délai opiniâtre de la promotion d'Albéroni excita les plaintes les
plus amères du roi et de la reine d'Espagne, et les avis les plus
fâcheux à Aldovrandi en chemin vers l'Espagne. Les agents qu'il y avait
laissés désespéraient qu'on l'y laissât rentrer, et du départ de
l'escadre. Le premier ministre voulait intimider le pape comme le plus
sûr moyen d'accélérer sa promotion, mais il n'avait garde de se
brouiller avec celui dont il attendait uniquement toute sa solide
grandeur, qu'il ne se pouvait procurer par aucun autre. Il sentait aussi
que le roi d'Espagne avait besoin de ménager les favorables dispositions
du pape pour lui, qui disait souvent à Acquaviva qu'il le regardait
comme l'unique soutien de la religion prête à périr en France,
uniquement pour l'intérêt particulier du régent, contradictoire à celui
du roi d'Espagne, tant il était bien informé par Bentivoglio et ses
croupiers.

Acquaviva ne cessait donc d'exhorter le roi d'Espagne de former une
liaison étroite avec le pape pour le bien de la religion. Il disait que
les Français n'avaient pas souffert moins impatiemment que les Allemands
le long séjour d'Aldovrandi à Rome, dans le désir pour l'intérêt
personnel du régent, que la discorde eût duré entre les cours de Rome et
de Madrid\,; qu'on voyait enfin à découvert que la triple alliance était
moins contraire à l'empereur qu'au roi d'Espagne\,; que le pape en avait
fait porter ses plaintes au régent, et chargé son nonce d'engager les
cardinaux de Rohan et de Bissy et les évêques qui avaient le plus de
crédit d'appuyer ses remontrances, même les admonitions que Sa Sainteté
était obligée de lui faire. Elle ne se contenta point de ce que le
cardinal de La Trémoille lui put dire sur la triple alliance. Elle
voulait rassembler plusieurs sujets de plaintes. L'abandon du Prétendant
en eût été un, en forme si elle n'eût pas compris tous les princes
catholiques de l'Europe. Le pape se réduisit à la compassion, et à faire
assurer la reine sa mère qu'il ne l'abandonnerait point, que ses États
lui seraient ouverts, et qu'il souhaitait de l'y pouvoir recevoir et
traiter d'une manière qui répondît à son rang et à sa condition. Rome
était généralement persuadée que la triple alliance avait pour premier
objet de priver le roi d'Espagne de ses droits\,; on y disait tout haut
que trois rois y étaient sacrifiés pour deux injustes successions, l'une
contre la loi divine, l'autre contre la loi de nature. Le pape on était
persuadé. Il déplorait l'état de la religion en France, car la religion
à Rome, l'infaillibilité du pape et toutes les prétentions de cette même
cour n'y sont qu'une seule et même chose. Le pape disait souvent à
Acquaviva qu'il ne voyait d'appui pour elle que le roi d'Espagne, et
qu'il espérait aussi que ce serait par la même main que Dieu la
rétablirait en France dans sa pureté avec les droits de la nature.
Aldovrandi avait ordre de s'expliquer plus clairement sur cette matière
importante lorsqu'il serait arrivé à la cour d'Espagne. Il avait reçu
les instructions et les pouvoirs nécessaires pour terminer les
différends des doux cours à leur satisfaction commune. Le pape, désireux
de lier une étroite union avec le roi d'Espagne, et persuadé que le
grand point dos différends était les biens patrimoniaux mis sous le nom
d'ecclésiastiques pour les affranchir de tout par l'immunité
ecclésiastique et les contributions du clergé des Indes, avait laissé
pouvoir à Aldovrandi d'étendre les facultés qu'il lui avait données, et
de se relâcher autant qu'il le verrait nécessaire pour la satisfaction
de la cour d'Espagne, et de se bien concerter avec le duc de Parme, en
passant à Plaisance pour assurer le succès de sa commission.

Ce nonce exposa donc ses instructions au duc de Parme\,; ils convinrent
que, puisque le pape ne voulait point accorder l'imposition perpétuelle
sur le clergé, le roi d'Espagne devait se contenter d'une imposition à
temps, fondé sur l'exemple des premières de cette sorte, qui peu à peu
s'étaient augmentées, et étaient enfin doyennes perpétuelles, comme ces
nouvelles seraient conduites par même voie à même fin\,; surtout
d'éviter que cette affaire fût remise à une junte, toujours plus occupée
de durer et de former des difficultés que de les aplanir, et de se tirer
de l'exemple des congrégations par dire que le pape n'en avait fait une
là-dessus que pour s'autoriser contre l'opinion de plusieurs qui ne
voulaient point d'accommodement. À l'égard du principal moyen, qui était
de choses secrètes que le nonce se réservait à lui-même, et qui très
vraisemblablement regardait la succession possible de France, il est
incertain si Aldovrandi les confia au duc de Parme, mais on sut
certainement que ce prince n'oublia rien pour convaincre Albéroni de la
nécessité de répondre aux bonnes dispositions du pape, de former avec
lui des liaisons stables et perpétuelles, et qu'on général il y avait
lieu d'espérer encore plus pour l'avenir.

Le personnel d'Albéroni ne fut pas oublié dans ces conférences.
Aldovrandi proposa au duc de Parme de commettre quelque personne
d'autorité à Rome pour y solliciter la promotion d'Albéroni, qui ne
dépendait, suivant les assurances du nonce, que du succès de
l'accommodement\,; et s'il pouvait en arrivant à Madrid promettre
positivement au pape la conclusion des différends entre les deux cours,
la promotion se ferait à l'arrivée du courrier qu'il dépêcherait à Rome.
Ensuite le duc de Parme pensa à soi\,; il était fort inquiet d'une
prétendue négociation qu'on disait que le pape conduisait entre
l'Espagne et l'empereur. Un petit prince tel que lui avait fort à se
ménager pour ne pas irriter une puissance telle que celle de l'empereur,
et ne pas perdre sa considération en Italie en perdant son crédit en
Espagne. Il avait recours aux conseils d'Albéroni pour se conduire dans
une conjoncture si délicate. Il comptait également sur son appui et sur
celui de la reine d'Espagne, dont il craignit les bizarreries et la
facilité à se fâcher, qu'elle faisait souvent sentir au duc et même à la
duchesse de Parme qui de son côté n'était pas moins impérieuse que la
reine sa fille. Son prodigieux mariage, qui lui avait fait oublier sa
double bâtardise du pape Innocent III\footnote{Le manuscrit de
  Saint-Simon porte Innocent III, mais c'est une erreur évidente pour
  Paul III. En effet, Pierre-Louis Farnèse, premier duc de Parme et
  Plaisance était fils naturel du pape Paul III. Octave Farnèse, fils et
  héritier de Pierre-Louis, épousa Marguerite, fille naturelle de
  Charles-Quint. Ainsi s'explique la double bâtardise dont parle
  Saint-Simon.} et de l'empereur Charles-Quint, lui fit trouver fort
étrange que le duc de Parme eût osé sans sa participation écouter des
propositions de mariage pour le prince Antoine son frère avec une fille
du prince de Liechtenstein et deux millions de florins de dot. Le duc de
Parme eut beaucoup de peine à l'apaiser et n'osa achever ce mariage.

Les ministres d'Angleterre étaient alarmés aussi de ces bruits d'un
traité ménagé par le pape entre l'empereur et le roi d'Espagne. Le roi
d'Angleterre voulait conserver son crédit en Espagne, pour s'autoriser
en Angleterre. Stanhope écrivit confidemment à Albéroni que les
ambassadeurs de Franco lui avaient parlé à la Haye des bruits de ce
traité\,; il lui mandait quo si le roi d'Espagne désirait effectivement
de faire la paix avec l'empereur, l'Angleterre et la Hollande lui
offriraient non seulement leur médiation, mais encore leur garantie du
traité, engagement que la faiblesse, le caractère et l'éloignement du
pape ne lui pouvaient laisser prendre, et que les deux nations
exécuteraient aisément. Il offrait encore les mesures nécessaires pour
empêcher l'empereur de s'emparer des États du grand duc. Albéroni
répondit que le roi d'Espagne était très sensible à ces propositions,
qu'il ne croyait pas que le pape eût entamé rien à Vienne, que Sa
Majesté Catholique ne s'éloignerait jamais de contribuer à mettre
l'équilibre dans l'Europe, et qu'en toutes occasions elle donnerait des
marques de sa modération.

Albéroni voulait voir de quelle manière Stanhope s'expliquerait sur
cette réponse générale. Beretti avait déjà donné le même avis du
prétendu traité par le pape, mais sans parler des ambassadeurs de
France, circonstance essentielle en toute affaire où l'Espagne prenait
quelque intérêt. Albéroni disait que le principal embarras pour le roi
d'Espagne était à l'égard des futurs contingents, véritable centre où
tendaient toutes les lignes qu'on tirait de tous les côtés, qu'il ne se
mettait point en peine des alliances, parce que Riperda l'assurait que
les Hollandais n'en feraient point avec l'empereur\,; que le roi
d'Espagne savait que les Anglais voulaient s'allier avec lui, et que,
comme il savait aussi qu'il n'y avait rien de la prétendue négociation
du pape à Vienne, il voulait mûrement examiner les conditions et les
engagements à prendre et à demander dans les traités à conclure avec
l'Angleterre et la Hollande. Beretti était lors celui de tous ceux quo
l'Espagne employait au dehors qui avait le plus la confiance
d'Albéroni\,; il en eut ordre de dresser un projet le plus convenable
qu'il jugerait pour servir de règle à la négociation quo l'Espagne
voulait faire avec la Hollande et l'Angleterre. Albéroni y voulait un
grand secret et la diriger lui-même. Il avait persuadé à Leurs Majestés
Catholiques que cette négociation ayant une liaison nécessaire avec les
événements qui pouvaient arriver en France, il n'y avait que lui seul
qui dût en avoir la confiance\,; qu'il fallait se défier de tout
Espagnol, qui tous auraient des motifs particuliers de se conduire
contre les intentions et l'intérêt du roi d'Espagne.

Ce prince ennuyé de la lenteur des États généraux à se déterminer sur
l'alliance qu'il leur avait fait proposer et des bruits qui couraient de
leur dessein de traiter avec l'empereur, dit à leur ambassadeur qui le
suivait à sa promenade dans les jardins du Retiro, qu'il ne pouvait
comprendre l'empressement que ses maîtres témoignaient de s'allier avec
le seul ennemi qu'il eût, sans se souvenir de toutes les démarches qu'il
avait faites pour les convaincre de son amitié, jusqu'à se porter
aveuglément à tout ce qu'ils avaient voulu\,; et comme les expressions
latines lui étaient familières, il ajouta celles-ci\,: \emph{Patientia
fit tandem furor}. Riperda venait alors de recevoir des ordres de sa
république qui protestait de son intention d'entretenir une vraie bonne
intelligence avec le roi d'Espagne, et de lui donner en toutes occasions
des témoignages de leur respect. Il s'en servit dans sa réponse qui
apaisa le roi d'Espagne.

\hypertarget{chapitre-xi.}{%
\chapter{CHAPITRE XI.}\label{chapitre-xi.}}

1717

~

{\textsc{Le roi d'Angleterre à Londres.}} {\textsc{- Intérieur de son
ministère.}} {\textsc{- Ses mesures.}} {\textsc{- Gyllembourg, envoyé de
Suède, arrêté.}} {\textsc{- Son projet découvert.}} {\textsc{- Mouvement
causé par cette action parmi les ministres étrangers et dans le
public.}} {\textsc{- Mesures du roi d'Angleterre et de ses ministres.}}
{\textsc{- L'Espagne, à tous hasards, conserve des ménagements pour le
Prétendant.}} {\textsc{- Castel-Blanco.}} {\textsc{- Le roi de Prusse se
lie aux ennemis du roi d'Angleterre.}} {\textsc{- Les Anglais ne veulent
point se mêler des affaires de leur roi en Allemagne.}} {\textsc{-
Goertz arrêté à Arnheim et le frère de Gyllembourg à la Haye, par le
crédit du Pensionnaire.}} {\textsc{- Sentiment général des Hollandais
sur cette affaire.}} {\textsc{- Leur situation.}} {\textsc{- Entrevue du
Prétendant, passant à Turin, avec le roi de Sicile, qui s'en excuse au
roi d'Angleterre.}} {\textsc{- Cause de ce ménagement.}} {\textsc{-
Réponse ferme de Goertz interrogé en Hollande.}} {\textsc{- L'Angleterre
et la Hollande communiquent la triple alliance au roi d'Espagne.}}
{\textsc{- Soupçons, politique et feinte indifférence de ce monarque.}}
{\textsc{- Mauvaise santé du roi d'Espagne.}} {\textsc{- Burlet, premier
médecin du roi d'Espagne, chassé.}} {\textsc{- Craintes de la reine
d'Espagne et d'Albéroni.}} {\textsc{- Ses infinis artifices pour hâter
sa promotion.}} {\textsc{- Clameurs de Giudice contre Aldovrandi,
Albéroni et Aubenton.}} {\textsc{- Angoisses du pape entraîné enfin.}}
{\textsc{- Il déclare Borromée cardinal seul et sans ménagement pour
Albéroni.}} {\textsc{- Mesures et conseils d'Acquaviva et d'Alexandre
Albani à Albéroni.}} {\textsc{- Nouveaux artifices d'Albéroni pour hâter
sa promotion, ignorant encore celle de Borromée.}} {\textsc{- Albéroni
fait travailler à Pampelune et à la marine\,; fait considérer
l'Espagne\,; se vante et se fait louer de tout\,; traite froidement le
roi de Sicile\,; veut traiter à Madrid avec les Hollandais.}} {\textsc{-
Journées uniformes et clôture du roi et de la reine d'Espagne.}}
{\textsc{- Albéroni veut avoir des troupes étrangères\,; hait
Monteléon.}} {\textsc{- Singulière et confidente conversation de
Stanhope avec Monteléon.}} {\textsc{- Dettes et embarras de
l'Angleterre.}} {\textsc{- Mesures contre la Suède.}} {\textsc{-
Conduite d'Albéroni à l'égard de la Hollande.}} {\textsc{- Le
Pensionnaire fait à Beretti une ouverture de paix entre l'empereur et le
roi d'Espagne.}} {\textsc{- L'Angleterre entame une négociation à Vienne
pour la paix entre l'empereur et le roi d'Espagne.}} {\textsc{- Lettre
de Stanhope à Beretti, et de celui-ci à Albéroni.}} {\textsc{- Son
embarras.}} {\textsc{- Ordres qu'il en reçoit et raisonnement.}}
{\textsc{- Vues et mesures de commerce intérieur et de politique au
dehors d'Albéroni.}} {\textsc{- Angoisses du roi de Sicile éconduit par
l'Espagne.}} {\textsc{- Venise veut se raccommoder avec le roi
d'Espagne.}}

~

Le roi d'Angleterre, en arrivant à Londres, avait donné ses premiers
soins à réunir ses principaux ministres qui ne songeaient qu'à
s'entre-détruire. Towsend avait promis d'accepter la vice-royauté
d'Irlande, et d'y demeurer trois ans si le roi ne le rappelait
auparavant\,; Mothwen avait été fait second secrétaire d'État. Le
département du sud lui avait été donné, quoique ce fût celui du premier,
pour laisser le nord à Stanhope et le soin des affaires d'Allemagne, qui
touchaient le roi d'Angleterre bien plus quo toutes les autres par
rapport à ses États patrimoniaux. Le parlement avait été prorogé
jusqu'au 20 février (vieux style), pour avoir le temps de disposer la
nation à la conservation des troupes, dont on ne serait pas venu à bout
si les ministres qui venaient de découvrir le projet des ministres de
Suède n'eussent fait alors éclater la conspiration. Gyllembourg, envoyé
de Suède, fut arrêté dans sa maison à Londres, le 9 février à dix heures
du soir. Vingt-cinq grenadiers posés à sa porte eurent ordre d'empêcher
que personne pût lui parler\,: on rompit ses cabinets et ses coffres\,;
ses papiers furent enlevés sans inventaire et sans scellé\,; on répandit
dans le public que le complot avait été découvert par trois lettres que
Goertz écrivait à Gyllembourg, avec ses réponses, et le chiffre dont ils
se servaient\,; qu'on y avait vu le projet d'une descente à faire en
Écosse\,; quo Goertz avait déjà touché cent mille florins en Hollande,
depuis dix mille livres sterling à Paris\,; que Gyllembourg avait reçu
vingt mille livres sterling à Londres.

Presque tous les ministres étrangers qui étaient à Londres sentiront les
conséquences de cet arrêt pour leur propre sûreté, et s'assemblèrent
chez Monteléon, ambassadeur d'Espagne, pour en délibérer. Ils convinrent
que le droit des gens était violé, principalement par l'enlèvement des
papiers de l'envoyé de Suède\,; mais n'ayant point d'ordres de leurs
maîtres chacun craignit de prendre un engagement, et ils concluront à
attendre les éclaircissements que le gouvernement d'Angleterre avait
promis de donner. Monteléon, moins content du ministère d'Angleterre
qu'il ne l'avait été autrefois, fut moins discret\,; il discourut sur ce
que le projet paraissait peu vraisemblable, qu'il y aurait peut-être
quelque idée particulière de Gyllembourg sans rien de réel ni de
concerté\,; que le roi d'Angleterre avait un pressant intérêt d'engager
la nation Anglaise à déclarer la guerre au roi de Suède, et à contribuer
à l'entretien des troupes et à l'armement des vaisseaux\,; que ce ne
serait pas la première fois qu'une conjuration, révélée au parlement au
commencement de ses séances, aurait produit des effets merveilleux pour
les volontés de la cour. Ces propos, qu'il croyait tenir sûrement à des
amis dans un intérêt commun, lui attirèrent une espèce de reproche des
ministres d'Angleterre, et Stanhope lui dit qu'il était fâché qu'il eût
désapprouvé ce qui s'était passé à l'égard de l'envoyé de Suède, mais
qu'ils espéraient qu'il changerait de sentiment quand il en saurait le
motif. En attendant de satisfaire la curiosité générale, les ministres
d'Angleterre laissaient répandre que les ducs d'Ormont et de Marr,
chargés de conduire le débarquement, étaient déjà dans le royaume. Sur
ces bruits et sur les preuves que le gouvernement promettait de publier
incessamment, tout devenait facile au roi, et il armait sans peine
trente navires, dont quinze étaient destinés pour la mer Baltique.

Quelques protestations d'intelligence et d'amitié qu'il y eût entre les
cours de Londres et de Madrid, cotte dernière ne laissait pas d'avoir
des ménagements pour le Prétendant. Le marquis de Castel-Blanco, dont le
nom était Rojas, et qui était des Asturies, avait épousé une fille du
duc de Melfort. Il s'était dévoué au Prétendant pour lequel il avait
dépensé de grandes sommes qu'il avait rapportées des Indes. Le
Prétendant l'avait fait duc en sortant d'Avignon, et le roi d'Espagne y
avait consenti avec la condition du secret, jusqu'au rétablissement de
ce prince sur le trône de ses pères ainsi, l'union n'empêchait pas le
roi d'Espagne de regarder comme très possible une révolution en
Angleterre, et peut-être prochaine, ce que bien des gens dans Londres
pensaient aussi. Le gouvernement, appliqué à faire connaître le crime de
Gyllembourg, désirait d'en faire un exemple en sa personne, et consulta
des juges pour savoir si le caractère public empêchait qu'on lui pût
faire son procès. L'animosité était pareille à l'intérêt du roi, comme
duc d'Hanovre, de faire déclarer la guerre à la Suède par les Anglais,
et à celui de ses ministres blâmés par le parti opposé, comme d'une
violence extravagante, et dont les découvertes ne répondaient ni à
l'éclat ni à l'attente du public.

Le roi d'Angleterre, qui prévoyait des suites, augmenta les troupes
qu'il entretenait pour la conservation de ses États en Allemagne\,: ce
n'était pas qu'il eût rien {[}à{]} y craindre de la part du roi de
Suède, qui avait perdu tout ce qu'il y possédait, et {[}était{]} très
pauvrement renfermé dans ses anciennes bornes. Mais le roi de Prusse,
gendre du roi d'Angleterre, piqué de sa froideur et de ses mépris, était
devenu son plus mortel ennemi. Il s'unissait étroitement avec le czar
qui était irrité au dernier point contre le roi d'Angleterre Le roi de
Prusse voulait la paix avec la Suède, pourvu que le Danemark, son allié,
y fût compris. Il sentait que l'intervention de la France en était la
voie la plus sûre. Il craignait en même temps l'union nouvellement
resserrée entre l'Angleterre et le régent, et il tâchait de l'affaiblir,
en avertissant ce dernier de la liaison intime dont le roi d'Angleterre
se vantait d'être avec l'empereur\,; et priait le régent de faire ses
réflexions là-dessus. Le czar, personnellement piqué contre le roi
d'Angleterre, ne se pressait point de tenir la parole qu'il avait donnée
de faire sortir ses troupes du pays de Mecklenbourg, et toutes ces
considérations éloignaient les Anglais de se mêler des affaires de leur
roi en Allemagne, où ils jugeaient qu'il en aurait beaucoup sur les
bras, et leur persuadaient de laisser à Bernstorff, seul auteur de la
violence exercée contre Gyllembourg, le soin de tirer son maître de
l'engagement où il l'avait jeté mal à propos. Les ministres Anglais
pensaient à peu près de même, et abandonnaient Bernstorff\,; et les amis
du roi de Suède, qui en avait beaucoup à Londres, l'exhortaient à
distinguer le roi et la nation, et de déclarer dans un manifeste qu'il
ne considérait que le duc d'Hanovre dans ce qui s'était passé, dont il
appelait aux deux chambres du parlement.

Quoique la Hollande n'approuvât point cette violence, Heinsius, toujours
attaché au roi d'Angleterre par ses anciennes liaisons, avait eu le
crédit aux États généraux de faire arrêter le baron de Goertz, ministre
du roi de Suède, à Arnheim, et le frère de Gyllembourg, à la Haye.
Slingerland, au contraire, traitait l'action de Londres d'attentat au
droit des gens, et, parlant à Beretti, blâma Stanhope d'avoir, dans sa
lettre circulaire aux ministres étrangers résidant à Londres, marqué que
la révolte serait appuyée d'un secours de troupes, parce que, les
troupes ne marchant que sur les ordres du souverain, c'était avouer que
l'envoyé de Suède était autorisé de son maître, et rendre ainsi
l'affaire personnelle au roi de Suède, rendre innocent son envoyé,
n'agissant que sur ses ordres, et ne laisser plus de doute à l'attentat
au droit des gens. On croyait en Hollande que ce qui avait le plus
engagé le roi d'Angleterre à demander aux États généraux de faire
arrêter Goertz, était l'opinion qu'il traitait la paix de la Suède avec
le czar. On disait même que la condition en était la restitution de
toutes les conquêtes du czar sur la Suède, excepté Pétersbourg et son
territoire, et que ce prince donnerait une de ses filles au jeune duc de
Holstein. L'empereur désirait ardemment la paix du nord, et les
Hollandais pour le moins autant, pour leur commerce et pour affermir la
paix dans toute l'Europe. Leurs dettes étaient immenses\,; la nécessité
d'épargner les avait obligés à une grande réforme de troupes, et à
manquer à la parole qu'ils avaient donnée, pendant la dernière guerre à
MM. de Berne de conserver en tout temps vingt-quatre compagnies de leur
canton. Ils avaient réformé trois mille Suisses. Les troupes qu'ils
avaient conservées se montaient à vingt-huit mille hommes d'infanterie,
deux mille cinq cents de cavalerie et quinze cents dragons\,; ce qui
leur parut suffisant dans un temps où ils ne voyaient plus de guerre
prochaine, surtout depuis la dernière liaison de la France avec
l'Angleterre, et le départ du Prétendant d'Avignon pour se retirer en
Italie.

Lorsque ce prince approcha de Turin, le roi de Sicile lui envoya le
marquis de Caravaglia et une partie de sa maison pour le recevoir et le
traiter. Il entra dans Turin, vit incognito le roi et la reine de
Sicile, et le prince de Piémont\,; demeura quelques heures dans la ville
sans cérémonies, et continua son chemin. Ce passage avait fort
embarrassé le roi de Sicile. Sa proche parenté avec le Prétendant, et
les droits qu'il en tirait dans l'ordre naturel pour la succession
d'Angleterre, ne lui permettaient pas de refuser passage à ce prince,
par conséquent {[}de refuser{]} de le faire recevoir et de le voir. Il
craignait de mécontenter l'Angleterre\,; il n'espérait que du roi
Georges son accommodement avec l'empereur. Trivié, son ambassadeur à
Londres, l'avait flatté que ce prince lui garantirait la Sicile\,; mais
quand son successeur La Pérouse en parla à Stanhope, celui-ci lui nia le
fait. Il lui dit que si le roi d'Angleterre se portait à lui garantir
les traités antérieurs à celui d'Utrecht, jamais il n'irait au delà, ni
à aucune garantie pour la Sicile\,; que l'empereur ne voulait entendre
parler de rien avant que la Sicile lui fût restituée\,; que le prince
Eugène même, si porté pour le chef de sa maison, s'expliquait que rien
ne se pouvait traiter sans cela. Ainsi le roi de Sicile, bien instruit
des volontés fixes de l'empereur, n'espérait se rapprocher de lui que
par le roi d'Angleterre, qu'il ménageait, par cotte raison, plus
qu'aucune autre puissance. Il n'oublia donc rien pour se justifier
auprès de lui à l'égard du Prétendant.

Le roi d'Angleterre reçut assez bien ses excuses, peut-être par la
conjoncture de l'embarras de l'affaire des ministres de Suède, et la
crainte où il était du nombre et de la force des jacobites, et de la
réponse de Goertz à l'interrogation qu'il avait subie en Hollande. Il
avait déclaré qu'il avait dressé un projet, approuvé par le roi son
maître, pour faire la guerre au roi d'Angleterre, son ennemi découvert,
mais une bonne guerre sans trahison\,; qu'à son égard, il n'avait à
répondre qu'au roi de Suède. Une flotte de charbon venant d'Écosse
effraya Londres, dans la fin de février. Le bruit s'y répandit qu'on
voyait trente vaisseaux du roi de Suède\,; rien n'était encore préparé
pour s'opposer à une descente, et l'alarme fut grande jusqu'à ce qu'on
eût bien reconnu que ce n'était que des charbonniers.

L'Angleterre et la Hollande ménageaient toujours le roi d'Espagne. À
l'imitation de la Franco, ils lui communiquèrent le traité de la triple
alliance. Ce monarque soupçonnait des articles secrets que le régent y
aurait fait mettre, et qui étaient la vraie substance du traité. Mais il
avait au dedans et au dehors trop d'intérêt à cacher ses pensées de
retour au trône de ses pères, pour ne pas montrer la plus entière
indifférence, qui fit douter en effet s'il s'intéressait à la ligue qui
venait de se conclure, et {[}fit{]} qu'on crut généralement en Espagne
et parmi les étrangers qu'il portait toutes ses vues sur l'Italie, et à
recouvrer une partie de ce qu'il y avait perdu. On en jugeait par
l'intérêt de la reine, qu'Albéroni en avait tant à servir, et par son
impatience de terminer tous les différends avec Rome. Il ne laissait pas
de s'y montrer ralenti par les délais de sa promotion, que la reine
irritée regardait, disait-il, comme un mépris pour elle, et qu'elle
sentait moins son affection pour un sujet qui lui était dévoué, que par
l'empressement, né des conjonctures, d'aimer celui en qui elle avait mis
toute sa confiance, d'une supériorité de représentation qui le mît en
état de la servir sans ménagement dans les occasions scabreuses dont
elle se voyait menacée. Cela désignait les vapeurs noires du roi
d'Espagne, retombé depuis peu dans une maigreur et une mélancolie qui
faisaient craindre la phtisie, et que sa vie ne fût pas longue.

Burlet, son premier médecin, fut chassé d'Espagne un mois après ces
derniers accidents, pour s'en être trop librement expliqué. Les suites
en étaient fort à craindre pour la reine si haïe des Espagnols, et pour
les étrangers qui ne tenaient rien que d'elle\,; mais le péril était
extrême pour Albéroni, parce que, maître de tout sous elle, il était en
butte à la jalousie et à la haine universelle, et que, n'ayant point
d'établissement, sa chute ne pouvait être médiocre. Il avait persuadé la
reine qu'il y allait de tout son honneur à elle, et que ce lui serait la
dernière injure, qu'après toutes les promesses du pape, une ombre de
protection de l'empereur élevât Borromée à la pourpre, en négligeant son
plus intime serviteur, pour lequel elle avait encore, en dernier lieu,
écrit de sa main, en termes si forts, qu'elle n'en pouvait employer de
plus pressants pour demander à Dieu le paradis. En même temps,
connaissant bien le pouvoir de la crainte sur le pape, il fit donner
ordre à Daubenton, par le roi d'Espagne, d'écrire à Aldovrandi que si la
reine n'était pas promptement satisfaite, ni lui ni Alexandre Albani
n'obtiendraient point la permission de venir à Madrid.

Albéroni comptait se cacher ainsi, et faire valoir son entière
soumission aux volontés du pape sans aucune impatience, et qu'il
regardait comme le dernier des malheurs d'être la cause éloignée de la
moindre brouillerie entre les deux cours, tandis qu'il ne laissait
échapper aucune occasion, ni aucune circonstance de l'intérêt, de la
volonté, de la vivacité de la reine. Il fortifiait ces artifices de la
peinture la plus avantageuse de l'état où il avait mis l'Espagne, tel
qu'elle pouvait se rire de ses ennemis, reconnaître les bienfaits, et se
venger de ceux dont il ne serait pas content. Ainsi, rien à espérer pour
Aldovrandi ni pour don Alexandre, pas même la permission d'aller à
Madrid, s'ils n'apportaient la satisfaction des désirs de la reine,
comme, au contraire, tout aplani en rapportant. Il protestait qu'il
n'oserait plus ouvrir la bouche là-dessus\,; que la reine lui avait déjà
reproché que six mois plus ou moins lui étaient indifférents, tandis que
son honneur était en continuel spectacle d'un mépris pour elle si
insupportable\,; que le roi et elle avaient fort approuvé les nouvelles
instances qu'Acquaviva avait faites à l'occasion de la mort du cardinal
del Verme, et qu'ils étaient l'un et l'autre certainement déterminés à
rejeter toute proposition de Rome, si la grâce qu'ils avaient demandée
n'était auparavant accordée. Le dernier courrier avait porté au cardinal
Acquaviva des ordres dressés dans cet esprit, et menaçants pour le pape.
Néanmoins Albéroni voulait ménager les parents du pape\,; il pensait à
faire donner, par le roi d'Espagne, une pension au cardinal Albani,
qu'il savait, par Acquaviva, disposé à la recevoir. Il se voulait ainsi
réserver les grâces, et laisser au contraire au roi d'Espagne les
démonstrations et les effets de rigueur. Aldovrandi, informé en chemin
de la colère de la reine par Aubenton, craignit pour sa fortune une
rupture ouverte entre les deux cours. Le confesseur lui avait mandé que
la reine ordonnerait peut-être à Acquaviva de se désister de sa demande.
C'était fermer au prélat la nonciature, par conséquent le chemin au
cardinalat. Il écrivit donc à Albéroni que ce serait donner à rire à ses
envieux, et tout ce qu'il jugea le plus propre à lui en faire craindre
l'événement et à lui faire prendre patience.

Le pape, impatient de l'arrivée de l'escadre d'Espagne dans les mers
d'Italie, et facilement épouvanté par les Vénitiens, qui lui
représentaient les Turcs prêts d'en envahir ce qu'ils voudraient, avait
trouvé son nonce trop lent en sa route, mais toutefois sans pouvoir se
résoudre à la promotion d'Albéroni, sans être sûr de l'accommodement de
ses différends avec l'Espagne, suivant le projet qu'il en avait fait. Un
des principaux moyens que ses amis avaient imaginé était de procurer à
don Alexandre. Albani le voyage d'Espagne, pour y signer l'accommodement
qu'Aldovrandi aurait dressé suivant les intentions du pape. Don
Alexandre désirait avec passion cet honneur depuis longtemps. La
princesse des Ursins, et Albéroni après elle, s'y étaient toujours
opposés\,; enfin le dernier y avait consenti, et permis à Acquaviva d'en
parler au pape. Il le fit dans un temps où don Alexandre était à la
campagne. À son retour le pape lui en dit un mot, et remit à une autre
fois à lui en parler plus au long. Il parut que ces délais étaient un
peu joués entre l'oncle et le neveu. Le pape s'était engagé à l'envoyer
nonce extraordinaire à Vienne porter les langes bénits au prince dont
l'impératrice accoucherait. Mais ce prince étant mort avant que la
fonction eût été exécutée, le cardinal Albani, dévoué à la maison
d'Autriche, prétendit que le même engagement subsistait, et soit que ce
fût de concert ou de jalousie, le pape trouva des difficultés
insurmontables au voyagé de don Alexandre à Madrid. Albéroni se vit
ainsi privé des avantages de traiter et de terminer avec le neveu du
pape les différends entre les deux cours. Il trouva encore d'autres
traverses.

Le cardinal del Giudice, avant d'arriver à Rome, la remplissait de ses
plaintes contre Aldovrandi, et demandait des réparations des discours
qu'il avait tenus contre son honneur. Il avertissait le pape de ses
fourberies et de celles d'Aubenton et d'Albéroni qu'il accablait de
railleries piquantes, et le représentait comme ne pouvant maintenir
longtemps sa faveur\,; qui était le meilleur moyen de nuire à sa
promotion, et c'était aux cardinaux Albani et Paulucci à qui il
s'adressait. Le pape se trouvait en d'étranges angoisses. La maison
Borromée le pressait pour son maître de chambre, dont le neveu avait
épousé sa nièce, et dont la promotion avait été arrêtée par Acquaviva le
matin même qu'elle allait être faite.

Le pape comprenait quelle colère cette promotion allumerait en
Espagne\,; il craignait mortellement que l'escadre espagnole n'en fût
arrêtée, et de voir l'Italie exposée aux Turcs. Néanmoins il fallut
céder à ses neveux\,: Borromée fut déclaré cardinal le 16 mars, et le
pape ne donna pas même la satisfaction à Albéroni de lui faire espérer
le second chapeau qui vaquerait, ni de le réserver \emph{in petto}. Rien
n'était plus contraire aux espérances qu'Acquaviva avait données à
Albéroni de sa promotion certaine et prochaine. Ce cardinal fit savoir
au duc de Parme par un courrier la promotion unique de Borromée, en le
priant d'en dépêcher un en Espagne pour y porter cette fatale nouvelle.
En même temps il écrivit à Albéroni qu'il savait que le pape le ferait
cardinal s'il voulait dépêcher un courrier portant parole positive que
le roi d'Espagne mettrait Aldovrandi en possession de toutes les
prérogatives de la nonciature, et qu'il enverrait incessamment son
escadre en Levant pour agir contre les Turcs\,; que le lundi d'après
l'arrivée du courrier le pape tiendrait un consistoire, dans lequel il
conférerait la seule place vacante à Albéroni, mais qu'il fallait se
presser et n'attendre pas d'autres vacances, qui donneraient lieu au
pape de se trouver embarrassé par d'autres demandes, et par les
couronnes, enfin que le pape se contenterait de deux lignes de la main
du roi d'Espagne, qui confirmeraient ces promesses. Don Alexandre voulut
aussi justifier à Albéroni la promotion de Borromée. Il la maintint
indispensable et sans préjudice pour Albéroni. Il devait regarder ce
délai, non comme exclusion, mais comme un effet malheureux de la
contrainte du pape, qui ne voulait pas s'exposer à une compensation que
les couronnes lui demanderaient pour le chapeau accordé à l'Espagne\,;
mais que le prétexte sûr de le tirer de cet embarras, serait le service
signalé rendu à l'Église par l'accommodement des différends des deux
cours, et l'envoi de l'escadre contre les Turcs. C'est ainsi que Rome
sait profiter de l'ambition des ministres, et les gagner par l'appât
d'une dignité étrangère. Don Alexandre qui n'avait pas abandonné
l'espérance de sa mission en Espagne, n'épargna pas les protestations
d'attachement pour Leurs Majestés Catholiques et de respect pour leur
premier ministre. Il y avait déjà quelque temps qu'il regardait sa
promotion comme sûre, qu'il en attendait la nouvelle avec impatience,
sans cesser de la faire presser par la reine, et d'en faire l'affaire
particulière de cette princesse. Comme la difficulté principale était la
défiance réciproque, que le pape voulait être satisfait avant la
promotion, et qu'Albéroni, au contraire, voulait que sa promotion
précédât la satisfaction du pape\,: il représentait de la part de la
reine au duc de Parme, son principal agent dans cette affaire à Rome,
deux raisons invincibles qui engageaient la reine à vouloir que sa
promotion précédât la satisfaction du pape le point d'honneur était la
première, l'autre était d'empêcher les Espagnols de dire que la
promotion d'Albéroni serait la condition secrète d'un accommodement
préjudiciable au roi et au royaume d'Espagne. Il voulait que sa
promotion ne parût fondée que sur la reconnaissance de tout ce que la
reine avait fait en faveur du saint-siège, qu'il rappelait en détail,
ainsi que la montre du secours maritime qu'il étalait aux yeux du pape,
et qu'il promettait d'envoyer d'abord après sa promotion, et la reine,
de terminer en même temps les différends des deux cours, mais pas un
clou sans sa promotion. C'était ses termes, mais toujours désintéressé
et se couvrant du voile du caractère de la reine.

Comme il ne craignait point d'être contredit en rien, et qu'il était
maître de faire parler la reine comme il voulait, il chargea le duc de
Parme de se porter pour garant au pape de sa totale satisfaction, au
moment que la promotion serait faite. Il en fit en même temps assurer
directement le pape par Acquaviva, mais avec un mélange de menaces. Tout
de suite il avertit Aldovrandi qu'il serait mal reçu s'il s'avançait
sans la nouvelle de sa promotion, et dépêcha un courrier pour le retenir
sur la frontière du royaume. Mais dans l'incertitude de sa route, qui
lui pouvait faire manquer le courrier, il fit résoudre le roi d'Espagne
que, si Aldovrandi arrivait à Madrid, il lui serait fixé un terme pour
en sortir. Parmi toutes ces mesures, c'était toujours la même fausseté.
Il protestait un désintéressement parfait\,; sa promotion ne servirait
jamais de condition honteuse à raccommodement\,; il ne voulait pas être
cardinal aux dépens de la réputation de la reine\,; que cette princesse,
en lui procurant cet honneur, joignait à la satisfaction de l'élever des
vues bien plus considérables que le roi et elle voulaient faire tomber
un chapeau sur celui qu'elles honoraient de toute leur confiance,
dépositaire de tous leurs secrets, le seul qui les pût servir en des
événements de la dernière importance\,; mais que, puisque le pape,
nonobstant le besoin qu'il avait de leur secours, témoignait tant de
répugnance, elles n'avaient d'autre parti à prendre que celui de se
désister d'une telle demande, et de regarder comme un affront la
préférence donnée à l'empereur, et les ménagements pour un sujet tel que
Borromée. Il ajoutait qu'en la place du roi d'Espagne, il mépriserait
également toutes les concessions sur le clergé, dont il ne retirerait
jamais qu'une modique somme, après avoir défalqué ce que la nécessité et
l'usage en déduisait\,; que c'était demander l'aumône à une cour
orgueilleuse qui la faisait tant valoir, et s'en rendre esclave pour
chose qui était due en justice rigoureuse\,; qu'il n'y avait qu'une
bonne règle à établir aisément dans les Indes pour se passer des
subsides du clergé, par conséquent de tout accommodement avec Rome, qui
souffrirait bien plus que l'Espagne de la prolongation des différends,
qui certainement ne seraient point terminés que la promotion n'eût
précédé. Il observait que le pape était bien mal conseillé de faire un
si grand tort à la religion, dont la défense à tous égards semblait
réservée au roi d'Espagne, ayant lieu de s'assurer qu'en usant
généreusement envers la reine, elle y saurait répondre avec usure. La
reine accoucha d'un cinquième prince, qui mourut bientôt après.

Albéroni crut que l'Espagne devait se fortifier du côté de la France\,;
il fit travailler à Pampelune. Il compta y avoir tout achevé dans le
courant de l'année et y mettre cent cinquante pièces de canon. Il
travaillait en même temps aux ports de Cadix et de Ferrol, en Galice,
dont les ouvriers étaient exactement payés. Il comptait avoir en mer
vingt-quatre vaisseaux vers le 15 mai. On en construisait un en
Catalogne de quatre-vingts pièces de canon, qui devait être prêt à la
fin d'avril\,; enfin les puissances étrangères commençaient à chercher
avec empressement l'Espagne. Il y en avait qui s'inquiétaient des bruits
répandus depuis quelque temps de négociations commencées entre
l'empereur et le roi d'Espagne. Albéroni avait averti les ministres
d'Espagne au dehors, de n'avoir aucune inquiétude de tout ce qui s'en
pourrait débiter. Le roi de Sicile, toujours mal avec l'empereur,
craignait d'en être exclu. Le moyen sûr d'y être compris, s'il s'en
faisait un, était de l'être dans tous les traités que ferait le roi
d'Espagne. Il donna donc ordre à son ambassadeur à Madrid de le faire
comprendre dans le traité dont il s'agissait entre l'Espagne et les
États généraux. Cet ambassadeur en parla à Albéroni, et n'en reçut que
des réponses courtes et vagues. Il voulait engager les États généraux à
traiter avec l'Espagne\,; il prenait toutes ses mesures pour en avoir
l'honneur, et que ce fût à Madrid. Il se louait et se faisait louer sans
cesse avec tout l'artifice imaginable, de la sagesse et du secret de son
gouvernement, du bon ordre qu'il avait mis dans les affaires de la
monarchie, et de la vigueur qu'il y avait fait succéder à toute sorte de
faiblesse\,; il ne songeait qu'à bien rétablir la marine et le commerce.
Surtout il déplorait la conduite des précédents ministres, qui avaient
offusqué les grands talents de Philippe V pour le gouvernement, dont il
louait la vie uniforme toute l'année, que lui-même avait établie pour le
tenir avec la reine sous sa clef, et que personne n'en pût approcher que
par sa volonté, et dont il ne pût prendre aucun ombrage. Cette suite de
journées qui a toujours duré depuis, par s'être tournée en habitude,
mérite la curiosité, et d'être rapportée d'après Albéroni même.

Le roi et la reine qui, en maladie, en couches, en santé, n'avaient
jamais qu'un même lit, s'éveillaient à huit heures, et aussitôt
déjeunaient ensemble. Le roi s'habillait et revenait après chez la reine
qui était encore au lit (je marquerai lors de mon ambassade les légers
changements que j'y trouvai), et il passai un quart d'heure auprès
d'elle. Il entrait après dans son cabinet, y tenait son conseil, et
quand il finissait avant onze heures et demie, il retournait chez la
reine. Alors elle se levait, et pendant qu'elle s'habillait le roi
donnait divers ordres. La reine étant prête, elle allait avec le roi à
la messe, au sortir de laquelle ils dînaient tous deux ensemble. Ils
passaient une heure de l'après-dînée en conversation particulière,
ensuite ils faisaient ensemble l'oraison, après laquelle ils allaient
ensemble à la chasse. Au retour le roi faisait appeler quelqu'un de ses
ministres, et pendant son travail en présence de la reine, elle
travaillait en tapisserie ou elle écrivait. Cela durait jusqu'à neuf
heures et demie du soir qu'ils soupaient ensemble. À dix heures Albéroni
entrait et restait jusqu'à leur coucher, vers onze heures et demie. Les
premiers jours d'une couche, leurs lits séparés étaient dans la même
chambre. À ce détail il faut ajouter que peu à peu les charges n'eurent
plus aucune fonction, et personne n'approcha plus de Leurs Majestés
Catholiques\,; ce qui a duré toujours depuis. J'en expliquerai le
détail, si j'arrive jusqu'au temps de mon ambassade.

Beretti ne recevait point de réponse de Stanhope, sur la permission
qu'il avait demandée, à son passage à la Haye, pour la levée de trois
mille Irlandais. Il eut ordre de demander trois régiments écossais que
les États généraux avaient à leur service, et qu'ils voulaient réformer.
Il eût été plus naturel d'en charger Monteléon à Londres, mais il avait
déplu par ses représentations sur les affaires, et par ses plaintes sur
le payement de ses appointements, et il pouvait bien aussi être trop
éclairé et trop fidèle, au compte d'Albéroni. Stanhope qui, par cette
même raison s'en était trouvé embarrassé, et qui, pour s'en défaire,
l'avait desservi auprès d'Albéroni, ne laissait pas de s'ouvrir fort à
lui.

Nonobstant les liaisons si étroites que l'Angleterre venait de prendre
avec la France, Stanhope n'hésitait pas de dire à Monteléon que les
véritables liaisons et la véritable amitié de l'Angleterre seraient
toujours avec l'Espagne\,; que le roi son maître était prêt de faire un
traité d'alliance si le roi d'Espagne y voulait entrer\,; qu'il ne
trouverait pas la même facilité avec les États généraux dont le traité,
généralement désiré par eux avec la France, avait été fort combattu, et
qui, sans faire d'alliance nouvelle avec l'Espagne, lui proposeraient
peut-être d'entrer dans celle qu'ils venaient de faire avec l'Angleterre
et la France, et pour faire remarquer à Monteléon la différence du
procédé de l'Angleterre à l'égard de l'Espagne d'avec celui des États
généraux, il ajouta qu'aussitôt que la France eut proposé de traiter
avec l'Angleterre, le roi d'Angleterre ordonna à son ministre à Madrid
d'en faire part au roi d'Espagne, et de l'inviter d'entrer dans la
négociation\,; qu'il ne fit point de réponse\,; que toutefois le roi
d'Angleterre, supposant qu'il entrerait dans le traité, fit communiquer
la proposition de l'abbé Dubois, employé dans le traité. De cette
confidence, Stanhope passa à une autre bien moins innocente. Il lui dit
tout de suite que l'abbé Dubois avait paru très embarrassé, et fort peu
content de la proposition qu'il lui avait faite de comprendre le roi
d'Espagne dans l'alliance\,; qu'en effet on avait vu pondant tout le
cours de la négociation qu'il ne s'agissait que d'un traité particulier,
uniquement pour les intérêts du régent\,; que plus les ministres Anglais
avaient insisté à ne faire mention ni de succession respective, ni des
traités d'Utrecht, plus l'abbé Dubois, au contraire, avait désiré et
sollicité que cotte condition réciproque fût clairement exprimée\,; que
c'était à ce prix qu'il avait offert de signer tous les articles et
avantages demandés par l'Angleterre\,; qu'il avait employé toutes sortes
de moyens pour parvenir à la conclusion du traité\,; qu'il avait enfin
gagné les ministres d'Hanovre, en les assurant que la France garantirait
à cette maison la possession de Brème et de Verden, et qu'elle
s'engagerait à ne donner désormais aucun subside à la Suède. Stanhope
avouait que, depuis la conclusion du traité, le régent témoignait
beaucoup d'attention et d'empressement pour les intérêts et pour les
avantages du roi d'Angleterre\,; que même l'abbé Dubois avait donné des
avis de la dernière importance\,; mais comme bon Anglais, il disait que,
lorsqu'il s'agissait de se fier à la France, il fallait suivre le
conseil donné à celui qui se noyait au sujet de l'invocation de saint
Nicolas. Cette maxime établie, Stanhope assura Monteléon que le roi
d'Espagne éprouverait en toutes choses l'amitié du roi d'Angleterre\,;
qu'il pouvait arriver de grands événements et des révolutions imprévues,
où les secours du roi d'Angleterre ne lui seraient pas inutiles. Il en
aurait peut-être dit davantage, mais Monteléon jugea de la prudence de
ne pas marquer trop de curiosité (et la chose était assez intelligible),
et d'attendre d'autres conjonctures pour le faire parler encore sur la
même matière. Stanhope lui confia qu'il attendait l'abbé Dubois, et que
vraisemblablement il résiderait quelque temps en Angleterre.

Ce royaume menaçait de nouveaux remuements. L'état de ses dettes passai
cinquante millions sterling. On se proposait d'en réduire les intérêts
de six à cinq pour cent, et cette contravention aux obligations passées
sous l'autorité des actes du parlement, n'était pas une entreprise sans
danger. On murmurait déjà beaucoup de la prorogation en pleine paix de
quatorze schellings pour livre sur le revenu des terres, établie
seulement pour le temps de la guerre. Le mécontentement était général.
Ainsi il importait fort au roi d'Angleterre de persuader aux Anglais
qu'ils étaient effectivement en guerre avec la Suède, et qu'il lui
fallait de nouveaux secours pour se garantir des entreprises. On
publiait donc que la flotte Anglaise serait de trente-six à trente-huit
vaisseaux de guerre, et que les Hollandais y en joindraient douze. Les
ministres d'Angleterre attendaient avec beaucoup d'inquiétude le parti
que prendrait le roi de Suède sur l'arrêt de son envoyé à Londres, qui
avait depuis été conduit à Plymouth. Ils prièrent Monteléon de demander
de la part du roi d'Angleterre au roi d'Espagne de ne pas permettre aux
Suédois de vendre dans ses ports leurs prises Anglaises, et firent en
France la même demande. On n'eut pas de peine à y répondre, les
ordonnances de marine ne permettant pas à un armateur de nation amie de
demeurer plus de vingt-quatre heures dans nos ports. La même loi n'étant
pas établie en Espagne, il y fallait une réponse décisive. Mais on n'y
jugea pas à propos d'accorder cotte demande.

Albéroni désirait toujours un traité avec l'Angleterre et la Hollande,
mais il y paraissait fort ralenti. Il croyait avoir reconnu que trop
d'empressement de sa part éloignerait l'effet de ses désirs, et qu'il
fallait moins en solliciter ces deux nations que s'en faire rechercher,
et seulement se proposer d'empêcher une nouvelle union des Hollandais
avec l'empereur. Il y était confirmé par Beretti, qui le rassurait à
l'égard de l'union qu'il craignait par les nouveaux sujets de
brouilleries que les affaires des Pays-Bas et l'exécution du traité de
la Barrière élevaient sans cesse entre l'empereur et les États généraux.
L'extrême épuisement où la dernière guerre avait jeté la Hollande lui
faisait ardemment souhaiter la continuation de la paix.

Le Pensionnaire, dont l'entêtement contre la France et l'attachement au
feu roi Guillaume et à la maison d'Autriche en était cause, ne respirait
aussi que le repos de l'Europe, mais avait au fond toujours le même
penchant à favoriser la maison d'Autriche. Il tint à Beretti quelques
propos sur la paix à faire entre l'empereur et le roi d'Espagne. Il lui
dit même que le baron de Heems, envoyé de l'empereur en Hollande, lui
avait laissé entendre que ce monarque la désirait sincèrement, et qu'il
attendait au premier jour des ordres pour parler plus positivement.
Beretti paraissant douter de la sincérité impériale, Heinsius lui dit
que, après que ses maîtres auraient proposé à l'empereur des conditions
raisonnables, ils n'auraient plus d'égard à ses prétentions, s'ils
s'apercevaient qu'il ne voulût que traîner les affaires en longueur\,;
qu'alors ils ne songeraient qu'à plaire au roi d'Espagne\,; qu'ils
connaissaient que son amitié leur était nécessaire\,; qu'ils la
voulaient obtenir\,; que déjà Amsterdam et Rotterdam avaient applaudi à
la proposition d'une alliance avec l'Espagne, et que la province de
Zélande était du même avis.

Stanhope, par ordre du roi d'Angleterre, avait entamé une négociation à
Vienne pour traiter la paix entre l'empereur et le roi d'Espagne. Il fit
savoir à Beretti que ceux qui avaient le plus de part en la confiance de
l'empereur goûtaient les idées qu'il leur avait suggérées. Un des points
qui touchait le plus le roi d'Espagne était d'empêcher que les États du
grand-duc et ceux du duc de Parme tombassent jamais dans la maison
d'Autriche, et d'assurer au contraire ceux de Parme et de Plaisance aux
fils qu'il avait de la reine d'Espagne, faute d'héritiers Farnèse.
Stanhope espérait d'obtenir cet article, trouvait difficile et long de
traiter par lettres, et pour le secret même trouvait nécessaire que
l'Espagne et la France envoyassent des ministres de confiance pour
traiter à Londres par l'entremise du roi d'Angleterre. Il manda à
Beretti que le régent, persuadé de l'utilité de cette paix pour le bien
et le repos de l'Europe, y concourrait de tout son pouvoir, et qu'il
enverrait l'abbé Dubois à Londres dès qu'il saurait l'affaire en
maturité. Stanhope comptait que Penterrieder y viendrait pour le même
effet de la part de l'empereur. Il exhortait Beretti de demander la même
commission, parce qu'il y fallait employer un homme qui eût la confiance
d'Albéroni, dont il prodigua les louanges que Beretti eut soin de ne pas
affaiblir, et de ne pas oublier les siennes propres en rendant compte à
Albéroni. Stanhope ajoutait l'offre de le faire demander par le roi
d'Angleterre, parce qu'il était impossible que ses ministres pussent
prendre aucune confiance en Monteléon, ambassadeur ordinaire d'Espagne à
Londres.

Beretti, instruit alors fort superficiellement des intentions de
l'Espagne, se trouva embarrassé à plusieurs égards. Il ne pouvait
répondre que vaguement à des propositions précises. Il craignait que
l'intérêt qu'il avait de se voir chargé de la plus grande affaire que
pût avoir le roi d'Espagne ne décréditât sa relation. Il savait
qu'Albéroni qui voulait traiter à Madrid était très susceptible de
jalousie, et de le soupçonner d'inspirer aux Anglais de traiter à
Londres pour que toute la négociation demeurât entre ses mains. Il
remarquait que les propositions de Stanhope avaient été concertées avec
la France, puisque le régent y entrait si pleinement. Il marchait donc
sur des charbons en rendant compte à Albéroni. Il protestait de son
insuffisance à traiter une si grande affaire, et de la peine qu'il
aurait d'en faire à Monteléon. Il représentait que les chefs de la
république des Provinces-Unies, qui se portaient alors pour pacifiques
et pour vouloir une ligue avec l'Espagne, se garderaient bien de la
conclure avant que le traité du roi d'Espagne le fût avec l'empereur, de
peur de s'attirer pour toujours l'inimitié de ce dernier monarque\,;
qu'il avait remarqué qu'accoutumés à voir faire tous les grands traités
chez eux, et y croyant leur situation la plus propre, ils craignaient
encore que la négociation en étant portée à Londres elle ne fût occasion
aux Anglais d'obtenir quelque prérogative avantageuse du roi d'Espagne à
leur commerce, et que, si cette paix ne se traitait pas chez eux, ils
aimeraient mieux encore qu'elle la fût à Madrid qu'à Londres. Il
finissait par demander des instructions et des ordres à Albéroni, bien
résolu suivant ceux qu'il en avait précédemment reçus d'insister
fortement sur la sûreté de l'Italie, et de déclarer dans le temps que le
roi d'Espagne ne consentirait à la paix qu'avec la remise actuelle de la
ville de Mantoue des mains de l'empereur en celles des héritiers
légitimes. Beretti, bien informé de l'importance de cette place, et que
l'article en était essentiel, était particulièrement chargé de ne rien
oublier pour engager les Hollandais à faire en sorte qu'elle fût
restituée au duc de Guastalla qui en était injustement privé\,; à leur
faire peur de l'ambition et de la puissance de l'empereur, qui, s'il se
rendait maître de l'Italie, les leur ferait bientôt sentir aux
Pays-Bas\,; qui se montrait pacifique tandis qu'il avait les Turcs sur
les bras, mais que, s'il faisait la paix avec eux, il ne se trouverait
personne qui pût résister à ses armées victorieuses qui auraient abattu
les Ottomans.

Albéroni lui prescrivait en même temps de témoigner une extrême
indifférence pour la paix avec l'empereur, et de se borner à faire
connaître que l'Espagne était disposée à concourir à tout ce qui pouvait
maintenir l'équilibre dans l'Europe. Il lui mandait qu'il lui suffisait
de savoir que les Hollandais disposés à traiter avec l'Espagne ne
traiteraient pas avec l'empereur\,; qu'il fallait laisser faire au
temps, attendre tranquillement les propositions que l'Angleterre et la
Hollande voudraient faire. Il trouvait la lettre de Stanhope vague, et
la conclusion d'un traité d'autant moins pressée qu'il ne voyait pas
l'utilité que l'Espagne en pouvait retirer. Le roi d'Espagne ne pensait
pas à recouvrer par les armes les États qu'il avait perdus. Il
connaissait que les Pays-Bas et l'Italie avaient dépeuplé l'Espagne et
les Indes. Il trouvai sa situation présente plus avantageuse que celle
d'aucune autre puissance. Ses frontières étaient bien garnies, la
citadelle de Barcelone devait être achevée dans la fin de l'année, et
garnie de cent pièces de canon. Si ses ennemis pensaient à l'attaquer
avec des armées nombreuses, elle périraient faute de subsistance\,; si
avec de médiocres, celles d'Espagne seraient suffisantes pour la
défense. Il n'y avait que trois ou quatre années de paix à désirer pour
donner à la nation espagnole le loisir de respirer, et ne rien négliger
en attendant pour faire fleurir son commerce.

Un des principaux moyens que le premier ministre s'en proposait était
des manufactures de draps, pour lesquelles il voulut faire venir des
ouvriers de Hollande. Il en parla à Riperda qui lui dit en grand secret
qu'il fallait que Beretti fît en sorte d'en envoyer un de ceux qui
travaillaient à Delft, en lui faisant envisager une récompense et une
fortune considérable en Espagne. Comme il y manquait plusieurs choses,
il fit remettre cent cinquante mille livres à Beretti pour un achat de
bronzes. Il prétendait qu'il ne songeait qu'à mettre le roi d'Espagne en
état de se faire respecter, sans causer de préjudice ni de tort à
personne, mais de procurer du bien à ses amis et à ses alliés. Les
ministres d'Espagne au dehors assuraient aussi que la triple alliance
n'avait pas fait la moindre peine au roi d'Espagne\,; qu'il n'avait
aucune vue sur le trône de France, quelque malheur qui pût y arriver, et
qu'étant naturellement tranquille, il se contenterait de régner en
Espagne.

Le roi de Sicile ne se lassait point de presser, ce monarque de veiller
à la sûreté des traités d'Utrecht. Il craignait tout de l'empereur pour
l'Italie et pour la Sicile, dès qu'il aurait fait la paix avec la Porto.
Il ne comptait point sur l'Angleterre, dont le roi, par ses ménagements
pour l'empereur, n'osait envoyer un ministre à Turin, et parce que le
gouvernement s'y était hautement déclaré contre le traité d'Utrecht\,;
qu'il n'avait consenti à la triple alliance que pour en réparer les
défauts\,; que, content d'y avoir remédié de la sorte, il
s'embarrasserait peu de ses derniers engagements, à ce que les whigs
publiaient hautement, et que jamais ils n'entreprendraient une guerre
nouvelle pour la garantie de ce qu'il venait de promettre. Monteléon,
qui en était bien persuadé, avait conseillé à ce prince de s'adresser au
roi d'Espagne\,; mais il trouva dans Albéroni un ministre qui le
connaissait bien, ainsi que toute l'Europe, et qui disait qu'il voulait
tirer les marrons du fou avec la patte du chat, et à qui il ne fallait
donner que de belles paroles.

La correspondance avec Venise, interrompue par la nécessité où cette
république s'était trouvée de reconnaître l'empereur comme roi
d'Espagne, était prête à se rétablir par les excuses que le noble
Mocenigo, envoyé exprès à Madrid, en devait faire au roi d'Espagne dans
une audience publique. Les Vénitiens avaient enfin pris ce parti, par
leur frayeur commune avec le pape de voir les Turcs sur les côtes de
l'Italie et l'impatience d'y voir arriver au plus tôt les secours
maritimes promis au pape par l'Espagne.

\hypertarget{chapitre-xii.}{%
\chapter{CHAPITRE XII.}\label{chapitre-xii.}}

1717

~

{\textsc{Le régent livré à la constitution sans contre-poids.}}
{\textsc{- Le nonce Bentivoglio veut faire signer aux évêques que la
constitution est règle de foi, et y échoue.}} {\textsc{- Appel de la
Sorbonne et des quatre évêques.}} {\textsc{- J'exhorte en vain le
cardinal de Noailles à publier son appel, et lui en prédis le succès et
celui de son délai.}} {\textsc{- Variations du maréchal d'Huxelles dans
les affaires de la constitution.}} {\textsc{- Entretien entre M. le duc
d'Orléans et moi sur les appels de la constitution, tête à tête, dans sa
petite loge à l'Opéra.}} {\textsc{- Objection du grand nombre.}}
{\textsc{- Le duc de Noailles vend son oncle à sa fortune.}} {\textsc{-
Poids des personnes et des corps.}} {\textsc{- Conduite à tenir par le
régent.}} {\textsc{- Raisons personnelles.}} {\textsc{- Le régent arrête
les appels et se livre à la constitution.}}

~

Je ne continuerai à mon ordinaire à ne parler de la constitution
qu'autant que la place où j'étais m'obligeait rarement de m'en mêler. Je
connaissais la faiblesse du régent, et, quoiqu'il crût malgré lui, le
peu de cas qu'il se piquait de faire de la religion. Je le voyais livré
à ses ennemis sur cette affaire comme sur bien d'autres\,: aux jésuites
qu'il craignait, au maréchal de Villeroy qui lui imposait dès sa
première jeunesse, et qui dans la plus profonde ignorance se piquait de
la constitution pour faire parade de sa reconnaissance pour le feu roi
et pour M\textsuperscript{me} de Maintenon\,; à l'Effiat livré à M. du
Maine et au premier président, qui ne cherchaient qu'à lui susciter
toute espèce d'embarras pour qu'il eût besoin d'eux, et pour leurs vues
particulières\,; à la bêtise de Besons gouverné par d'Effiat, qui le
lâchait comme un sanglier au besoin, et qui faisait impression par
l'opinion que le régent avait prise de son attachement pour lui\,; à
l'abbé Dubois qui dans les ténèbres songeait déjà au cardinalat et à
s'en aplanir le chemin du côté de Rome\,; enfin aux manèges du cardinal
de Rohan, aux fureurs du cardinal de Bissy, et à la scélératesse de
force prélats qui se faisaient une douce chimère d'arriver au chapeau,
et une réalité, en attendant, de briller, de se faire compter et
craindre, de se mêler, d'obtenir des grâces enfin à ce cèdre tombé, à ce
malheureux évêque de Troyes que le retour au monde avait gangrené jusque
dans les entrailles, sans objet, sans raison, et contre toutes les
notions et les lumières qu'il avait eues et soutenues toute sa vie
jusqu'à son entrée dans le conseil de régence. De contre-poids, il n'y
en avait point.

Le duc de Noailles avait vendu son oncle à sa fortune. Le cardinal de
Noailles avait trop de droiture, de piété, de simplicité, de vérité\,;
les évêques qui pensaient comme lui s'éclaircissaient tous les jours à
force d'artifices et de menaces. Ils demeuraient concentrés, ils
n'avaient ni accès ni langage, ils se confiaient et s'offraient à Dieu,
ils ne pouvaient comprendre qu'une affaire de doctrine et de religion en
devînt une d'artifices, de manèges, de pièges et de fourberies\,; aucun
n'était dressé à rien de tout cela. Le chancelier, lent, timide, suspect
sur la matière, n'avait pas la première teinture de monde ni de cour\,;
toujours en brassière et en doute, en mesure, en retenue, arrêté par le
tintamarre audacieux des uns, et par les doux et profonds artifices des
autres, incapable de se soutenir contre les premiers à la longue, et de
jamais subodorer les autres, médiocrement aidé du procureur général, qui
ne faisait bien que quand il le pouvait sans crainte d'y gâter son
manteau, tous déconcertés à l'égard du parlement par les adresses du
premier président, et suffoqués de ses grands airs de la cour et du
grand monde, par son audace, et par des tours de passe-passe où il était
un grand maître. Bentivoglio, depuis les premiers jours de la régence,
ne cessait de souffler le feu en France, et de faire les derniers
efforts à Rome pour porter le pape aux dernières violences. Il était
fort pauvre, fort ambitieux, fort ignorant, sans moeurs, comme on a vu
qu'il en laissa des marques publiques, dont il ne prenait même pas grand
soin de se cacher, et par ce qu'on vit sans cesse de ce furieux nonce,
sans religion que sa, fortune. Il croyait son chapeau et de quoi en
soutenir la dignité attaché aux derniers embrasements que la bulle pût
susciter en France, et il n'épargnait rien pour y parvenir, jusque-là
que le pape le trouvait violent au point d'être importuné de ses
exhortations continuelles, et que les prélats les plus attachés à
Rome\,; soit par leur opinion, soit par leur fortune, s'en trouvaient
pour la plupart excédés, même les cardinaux de Rohan et de Bissy, hors
un petit nombre de désespérés, qui avec les jésuites ne respiraient que
sang, fortune et subversion de l'Église gallicane. De degré en degré et
de violence en violence qu'ils extorquaient du régent malgré lui,
l'affaire en vint au point de faire de la constitution une règle de foi.

Le pape, roidi, contre l'usage de ses plus grands et plus saints
prédécesseurs, à ne vouloir donner aucune explication de sa bulle, ni
souffrir que les évêques y en donnassent aucune de pour d'attenter à sa
prétendue infaillibilité, encore plus dans l'embarras de donner une
explication raisonnable, ou d'en admettre une, ne voulait ouïr parler
que d'obéissance aveugle, et son nonce, à la tête des jésuites et des
sulpiciens, trouvait l'occasion trop belle d'abroger les libertés de
l'Église gallicane, et de la soumettre à l'esclavage de Rome, comme
celles d'Italie, de l'Espagne, du Portugal, des Indes, pour en manquer
l'occasion. Il se mit donc à bonneter les évêques par lui, et par les
jésuites et les sulpiciens, pour faire déclarer la constitution règle de
foi. Les plus attachés à Rome d'entre les évêques se révoltèrent d'abord
contre une proposition si absurde, et que Rome même avait trouvée telle,
comme ils s'étaient révoltés d'abord contre la constitution à son
premier aspect. La règle de foi eut le même sort qu'avait eu
l'acceptation de la constitution, et à force d'intrigues et de manèges
quelques évêques y consentiront, et le nombre parut s'en grossir.

Dans cette extrémité d'un nouvel article de foi si destitué de toute
autorité légitime, puisqu'elle n'est donnée qu'à l'assemblée libre et
générale de l'Église, à qui seule les promesses de Jésus-Christ
s'adressent d'être avec elle jusqu'à la consommation des siècles\,; la
Sorbonne et quatre évêques crurent qu'il était temps d'avoir recours au
dernier remède que l'Église a toujours présenté, et approuvé que ses
enfants en usassent comme suspensif, en attendant des temps où la vérité
pourrait être écoutée, et dont jusqu'au feu roi inclusivement on s'était
publiquement servi dans les parlements et parmi les évêques, les
docteurs, etc., pour se dérober aux entreprises de Rome. Ce fut l'appel
au futur libre concile général. Bentivoglio et toute la constitution
jetèrent les hauts cris. Ils sentaient le poids en soi de cette grande
démarche\,; ils gémissaient sous son poids suspensif. Ils sentaient
l'effet terrible pour leur entreprise de la suite qu'ils devaient
craindre de cet exemple, et remuèrent l'enfer pour l'arrêter. Le régent
prompt à s'effrayer, facile à se laisser entraîner par ses confidents
perfides, s'abandonna à eux pour sévir contre la Sorbonne et contre les
quatre évêques, qu'il exila, puis qu'il renvoya dans leurs diocèses.

Ce fut alors que le cardinal de Noailles manqua un grand coup, comme il
en avait déjà manqué plusieurs fois. Je le voyais souvent chez lui et
chez moi. Il y vint dans cette occasion raisonner avec moi. Je
l'exhortai à l'appel. Il était sûr des chapitres et des curés de Paris,
des principaux ecclésiastiques et des plus célèbres et nombreuses
congrégations de communautés séculières et régulières. Il l'était aussi
de plusieurs évêques qui n'attendaient que son exemple\,; et de tous
ceux-là il était pressé de le donner. Je lui représentai qu'après s'être
inutilement prêté à tout, il devait demeurer convaincu de la perfidie,
dès artifices, du but du parti, qui, sous l'apparence d'obéissance à
Rome, forçait la main au pape pour triompher en France, et ne
consentirait jamais à rien qu'à l'obéissance aveugle\,; qu'il avait
suffisamment montré raison, patience, douceur, modération\,; désir de
pouvoir sauver l'obéissance avec la vérité et les libertés de l'Église
gallicane\,; qu'il était enfin temps d'ouvrir les yeux, et de mettre des
bornes aux fureurs et aux artifices\,; et qu'appelant à la tête de tous
ceux que je viens de désigner\,; ce groupe deviendrait d'autant plus
formidable aux entreprises et aux violences qu'il se trouverait
nombreux, illustre, et à couvert par les règles de l'Église les plus
anciennes, les plus certaines, les plus en usage, respecté depuis les
premiers temps qu'on y avait eu recours jusqu'aux derniers du règne du
feu roi\,; qu'un appel si général et si canonique inspirerait du courage
aux abattus, de la crainte et un extrême embarras aux violences, une
salutaire neutralité à ceux qui penchaient à la constitution dans la
simplicité de leur coeur\,; que cette démarche aurait un grand effet sur
les parlements, qui ne demandaient pas mieux que d'appeler, et qui n'en
étaient retenus que par l'autorité du gouvernement, et encore par art et
pan machines\,; et que si ces compagnies s'unissaient enfin à lui, comme
toutes les apparences y étaient, par leur appel, c'en serait fait de la
constitution, et que Rome ne pourrait plus songer qu'à la retirer, à
étouffer doucement cette affaire, et se trouverait heureuse de donner de
bonnes sûretés qu'il n'en serait plus parlé.

J'ébranlai le cardinal de Noailles. Il me confia que son appel était
tout fait et tout prêt\,; mais qu'il croyait qu'il en fallait encore
suspendre l'éclat, et n'avoir pas à se reprocher de n'avoir pas eu assez
de patience. Jamais je ne pus le sortir de là, ni lui m'en alléguer de
raisons que ce vague. Au bout d'un long débat, je lui prédis que sa
patience serait funeste, qu'il viendrait à la fin à l'appel\,; mais trop
tard\,; qu'il trouverait tout ce qui était prêt actuellement d'appeler
avec lui séduit, intimidé, divisé par le temps\,; qu'il en donnerait aux
artifices et à l'autorité séduite du régent, qu'il éprouverait contraire
avec force\,; qu'étourdie alors du coup, il n'en aurait rien à craindre,
surtout avec les parlements qu'il aurait avec lui\,; au lieu qu'ils
seraient gagnés, divisés, intimidés par le loisir qu'il donnerait de le
faire, et que, quand il voudrait déclarer son appel, il se trouverait
abandonné. Je ne fus que trop bon prophète.

Le maréchal d'Huxelles, ministre nécessaire dans toute cette affaire, y
variait souvent. Tout lui en montrait la friponnerie, et le danger en
croupe de l'anéantissement des libertés de l'Église gallicane, qui était
le but auquel tendaient les véritables abandonnés à Rome, tels que le
nonce, les jésuites, les sulpiciens et les évêques de leur faciende, et
plusieurs antres qui ne le voyaient pas, mais que les autres
entraînaient par ignorance et par bêtise. Ainsi le maréchal faisait
souvent des pointes qui déconcertaient les projets. Mais bientôt après,
le premier président et d'Effiat le prenaient, tantôt par caresses,
tantôt sur le haut ton, souvent par des raisons d'intérêts particuliers,
qui n'étaient pas ceux de l'Église ni de l'État, moins encore du régent,
et le ramenaient, de sorte que l'irrégularité de cette conduite du
maréchal d'Huxelles entravait souvent les deux partis et le régent
lui-même. Ce prince qui, dès les temps du feu roi savait ce que je
pensais sur la constitution, et, comme je l'ai rapporté en son temps, ce
que lui-même en pensait, en était embarrassé avec moi. Il évitait
d'autant plus aisément de me parler de cette matière que je ne l'y
mettais jamais, et qu'à l'exception de quelques adoucissements que j'en
obtenais quelquefois des violences qu'on extorquait de lui sur des
particuliers, je ne cherchais point à entrer en rien de toute cette
affaire avec lui, depuis que j'avais reconnu l'entraînement où il
s'était laissé aller. Mais quand il se sentait embarrassé et pressé à un
certain point, il ne pouvait s'empêcher de revenir à moi avec une
entière ouverture, dans les occasions et sur les choses même où ses
soupçons ou les influences de gens qui l'approchaient me rendaient le
plus suspect à ses yeux. Pressé donc, et embarrassé entre les appels et
les fureurs opposées dont je viens de parler, il m'arrêta, une
après-dînée, comme je resserrais des papiers, et que je me préparais à
le quitter après avoir travaillé avec lui tête-à-tête, comme il
m'arrivait une ou deux fois la semaine. Il me dit qu'il s'en allait à
l'Opéra, et qu'il voulait m'y mener pour m'y parler de choses
importantes. «\,L'Opéra, monsieur\,! m'écriai-je\,; eh\,! quel lieu pour
parler d'affaires\,! parlons-en ici tant que vous voudrez, ou si vous
aimez mieux aller à l'Opéra à la bonne heure\,; et demain ou quand il
vous plaira je reviendrai.\,» Il persista, et me dit que nous nous
enfermerions tous deux dans sa petite loge, où il allait à couvert et de
plain-pied tout seul de son appartement, et que nous y serions aussi
bien et mieux que dans son cabinet. Je le suppliai de songer qu'il était
impossible de n'être pas détournés par le spectacle et par la musique\,;
que tout ce qui voyait sa loge nous examinerait parlant, raisonnant et
n'être point attentifs à l'Opéra, chercherait à pénétrer jusqu'à nos
gestes\,; que les gens qui venaient là lui faire leur cour
raisonneraient de leur côté de le voir dans sa petite loge enfermé avec
moi\,; que chacun en compterait la durée\,; qu'en un mot, l'Opéra était
fait pour se délasser, s'amuser, voir, être vu, et point du tout pour y
être enfermé et y parler d'affaires et s'y donner en spectacle au
spectacle même. J'eus beau dire, il se mit à enfin à rire, prit d'une
main son chapeau et sa canne sur un canapé, moi par le bras de l'autre,
et nous voilà allés. En entrant dans sa loge, il défendit que personne
n'y entrât, qu'on l'ouvrît pour quoi que ce pût être, et qu'on laissât
approcher personne de la porte. C'était bien montrer qu'il ne voulait
pas s'exposer à être écouté, mais bien montrer aussi qu'enfermé là avec
moi, qui n'étais pas un homme de spectacles et de musique, il y était
moins à l'Opéra que dans un cabinet en affaires. Aussi cela réussit-il
fort mal à propos à faire une nouvelle que tout ce qui se trouva à
l'Opéra en sortant distribua par Paris, comme je l'avais bien prévu et
prédit à M. le duc d'Orléans. Il se mit où il me dit qu'il avait
accoutumé de se mettre, regardant le théâtre, auquel il me fit tourner
le dos pour être vis-à-vis de lui. Dans cette position nous étions vus
en plein, lui de tout le théâtre et des loges voisines, et d'une partie
du parterre, moi du théâtre par le dos, et de côté et presque en face de
presque tout ce qui était à l'Opéra du côté opposé pour les loges, mais
de tout le parterre et de tout l'amphithéâtre de côté et presque en
face. Ce m'était un pays inusité, où on eut peine d'abord à me
reconnaître, mais où quelques yeux, le tête-à-tête et l'action de la
conversation me décelèrent bientôt. L'Opéra ne faisait que commencer\,;
nous ne fîmes que regarder un moment le spectacle en nous plaçant, qui
était fort plein, après quoi nous n'en vîmes ni n'en ouïmes plus rien
jusqu'à sa fin tant la conversation nous occupa.

D'abord M. le duc d'Orléans m'expliqua avec étendue l'embarras où il se
trouvait entre les appels dont il était pressé par le parlement qui le
voulait faire, plusieurs évêques et tout le second ordre de Paris, à
l'exemple de la Sorbonne et de plusieurs corps réguliers et séculiers
entiers. Je l'écoutai sans l'interrompre, puis je me mis à raisonner.
Peu après que j'eus commencé, il m'interrompit pour me faire remarquer
que le grand nombre était pour la constitution, et le petit pour les
appels\,; que la constitution avait le pape, la plupart des évêques, les
jésuites, tous les séminaires de Saint-Sulpice et de Saint-Lazare, par
conséquent une infinité de confesseurs, de curés de vicaires répandus
dans les villes et les campagnes du royaume qui y entraînaient les
peuples par conscience, tous les capucins et quelque petit nombre
d'autres religieux mendiants\,; et que telle chose pouvait arriver en
France où tous ces constitutionnaires se joindraient au roi d'Espagne
contre lui, et par le nombre seraient les plus forts, ainsi que par
l'intrigue et par Rome, et de là se jeta dans un grand raisonnement. Je
l'écoutai encore sans l'interrompre, et je le priai après de m'entendre
à son tour. Je commençai par lui dire qu'avec lui il ne fallait pas
raisonner par motif de religion ni de bonté de la cause de part ni
d'autre\,; que je ne pouvais pourtant m'empêcher de lui dire combien il
était étrange de traiter une affaire de doctrine et de religion, poussée
jusqu'à vouloir faire passer en article, au moins en règle de foi, qui
en expression plus douce n'est que synonyme à l'autre, tant de si
étranges points, et trouvés d'abord si étranges en effet par ceux-là
mêmes qui en sont devenus les athlètes\,; de traiter, dis-je, une telle
affaire par des vues et des moyens uniquement politiques, qui ne
pouvaient être bons qu'à attirer la malédiction de Dieu sur le succès,
sur les personnes qui s'en mêlaient de la sorte et sur tout le
royaume\,; que je ne pouvais aussi me passer de lui rappeler ce qu'il
avait pensé de l'iniquité du fond et de la violence des moyens du temps
du feu roi, et ce que lui et moi nous nous étions confié l'un à l'autre
quand on se crut sur le point d'aller au parlement avec le feu roi, qui
n'en fut empêché que par l'augmentation subite du mal qui l'emporta peu
de temps après\,; que me contentant de lui avoir remis en deux mots
devant les yeux des choses si déterminantes pour un autre que lui par
les seuls vrais, grands et solides principes qui devraient, uniquement,
conduire, surtout en matière de religion, je n'en ferais plus aucune
mention, et ne lui parlerais que le langage duquel seulement il était
susceptible.

Je lui montrai qu'il se trompait sur le grand nombre, et pour s'en
convaincre, je le suppliai de se transporter au temps du feu roi, où
toute sa terreur, ses menaces, les violences qu'on lui avait fait
employer n'avaient pu attirer le grand nombre qu'avec une répugnance et
une variété d'expressions toutes captieuses, qui montraient évidemment
qu'on ne cherchait qu'à se sauver, en abandonnant ses sentiments sous un
voile, et sauvant la vérité autant que la frayeur le pouvait permettre à
la faiblesse, d'où on pouvait juger de ce qui serait arrivé de la
constitution, si un roi aussi redouté qu'il était n'y eût déployé toute
sa puissance. Je convins ensuite des progrès que la constitution avait
faits depuis, mais par la crainte, l'industrie, la calomnie, la cabale,
les espérances ou de fortune ou de paix\,; mais j ajoutai qu'en ôtant
tous ces artifices, comme ils le seraient du moment que son autorité ne
les soutiendrait plus, tout ce qui avait tâché, de demeurer dans le
silence éclaterait, et que les trois quarts de ce qui s'était laissé
prendre en ces différents filets s'en secouerait, et chanterait la
palinodie, comme l'entrée de sa régence le lui avait montré en plein,
pendant le peu de temps qu'avait duré l'étourdissement des chefs du
parti constitutionnaire, et de la protection qu'il avait donné au parti
opprimé. Je lui fis sentir quelle différence mettait, pour le nombre
entre deux partis, la pesanteur de la puissance temporelle, unie avec
l'apparence de la spirituelle, le grand nom de chef de l'Église,
d'unité, d'obéissance, de parti le plus sûr à l'égard des simples et des
ignorants, qui font le grand nombre des ecclésiastiques comme des
laïques, la crainte des peines et l'espérance des récompenses pour
beaucoup, et pour tous de ne point trouver d'obstacles dans leur chemin,
enfin la licence de tout entreprendre d'une part, avec impunité tout au
moins, et très rarement sans succès\,; de l'autre, trouver tous les
tribunaux fermés à leurs plaintes, et impuissants à leurs plus justes
défenses qu'outre l'odieux d'un si prodigieux contraste, et qui n'avait
d'exemple que celui des temps de persécution des princes idolâtres ou
hérétiques, cette disparité écrasait les plus sages et les plus
religieux, et persuadait aux courages abattus, qui n'envisageaient
aucune étincelle de protection ni d'espérance, de se prêter au temps, et
de rejeter sur la violence les mensonges auxquels on les forçait\,; que
c'était ainsi que Henri VIII s'était fait chef de la religion en Si peu
de mois en Angleterre, avait chassé Rome, et envahi les biens immenses
des ecclésiastiques de son royaume, et que les régents de la minorité de
son fils, malgré leurs divisions et leurs troubles domestiques, avaient
en si peu de temps achevé le saut, embrassé l'hérésie après le schisme,
et s'étaient composé une religion qui avait chassé la catholique sous
les dernières peines\,; que c'était ainsi qu'en si peu de temps les rois
du Nord, dont l'autorité chez eux était alors si nouvelle et si peu
affermie, avaient rendu leurs royaumes protestants, et que presque tous
les souverains du nord d'Allemagne en avaient fait autant dans leurs
États\,; que le grand nombre présenté de là sotte par une telle
inégalité de balance dans le gouvernement, n'était donc qu'un leurre et
une tromperie manifeste, dont l'appel se trouverait la véritable
correctif\,; qu'alors les tribunaux rendus à l'exercice de la justice à
cet égard, l'autorité royale à embrasser tous ses sujets avec égalité,
le gros du monde en liberté de voir, de parler, de s'instruire, et de
discerner, les simples et les ignorants, éclairés par les appels des
évêques, d'un nombre infini d'ecclésiastiques du second ordre, de
religieux, de corps entiers séculiers et réguliers, enfin par celui des
parlements, reviendraient de la crainte servile qui les avait
enchaînés\,; et qu'alors il verrait avec surprise que le grand nombre
serait des appelants, et le très petit, et encore méprisé et honni comme
celui des tyrans renversés, se trouverait des constitutionnaires.

En cet endroit le régent m'interrompit, et avec une sorte d'angoisse\,:
«\,Mais, monsieur, me dit-il, que voulez-vous que je croie, quand le duc
de Noailles lui-même m'arrête sur les appels, et me maintient que j'y
hasarde tout, parce que le très grand nombre est pour la constitution,
et qu'il n'y a qu'une poignée du parti opposé\,; et si vous ne nierez
pas combien il y est intéressé pour son oncle\,? --- Monsieur,
repris-je, cela est horrible, mais ne me surprend pas. Vous savez que je
ne vous parle jamais du duc de Noailles depuis les premiers temps de ce
qui s'est passé entre nous\,; mais puisque vous me le mettez en jeu et
en opposition si spécieuse, si faut-il aussi que je vous y réponde. M.
de Noailles, monsieur, est un homme qui n'a ni religion ni honneur, et
qui jusqu'à toute pudeur, l'a perdue, quand il croit y trouver le plus
petit avantage. Du temps du feu roi, rappelé d'Espagne, brouillé avec
lui, avec M\textsuperscript{me} de Maintenon, avec M\textsuperscript{me}
la duchesse de Bourgogne, craint et mal voulu de tout le monde, en un
mot perdu en Espagne et ici, il n'avait d'appui ni d'existence que son
oncle, et par lui, ce qui s'appelait son parti, ainsi il y tenait.
Depuis qu'il vole de ses ailes, ce même oncle et son parti, ne lui
servant plus à rien, lui pèsent\,; ainsi il veut en tirer le fruit de se
faire considérer de l'autre comme un homme impartial, traitable sur un
point qui lui doit être si sensible, éteindre de ce côté-là craintes et
soupçons, ranger ainsi les obstacles qu'il en appréhende dans le chemin
de la fortune, et de la place de premier ministre, qui lui a fait
commettre un crime si noir et si pourpensé à mon égard, de laquelle il
n'abandonnera jamais le désir et l'espérance, tandis que misérablement
adoré par son oncle, qui ne voit pas assez clair pour le connaître, il
l'entraîne dans les panneaux pour se faire valoir de l'autre côté,
pendant que son oncle le vante dans le sien, que lui, de son côté,
trompe et cajole. Son compte est de faire durer la querelle pour se
faire admirer des deux côtés, et vous parler comme il fait pour vous
persuader d'un attachement pour vous, et d'une vérité pour la chose, à
l'épreuve du sang, de l'amitié et de tout intérêt. Voilà, monsieur, quel
est le duc de Noailles, et, puisque vous m'y forcez, jusqu'à quel point
vous êtes sa dupe. Mais moi, qui suis plus vrai, plus droit et plus
franc, je vous parlerai sur un autre ton\,: c'est que je ne me cache à
vous, à personne ni à lui-même, que le plus beau et le plus délicieux
jour de ma vie ne fût celui où il me serait donné par la justice divine
de l'écraser en marmelade, et de lui marcher à deux pieds sur le ventre,
à la satisfaction de quoi il n'est fortune que je ne sacrifiasse. Je ne
suis pas encore assez dépourvu de sens et de raisonnement pour ne pas
voir que, quelque mobilité, quelque adresse, quelque finesse et quelque
art qu'ait le duc de Noailles, il ne peut éviter de se trouver perdu si
son oncle est perdu, et que Rome et les constitutionnaires viennent à
bout de le traiter comme ils ont été si près de faire sous le feu roi,
et comme ils travaillent tous les jours à y revenir. Ce que j'avance est
manifeste. S'ils vous persuadent par degrés de le leur abandonner, et
qu'ils le dépouillent de la pourpre et de son siège, voilà un homme au
moins anéanti, si pis ne lui arrive, par être confiné quelque part, ou
envoyé à Rome. Dans cet état, de deux choses l'une nécessairement ou le
duc de Noailles suivra la fortune de son oncle, ou il l'abandonnera pour
conserver la sienne. S'il suit la fortune de son oncle, le voilà retiré,
hors de place, ne voulant plus se mêler de rien sous un prince qui
égorge son oncle, ou qui du moins l'abandonne à la boucherie et à la
rage de ses ennemis. Voilà où le sang, l'amitié, l'honneur le
conduisent, et moi, par conséquent, nageant dans la joie de le voir
entraîné et noyé sans retour par le torrent qui emporte son oncle. Si,
au contraire, avec des tours et des distinctions d'esprit, il abandonne
son oncle pour se cramponner en place, il devient l'homme le plus
publiquement et le plus complètement déshonoré\,; il devient, de plus,
suspect au parti qu'il ménage au prix du sang de son ondé, et à
vous-même qui n'oserez jamais vous fier à lui de quoi que ce soit\,; il
devient l'horreur du monde, et l'exécration du parti de son oncle, qui
tout entier ne saurait périr avec lui\,; il devient enfin l'opprobre et
le mépris de toute la terre\,; et moi, par conséquent, jouissant d'un
état dont l'infamie ne laisse plus rien à faire ni à désirer à ma
vengeance. Mon intérêt le plus vif et le plus cher, si j'étais aussi
scélérat que le duc de Noailles, aurait donc été, dès les premiers jours
de votre régence, de répondre aux empressements des cardinaux de Rohan
et de Bissy, et de leurs consorts, de m'unir étroitement à eux, de les
servir auprès de vous de toutes mes forces. La bonté et la confiance
dont vous m'honorez m'aurait rendu parmi eux l'homme laïque le plus
principal, le conseil et le modérateur du parti, avec une intimité et
une considération d'autant plus solides que nous aurions travaillé de
toutes nos forces au même but, et que nous y serions peut-être déjà
parvenus. Ne croyez pas que cette réflexion me soit nouvelle, ni que ces
messieurs-là soient demeurés jusqu'à présent à me la faire suggérer,
jusqu'à me faire dire de leur part, et plus d'une fois, qu'ils ne
comprenaient pas comment, avec toute ma haine publique pour le duc de
Noailles, que je pouvais perdre sûrement et solidement en perdant son
oncle, je demeurais l'ami du cardinal de Noailles, et, pour user de
l'abus de leurs termes, son plus puissant protecteur. Mais si je suis
encore incapable de cette vertu qui ne vous coûte rien, et que sans nul
mérite vous portez souvent au plus pernicieux abus, qui est le pardon
des ennemis, à Dieu ne plaise que je succombe assez au plaisir de la
vengeance, et devienne assez scélérat pour me tourner contre la vérité
connue, la droiture et l'innocence manifeste, et le bien de la religion
et de l'État, et que je cesse de vous les représenter de toutes mes
forces, et tout votre intérêt personnel qui y est attaché, tant que vous
voudrez bien m'écouter sur un si grand chapitre\,!» Je conclus ce propos
péremptoire par lui dire que c'était à lui {[}à{]} discerner qui, du duc
de Noailles ou de moi, lui parlait avec plus de désintéressement et de
vérité sur l'appel.

Revenant tout court au fond de la chose, je lui dis qu'avec le nombre il
fallait aussi peser la qualité\,; qu'il devait voir que d'un côté
étaient tous les ambitieux, les mercenaires et les ignorants, séduits
par quelques savants et quelques simples de bonne foi\,; que de l'autre,
étaient les prélats les plus doctes, les plus vertueux, les plus
désintéressés, les plus pieux et des meilleures moeurs, enfin de vrais
pasteurs, résidant, travaillant, adorés dans leur diocèse, et en exemple
non contredit à toute l'Église de France, toutes les écoles et les
universités, les collèges, les curés et les chapitres de Paris et de
presque toute la France, en un mot, la presque totalité du second ordre,
non des abbés aboyants, mais de ce second ordre, pieux, éclairé, qui ne
prétendait à rien et qui ne vendait point sa foi et sa doctrine\,; enfin
les parlements, qui en ce genre formaient un groupe respectable, et que
Rome redouterait toujours\,; que le gros de la cour, du monde, du public
par tout le royaume était encore du même côté soit lumière ou
prévention, et grand nombre aussi par indignation des violences, et des
moeurs, de l'ambition, de la conduite du plus grand nombre des évêques
du parti opposé, et d'abominables intrigues dont le temps avait fait la
découverte\,; qu'avec les lois de l'Église et de l'État pour lui, avec
les évêques, les docteurs, le clergé séculier et régulier le plus estimé
et le plus distingué, les corps entiers séculiers et réguliers les plus
vénérables, et les compagnies supérieures qui se feraient toutes honneur
de suivre les parlements, qui sont en ce genre les gardiens et les
protecteurs des lois, il se trouverait à la tête d'un bien autre parti
que ne serait celui de la constitution, d'un parti sur qui la religion,
la vérité, les canons de l'Église, ses règles immuables, les lois de
l'État, les libertés de l'Église gallicane, qui ne sont que la
conservation de l'ancienne discipline de l'Église envahie ailleurs par
l'usurpation des papes et la despotique tyrannie de Rome, sur qui enfin
la conscience pouvait tout, l'ambition, l'intérêt rien, comme tant et de
si vives persécutions si grandement souffertes le démontraient avec la
dernière évidence, parti, puisqu'il faut {[}se{]} servir de ce terme
quoiqu'il ne convienne qu'à celui qui lui est opposé, parti qui lui
serait solidement et inviolablement attaché par les liens de sa
conscience, de la religion, de la vérité, de la reconnaissance, et que
nul intérêt temporel n'en pourrait débaucher, qui grossirait sans cesse
de tous les ignorants de l'autre, à qui alors il serait libre de parler,
et de les éclairer, à eux d'écouter et d'être instruits, et d'une foule
de mercenaires dont il avait vu les variations à mesure de celles du
crédit de leur parti, et qui étaient incapables d'en suivre aucun que
pour des vues humaines. Alors que deviendrait le parti opposé, chargé du
mépris de ses artifices, de la haine de ses violences, dépouillé du
pouvoir d'en commettre, et de l'affranchissement du pouvoir des lois et
des tribunaux, et de la censure des doctes, de cette foule de
personnages de la plus grande réputation chacun dans leur état\,?
Comment soutenir une cause qui arme la raison et toutes les lois contre
elle, qui s'est noircie de tout ce que l'artifice et la persécution ont
de plus odieux, et opposer la honte de l'épiscopat et du sacerdoce en
tout genre pour la plupart à l'élite qui forme tout l'autre parti,
décorée de ses souffrances et purifiée par le feu de la persécution\,?
Que pourraient opposer à tant de savoir et de vertu les grâces alors
flétries par faute de pouvoir, et les mines de protection du premier de
ses chefs, et les repoussantes clameurs de l'autre\,; les rusés si
reconnues de leurs principaux ouvriers du premier et du second ordre,
dont les moeurs de la plupart, la conduite et l'ambition de tous, les
ont rendus l'abomination du monde jusque dans l'usage le plus effréné de
leur crédit et de leur pouvoir et Rome qui recule devant un roi de
Portugal, et pour une grâce qui ne dépend que d'elle, qui ne tient ni à
vérité ni à religion, grâce injuste, même scandaleuse, sera-t-elle plus
audacieuse contre un groupe si vénérable du premier et du second ordre,
soutenu de la multitude rendue à la liberté, et des parlements engagés
par leur appel dans la même cause, Rome, dis-je, dépouillée de
l'autorité royale, qui faisait tout trembler sous elle, mais qui avec ce
terrible avantage n'a pourtant jamais osé que menacer.

J'ajoutai à cette peinture que son personnage, à lui régent, était bien
honnête et bien facile. Il n'avait qu'à laisser faire et jouir de ce qui
se ferait et des appels en foule qu'il verrait éclater. Dire au pape et
aux chefs de la constitution qu'ils ne devaient pas attendre du pouvoir
précaire d'un régent plus qu'ils n'avaient pu obtenir de la redoutable
et absolue autorité du feu roi, qui l'avait si longtemps déployée on
leur faveur tout entière\,; qu'il y a de plus, bien loin de ce dont il
s'agissait alors à ce qui s'entreprenait aujourd'hui. Alors il ne
s'agissait que de la condamnation d'un livre, et de se taire sur la
constitution. Aujourd'hui que, les desseins croissant avec le pouvoir,
il ne s'agit de rien moins que d'embraser la France par toutes les
intrigues imaginables, jusqu'à y vouloir faire entrer les premières
puissances étrangères, et faire recevoir, signer, croire et jurer comme
articles définis de foi, au moins en attendant comme règle de foi qui en
est le parfait synonyme, tout ce qui est dans la constitution\,; ce
comble de pouvoir qui n'est permis et donné qu'à l'Église assemblée,
appliqué à une bulle qui bien ou mal à propos a soulevé toute la France
dès qu'elle a paru, que les uns trouvent inintelligible, les autres non
recevable dans ce qui s'en entend, bulle dont le pape, contre la coutume
de ses plus saints et plus illustres prédécesseurs, n'a jamais voulu ni
expliquer, ni souffrir que les évêques l'expliquassent, depuis tant
d'années qu'il on est supplié et conjuré avec tout le respect et
l'humilité possible, il n'est pas étonnant que, poussées enfin à bout,
les consciences se révoltent, forcent la main au régent, et aient enfin
recours au dernier remède de tout temps établi dans l'Église, et dont
les plus saints et les plus grands papes ne se sont jamais offensés.
Ajouter que vous êtes affligé d'un si grand éclat, et impuissant pour
l'arrêter, mais qu'étant régent du royaume, et n'ayant jusqu'à ce jour
omis travail, peine, ni soin pour procurer la satisfaction du pape, et
votre vénération personnelle, jusqu'à y employer l'autorité dont vous
êtes dépositaire plus encore que le feu roi n'avait fait et,
malheureusement vous ne mentirez pas, vous n'êtes pas résolu aussi à ne
pas protéger les lois de tout temps on usage, auxquelles le feu roi
lui-même à ou recours en d'autres occasions, ni à laisser mettre le feu
et le trouble dans le royaume. Faire en même temps avertir le nonce
d'être sage, et de ne vous pas forcer par sa conduite à des démarches
qui lui seraient désagréables, et dont les suites pourraient arrêter sa
fortune\,; et prendre des précautions mesurées mais justes pour rendre
ses communications difficiles avec les chefs et les enfants perdus du
parti. Écrire aussi en même sens au cardinal de La Trémoille, d'une
façon à faire peur au pape s'il pensait aller plus loin, tant sur la
chose en général que sur le cardinal de Noailles et aucun autre en
particulier\,; et lui envoyer une lettre pour le pape remplie des plus
beaux termes d'attachement, de douleur, de vénération, mais imprimée
vaguement d'une teinture de fermeté qui soutînt la lettre au cardinal de
La Trémoille\,; surtout n'oublier, pas de faire parler français aux
principaux jésuites d'ici à leur général à Rome, et aux supérieurs de
Saint-Sulpice et de Saint-Lazare\,; puis demeurer fermé à quelque
proposition que ce pût être, et les plus spécieuses. Ouvrir les prisons,
et rappeler et rétablir les exilés, et la liberté, mais parler ferme aux
principaux, et donner au cardinal de Noailles et aux parlements des
ordres sévères et y être inexorable, pour que la liberté, bien loin de
se tourner en licence et en triomphe\,; se contienne dans les plus
étroites bornes de sagesse, de prudence, de modestie, de charité, de
respect pour l'épiscopat et pour les évêques, de mesure à l'égard de la
personne du nonce, de vénération pour celle du pape, de soumission pour
le saint-siège, et de toutes les précautions nécessaires pour éviter
toute occasion de donner prise à l'autre parti, et tout prétexte de
crier au schisme ou le faire craindre avec la plus légère apparence.

Après ce discours, que M. le duc d'Orléans écouta fort attentivement et
qu'il me parut goûter, je vins au point sensible. Je lui remis devant
les yeux le défaut des renonciations, où on n'avait voulu souffrir ni
formes ni apparence de liberté\,; et je lui répétai, ce que je lui avais
dit souvent, qu'il ne pouvait tirer aucun fruit de ces actes, si le
malheur du cas en arrivait, que de l'estime et de l'affection de la
nation par la sagesse, la douceur, l'estime de son gouvernement\,; que
ce que je lui proposais en était une des voies la plus assurée en
protégeant les lois, la raisonnable et juste liberté, et se rendant le
conservateur de ce qui dans l'ecclésiastique et le civil était en la
plus grande et solide réputation par la doctrine et la vertu, et
s'amalgamant les parlements et les autres tribunaux\,; tandis qu'en
prenant l'autre parti c'était un chemin de continuelles violences aux
consciences, aux lois ecclésiastiques et civiles, une suspension
continuelle de l'exercice et des fonctions de la justice, des exils et
des prisons sans fin, pour plaire à une cour impuissante, ingrate, qui
ne voulait que soumettre la France comme l'Espagne, le Portugal,
l'Italie, avec les inconvénients temporels et si serviles qu'en
éprouvent ces souverains rendus si dépendants de Rome en autorité et en
finance par les excès de l'immunité ecclésiastique, et pour des
mercenaires qui, de concert avec Rome, demanderaient toujours pour
régner, et ne sauraient gré d'aucun succès général ou particulier qu'à
leur artifice et à leur audace.

Je lui dis qu'il ne devait pas se faire illusion à lui-même, mais qu'il
devait bien comprendre et bien se persuader que les hommes ne se
conduisent jamais que par leur intérêt, excepté quelques rares exemples
de gens consommés en vertu\,; qu'il ne fallait donc pas qu'il s'imaginât
que quoi qu'il pût faire pour Rome, pour les jésuites et pour le parti
de la constitution, il pût jamais les gagner contre le roi d'Espagne\,;
que, pour peu qu'il fît de comparaison entre ce prince et lui, il
sentirait bien tôt lequel des deux emporterait tous leurs voeux et leur
choix, par conséquent tous leurs efforts\,; que leur but était de
régner, de dominer, de subjuguer la France comme sont l'Espagne, le
Portugal et l'Italie, à quoi ils n'avaient jamais eu plus beau jeu que
par le moyen de l'état où ils avaient su porter l'affaire présente\,;
qu'il n'y avait point aussi de prince plus expressément formé à leur gré
pour ce dessein, qu'un esprit accoutumé à se reposer de tout sur autrui,
dans l'habitude de tant d'années de règne sous le joug entier qu'ils
voulaient imposer ici, d'une conscience sans lumière, toujours
tremblante au nom de Rome et de l'inquisition, livré entièrement à
toutes les prétentions ultramontaines tournées en lois dans ses vastes
États, abandonné depuis toute sa vie aux jésuites, et à deux reprises,
dont la dernière était lors dans sa vigueur, au fabricateur de la
constitution, enfermé de plus par habitude et par goût, et inaccessible
à tout excepté à une épouse italienne pétrie des mêmes maximes romaines,
à son confesseur et à son ministre, et incapable par ses moeurs de
laisser aucun lieu de craindre rien qui puisse déranger des préventions
si favorables aux projets de Rome et des constitutionnaires et des
maximes ultramontaines qu'il tient être des parties intégrantes de la
religion. Avec un prince fait de la sorte, il n'y a qu'à vouloir et
faire\,; et l'État absolu et sans forme auquel il est accoutumé de
régner en Espagne joignant en lui, revenu en France, la jalousie de
l'autorité à ce qu'il croirait de si étroite obligation de sa
conscience, jusqu'à quels excès ne pourrait-il pas être mené sans autre
peine que de vouloir et de dire\,! «\,Croyez-vous, monsieur,
continuai-je, être en même parallèle avec tout votre esprit, votre
savoir, votre discernement, vos lumières, le dérèglement affiché de
votre vie, votre accès libre à tout le monde, vos connaissances étendues
et si extraordinaires à votre naissance, enfin avec ce mépris de la
religion, et ce libertinage d'esprit dont vous affectez de tout temps
une profession si publique\,? Pour peu que vous y pensiez un moment,
vous serez intimement convaincu que vous ne pouvez jamais devenir
l'homme de Rome et des jésuites, et qu'il ne manque au roi d'Espagne
aucune des qualités qui le rendent un roi fait et formé tout exprès pour
eux. Ôtez-vous donc bien exactement de la tête que, quoi que vous
puissiez faire, vous ayez jamais Rome, jésuites, constitutionnaires,
dans votre parti. Si le malheureux cas arrive, persuadez-vous au
contraire bien fortement que vous les aurez pour vos plus grands
ennemis, et qui n'auront rien de sacré contre vous. Si avec cela vous
allez prendre le parti qui leur est opposé, qui est celui des lois et de
l'estime publique\,; Si vous négligez de vous rapprocher les parlements
en cessant de les irriter par des violences à cet égard, des défenses de
recevoir des plaintes et d'y prononcer des évocations sans fin dès qu'il
y a le moindre trait véritable ou supposé à l'affaire de la
constitution, des cassations d'arrêts au gré des constitutionnaires, qui
est la chose qui blesse le plus les parlements, la totalité de la
magistrature, tout le public même le plus neutre et le plus indifférent,
et ce qui le révolte encore plus sans mesure\,; Si vous continuez et
redoublez même, comme l'extrémité où les choses se portent vous y
forceront, les exils, les prisons, les saisies de temporel, les inouïs
expatriments, les privations d'emplois et de bénéfices\,; qui aurez-vous
pour vous, si le malheureux cas arrive, de l'un ou de l'autre parti, ou,
s'il en reste encore, dans les termes où on viennent les choses, des
neutres et des indifférents\,?»

Je m'arrêtai là et n'en voulus pas dire davantage, pour juger de
l'impression que j'avais faite. Elle passa mon espérance sans toutefois
me rassurer\,; je vis un homme pénétré de l'évidence de mes raisons (il
ne fit pas difficulté de me l'avouer)\,; en même temps en brassière et
dans l'embarras d'échapper à ceux que j'ai nommés, et qui, dans ces
moments critiques de laisser aller le cours aux appels ou de les
arrêter, se relayaient pour ne le pas perdre de vue. Il raisonna sur
l'état présent de l'affaire et les inconvénients des deux côtés\,; il
convint de toute la force de ce que je lui avais représenté. Je ne
disais alors que quelques mots de traverse pour le laisser parler, et le
bien écouter\,; et je ne vis qu'un homme, convaincu à la vérité, et de
son aveu, sans réponse à pas une des raisons que je lui avais
représentées, mais un homme dans les douleurs de l'enfantement. Nous en
étions là, quand la toile tomba. Nous fûmes tous deux surpris et fâchés
de la fin du spectacle. Malgré le brouhaha qu'il produisit par
l'empressement de chacun pour sortir, nous demeurâmes encore quelques
moments sans pouvoir cesser cette conversation. Je la finis on lui
disant que le nonce ne le connaissait que trop bien quand il disait que
le dernier qui lui parlait avait raison\,; que je l'avertissais qu'il
était veillé par des gens qu'il se croyait affidés et qui ne l'étaient
qu'à eux-mêmes, à leurs vues, à leurs intrigues, à leurs intérêts, et
veillé comme un oiseau de proie\,; qu'il serait la leur s'il ne prenait
bien garde à lui, parce que la vérité n'avait pas auprès de lui des
surveillants si à portée ni si empressés\,; qu'il prît donc garde au
trop vrai dire du nonce, et qu'il ne se laissât pas misérablement
entraîner. Là-dessus il sortit de sa petite loge, et moi avec lui. Tout
le dehors était rempli de tout ce qui successivement s'y était amassé
pour entrer dans sa loge ou l'en voir sortir, dont la plupart le
regardèrent attentivement, et moi encore plus. Il était si concentré de
tout ce que nous venions de dire qu'il passa assez sombrement. Il alla
dans son appartement avec tout ce monde, dans le fond duquel j'aperçus
Effiat et Besons. Effiat avait été apparemment averti du tête-à-tête de
l'Opéra, et s'était fortifié de Besons pour saisir le court moment de la
fin de la journée publique, et du commencement de la soirée des roués,
pour explorer ce qui s'était passé et le détruire à la chaude. Je ne
sais ce qu'ils devinrent, car je m'en allai aussitôt.

Mais pour ne pas revenir aux appels, je ne dis que trop vrai au régent
en sortant de la petite loge. Il fut si bien veillé, relayé, tourmenté
qu'ils l'emballèrent. D'Effiat, le premier président et les autres
l'emportèrent. Le régent arrêta les appels, mit toute son autorité à
empêcher celui du parlement, et lui fit suspendre un arrêt contre des
procédures monstrueuses de l'archevêque de Reims, et contre d'autres
fureurs d'évêques constitutionnaires. Je me contentai d'avoir convaincu,
et puis je laissai faire, sans courir ni recommencer à raisonner avec un
prince que je savais circonvenu de façon que sa facilité et sa faiblesse
serait incapable de résistance. Il devint enfin tout ce qu'ils
voulurent, entraîné par leur torrent\,; et il en arriva dans les deux
partis le fruit que je lui avais prédit par leurs sentiments à son
égard. S'il m'avait cru, ou plutôt s'il en avait eu la force, la
constitution tombait avec toutes ses machines et ses troubles, l'Église
de France serait demeurée en paix, et Rome de plus eût appris par un si
fort exemple à ne la plus troubler de ses artifices et de ses
ambitieuses prétentions. Le pape, si soutenu par tant d'évêques en
France, ou ignorants, ou simples, ou ambitieux, et si continuellement
pressé et tourmenté par son nonce et par les autres boutefeux de se
porter à des démarches violentes, n'avait jamais osé s'y commettre. Il
avait menacé trop souvent pour qu'on n'y fût pas accoutumé. Il ne
s'agissait pourtant que de sévir contre la personne du cardinal de
Noailles en particulier, et en gros contre d'autres de son parti, en
dernier lieu contre les appelants. Rien ne fut oublié de la part de
Bentivoglio et des furieux pour l'y engager, sans que jamais il ait osé
passer les menaces, et encore sans s'en expliquer. Pouvait-on craindre
qu'il se fût porté à des extrémités contre ce nombre immense d'appelants
en corps et en particuliers, écoles célèbres et nombreuses, diocèses
entiers, congrégations fameuses et étendues, contre les parlements qu'il
a toujours redoutés, on un mot contre le régent à la tête de tout le
royaume, armé de ses lois, des canons, de la discipline de l'Église
reconnue et pratiquée jusque sous le feu roi. Rien de schismatique en
cette démarche de l'appel de tout temps, encore une fois pratiquée et
suspensive dans l'Église\,; on ne le devient point quand on ne veut pas
l'être, et le pape se serait bien gardé de se risquer la France pour un
sujet aussi dépourvu de tout fondement après les pertes que Rome a
faites de plus de la moitié de l'Europe. Il se serait donc réduit à des
plaintes, à se contenter des respects qu'on ne lui aurait pas épargnés,
et à se satisfaire comme d'un gain des assurances qu'il aurait exigées
qu'on ne parlant plus de sa bulle, personne aussi n'aurait la témérité
de la combattre en aucune sorte ni occasion, puisqu'il ne s'en agirait
plus\,; que de part et d'autre on laisserait tomber tout ce qui s'était
fait là-dessus, et qu'il serait même remercié de sa condescendance. Ce
qu'on verra bientôt qui arriva sur les bulles est une démonstration que
les choses se seraient passées aussi doucement que l'opinion que j'en
avais, et que je rapporte ici. Je n'ajouterai rien sur la façon dont
parut peu après l'appel du cardinal de Noailles, ni des divers succès
qu'il eut, qu'on a vu que je lui avais prédits pour l'avoir trop
différé\,; cela appartient à la constitution sans avoir produit
d'occasion qui me regarde.

\hypertarget{chapitre-xiii.}{%
\chapter{CHAPITRE XIII.}\label{chapitre-xiii.}}

1717

~

{\textsc{M\textsuperscript{lle} de Chartres prend l'habit à Chelles.}}
{\textsc{- Mort d'Armentières.}} {\textsc{- Mort du duc de Béthune.}}
{\textsc{- Mort de M\textsuperscript{me} d'Estrades.}} {\textsc{- Son
beau-fils va en Hongrie avec le prince de Dombes.}} {\textsc{- Indécence
du carrosse du roi expliquée.}} {\textsc{- Maupeou président à mortier,
depuis premier président.}} {\textsc{- Nicolaï obtient pour son fils la
survivance de sa charge de premier président de la chambre des
comptes.}} {\textsc{- Bassette et pharaon défendus.}} {\textsc{- Mort et
famille de la duchesse douairière de Duras.}} {\textsc{- Mort de la
duchesse de Melun.}} {\textsc{- Mort de la comtesse d'Egmont.}}
{\textsc{- Mort de M\textsuperscript{me} de Chamarande.}} {\textsc{-
Éclaircissement sur sa naissance.}} {\textsc{- Mort de l'abbé de
Vauban.}} {\textsc{- Mariage d'une fille de la maréchale de Boufflers
avec le fils unique du duc de Popoli.}} {\textsc{- Le duc de Noailles
manque le prince de Turenne pour sa fille aînée, et la marie au prince
Charles de Lorraine, avec un million de brevet de retenue sur sa charge
de grand écuyer\,; et un triste succès de ce mariage.}} {\textsc{- M. le
comte de Charolais part furtivement pour la Hongrie par Munich.}}
{\textsc{- Personne ne tâte de cette comédie.}} {\textsc{- Il ne voit
point l'empereur ni l'impératrice, quoique le prince de Dombes les eût
vus, dont M. le Duc se montre fort piqué.}} {\textsc{- L'abbé de La
Rochefoucauld va en Hongrie et meurt à Bude.}} {\textsc{- Conduite de M.
et de M\textsuperscript{me} du Maine dans leur affreux projet.}}
{\textsc{- Causes et degrés de confusion et de division dont ils savent
profiter pour se former un parti.}} {\textsc{- Formation d'un parti
aveugle composé de toutes pièces sans aveu de personnes, qui ose de
soi-même usurper le nom de noblesse.}} {\textsc{- But et adresse des
conducteurs.}} {\textsc{- Folie et stupidité des conduits.}} {\textsc{-
Menées du grand prieur et de l'ambassadeur de Malte pour en exciter tous
les chevaliers, qui reçoivent défense du régent de s'assembler que pour
les affaires uniquement de leur ordre.}} {\textsc{- Huit seigneurs
veulent présenter au nom de la prétendue noblesse un mémoire contre les
ducs.}} {\textsc{- Le régent ne reçoit point le mémoire et les traite
fort sèchement.}} {\textsc{- Courte dissertation de ces huit
personnages.}} {\textsc{- Embarras de cette noblesse dans
l'impossibilité de répondre sur l'absurdité de son projet.}}

~

M\textsuperscript{lle} de Chartres ayant persévéré longuement à vouloir
être religieuse contre le goût et les efforts de M. le duc d'Orléans, il
consentit enfin qu'elle prît l'habit à Chelles, dont une soeur du
maréchal de Villars était abbesse. M. {[}le duc{]} et
M\textsuperscript{me} la duchesse d'Orléans y allèrent, et n'y voulurent
personne. L'action fut ferme et édifiante, et tout s'y passa avec le
moins de monde et le plus de simplicité qu'il fut possible.

Armentières mourut chez lui en Picardie, assez jeune, d'une fort longue
maladie. Il était premier gentilhomme de la chambre de M. le duc
d'Orléans, qui donna cette place à son frère Conflans, qui était aussi
son beau-frère, comme on l'a vu ailleurs. Il était surprenant de trouver
en ce M. d'Armentières un homme aussi parfaitement bouché, avec deux
frères qui avaient tant de savoir et d'esprit\,; d'ailleurs bon et
honnête homme.

Le duc de Béthune mourut à soixante-seize ans. C'était un bon et
vertueux homme. J'ai parlé plus d'une fois de la fortune de son père et
de lui, qu'il vit refleurir en lui et en son fils et son petit-fils
après une légère éclipse, et qui après lui augmenta encore beaucoup.

M\textsuperscript{me} d'Estrades mourut aussi. Elle était soeur de
Bloin, premier valet de chambre du roi, et avait été fort belle. Le fils
aîné du maréchal d'Estrades l'avait épousée en secondes noces par amour.
Elle était mère de M\textsuperscript{me} d'Herbigny. La considération
que M. le duc d'Orléans conserva toujours pour la famille du maréchal
d'Estrades, qui avait été son gouverneur, et un homme illustre dans les
armes et dans les négociations, dont M\textsuperscript{me} d'Herbigny
était petite-fille, fit uniquement son mari conseiller d'État. Le comte
d'Estrades lieutenant général, de la belle-mère de qui en vient de dire
la mort, se laissa engager par M. du Maine à aller en Hongrie avec le
prince de Dombes. C'était un honnête homme et de distinction à la
guerre. Le régent le lui permit, mais le roi ni lui n'y entrèrent pour
rien.

Le roi s'alla promener au cours. Il était au fond de son carrosse, serré
entre le duc du Maine et le maréchal de Villeroy avec la dernière
indécence. Tant que le feu roi admit des hommes dans son carrosse,
jamais aucun prince du sang n'y a été à côté de lui. C'était un honneur
réservé aux seuls fils de France. M. le Prince le dernier donnant au roi
une fête à Chantilly, où était toute la cour, il se trouva pendant le
voyage une fête d'Église solennelle, pour laquelle le roi alla à la
paroisse du lieu, seul, dans sa calèche qui n'était qu'à deux places sur
le derrière, le devant étant accommodé pour y mener des chiens
couchants. Jamais personne n'y montait avec lui, sinon Monseigneur ou
Monsieur, encore si rarement qu'il ne se pouvait davantage. On regarda
comme une distinction fort grande due à la magnificence de la fête de
Chantilly, et à la nouveauté du mariage de M\textsuperscript{me} la
Duchesse, que le roi sortant de l'église, et monté dans sa calèche,
voyant M. le Prince à la portière, lui ordonna d'y monter et de se
mettre auprès de lui, parce qu'il n'y avait point d'autre place. C'est
l'unique fois que cela soit arrivé. Le maréchal de Villeroy avait bien
dans le carrosse du roi, comme son gouverneur, une place de préférence,
mais non pas de préséance sur le grand écuyer, ni sur le grand
chambellan, ni même sur le premier gentilhomme de la chambre en année.
Mais tout était en pillage et en indécence, qui s'augmenta sans cesse en
tout de plus en plus.

Maupeou, maître des requêtes, fit un marché extraordinaire avec Menars,
président à mortier, pour s'assurer sa charge et lui en laisser la
jouissance sa vie durant à certaines conditions. Le prix fut sept
cent-cinquante mille livres et vingt mille livres de pot-de-vin. Je ne
marque cette bagatelle que parce que le même Maupeou est devenu premier
président\footnote{René-Charles de Maupeou fut reçu président à mortier
  le 23 mars 1718, devint premier président en 1743, et, longtemps après
  la mort de Saint-Simon, garde des sceaux et vice-chancelier de France
  en 1763, enfin chancelier en 1768 (15 septembre). Il céda presque
  immédiatement cette charge à son fils, René-Nicolas-Charles-Augustin
  de Maupeou, qui s'est rendu célèbre par sa lutte contre les
  parlements.}, et a fait passer à son fils sa charge de président à
mortier, tous deux avec réputation. Peu de jours après, Nicolaï, premier
président de la chambre des comptes, obtint la survivance de cette
charge pour son fils. Ce fut comme bien d'autres une grâce perdue pour
M. le duc d'Orléans, qui ne trouva pas ce magistrat par la suite moins
singulièrement audacieux à son égard. Ce prince fit plus utilement par
la défense sévère qui fut publiée de la bassette et du pharaon sans
distinction de personne. Ce débordement de ces sortes de jeux quoique
défendus était devenu à un point, que les maréchaux de France avaient
établi à leur tribunal qu'on ne serait point obligé à payer les dettes
qu'on ferait à ces sortes de jeux.

La duchesse de Duras mourut à Paris à cinquante-huit ans d'une longue
maladie\,; elle était veuve dès 1697 du duc de Duras, fils et frère aîné
des deux maréchaux de Duras. Il n'avait que vingt-sept ans, et ne lui
avait laissé que deux filles, dont elle avait marié l'aînée, comme on
l'a vu en son temps, au prince de Lambesc, petit-fils de M. le Grand, et
avait, comme on le verra, arrêté le mariage de l'autre lorsqu'elle
mourut. Son nom était Eschallard\,; elle était fille de La Boulaye, qui
fit un moment tant de bruit à Paris dans le parti de M. le Prince, et
qui est si connu dans les histoires et les mémoires de la minorité de
Louis XIV. La Boulaye avait épousé une fille unique du baron de Saveuse,
et il fut tué maréchal de camp au malheureux combat du maréchal de
Créqui à Consarbruck, en 1675. Son père avait épousé en 1633 une fille
d'Henri-Robert de La Marck, comte de Brame, capitaine des Cent-Suisses
de la garde du roi, mort en 1652, fils de Charles-Robert, comte de
Maulevrier et chevalier du Saint-Esprit, aussi capitaine des
Cent-Suisses, frère puîné du père de l'héritière de Bouillon, Sedan,
etc., qu'épousa le vicomte de Turenne, dit depuis le maréchal de
Bouillon, contre lequel après la mort sans enfants de l'héritière, il en
prétendit la succession, se fit appeler duc de Bouillon, disputa toute
sa vie et précéda partout le maréchal de Bouillon. On a assez parlé
ailleurs de toute cette grande affaire et de toute cette descendance. Le
marquis de Mauny, frère cadet du beau-père de La Boulaye, qui était
chevalier du Saint-Esprit, capitaine des gardes, puis premier écuyer de
la reine-mère et la marquise de Choisy-L'Hospital si connue dans le
grand monde, soeur de M\textsuperscript{me} de La Boulaye, n'ayant point
eu d'enfants, ni cette dernière de frère, La Boulaye son mari prit
hardiment le nom et les armes de La Marck, que sa postérité a conservés,
quoiqu'il restât une branche de la maison de La Marck, comtes de Lumain
en Wéteravie, dont est demeuré seul de ce grand nom le comte de La
Marck, chevalier du Saint-Esprit et de la Toison, grand d'Espagne, connu
par ses ambassades, dont le fils unique a épousé une fille du duc de
Noailles.

La duchesse de Melun, fille du duc d'Albret, mourut dans la première
jeunesse, étouffée dans son sang en couches, pour n'avoir point voulu
être saignée dans sa grossesse qui était la première. La fille dont elle
accoucha ne vécut pas.

La comtesse d'Egmont mourut aussi à Paris. Elle était nièce de
l'archevêque d'Aix, si connu par les aventures de sa vie et commandeur
de l'ordre, et parente proche des Chalais. M\textsuperscript{me} des
Ursins, qui aimait fort tout ce qui appartenait à son premier mari,
étant à Paris avant la mort de son second mari, l'avait fait venir de sa
province chez elle, où elle demeura jusqu'à son mariage avec le dernier
de la maison d'Egmont, dont elle n'eut point d'enfants, et dont elle
était veuve.

Chamarande perdit sa femme, qui avait du mérite, et qui était fille du
comte de Bourlemont, lieutenant général et gouverneur de Stenay, frère
de l'archevêque de Bordeaux. J'observerai, pour la curiosité, qu'on
disait que ces Bourlemont portaient le nom et les armes d'Anglure, dont
ils n'étaient point\,; que leur nom est Savigny, qui sûrement ne vaut
pas l'autre. Chrestien de Savigny, seigneur de Rosne, s'attacha au duc
d'Alençon, dont il fut chambellan, et par sa valeur et ses talents
s'éleva dans les emplois et se fit un nom. À la mort de son maître, il
s'attacha aux Guise, alors tout-puissants, et devint, par son esprit, un
de leurs principaux confidents et un des chefs de la Ligue sous eux.
Lorsque, après le meurtre d'Henri III, le duc de Mayenne attenta à tout,
jusqu'aux fonctions de la royauté, de Rosne fut un des maréchaux de
France qu'il fit, avec MM. de La Châtre et de Brissac, et d'autres qui
le demeurèrent par leurs traités avec Henri IV\,; mais de Rosne n'en eut
pas le temps. Il était lieutenant général de Champagne et commandait à
Reims pour la Ligue\,; il était devenu fort audacieux, et son
attachement pour le duc de Mayenne, dont il tenait son prétendu bâton de
maréchal de France, ne lui avait {[}point{]} donné d'affection pour le
jeune duc de Guise qui, par s'être échappé de la prison où il avait été
mis lorsque son père et le cardinal son oncle furent tués à Blois, avait
ôté toute espérance au duc de Mayenne de faire couronner son fils avec
l'infante d'Espagne par les prétendus états généraux assemblés à Paris.
Le duc de Guise, allant en Champagne, y donna ses ordres que Rosne ne se
crut pas obligé de suivre. Étant l'un et l'autre à Reims, les disputes
s'échauffèrent tellement, qu'en pleine place publique le duc de Guise,
poussé à bout de son insolence, lui passa son épée à travers du corps et
le tua roide. C'est ce même de Rosne qui avait épousé la fille unique et
héritière de Jacques d'Anglure, seigneur d'Estoges, en qui cette branche
d'Estoges finit, et qui était frère aîné {[}de{]} René d'Anglure,
seigneur de Givry en Argonne, qui a fait la branche de Givry. Pour
revenir au prétendu maréchal de Rosne, il eut un fils que son grand-père
maternel substitua aux nom et armes d'Anglure\,; mais ces faux Anglure
n'ont point prospéré et sont demeurés obscurs. Le comte de Bourlemont,
ami de mon père, frère des archevêques de Toulouse et de Bordeaux, et
père de la femme de Chamarande, était fils puîné de Nicolas d'Anglure,
quatrième descendant d'autre Nicolas d'Anglure, chef de la branche de
Bourlemont et d'Is, du Châtelet, lequel était puîné de Simon d'Anglure,
vicomte d'Estoges, mort en 1499. En voilà assez pour revendiquer cette
vérité.

En même temps mourut l'abbé de Vauban, uniquement connu pour avoir été
frère du célèbre maréchal de Vauban.

La maréchale de Boufflers, qui n'avait pas grand'chose à donner à sa
seconde fille, conclut son manège avec le fils unique du duc de Popoli,
duquel il a été parlé plus d'une fois. Excepté d'aller en Espagne, le
nom, les établissements, les biens, tout était à souhait. Une place de
dame du palais de la reine d'Espagne attendait la nouvelle mariée en
arrivant. Popoli, toujours épineux, ne voulut pas que le prince de
Pettorano vînt jusqu'à Pâris, parce que les fils aînés des grands ont en
Espagne des distinctions qui sont inconnues en France. Il s'arrêta donc
à Blois, et y attendit six semaines la maréchale de Boufflers, qui y
mena sa fille. Le mariage s'y fit, et les deux époux partirent deux
jours après pour Madrid. Si Dieu me donne le temps d'écrire mon
ambassade en Espagne, j'aurai lieu de dire quel fut le triste succès de
ce mariage.

Il s'en fit un autre en même temps, qui ne réussit pas mieux, mais qui
ne fit le malheur de personne. La faveur du duc de Noailles, et beaucoup
plus la place et l'autorité entière qu'il avait dans les finances,
tentèrent le duc d'Albret de finir par une alliance les longs et fâcheux
démêlés des deux maisons. Le comte d'Évreux, qui en sentit l'importance
pour un rang et un échange aussi peu solide que le leur, n'oublia rien
pour y réussir. L'affaire fut même si avancée, qu'ils la crurent faite,
et que des deux côtés elle fut donnée comme telle. Néanmoins elle se
rompit par tout ce que le duc d'Albret ne cessa de prétendre, dont son
frère le blâma au point que, pour ne pas irriter le crédit du duc de
Noailles, il demeura toujours de ses amis. Le duc d'Elboeuf, qui n'avait
pas les mêmes raisons, mais qui fut toute sa vie fort avide, avait envie
de marier le prince Charles qu'il regardait comme son fils, et qui, avec
ses grands établissements en survivance, n'avait point de bien. Il crut
trouver dans ce mariage une alliance convenable et tous les avantages
d'une affaire purement d'argent pour le prince Charles, et pour soi-même
le moyen de puiser dans les finances.

Le duc de Noailles, piqué de la rupture du duc d'Albret, se trouva
flatté de trouver sur-le-champ un prince véritable, au lieu du faux qui
lui manquait avec des établissements extérieurs encore plus éblouissants
qui le firent passer pardessus l'inconvénient des biens, immenses chez
les Bouillon, nuls dans le prince Charles. Ainsi le mariage également
désiré fut bientôt arrêté, moyennant huit cent mille livres, et ce que
l'on ne disait pas, et la patte du duc d'Elboeuf largement graissée. Les
deux familles obtinrent pour le prince Charles un million de brevet de
retenue sur la charge de grand écuyer, publiquement volée à mon père, et
qui ne leur avait jamais rien coûté, comme on l'a vu au commencement de
ces Mémoires. Jamais on n'avait ouï parler d'un pareil brevet de
retenue, qui assurait à toujours la charge dans la famille\,; parce que
personne ne pouvait être en état de le payer. Le cardinal de Noailles
les maria dans sa chapelle, et donna un grand dîner à l'archevêché, et
le soir il y eut une fête à l'hôtel de Noailles, où sur le minuit M. le
duc d'Orléans alla donner la chemise au prince Charles, qui voulut
continuer d'être nommé ainsi, et sa femme la comtesse d'Armagnac, comme
on appelait la femme de M. Le Grand. Celle-ci n'avait pas encore treize
ans\,; ainsi le mari ne fut au lit avec elle qu'un moment pour la
cérémonie, et chacun demeura chez soi jusqu'à un temps fixé, qu'elle
alla chez son mari, où elle ne demeura pas longtemps. Tant que le duc de
Noailles eut les finances, tout alla à merveilles\,; vers leur déclin,
les rats le sentirent, et se hâtèrent de dénicher. Une très légère
imprudence de M\textsuperscript{me} d'Armagnac causa un éclat qui dure
encore. Elle entra aux filles de Sainte-Marie du faubourg Saint-Germain,
où une soeur de son père était religieuse, et où elle vécut plusieurs
années très régulièrement. Elle y reçut toute la maison de Lorraine,
hommes et femmes, qui prirent son Parti contre son mari,
M\textsuperscript{me} d'Armagnac même, qui en demeurèrent brouillés avec
lui, et des compliments de M. {[}le duc{]} et de M\textsuperscript{me}
la duchesse de Lorraine. Il n'y eut que le duc d'Elboeuf qui ne vit plus
aucun Noailles, et qui ne les épargna pas. Le prince Charles ne salua
même plus son beau-père, et ils en sont demeurés là. Au bout de quelques
années, M\textsuperscript{me} d'Armagnac alla demeurer à l'hôtel de
Noailles. Elle aborda la haute dévotion, et à là fil a pris une maison à
elle fort éloignée de toutes celles de ses parents. La dévotion n'y nuit
point à l'intrigue si naturelle aux Noailles. Mais il n'y a jamais en
moyen d'obtenir du prince Charles qu'elle mît les pieds à la cour.

M. le comte de Charolais, étant à Chantilly, fit semblant le 30 avril
d'aller courre le sanglier dans la forêt d'Halatre, suivi de Billy tout
seul, qui était un gentilhomme de M. le Duc, qui avait beaucoup de sens
et de mérite, et ils ne revinrent plus. M. le Duc, qui était à
Chantilly, revint à Paris le lendemain essayer de persuader M. le duc
d'Orléans et le monde qu'il n'avait aucune part à cette équipée, dont il
n'avait pas su un mot. M\textsuperscript{me} la Duchesse tint le même
langage. Deux jours après, ils reçurent tous des lettres datées de Mons
de M. de Charolais et de Billy, remplies {[}de demandes{]}, de pardons
de son départ sans leur permission, et d'excuses de Billy sur les
serments du secret que M. de Charolais lui avait fait faire avant que de
lui, déclarer de quoi il s'agissait. Il ajoutait que ce prince prendrait
incognito, sous le nom de comte de Dammartin, la route de Munich, où il
attendrait leurs ordres et leurs secours. Personne ne fut un moment la
dupe de cette partie de main, dont la maison de Condé ne tira pas le
fruit qu'elle s'en était promis. M\textsuperscript{me} la Princesse et
la duchesse d'Hanovre, mère de l'impératrice, étaient soeurs.
M\textsuperscript{me} la Duchesse et M. le Duc espérèrent intimider M.
le duc d'Orléans par ce voyage à Vienne et en Hongrie, et par cet air de
fuite et de secret n'avoir point à répondre de ce qui s'y passerait.
L'artifice était trop grossier pour laisser imaginer à qui que ce fût
qu'un prince du sang de dix-sept ans fût parti de Chantilly pour la
Hongrie sans l'aveu d'une mère et d'un frère aîné tels que
M\textsuperscript{me} la Duchesse et M. le Duc. Le seul accompagnement
de Billy, connu pour avoir leur confiance, aurait levé le voile. M. le
duc d'Orléans ne prit aucune inquiétude de cette disparate, qui en effet
n'en pouvait donner la plus légère. Il se contenta de n'y prendre aucune
part, et ne fut pas fâché de plus de se trouver par là hors d'atteinte
des attaques de bourse pour fournir aux frais. M. de Charolais fut
magnifiquement reçu à Munich par l'électeur de Bavière, qui avait
continuellement vécu avec M\textsuperscript{me} la Duchesse dans tous
ses voyages à Paris et à la cour. Il fit présent à ce prince de beaucoup
de chevaux tant pour sa personne que pour ses gens. Mais à Vienne, il ne
put voir ni l'empereur ni l'impératrice. M. le Duc en fut extrêmement
piqué et s'en prit vainement à Bonneval, qu'il crut l'avoir empêché. On
ne comprit point quelle en fut la difficulté, puisque le prince de
Dombes, arrivé auparavant, les avait vus. Quelque différence réelle
qu'il y eût entre eux deux, il n'y en avait alors aucune pour le rang et
pour tout l'extérieur. Le prince de Dombes avait bien sûrement sa leçon
très distincte, et M. du Maine était trop attentif à la qualité de
prince du sang, dont il jouissait alors en plein, et qu'il avait
conquise pour soi et pour ses enfants, pour en avoir commis la moindre
chose sur un si grand théâtre. Apparemment que M. le comte de Charolais
en voulut plus qu'on n'avait donné à M. de Dombes\,; cependant
l'incognito couvrait tout. Il est vrai que MM. les princes de Conti
n'avaient point vu l'empereur Léopold à leur voyage de Hongrie, ni en
allant ni revenant, qui ne voulut pas leur donner le fauteuil comme aux
électeurs\,; mais il est vrai aussi qu'ils passèrent à Vienne à visage
découvert.

On a vu, on son temps, tout ce que l'abbé de La Rochefoucauld eut à
essuyer de sa famille, à la fin du règne du feu roi\,; et depuis, qui le
voulait forcer, lorsqu'il fut devenu l'aîné, à céder tous ses droits
d'aînesse à son frère, ou à quitter tous ses riches bénéfices, sans lui
en donner de dédommagement. Enfin, ils le résolurent à s'en aller en
Hongrie avec une dispense du pape de porter l'épée trois ans en gardant
ses bénéfices. Le prince Eugène, le chevalier de Lorraine, le marquis de
Forbin lieutenant général et capitaine des mousquetaires gris, et bien
d'autres, ont toujours servi avec des abbayes sans dispenses, et ont
porté l'épée et gardé leurs bénéfices jusqu'à la mort, sans être
chevaliers de Malte ni de Saint-Lazare\,; mais le scrupule convenait aux
desseins de M. et M\textsuperscript{me} de La Rochefoucauld. Il n'a pas
paru que Dieu y ait répandu sa bénédiction\,; mais en attendant, ils
furent tous bien soulagés. L'abbé de La Rochefoucauld partit mal
volontiers peu de jours après M. de Charolais\,; il arriva à Bude, où,
avant d'avoir joint l'armée impériale, il fut pris de la petite vérole,
et en mourut.

On a vu à la mort du roi le succès de la noire et profonde scélératesse
du duc de Noailles à mon égard\,; par une calomnie et une perfidie qui
a, je crois, peu d'exemples, et combien elle seconda le projet du duc et
de M\textsuperscript{me} la duchesse du Maine, résolue à bien tenir les
épouvantables paroles qu'elle avait dites à Sceaux aux ducs de La Force
et d'Aumont. On les a vues, t. XI, p.~421, et à propos de quoi elles
furent dites\,; mais il est nécessaire ici de les répéter. Les voici\,:
«\,Qu'elle voulait bien leur dire, pour qu'ils ne prétendissent pas en
douter, que quand on avait une fois acquis l'habilité de succéder à la
couronne, il fallait plutôt que se la laisser arracher, mettre le feu au
milieu et aux quatre coins du royaume.\,» Ces furieuses paroles furent
les dernières de cette belle conférence qui fut unique. Ce fut dans la
vue d'une si monstrueuse exécution, si besoin en était, qu'ils
continuèrent plus que jamais d'échauffer tout ce qu'ils purent contre
les ducs\,; premièrement pour effrayer et se maintenir dans leurs
usurpations contre eux, en empêchant par ce bruit, tout jugement dans la
suite\,; secondement pour, sous prétexte de l'objet des ducs, s'attacher
et se former un parti, dont ils pussent faire à leur gré toutes sortes
d'autres usages, à quoi ils ne cessèrent de travailler tant que le roi
vécut, surtout sur la fin.

Une image d'ordre et de distinction s'était soutenue jusqu'à la mort du
roi, au milieu de toutes les entreprises et de toute décadence. Après
lui, le peu de dignité de M. le duc d'Orléans jusque pour lui-même, sa
légèreté, sa facilité, sa politique si favorite, \emph{divide et
impera}, confondirent tout à son avènement à la régence. Plus de cour,
un roi enfant, ni reine ni dauphine, et deux uniques veuves de fils de
France\,: Madame, toujours enfermée, sa toilette et son dîner fort
déserts\,; M\textsuperscript{me} la duchesse de Berry renfermée ou en
parties, voulant et ne voulant point de cour, et se trouvant fort
abandonnée, imagina d'en réchauffer une, en permettant aux dames d'y
venir en robes de chambre\,; établit des tables, de jeu, et on retint
plusieurs à souper tous les soirs. Cela éclipsa les tabourets, parce
que, y ayant cette heure commode de la voir, on ne tint plus compte
d'aller à sa toilette, ni guère plus d'aller aux audiences qu'elle
donnait aux ambassadeurs, ni à celles de Madame, laquelle on avait
négligée assez de tout temps. Dès les dernières années du roi, les
princes et les princesses du sang, dont le temps n'a voit pu diminuer le
dépit du rang de M. {[}le duc{]} et de M\textsuperscript{me} la duchesse
d'Orléans, qu'en dernier lieu la prétention pour ses filles avait encore
aigri, s'étaient établis sur de petites chaises à des de paille, plus
mobiles, disaient-elles, et plus légères et commodes pour travailler et
pour jouer. Par ce moyen, plus de distinction de sièges, et ils ne
prenaient et ne donnaient des fauteuils à qui ils en devaient, que
lorsqu'ils ne pouvaient s'en dispenser en des visites de cérémonie,
comme de mort, de mariage, etc. Les gens de qualité, accoutumés ainsi à
ne trouver plus de différence d'avec les gens titrés, commencèrent
bientôt à ne plus donner puis offrir leurs places, en quoi les gens
titrés leur avaient montré un fort sot exemple depuis plus longtemps,
qu'ils avaient cessé entre eux le même usage presque tous. Je l'avais
trouvé établi en entrant dans le monde\,; il ne cessa peu à peu que
longtemps depuis. Moi et quelques autres ducs et duchesses l'avions
toujours conservé\,; la maison de Lorraine l'avait continué par aînesse,
et ses singes de Rohan et de Bouillon n'y manquaient pas non plus
chacune entre elles. Mais toutes trois eurent à cet égard la même
nouvelle conduite à essuyer que les ducs et les duchesses.

Rien ne pouvait être plus agréable à M. et à M\textsuperscript{me} du
Maine. La division était leur salut. Ils l'avaient procurée et mise au
comble entre les ducs et le parlement, ils n'oublièrent rien pour la
porter aussi loin qu'elle put aller entre les ducs et tous ceux qui ne
l'étaient pas, en même temps pour profiter de l'une et de l'autre à
lier, unir et amalgamer ensemble le parlement, et tout ce qu'ils
pouvaient animer de gens contre les ducs. Ils y parvinrent bientôt, et
dès que leurs mesures là-dessus eurent réussi, ils commencèrent à former
et à organiser leur parti sans y paraître à découvert.

Ce mélange de gens de qualité de moindre, et des plus petits compagnons,
ne blessa point ceux de la plus grande naissance, et pour faire nombre
tout leur fut bon. Quelques gens d'esprit de la première qualité
passèrent là-dessus pour parvenir à grossir assez, pour, après le
prétexte des ducs, venir à des choses plus importantes, à ventiler le
gouvernement et parvenir à ce que se proposent ceux qui s'élèvent contre
le roi ou le régent ou le premier ministre, comme on a vu dans tous les
troubles domestiques et les guerres civiles de tous les âges de la
monarchie. Le grand nombre de ces gens de toutes qualités étaient menés
par le nez, comme il arrive toujours, par le chef ou les chefs, et le
petit nombre de leurs confidents, qui détachent des émissaires, et qui
tournent les esprits, sous divers prétextes, à faire tout ce qui leur
convient, et ce qui ne convient qu'à eux\,; et qui se rient et se
moquent de ce grand nombre d'instruments dont ils font la même sorte de
cas qu'un artisan et un ouvrier font de leurs outils, dont tout le
travail n'est utile qu'à eux, et est inutile aux outils mêmes, qui,
après avoir bien servi leurs maîtres, deviennent usés, ébréchés, cassés,
et ne sont plus de nul usage, ni ramassés par personne. Tel fut ce
groupe qui, depuis les Châtillon, les Rieux, etc., jusqu'aux Bonnetot et
autres fils de secrétaires du roi ou de fermiers, osèrent se produire
comme un corps sous l'auguste nom du second des trois États du royaume,
de leur unique autorité. Ce fut donc ce monstre sans titre légitime, ni
même l'ombre illégitime, sans convocation, sans élection, sans pouvoir,
ni instruction ni commission, {[}qui{]} se donna sous le nom de la
noblesse, dont les trois quarts auraient eu grande peine à prouver la
leur. Je n'en nomme aucun, parce que je ne prétends pas entrer en des
généalogies, qui n'ont d'autre fruit que de désoler ceux qui ne peuvent
montrer de vérité, et si j'ai nommé ce Bonnetot, c'est par le contraste
d'avoir pour sa richesse épousé une fille de M. de Châtillon, et
{[}avoir été{]} admis par lui, et en sa considération, par tous les
autres, à être indistinctement regardé comme M. de Châtillon même, et à
son exemple, tous les gens de peu ou de rien qui s'empressèrent d'y
entrer, pour se faire un titre dans les suites d'avoir été de ces
assemblées de la noblesse qui commencèrent à se tenir tantôt chez l'un,
tantôt chez l'autre.

Mais dans ces assemblées où sans savoir pourquoi on rugissait contre les
ducs d'impulsion du duc et de la duchesse du Maine, l'embarras fut
longtemps d'un objet particulier. Ils éclataient en plaintes qu'ils
faisaient retentir partout avec une sorte de tumulte tantôt que les ducs
prétendaient faire un corps à part de la noblesse, tantôt que la
noblesse ne voulait plus que les ducs fissent corps avec elle. On
débitait des choses qui ne se pouvaient appeler que de véritables
pauvretés, sans nombre, sans vérité, sans la moindre apparence, sans
aucune sorte d'existence, de tentatives des ducs, les unes ridicules,
les autres parfaitement inutiles ou indifférentes, quand même elles
auraient existé, telles qu'on aurait honte de les rapporter et de les
réfuter. Elles tombaient aussi d'elles-mêmes à mesure qu'elles étaient
alléguées, mais pour faire place à d'autres aussi faussement et
misérablement inventées, et qui ne vivaient pas plus longtemps. La
fécondité en substituait d'autres pour entretenir l'effervescence et le
bruit, qui ne duraient pas plus longtemps, mais auxquelles on en faisait
succéder d'autres, qui n'avaient pas plus de fondement ni un meilleur
sort. Quand des ducs ou gens de qualité, et de différentes qualités, car
il s'en fallait bien que tous se fussent laissé ensorceler, demandaient
à des parents et à des amis de cette noblesse (car pour s'entendre, il
les faut bien désigner par le nom qu'ifs avaient usurpé), quand, dis-je,
on leur demandait de quoi ils se plaignaient, ce qu'ils voulaient, et
que par amitié, ou pour ne pas montrer qu'ils ne le savaient pas
eux-mêmes, ils voulaient répondre, ils balbutiaient et ne savaient
qu'articuler. Quand on leur démontrait combien on se jouait d'eux par
toutes les puérilités sans vérité et sans vraisemblance dont on les
abusait, ils demeuraient muets et honteux. Quand on leur faisait sentir
que les ducs ne pouvaient pas n'être point du corps de la noblesse, et
qu'il {[}était{]} absurde de les accuser de n'en vouloir pas être, et
impossible de les en exclure, parce que, n'y ayant que trois ordres dans
l'État, il fallait bien qu'ils fussent de l'un des trois par leur
naissance et leur dignité française, et qu'ils ne pouvaient pas être du
premier ni du troisième, quelques-uns semblaient se rendre\,; mais la
plupart, ne sachant que répondre à ce dilemme, se mettaient en fureur.
En un mot, ils ne savaient que dire, ils y suppléaient par crier et
parler à tort et à travers.

L'affaire n'était pas assez mûre ni assez préparée pour aller plus loin.
On y travaillait sans relâche, on cabalait les provinces pour en attirer
des députations en y soufflant le même feu\,; et, pour l'entretenir et
l'augmenter à Paris, on prépara un mémoire contre le rang et les
honneurs des ducs et des duchesses. Ce n'était pas que les moteur de
cette requête en imaginassent aucun succès, mais il fallait tenir cette
noblesse ensemble et en mouvement, se l'attacher de plus en plus,
l'encourager à des tentatives hardies, la piquer par lui faire recevoir
des refus, et pour cela lui donner de la pâture par des prétentions
absurdes qui flattassent leur vanité. Quand ce mémoire fut prêt, et
qu'il fut question de le présenter, les directeurs jugèrent à propos de
se servir de ce qui était sous leur main pour augmenter le nom et le
nombre. Le grand prieur était intéressé pour ses propres entreprises de
n'en pas voir tomber les fondements, et les princes du sang pressaient
le régent sans relâche de leur tenir parole et de les juger\,; le
premier président, le plus envenimé de tous contre les ducs par les
perfidies qu'il leur avait faites dans l'affaire du bonnet, publiquement
déshonoré, par l'amas de scélératesses qu'il y avait commises, et que
les ducs avaient exposées fidèlement au plus grand jour, esclave
d'ailleurs de M. et de M\textsuperscript{me} du Maine, disposait de son
misérable frère non moins déshonoré que lui, mais par d'autres endroits,
que M. du Maine avait par le feu roi fait ambassadeur de Malte ainsi
joints dans cette affaire avec le grand prieur, ils soulevèrent tout ce
qui était à Paris de l'ordre de Malte qui se joignit à cette noblesse,
et ils convoquèrent tout ce qui en portait la croix pour accompagner la
présentation du mémoire. Le régent qui en fut averti, sentit
l'inconvénient de cet attroupement, et manda l'ambassadeur de Malte la
veille de la présentation du mémoire, auquel il dit qu'il défendait
toutes assemblées des chevaliers de Malte, à moins que ce ne fût
uniquement pour les affaires de leur ordre.

Le samedi 18 avril, MM. de Châtillon, chevalier de l'ordre, de Rieux, de
Laval, de Pons, de Baufremont et de Clermont vinrent au Palais-Royal, et
entrèrent ensemble pour présenter leur mémoire au régent qui ne voulut
pas {[}le{]} recevoir, leur dit deux mots de mécontentement fort secs,
leur tourna le dos, et entra dans une pièce de derrière. M. de Châtillon
avait fait sa fortune par sa figure chez Monsieur, dont peu à peu il
devint premier gentilhomme de la chambre\,; il le fut après de M. son
fils, qu'il suivit en Italie. À la figure près, qui était singulièrement
belle, et à la valeur\,; il n'y avait rien, et quoique cette figure
l'eût mis longtemps dans un certain grand monde, il n'y avait été
souffert que par ses qualités corporelles, et il y avait longtemps qu'il
menait une vie fort obscure. M. de Rieux avait beaucoup d'esprit, fort
avare, fort méchant, fort glorieux, fort pensant en dessous, fort
obscur, qui n'avait jamais vu ni guerre, ni cour, ni monde. Les
intendants, les impôts, le pouvoir absolu lui déplaisait infiniment par
gloire et par avarice, et il aurait voulu donner le ton au gouvernement,
ou se faire donner et compter avec lui sans se donner la peine de
paraître. Il n'était pas assez simple pour compter gagner rien sur les
ducs\,; il ne regardait cette entreprise que comme le chausse-pied
d'autres plus solides et plus importantes, mais par cela même des plus
vifs pour animer le gros à poursuivre le fantôme qui les ameutait. M. de
Laval, fils du frère de la duchesse de Roquelaure, était sur le même
moule que M. de Rieux, mais il avait vu la cour et le monde plus que
lui, et avait servi avec assez de distinction. Il avait tâché de tirer
un grand parti d'une blessure qu'il avait reçue à la mâchoire, et, pour
le distinguer des autres Laval, on l'appelait la Mentonnière, parce
qu'il en conserva une, toute sa vie, de taffetas noir, qui d'ailleurs ne
l'incommodait en rien, mais qu'il crut qui affichait son mérite
militaire. Cette mentonnière ne lui ayant pas valu ce qu'il avait
espéré, il quitta le service avec hauteur\,; et retomba dans l'obscurité
tant que le roi vécut, et ne songea qu'à s'enrichir. Il y parvint en
épousant la soeur de Turménies, veuve de Bayez, qui était fort riche, et
tous deux fort appliqués le devinrent de plus en plus par quantité
d'intrigues et d'affaires d'argent. Celui-là devint le bras droit de
M\textsuperscript{me} du Maine, le confident de tous ses ressorts et le
plus ardent de toute cette noblesse. On verra dans la suite que ses vues
étaient pernicieusement vastes, et qu'il ne put se rendre capable de ce
prélude, que par un chemin à des révolutions d'État après lesquelles il
soupirait sans cesse. M. de Pons était encore de même genre.

Comme MM. de Châtillon et de Laval et presque comme M. de Rieux, il
était né pauvre, mais si pauvre qu'il n'avait rien\,; il était parent de
M. de La Rochefoucauld le père, qui logeait chez lui un cadet de cette
maison, qui portait le nom de La Case, et qu'il avait défrayé longtemps,
jusqu'à ce que, devenu par le temps et les grades lieutenant des gardes
du corps, il les quitta avec un cordon rouge et le gouvernement de
Cognac, mais logé toute sa vie, et monté aux chasses par M. de La
Rochefoucauld. La Case lui parla du triste état de l'aîné d'une maison
si ancienne et si distinguée, et M. de La Rochefoucauld, qui était fort
noble et très bienfaisant, le fit venir de Saintonge, le mit avec ses
petits-fils, et on fit comme de l'un d'eux. Tout contribua à le faire
entrer agréablement dans le monde avec un tel appui, un grand nom, un
des plus beaux visages et des plus agréables qu'on pût voir dans la
fleur de quatorze ou quinze ans, beaucoup d'esprit, d'art et de tour,
qui surprennent infiniment à cet âge, et à cette arrivée de province,
enfin la compassion d'un abandon si total de fortune avec tant de
talents naturels. Il fut ainsi à la cour plusieurs années avant la mort
du roi, qui, à la prière de M. de La Rochefoucauld, lui donna enfin pour
rien un guidon de gendarmerie. Le fils aîné du maréchal de Tallard avait
épousé en 1704 la fille unique de Verdun, aîné de sa maison et
cousin-germain de son père, pour terminer de grands procès. Il mourut
sans enfants des blessures qu'il reçut à la bataille d'Hochstedt. Sa
veuve était également laide et riche. M. de Pons, qui n'avait rien, se
mit en tête de l'épouser. Il y parvint par ses charmes en 1710. Il
quitta la cour, MM. de La Rochefoucauld, dont il compta n'avoir plus
besoin, et le service, et montra plus de talent à faire valoir des
procès que pour la guerre\,; il désola le maréchal de Tallard, et il
montra souvent aux procureurs les plus lestes qu'il en savait plus
qu'eux. M\textsuperscript{me} de Montmorency-Fosseux s'étant bientôt
lassée d'être dame d'honneur de M\textsuperscript{me} la Duchesse
(Conti), M. le Duc et M\textsuperscript{me} sa mère se piquèrent de ne
pas déchoir, et mirent M\textsuperscript{me} de Pons en sa place. Rien
de si avare, de si glorieux, de si pointilleux, et si la naissance
permettait de le dire, de si audacieux que M. de Pons avec un air de
politesse et un débit sentencieux de maximes, et que
M\textsuperscript{me} de Pons avec l'aigreur et l'emportement d'une
femme qui connaissait peu le monde et les mesures. Leur règne fut donc
assez court à l'hôtel de Condé, d'où ils sortirent brouillés avec tout
ce qui y allait, et plus encore avec les maîtres. De ce moment on ne les
a plus vus dans le monde, uniquement appliqués à s'enrichir de plus en
plus, et M. de Pons raccroché par M\textsuperscript{me} du Maine à
former son parti, avec le même but et le même feu que M. de Laval\,;
mais comme ayant bien plus d'esprit et d'instruction, car il s'était
orné l'esprit de lecture, il garda plus de ménagements pour sa propre
sûreté, et en servant M\textsuperscript{me} du Maine avec autant et plus
même d'art que lui, et qu'aucun de ceux qui étaient dans la bouteille,
il eut celui de se préserver des accidents personnels.

M. de Baufremont, avec bien de l'esprit et beaucoup de bien et de
désordre, était un fort sérieux, très sottement glorieux, qui se piquait
de tout dire et de tout faire, et qui avait épousé une Courtenay plus
folle que lui encore en ce genre. Les conducteurs en savaient trop pour
s'en servir autrement que d'un pion avancé. Il n'en voulait qu'aux ducs,
et disait tout haut que, ne pouvant pas le devenir\,; il les voulait
détruire. En cela il faisait plus de justice à son mérite qu'à sa
naissance. M. de Clermont était un bellâtre tout à fait dépourvu de sens
et d'esprit, qui, débarqué du Mans par le coche, car il n'avait rien, se
targuait de son nom et de sa figure avec quoi il prétendait faire
fortune. Il épousa la seconde fille de M. et de M\textsuperscript{me}
d'O\,; c'était la faim et la soif ensemble. Mais il espéra tout du
crédit de cette alliance par laquelle il vécut à la cour, et y attrapa
des emplois à la guerre. D'O bien plus au duc du Maine et à
M\textsuperscript{me} du Maine qu'au comte de Toulouse, mais à qui la
prudence ne permettait pas de se montrer, paya de ce gendre que sa
gloire et sa sottise enrôlèrent contre les ducs sans rien apercevoir au
delà, et qu'on se garda bien aussi de lui découvrir. Il se crut un homme
principal de se voir en si belle compagnie, où il aboya des mieux en
écho. Tels furent les chiens de confiance de cette meute, auxquels on
étaient sourdement joints d'autres, qui ne paraissaient pas à découvert,
tant du petit nombre du conseil à divers degrés de confiance du secret,
que de pions.

Cette levée de boucliers ne fit pas grand'peur aux ducs\,; ils virent le
mémoire par quelques amis, car on se garda bien de le laisser courir, et
ils le méprisèrent jusqu'à n'y pas faire la moindre réponse. Quand on
demandait à ces messieurs en quel pays civilisé des quatre parties du
monde il n'y avait point de grands avec des rangs distinctifs de
quiconque ne l'était pas, quand on leur demandait la date de leur
commencement partout sous quelque nom qu'ils fussent connus dans tous
les âges, quand on leur proposait d'expliquer ce que deviendrait en les
abolissant l'ambition et l'émulation, le service de l'État, le pouvoir
des rois et l'utilité des grandes récompenses, quand on les pressait sur
la possibilité des préférences par naissance parmi la noblesse sans
dignités et sans distinctions marquées, quand on les poussait sur ce qui
était le plus fâcheux à supporter, d'un rang distinctif par dignité que
tout homme de qualité pouvait posséder, dont il était capable, et qui
n'était presque composé que de gens de qualité comme eux, et qui
n'étaient que tels avant que cette dignité leur eût été donnée, ou d'un
rang distinctif par naissance hors la maison régnante, qui s'étend à
toute une maison mâles et femelles à l'infini, et qui dit tacitement
sans cesse à tous les gens de qualité, mais très clairement et très
palpablement, qu'ils sont et ont ce que les gens de qualité ne peuvent
jamais être par la disproportion de naissance qui est entre eux\,; à ces
courtes et pressantes considérations nulle réponse, les uns muets et
honteux, les autres furieux balbutiant de rage, et ne disant pas quatre
mots suivis. Quand on les poussait sur la comparaison de leurs pères ou
prédécesseurs, ou qu'on leur donnait la cause d'un changement du blanc
au noir si contradictoire, car ceux-ci ne disaient mot sur le rang de
princes étrangers, on apprenait à la plupart ce qu'ils ignoraient, qui
en ouvraient la bouche et de grands yeux, et en demeuraient stupéfaits,
et les autres ne savaient où se mettre. Ce contraste mérite bien place
ici pour ne le pas laisser périr dans l'oubli, et au moins en rafraîchir
la mémoire.

\hypertarget{chapitre-xiv.}{%
\chapter{CHAPITRE XIV.}\label{chapitre-xiv.}}

1717

~

{\textsc{Différence diamétrale du but des assemblées de plusieurs
seigneurs et gentilshommes en 1649, de celles de cette année.}}
{\textsc{- Copie du traité original d'union et association de plusieurs
de la noblesse en 1649, et des signatures.}} {\textsc{- Éclaircissement
sur les signatures.}} {\textsc{- Requête des pairs au roi à même fin que
l'association de plusieurs de la noblesse en 1649.}} {\textsc{-
Comparaison de la noblesse de 1649 avec celle de 1717.}} {\textsc{-
Succès et fin des assemblées de 1649.}} {\textsc{- Ma conduite avec le
régent sur l'affaire des princes du sang et des bâtards, et sur les
mouvements de la prétendue noblesse.}} {\textsc{- Les bâtards ne
prétendent reconnaître d'autres juges que le roi majeur ou les états
généraux du royaume, et s'attirent par là un jugement préparatoire.}}
{\textsc{- Excès de la prétendue noblesse trompée par confiance en ses
appuis.}} {\textsc{- Conduite et parfaite tranquillité des ducs.}}
{\textsc{- Arrêt du conseil de régence portant défense à tous nobles de
s'assembler, etc., sous peine de désobéissance.}} {\textsc{- Ma conduite
dans ce conseil suivie par les ducs, puis par les princes du sang et
bâtards.}} {\textsc{- Succès de l'arrêt.}} {\textsc{- Gouvernement de
Saint-Malo à Coetquen, et six mille livres de pension à Laval.}}
{\textsc{- Mensonge impudent de ce dernier prouvé, et qui lui demeure
utile, quoique sans nulle parenté avec la maison royale.}} {\textsc{-
Maison de Laval-Montfort très différente des Laval-Montmorency,
expliquée.}} {\textsc{- Autre imposture du même M. de Laval sur la
préséance sur le chancelier.}} {\textsc{- Premier exemple de mariage de
fille de qualité avec un secrétaire d'État.}}

~

On ne répétera pas ce qui se trouve répandu en plusieurs endroits de ces
Mémoires à mesure que l'occasion naturelle s'est présentée d'expliquer
comment le rang de prince étranger s'est formé à l'appui de la Ligue,
puis {[}a été{]} accordé par degrés à d'autres maisons que les
souveraines\,; on se contentera de rapporter ici le traité d'union de
ceux qui, comme cette noblesse dont on parle, en prirent de même le nom
sans aveu ni mission, mais pour chose réelle et non imaginaire, et chose
si radicalement contraire aux lois et usages de ce royaume, à ce qui est
établi dans tous les États, et qui offense si personnellement tout le
second ordre du royaume en général et en particulier. Ces assemblées de
noblesse, et ce traité entre elle, se firent à Paris en 1649 après le
rang accordé à MM. de Bouillon, et le tabouret à la princesse de Guéméné
qui enfanta depuis par longs degrés le même rang, et deux autres
tabourets à la marquise de Senecey et à la comtesse de Fleix mère et
fille, toutes deux veuves, et toutes deux dames d'honneur, l'une en
titre et l'autre en survivance, de la reine mère pour les intérêts de
laquelle elles avaient été longtemps exilées à Randan en Auvergne, et
M\textsuperscript{me} de Brassac mise dame d'honneur en la place de
M\textsuperscript{me} de Senecey qui fut rappelée à la mort de Louis
XIII, M\textsuperscript{me} de Brassac renvoyée, et
M\textsuperscript{me} de Senecey rétablie avec sa fille en survivance.
On verra dans ce traité ce que la noblesse d'alors pensait si
différemment de celle d'aujourd'hui\,; mais elle était encore instruite
dans ces temps-là, connaissait son intérêt et ne se laissait pas mener
par le nez à ce qui y est le plus directement contraire. J'ai eu entre
les mains l'original signé de ce traité, et j'en donne ici la copie que
j'en ai faite. Il est étonnant en quelles mains tombent par la suite des
temps les pièces originales souvent les plus curieuses et les plus
importantes, et les titres les plus précieux\,; il n'est pas rare d'en
trouver chez des beurrières, et entre de pareilles mains. La pièce dont
il s'agit, qui n'est pas de cet ordre, mais qui a sa curiosité, était
tombée entre celles d'un vieux médecin de Chartres, qui était excellent
médecin, encore plus philosophe, savant en belles-lettres, curieux et
très instruit de l'histoire, qui, content de peu, n'avait jamais voulu
quitter sa patrie, ni chercher à paraître et à s'enrichir à Paris. Il
s'appelait Bouvard\,; il avait infiniment d'esprit et une mémoire
prodigieuse. Le malheureux état de mon fils aîné me fit appeler ce
médecin à la Ferté sur le témoignage de M. de Chartres
(Mérinville)\footnote{L'évêque de Chartres était de la famille de
  Mérinville. On a fait, dans les précédentes éditions, deux personnages
  distincts de l'évêque de Chartres et de Mérinville.} et d'autres
encore. Il demeura quelque temps avec nous à plusieurs reprises, et je
trouvai fort à m'amuser, et même à m'instruire dans sa conversation qui
d'ailleurs avait encore l'agrément de la gaieté. Nous tombâmes sur des
matières qui l'engagèrent à me parler de ce traité de la noblesse. Il me
dit qu'il l'avait original, et, en effet, il me l'apporta quand il
revint. Je le copiai avec les signatures dans le même ordre que je les y
trouvai, et, j'eus toutes les peines du monde à le lui faire reprendre.
Il voulait absolument me le donner\,; il me le rapporta même une seconde
fois dans le même dessein, mais je ne crus pas devoir profiter de son
honnêteté et priver un curieux savant et un fort honnête homme d'une
pièce\footnote{Il a déjà été question de cette querelle des tabourets,
  t. V, p.~438. Voy. la note III à la fin du volume.} originale. La
voici\,:

TRAITÉ D'UNION ET ASSOCIATION FAITE PAR LES SEIGNEURS DE LA PLUS HAUTE
NOBLESSE DU ROYAUME, TENUE À PARIS EN L'ANNÉE 1649.

Nous, soussignés, pour obvier aux divisions et désordres qui pourraient
naître de la marque d'honneur extraordinaire qu'on témoigne vouloir
accorder à quelques gentilshommes et maisons particulières au préjudice
de toute la noblesse de ce royaume et notamment de plusieurs des plus
signalés de cet ordre, lequel, pour être le vrai et plus ferme appui de
cette monarchie, doit être par tous moyens conservé dans une parfaite
union sans qu'on laisse établir aucune différence de maisons, avons
déclaré par cet écrit, juré et promis unanimement sur notre foi et
honneur, qu'après avoir fait nos très humbles remontrances à Sa Majesté,
à Son Altesse royale et à Mgrs les princes du sang, et au cas qu'elles
ne soient suivies de l'effet que nous espérons de leur justice, nous
tâcherons par toutes sortes de voies et de ressentiments justes,
honnêtes et généreux, et qui n'iront point contre le service du roi et
de la reine, que semblables distinctions n'aient lieu, consentant que
celui de nous qui s'éloignera de la présente union soit réputé homme
sans foi et sans honneur, et ne soit point tenu pour gentilhomme parmi
nous. Seront suppliés de notre part tous les gentilshommes du royaume
absents de s'unir avec nous par députés, pour maintenir l'intérêt
général de toute la noblesse, et joindre leurs très humbles
supplications aux, nôtres. Le présent écrit a été signé sans distinction
ni différences de rang et de maisons, afin que personne n'y puisse
trouver à redire. De plus, nous promettons que si quelqu'un des
soussignés et intéressés est troublé et attaqué en quelque sorte que ce
soit dans la suite de cette affaire, nous prendrons ses intérêts comme
communs, et tous en général et en particulier, sans nous en pouvoir
séparer par aucune considération\,; et sera déclaré infâme et sans
honneur celui qui en userait autrement. En expliquant ce dernier
article, s'il arrive sur le sujet de l'affaire dont il s'agit, et pour
lequel nous sommes assemblés, qu'aucun de ceux qui se seront unis, soit
par mauvais offices ou autrement, tombe dans le malheur d'être attaqué
en sa personne, sa liberté et ses biens, tous les autres s'obligent sous
peine d'une honte publique et perte de leur réputation, de faire toutes
les choses nécessaires pour le tirer de l'état auquel il se serait mis
pour l'intérêt de leur cause commune, jusqu'à périr. Plutôt qu'il restât
opprimé.

S'engagent non seulement, sous les mêmes conditions de leur honneur, de
s'opposer dans l'occasion présente pour empêcher que nul obtienne les
privilèges des princes qui n'aura pas cet avantage par sa naissance,
mais promettent de former pour l'avenir les mêmes oppositions, afin
qu'aucun, de quelque qualité et sous quelque prétexte que ce puisse
être, n'étant pas né prince, ne parvienne à une semblable prérogative,
qui serait une distinction injurieuse à la noblesse, principalement
entre personnes dont les conditions ont toujours été égales, et de qui
les prédécesseurs ont tenu le même rang et vécu sans se déférer les uns
aux autres, ni dans la cour ni dans les provinces.

Promettent et s'engagent sur leurs mêmes paroles et sur leur honneur de
ne point se retirer de la foi qu'ils se sont donnée les uns aux autres,
de n'alléguer aucunes excuses, prétextes ni raisons qui les puissent
directement ni indirectement séparer de l'association générale et
particulière que porte cet écrit qu'ils ont signé pour le maintenir
inviolablement dans tous les articles qu'il contient, et courir tous la
même fortune.

Promettent pareillement de ne se point désister de la poursuite qu'ils
ont entreprise, qu'ils n'aient reçu la satisfaction qu'ils doivent
légitimement espérer de la bonté et de la justice de Leurs Majestés ou
que le parlement n'y ait apporté les règlements nécessaires suivant les
lois, les exemples et les constitutions du royaume, ne s'excluant point
de se pourvoir où ils jugeront bon être, et par les moyens que
l'assemblée trouvera justes et raisonnables.

Et pour expliquer nettement l'intention de tous, les intéressés en cette
affaire sont demeuré d'accord de former leur opposition conformément à
ce que porte cet écrit sur ce qui a été concédé et prétendu de cette
nature, depuis l'année 1643. Saint-Symon Vermandois,
Halluyes-Schomberg\footnote{Voy. les notes à la fin du volume.},
L'Hospital, le commandeur de Rochechouart, d'Aumont de Chappes, Vassé,
Orval, Leuville, Frontenac, Saujon de Campet, Vardes, Brancas,
Montrésor, Clermont-Tonnerre, comte de Vence, Charles-Léon de Fiesque,
Louis de Mornay-Villarceaux, Sévigné, Montesson, Argenteuil, Boubet,
Mallet, Moreuil-Caumesnil, Mauléon, de Clermont-Monglat, Congis,
Canaples, H. de Béthune, Roussillon, Savignac, Fr.~Gard, le chevalier de
Caderousse, Montmorency, Sigoyer, Leiden, Rouville, Bourdonné, Humières,
d'Aydie, Beauxoncles, Ligny, \footnote{Nom en blanc dans le manuscrit.},
Cormes-Spinchal, Houdancourt, Villeroy, L'Hospital-Sainte-Mesmes,
Longueval, Hautefort, Gasnières, Chasteauvieux, de Vienne, Montrésor,
d'Auteuil, de Crevant, G. Rouxel de Médavy, Maugiron, du Hamel,
d'Alemonis, le chevalier de La Vieuville, de L'Hospital, Bar, de Lanion,
Nantouillet, Froullay, Laigue, Gouffier, Maulevrier, Matha,
Saint-Germain, du Perron, Montiniac d'Hautefort, le comte de La
Chapelle, le comte de Saint-Georges, Thiboust de Boyvy, de Castres,
Fr.~de Montmorency, de Beringhen, Bruslart, Guenes, du Houvray, Damigny
de Meindrac, Lostellemans, Cl. Mohunt, du Monteil, Cl. Dendre de La
Massardière, de Guervon de Dreux, Felleton-Lamechan, Roger de Longueval,
Trésiguidy, Arcy, La Bourlie de Guiscard, de Grailly, Carnavalet,
Saint-Abre, du Mont, Saint-Hilaire, Pascheray, le chevalier de
Carnavalet, Jos, chevalier d'Ornano, J. de Lambert, le vicomte de Melun,
Beaumont, de Lessins, Valernod, Termes, d'Amboise-Aubijoux, Lussan,
Savignac de Gondrin, La Baulme de Vallon, de Voisins-Dusseau,
d'Estourmel, Cressay, Le Plessis d'Andigny, Chouppes, de Torson-Fors,
Chaisenisse, Villiers, Verderonne, Crissé, de La Roque, La Rousselière,
Guitaud, Pradel, Lurmont, Bussy-Rabutin, La Salle, Grammont de Vacher,
le chevalier de Grammont, d'O, Grenan, Maseroles, de Besançon, de
Rémond, Le Plessis-Besançon, Boyer, Montégu, le chevalier de Roquelaure,
Barthélemy Quelen du Broutay, Chollet, chevalier Dailly, Saint-Remy,
Annery, de Boyer, de Cominges de Guitaud, Thomas de Saint-André, de
Melville, Guadagne, La Guerche, Saint-Georges, Pirraud, de
Harlay-Chanvallon, de Monthas, Sabran, Droüe, Fontaine-Martel, Cussant
de Veronil, Fr.~de Rousselet de Châteaurenault, Henescors, Fontenailles,
Saint-Étienne, Achy, Mayac, Morainvillier.

De ces cent soixante-sept noms, il y en a peu de grands, plusieurs
moindres, force petits, assez d'inconnus, beaucoup pour faire nombre\,;
quelques-uns de surprenants\,; et presque aucun qui joigne à la grandeur
ou même à la bonté du nom, la distinction personnelle. Cela ne peut être
autrement, quand on veut du nombre, et qu'il n'y a point de barrière
pour s'arrêter. Les deux premières signatures demandent explication. Mon
oncle, frère aîné de mon père, signait toujours Saint-Symon, et par un
\emph{y}, mon père par un \emph{i}, et n'a jamais signé nulle part que
le duc de Saint-Simon, depuis qu'il l'a été. Cette première signature
est constamment de mon oncle, peu endurant sur les faux princes, encore
moins par son alliance, qui de plus le liait à la maison de Condé, avec
qui il était fort bien, et laquelle cherchait à embarrasser la cour. La
seconde paraît d'une autre main, et n'est pas en ligne, mais au-dessous
de la dernière. Je ne connais personne de ma maison qui ait jamais signé
Vermandois seul ou joint au nom de Saint-Simon, et cela me ferait croire
que cette signature serait du héraut d'armes Vermandois au lieu de
notaire. Il faut remarquer que la plupart de ces signatures sont très
difficiles à déchiffrer. J'en ai laissé une en blanc qui paraît
Villeroy. La même se retrouve trois signatures après. Il n'y en pouvait
avoir deux, car il n'y a pas eu deux branches. M. d'Alincourt, qui de
plus n'a jamais porté le nom de la terre de Villeroy, était mort en
1634\,; il était fils unique du secrétaire d'État, et il n'a eu que
quatre fils le premier maréchal de Villeroy, un comte de Bury, mort sans
enfants en 1628, l'archevêque de Lyon et l'évêque de Chartres,
ecclésiastiques dès leur première jeunesse, et un chevalier de Malte,
mort devant Turin, en 1629. Le premier maréchal de Villeroy fut, en mars
1646, gouverneur de la personne du feu roi, en octobre même année
maréchal de France, duc à brevet en 1651. Il est difficile de croire
qu'un gouverneur du roi entièrement dévoué à la reine mère et au
cardinal Mazarin, ait signé une pièce aussi contraire à leurs
volontés\,; il ne l'est pas moins de penser qu'ils la lui avaient fait
signer pour avoir un homme à eux de ce poids parmi cette noblesse pour
déconcerter ses projets et ses démarches et en être instruits à temps.
Premièrement le gouverneur du roi, surtout en ces temps de trouble, ne
quittait point le roi, ou si peu que sa présence aurait été trop rare
parmi cette noblesse pour en faire l'usage qui vient d'être dit\,;
secondement cette noblesse, qui n'ignorait ni l'attachement ni les
allures du maréchal de Villeroy, ne se serait pas fiée à lui. Son fils
aîné était mort jeune dès 1645, et le second maréchal de Villeroy, resté
unique, était né en avril 1644. Il y a donc sûrement erreur dans ce nom.
Celui de Schomberg est aisé à expliquer. Ce ne peut être le duc
d'Halluyn qui était aussi le maréchal de Schomberg, fils d'autre
maréchal de Schomberg, mort en 1632 à Bordeaux. Ce duc
d'Halluyn-Schomberg prit Tortose d'assaut en juillet 1648\,; il était
vice-roi de Catalogne, et y demeura longtemps depuis de suite. La pierre
le contraignit enfin au retour, dont il mourut à Paris en juin 1656. Il
n'avait ni frère ni enfants. Ce ne peut donc être que le comte de
Schomberg, Allemand comme les précédents, mais sans aucune parenté entre
eux, qui lors de cette affaire de la noblesse commençait à s'avancer, et
qui pouvait déjà être capitaine-lieutenant des gens d'armes écossais, le
même qui après la paix des Pyrénées alla en Portugal commander contre
les Espagnols, qui fut maréchal de France en 1675, qui étant huguenot se
retira en Brandebourg, après la révocation de l'édit de Nantes, puis en
Hollande où il entra dans toute la confidence du prince d'Orange pour
l'affaire d'Angleterre, y passa avec lui, puis avec lui encore en
Irlande, où il fut tué à la bataille de la Boyne, que le prince d'Orange
gagna complète contre le roi, son beau-père.

Il se trouve plusieurs signatures L'Hospital\,; elles ne peuvent être
d'aucun des deux frères tous deux maréchaux de France. L'aîné des deux
mourut en 1641, l'autre était, lors de ces assemblées, gouverneur de
Paris et ministre d'État. Il est donc sans apparence qu'avec ces
qualités qui marquaient l'entière confiance en lui de la reine et du
cardinal Mazarin en ces temps de troubles, où même il pensa être assommé
à l'hôtel de ville, cette signature puisse être de lui. Il ne laissa
point d'enfants. Ce ne peut-être aussi le fils aîné du maréchal son
frère, qui fut duc à brevet de Vitry en juin 1650, et qui s'appelait
auparavant, et lors de ces assemblées, le marquis de Vitry, et qui
aurait signé Vitry, quand ce n'aurait été que pour éviter la confusion
des autres signatures L'Hospital dont il y avait lors deux autres
branches. C'est, pour le dire en passant, ce même duc de Vitry, employé
jeune en diverses ambassades, qui fut fait conseiller d'État d'épée, et
qui comme duc à brevet, et non vérifié, ne laissa pas de précéder le
doyen des conseillers d'État au conseil, et d'y être salué du chapeau
par le chancelier en prenant son avis. Sur les autres signatures, il y a
peu de choses à remarquer. On y voit seulement que la reine et le
cardinal Mazarin d'une part, Monsieur et M. le Prince d'autre, qui
étaient liés on ce temps-là, avaient eu soin de fourrer dans cette
assemblée des personnes entièrement à eux, et quelques noms encore
d'entre les importants de la Fronde. Il s'y trouve entre ces derniers
deux signatures Montrésor. Il n'y avait alors qu'un Bourdeille, qui
portât ce nom, qui fut un des plus avant dans la direction de la Fronde
avec le coadjuteur et la duchesse de Chevreuse, et qui est mort très
vieux à l'hôtel de Guise, chez M\textsuperscript{lle} de Guise, qui
l'avait épousé secrètement. Ainsi il y a faute nécessairement on l'une
de ces deux signatures.

Mon père signa aussi avec plusieurs autres ducs et pairs, sans autres,
une requête au roi tendante à empêcher ces concessions dont j'ai la
copie que je ne donne pas, parce qu'il ne s'agit pas ici de dissertation
sur les rangs, mais simplement des événements de mon temps, à propos
desquels j'ai cru devoir faire mention de ces mouvements de 1649, et de
cette association ou traité qui demande quelques réflexions avant que
d'achever de raconter en deux mots ce qu'elle devint et quel en fut le
succès.

Ces messieurs de 1649 ne se proposent point d'attaquer ce qui est
établi, non seulement de tous les temps et en tous les pays du monde
comme en France, mais ce qui l'est depuis plusieurs règnes, et qui, bien
ou mal fondé, l'est sur la naissance à laquelle le nom de prince est
affecté, c'est-à-dire des personnes issues, de mâle en mâle, d'un
véritable souverain, et dont le chef de la maison l'est actuellement, et
reconnu pour tel dans toute l'Europe. On ne voit nulle part, dans
l'association que ces messieurs approuvent, rien de ce qui a été toléré,
puis accordé aux véritables princes étrangers. L'écrit se contente de
passer à côté et ne va qu'au but qui l'a fait faire, qui est de
s'opposer à des concessions de rangs et d'honneurs à des seigneurs et à
des maisons jusqu'alors semblables d'origine à eux, qui n'ont jamais
rien eu ni prétendu de différence, et auxquelles aussi nulle autre n'a
déféré nulle part distinction humiliante et outrageante que l'écrit sait
expliquer dans toute sa force, mais avec dignité. Il allègue donc les
plus pressantes et les plus invincibles raisons, les plus solides et les
plus évidentes, qu'a la noblesse à s'y opposer. Rien n'est plus éloigné
de battre l'air, et de ne savoir que répondre sur le but qu'on se
propose. Cet écrit est respectueux pour le roi et pour toute la maison
régnante, plein de protestation de fidélité, qui est toujours la
première exception pour n'y manquer jamais. Il n'est pas moins rempli
d'égards et de ménagements sur les personnes qu'il attaque. Pas un mot,
pas une expression qui les puisse le plus légèrement blesser, et la
discrétion y est portée jusqu'à éviter avec soin d'y nommer aucun nom.
En même temps, il s'exprime avec une dignité infinie, et sans
s'échapper, il se contente d'employer les armes naturelles de la
noblesse, l'honneur et la réputation, et s'il descend jusqu'à montrer un
recours au parlement, il faut se souvenir que cette compagnie s'était
alors rendue le fléau et le fouet du cardinal Mazarin, qui en mourait de
peur. Du reste, parmi ces messieurs point d'aboiement, point de rumeur
populaire, rien d'indécent, tout mesuré avec sagesse et dignité, comme
personnes qui se sentent, qui se respectent, et qui sont incapables de
rien d'approchant du tumulte populaire ni des mouvements des halles.
Enfin, pour dernière différence parfaite, toute contradictoire de ces
messieurs de 1649 d'avec ceux de 1717, c'est qu'ils n'usurpent point un
faux titre, et ne donnent point droit sur eux de demander qui ils sont
et par quelle autorité ils agissent. Ils ne prétendent point être la
noblesse, mais seulement être de ce corps. Ils ne se donnent ni pour le
second ordre de l'État, ni pour représenter ce second ordre\,; ils se
reconnaissent des membres et des particuliers de ce second ordre, qui,
pour un intérêt commun, effectif, palpable, pressant, s'associent. On ne
peut donc leur demander, comme à ceux de 1717, qui ils sont, ce qu'ils
veulent, par quelle autorité ils agissent. On voit clairement quels ils
sont, et ils ne se donnent pas pour autres. On sent pleinement ce qu'ils
veulent, et ce qu'ils ont raison de vouloir. Enfin l'autorité qui les
fait agir n'est ni fausse ni chimérique. C'est le plus évident et le
plus commun intérêt qui, sans mission et sans autorité de personne,
donne droit d'agir, de se défendre, de demander à quiconque en a raison
et nécessité effective, et qui le font, entièrement dégagés des
misérables inconvénients de la foule aveugle et du tourbillon. Quelle
disparité de 1649 à 1717\,! elle va jusqu'au prodige.

Néanmoins on ne saurait nier qu'avec tant de contraste il ne s'y trouve
quelques conformités. Le mélange des noms inévitable, comme on l'a dit,
quand on a besoin de nombre, et qu'il n'y a point de barrière, et le but
secret du très petit nombre de conducteurs. En 1649, M. le Prince
voulait embarrasser le cardinal Mazarin pour le rendre souple à ses
volontés\,; il avait entraîné la faiblesse de Monsieur par ceux qui le
gouvernaient, à ne pas s'opposer à ce dessein, qui n'allait alors à rien
de criminel. C'est ce qui donna lieu à ces assemblées, et ce qui les fit
durer. Mais, dès que la peur qu'en eut le cardinal Mazarin l'eut humilié
au gré de M. le Prince, il ne voulut pas aller plus loin, dont Monsieur
fut fort aise. Ils agirent donc en conséquence par ceux qu'ils avaient
dans leur dépendance en ces assemblées, mais ils ne voulurent pas
tromper l'association dans son but. Toutes les histoires et les mémoires
de ces temps-là racontent comment elle fut rompue. Tous ceux qui en
étaient furent mandés et conduits honorablement chez le roi, où ils
furent reçus avec beaucoup de distinction et d'accueil, la reine mère,
Monsieur, M. le Prince, le conseil, toute la cour présente. Monsieur les
présenta\,; la reine leur témoigna satisfaction de les voir, et opinion
de leur fidélité. Un secrétaire d'État leur lut tout haut la révocation
du rang et des honneurs accordés à MM. de Bouillon, et des tabourets de
la princesse de Guéméné, et de M\textsuperscript{me}s de Senecey et de
Fleix, et la montra aux principaux et à qui la voulut voir, pour que
leurs yeux les assurassent qu'il ne manquait rien à la forme de
l'expédition. La reine ensuite leur dit gracieusement que puisqu'ils
obtenaient ce qu'ils demandaient, il n'y avait plus de lieu à
association ni à assemblées, que le roi déclarait l'association finie,
et défendait les assemblées à l'avenir. La reine ensuite leur fit des
honnêtetés et le Mazarin des bassesses, et chacun se retira. Telle fut
la fin de cette affaire, bien différente aussi de celle de 1717. Cette
révocation subsista tant que les troubles firent craindre, après quoi
elle tomba. La reine remit MM. de Bouillon et les tabourets supprimés.
On a vu ailleurs comment celui de la princesse de Guéméné enfanta par
différents degrés les mêmes avantages à MM. de Rohan que MM. de Bouillon
avaient obtenus, et que celui de M\textsuperscript{me}s de Senecey et de
Fleix les fit enfin duchesses, et en même temps M. de Foix, leur fils et
petit-fils, duc et pair\footnote{Voy., sur la promotion de ducs et pairs
  où fut compris M. de Foix. t. Ier, p.~449. Gaston de Foix, qui prit le
  titre de duc de Randan, était fils de la comtesse de Fleix et
  petit-fils de la marquise de Senecey. La pairie de Randan fut déclarée
  mâle et femelle. Les précédents éditeurs ont changé le teste de
  Saint-Simon en mettant \emph{MM. de Foix, leurs fils et petits-fils}.}.
Après cette digression nécessaire, revenons en 1717.

Je me tenais avec M. le duc d'Orléans sur ces mouvements de la prétendue
noblesse et sur l'affaire des bâtards, qui lui était si connexe dans la
même conduite que je gardais avec lui sur le parlement\,; je m'étais
contenté de lui démontrer les intimes rapports de ces deux affaires et
leurs communs ressorts\,; quel était son plus puissant intérêt sur la
dernière, et qu'à l'égard de l'autre il éprouverait bientôt que le
prétexte frivole des ducs ne durerait que jusqu'à ce que le parti de M.
et de M\textsuperscript{me} du Maine fût assez bien formé et fortifié
pour aller à lui directement et à son gouvernement. Après cette
remontrance, je laissais aller le cours des choses, persuadé que ce que
je lui dirais ne ferait qu'augmenter ses soupçons que je ne lui parlais
que par intérêt et par passion, et que le duc de Noailles, Effiat,
Besons, Canillac et d'autres qui l'obsédaient rendraient inutiles les
plus évidentes raisons. À la fin pourtant il s'aperçut qu'il avait
laissé aller trop loin ces deux affaires, et du danger qui le menaçait.
Malgré mon silence avec lui là-dessus, il ne put s'empêcher de m'en dire
quelque chose. Je répondis avec un air d'indifférence que je lui avais
dit ce que je pensais là-dessus, que je n'avais rien à y ajouter, que
c'était à lui à juger de ce qu'il lui convenait de faire, et je changeai
aussitôt de discours. Il me parut qu'il le sentit, et il ne m'en dit pas
davantage. Cependant les princes du sang ne cessaient de le presser de
juger leur différend avec les bâtards, et à la fin il dit à M. le Duc
qu'il les jugerait incessamment\,; mais qu'il voulait prendre avis de
beaucoup de personnes, dont il choisirait plusieurs dans les différents
conseils. Cela fut su, et la duchesse du Maine alla se plaindre au
régent qu'il voulait faire juger cette affaire par des gens qui ne
savaient point assez les lois du royaume.

On ne peut qu'admirer que des doubles adultérins osent invoquer des lois
pour se maintenir dans une disposition sans exemple, faite directement
contre toutes les lois divines et humaines, contre l'honneur des
familles, contre le repos et la sûreté de la maison régnante, et de
toute société. Cette remontrance ne réussit pas, encore moins la
résolution prise par M. et M\textsuperscript{me} du Maine de ne
reconnaître d'autres juges que le roi majeur, ou les états généraux du
royaume\,; ils avaient bien leurs raisons pour cela. L'éloignement de la
majorité donnait du temps à leurs complots\,; et, avec ce parti qui se
formait et s'organisait de jour en jour, ils espéraient tout d'une
assemblée qu'ils comptaient bien parvenir à faire ressembler à celle que
la mort du duc et du cardinal de Guise déconcerta et dissipa. Mais M. du
Maine n'était en rien un Guise, sinon par l'excès de l'ambition. M. le
duc d'Orléans, poussé par les princes du sang, sentit enfin quelle
atteinte donnerait à son autorité de régent la résolution du duc du
Maine, si elle était soufferte, et quel exemple ce serait s'il différait
ce jugement. M. et M\textsuperscript{me} du Maine, qui, par d'Effiat et
par d'autres, savaient jour par jour ce que M. le duc d'Orléans pensait
sur leur affaire, comptèrent tellement sur son irrésolution, sa
facilité, sa faiblesse, qu'ils ne doutèrent pas de hasarder une
résolution si hardie, et qui comme leur affaire même était si opposée à
toute règle et à toute loi. Ils s'y méprirent, et ce fut ce qui
précipita leur jugement. Deux jours après la visite de
M\textsuperscript{me} la duchesse du Maine au Palais-Royal, il fut rendu
un arrêt au conseil de régence, où aucuns princes du sang, bâtards ni
ducs ne furent présents, qui ordonna aux princes du sang et aux bâtards
de remettre entre les mains des gens du roi les mémoires respectifs
faits et à faire sur leur affaire, et Armenonville, secrétaire d'État,
fut chargé de le leur aller communiquer\,: c'était bien s'engager à
juger incessamment et le leur déclarer d'une manière juridique.

Ces deux affaires marchaient ensemble, avec l'embarras pour le régent du
czar dans Paris. Cette prétendue noblesse faisait plus de bruit que
jamais avant sa députation. Elle comptait sur toute la protection du
régent qui la laissait dire et faire, et qui souffrait que M. de
Châtillon et beaucoup d'autres du Palais-Royal fussent à découvert ou
secrètement d'avec eux. Ils étaient poussés et soutenus par d'Effiat et
Canillac\,; et le duc de Noailles qui y avait à la mort du roi donné le
premier branle, se voulait faire élever par eux sur les pavais. Avec de
tels appuis auprès du régent, le parlement en croupe, M. et
M\textsuperscript{me} du Maine à leur tête, elle leur tourna entièrement
jusque-là qu'il y eut de leurs femmes qui se vantèrent qu'elles allaient
prendre des housses et des dais\,; mais il est vrai qu'aucune n'osa le
faire. Les ducs les laissaient s'exhaler et tirer leurs estocades en
l'air sans rien dire ni faire, et sans inquiétude, parce que de tels
glapissements n'en pouvaient donner. Ce fut dans ce tourbillon
d'emportement et de confiance que les huit seigneurs dont on a parlé
allèrent au Palais-Royal présenter leur mémoire, et qu'ils le
rapportèrent de la façon que je l'ai raconté. Le régent avait enfin
ouvert les yeux, et les ouvrit à plusieurs de ces messieurs par une
réception qu'ils en avaient si peu attendue. Le trouble se mit parmi
eux\,; la division, les reproches\,; plusieurs se plaignirent qu'on les
avait trompés, et dirent au régent, et à qui voulut l'entendre, qu'ils
ne s'étaient engagés que sur les assurances qui leur avaient été données
que tout se faisait du consentement et même pan les ordres secrets du
régent. Un grand nombre se détacha, lui fit des excuses\,; beaucoup
témoignèrent leur regret aux ducs de leur connaissance. Mais si les
sages prirent ce parti, ils ne furent pas le plus grand nombre. Les
conducteurs et le très peu de partisans du vrai secret redoublèrent
d'efforts et d'artifice pour retenir et rallier leur monde, et pour
l'irriter du mauvais succès de leur députation. Les huit députés surtout
s'y signalèrent, mais ils n'eurent plus le verbe si haut. Ils firent
parler au régent, mais comme à la fin il avait vu clair, il ne les
marchanda pas longtemps, avec toutefois ses adoucissements accoutumés
dont nulle expérience ne le pouvait défaire, et qu'il ne put refuser à
ceux qui l'obsédaient, et qui n'oubliaient rien pour lui faire peur\,;
il en eut en effet, et c'est ce qui précipita la fin du bruit de ces
belles prétentions.

Il fut rendu un arrêt l'après-dînée du samedi 14 mai, au conseil de
régence, qui est en ces termes\,: «\,Sa Majesté, étant en son conseil de
l'avis de M. le duc d'Orléans, régent, a fait très expresses inhibitions
et défenses à tous les nobles de son royaume, de quelque naissance, rang
et dignité qu'ils soient, de signer la prétendue requête, à peine de
désobéissance, jusqu'à ce qu'autrement en ait été ordonné par Sa
Majesté, suivant les formes observées dans le royaume, sans néanmoins
que le présent arrêt puisse nuire ni préjudicier aux droits,
prérogatives et privilèges légitimes de la noblesse, auxquels Sa Majesté
n'entend donner aucune atteinte, et qu'elle maintiendra toujours à
l'exemple des rois ses prédécesseurs, suivant les règles de la
justice.\,» Cet arrêt, tout emmiellé qu'il fût, sapait par le fondement
le chimérique objet qui avait ramassé cette prétendue noblesse. La
défense de signer la requête, qui était le mémoire porté au
Palais-Royal, tourné en requête toute prête, la mention d'observer les
formes du royaume, celle de l'exemple des rois prédécesseurs et des
règles de la justice, proscrivait d'une part une assemblée informe,
tumultueuse, sans nom qu'usurpé et faux, sans mission, sans autorité,
sans pouvoirs, et maintenait ce qui était des formes et de tout temps,
sous les rois prédécesseurs, tels que la dignité des ducs, dans toutes
leurs distinctions, rangs et prérogatives\,; aussi fut-ce un coup de
foudre sur cette prétendue noblesse. On parla de quelque autre affaire
courte au commencement de ce conseil, après laquelle celle-ci fut mise
sur le tapis par M. le duc d'Orléans. À l'instant, je regardai les ducs
du conseil, puis me tournant au régent, je lui dis que, puisqu'il
s'allait traiter de l'affaire de ces messieurs de la noblesse, je
n'oubliais point que nous étions tous du second des trois ordres du
royaume, et que je le priais de me permettre de n'être pas juge, et de
sortir du conseil. Je me levai en même temps, et quoique moi ni les
autres ducs n'y eussions été préparés en aucune sorte, regardant la
table quand j'eus fait quelques pas, je vis tous les ducs du conseil qui
me suivirent. Quittant ma place, le duc de Toulouse me dit tout bas\,:
«\,Et nous, que ferons-nous\,? --- Tout ce qu'il vous plaira, lui
dis-je\,; pour nous autres ducs, je crois que nous nous devons de
sortir.\,» Nous nous mîmes ensemble dans la pièce d'avant celle du
conseil pour y rentrer après l'affaire. Presque aussitôt nous vîmes les
princes du sang et les bâtards sortir. Cela fit un grand mouvement dans
ces dehors, où il y avait quelques personnes de cette noblesse qui se
tenaient éloignées dans des coins, qui avaient en apparemment quelque
vent qu'il serait question d'eux au conseil. Les ducs sortis avec moi me
remercièrent d'avoir pensé à ce à quoi ils ne songeaient pas, et de leur
avoir donné un exemple qu'ils avaient suivi aussitôt, et dont, comme
leur ancien à tous, j'étais plus en droit de faire. L'affaire dura
assez, après quoi M. le duc d'Orléans sortit, sans en entamer d'autres,
et nous sûmes aussitôt l'arrêt qui venait d'être rendu.

Dans tout le cours de ce long vacarme (car il ne se peut rendre que par
ce nom), les ducs, avec raison fort tranquilles sur leur dignité, ne
s'assemblèrent pas une seule fois, ni tous, ni quelques-uns, ne firent
aucun écrit, et ne députèrent pas une seule fois au régent. Par même
raison ils demeurèrent dans la même inaction sur cet arrêt, qui étourdit
étrangement cette prétendue noblesse à qui le régent fit en même temps
défendre de s'assembler désormais. Tout se débanda, la plupart en effet
et commença à ouvrir les yeux, et à avouer sa folie presque tous en
apparence. Ce fut à qui courrait au Palais-Royal s'excuser, où tous
furent reçus honnêtement, mais sèchement, ce qui diminua encore le
nombre, avec l'opinion que ces mouvements fussent du goût du régent, qui
donna place à la crainte de lui déplaire, au désespoir de réussir, et au
dépit d'avoir été trompés et menés par le nez. Mais les plus entêtés se
laissèrent persuader par les confidents de l'intrigue, à qui il
importait si fort de ne pas laisser démancher le parti, et qui
n'oublièrent rien pour en arrêter la totale dissipation, où pourtant il
ne se fit plus rien que dans les ténèbres.

M. de Noailles, pour le rassurer un peu, profita de la mort de Lannion,
lieutenant général, pour faire donner le gouvernement de Saint-Malo
qu'il avait à Coetquen, son beau-frère, son agent, et des plus avant
parmi cette noblesse, dont les fauteurs qui obsédaient le régent lui
persuadèrent dans la même vue d'en retirer M. de Laval par une pension
de six mille livres, grâce bien forte à un homme qui avait quitté le
service, et qui ne pouvait l'avoir méritée que par ses séditieuses
clameurs. Aussi verrons-nous combien le régent y fut trompé.

Ce M. de Laval si totalement enrôlé par M. et M\textsuperscript{me} du
Maine, et qui était avec M. de Rieux depuis longtemps dans le secret de
leurs vues et de leurs complots, était un homme à qui il ne coûtait rien
de tout prétendre et de tout hasarder. Dès la mort du roi, profitant de
la débandade de la draperie, il avait demandé et obtenu du régent la
permission de draper, à titre de parenté, sur ce que les Laval avaient
eu une duchesse d'Anjou, reine de Naples et de Sicile, qu'il faisait
extrêmement valoir. Il savait assez, et de plus il comptait assez sur
l'ignorance publique, pour ne craindre pas d'être démenti. Cette
effronterie en effet en avait besoin. Il est vrai que Jeanne de Laval,
fille de Guy XIII, épousa en septembre 1454, le bon René, duc d'Anjou et
comte de Provence, roi titulaire de Naples, Sicile, Jérusalem, Aragon,
etc., qui mourut à Aix en Provence, en juillet 1474\footnote{René
  d'Anjou mourut à Aix le 10 juillet 1480.}, et Jeanne de Laval, sa
femme, mourut au château de Beaufort en 1498. Mais malheureusement pour
cette grande alliance, il y a quelques remarques à faire, c'est
premièrement qu'il n'y eut point d'enfant de ce mariage\,; ainsi nulle
parenté entre ces princes et la maison de Jeanne de Laval.

Le bon roi René avait épousé en premières noces, en octobre 1420,
Isabelle, héritière de Lorraine, d'où s'ourdirent les guerres entre lui
et le comte de Vaudemont, qui se prétendit préférable comme mâle, qui
prit et retint longues années René prisonnier, ce qui lui coûta les
royaumes de Naples et de Sicile, qu'il ne put aller défendre contre les
Aragonnais. Isabelle mourut à Angers en 1452, et laissa Jean d'Anjou,
duc de Calabre et de Lorraine, qui fit la guerre en Italie et en
Catalogne, et qui mourut en 1471\footnote{Jean d'Anjou, duc de Calabre,
  mourut à Barcelone le 10 décembre 1471.} à Barcelone, avant le roi
René son père, laissant de Marie, fille aînée du duc Jean Ier de
Bourbon, Nicolas, successeur de ses États et prétentions, qui mourut à
Nancy, sans alliance, en 1473, laissant héritier de ses États et
prétentions, Charles IV son cousin germain, fils de Charles d'Anjou,
comte du Maine, etc., frère puîné du roi René, lequel fit le même
Charles son héritier, qui lui succéda, à qui il ne survécut pas six
mois\,; car il mourut à Marseille le 11 décembre 1480\footnote{D'après
  l'\emph{Art de vérifier les dates}, Charles de Maine ne mourut que le
  10 décembre 1481. Il avait donc survécu près de seize mois à René
  d'Anjou.}, sans enfants de Jeanne de Lorraine, fille du comte de
Vaudemont, morte en janvier précédent. Elle l'avait institué héritier de
tous ses biens, et lui, institua le roi Louis XI héritier de tous les
siens, États et prétentions. Marie d'Anjou, soeur de son père et du bon
roi René, était mère du roi Louis XI. En ce prince finit la branche
seconde d'Anjou-Sicile. On voit ainsi par toutes {[}sortes{]} d'endroits
qu'il n'y avait aucune parenté avec nos rois Bourbons ni même Valois, à
titre de mariage du bon roi René, duc d'Anjou, roi de Naples et de
Sicile, avec Jeanne de Laval-Montfort, de laquelle même il n'y a point
eu d'enfants. Secondement, et voici où l'effronterie est plus étrange,
c'est que M. de Laval, bien sûr de l'ignorance publique, n'a pas craint
le mensonge le plus net en se jouant du nom et des armes de Laval, dont
voici le fait et la preuve\,:

Matthieu II, seigneur de Montmorency, épousa en premières noces Gertrude
de Nesle, duquel mariage descend toute la maison de Montmorency jusqu'à
aujourd'hui. Le même Matthieu, connétable de France, épousa en secondes
noces Emme de Laval, héritière de cette ancienne maison, dont les armes
sont de gueules à un léopard passant d'or. Il n'en eut qu'un fils et une
fille. Ce fils fut Guy de Montmorency qui, succédant aux grands biens de
sa mère, quitta le nom de Montmorency, et prit pour soi et pour toute sa
postérité le seul nom de Laval\,; mais il retint les armes de
Montmorency, qu'il chargea pour brisures de cinq coquilles d'argent sur
la croix. De lui est descendue toute la branche de Montmorency qui,
depuis lui jusqu'à présent, n'a plus porté que le seul nom de Laval dans
toutes ses branches, avec les armes de Montmorency brisées des cinq
coquilles, qui font ce qu'on a appelé, depuis qu'elles ont été prises,
les armes de Laval. Ce Guy de Montmorency, dernier fils du connétable
Matthieu II, fils unique de sa seconde femme Emme, héritière de
Laval-Vitré, etc., prit non seulement le nom de Laval en héritant de sa
mère, mais le nom de baptême de Guy, que les pères de sa mère avaient
affecté. Ainsi il s'appela Guy VII de Laval, et fit passer d'aîné en
aîné cette même affectation du nom de Guy. Il eut cinq descendants
d'aîné en aîné, qui tous se nommèrent Guy VIII, Guy IX, Guy X, Guy XI et
Guy XII, seigneurs de Laval et de Vitré. Tous ceux-là, outre leurs
cadets qui firent des branches dont il y en a qui subsistent
aujourd'hui, étaient tous de la maison de Montmorency, mais ne portant
tous, aînés et cadets, que le seul nom de Laval, avec les armes de
Montmorency, brisées des cinq coquilles d'argent sur la croix. Guy XII
était frère de Guy XI, qui n'eut point d'enfants, et fut ainsi la
quatrième génération du dernier fils du connétable Matthieu II de
Montmorency, et sa seconde femme Emme, héritière de l'ancienne maison de
Laval.

Ce Guy XII n'ayant point d'enfants de Louise de Châteaubriant, morte en
novembre 1383, il se remaria six mois après à la veuve du fameux
connétable du Guesclin, qui était Laval comme lui, fille de J. de Laval
seigneur de Châtillon en Vendelais, fils d'André de Laval, oncle du père
de Guy XII. Il n'eut qu'un fils et une fille, Guy et Anne. Guy jouant à
la paume tomba dans un puits découvert, et en mourut huit jours après,
en mars 1403, tout jeune, et seulement fiancé à Catherine, fille de
Pierre, comte d'Alençon. Guy XII n'espérant plus d'enfants, quoiqu'il ne
soit mort qu'en 1412 et sa femme en 1433, choisit pour épouser Anne sa
fille et son unique héritière Jean de Montfort, seigneur de Kergorlay,
fils aîné de Raoul VIII, sire de Montfort en Bretagne, de Gaël, Loheac,
et La Roche-Bernard, et de Jeanne, dame de Kergorlay. Cette maison de
Montfort a été jusque-là assez peu connue. Le mariage se fit avec les
conditions et toutes les sûretés nécessaires que J. de Montfort
quitterait entièrement son nom, ses armes, etc., lui et toute sa
postérité, pour ne plus porter que le nom seul de Laval et les armes
seules de sa femme, qui étaient comme, on l'a vu, celles de Montmorency
brisées de cinq coquilles sur la croix. Cela a été si religieusement
exécuté que ces Montfort devenus Laval ont tous pris, quant aux aînés
seulement, le nom de baptême de Guy\,; en sorte que J. de Montfort, mari
de l'héritière Anne de Laval s'appela Guy XIII, seigneur de Laval,
Vitré, etc. Leurs enfants furent Guy XIV, André, seigneur de Loheac,
amiral et maréchal de France, et Louis seigneur de Châtillon, grand
maître des eaux et forêts, qui eut de grands gouvernements. Lui et le
maréchal son frère moururent sans enfants. Je laisse les soeurs. Guy XIV
fit ériger Laval en comté par le roi Charles VIII, en juillet 1429, et
mourut en 1486. D'Isabelle fille de Jean IV, duc de Bretagne, il eut Guy
XV, J. seigneur de Laroche-Bernard, Pierre, archevêque-duc de Reims,
cinq filles bien mariées dont Jeanne la seconde fut la seconde femme du
roi René de Naples et de Sicile, duc d'Anjou, comte de Provence, dont
elle n'eut point d'enfants, et qui a donné lieu à cette explication. De
Françoise de Dinan, sa seconde femme, Guy XV eut trois fils dont l'aîné
et le dernier n'eurent point d'enfants. Le second, François, seigneur de
Châteaubriant, eut de la fille unique de Jean de Rieux, maréchal de
Bretagne, deux fils dont le cadet n'eut point d'enfants. Jean l'aîné
seigneur de Châteaubriant fut gouverneur de Bretagne après son cousin
Guy XVI, comte de Laval. Il épousa Françoise, fille d'Odet de Grailly
dit de Foix, vicomte de Lautrec, maréchal de France, soeur de la femme
du comte Guy XVII de Laval, fils de son cousin germain, et qui toutes
d'eux n'eurent point d'enfants. C'est de cette dame de Châteaubriant
dont on a fait cette fable touchante. Elle mourut en 1537, et son mari
en 1542. Se voyant très riche et sans enfants, il dissipa une partie de
son bien et donna l'autre à ses amis. Il fit présent de Châteaubriant,
Condé, Chanseaux, Derval, Vioreau, Nosay, Issé, Rougé et d'autres terres
au connétable Anne de Montmorency, qui fut fort accusé de ne les avoir
pas eues pour rien.

Revenons maintenant à Guy XV, comte de Laval-Montfort, etc., frère de la
seconde femme du bon roi René, et d'une autre qui mourut jeune, fiancée
au comte de Genève, frère du duc de Savoie. Louis XI lui fit épouser à
Tours, en 1461, Catherine, fille de Jean II, comte d'Alençon, et Charles
VIII le fit grand maître de France. Se voyant sans enfants, il fit le
fils de Jean, seigneur de la Roche-Bernard, son frère de père et de
mère, son héritier, parce que ce frère était mort longtemps devant lui.
Ce frère avait eu de Jeanne Duperrier, comtesse de Quintin, un fils
unique qui, héritant en 1500 de son oncle Guy XV, quitta son nom de
baptême qui était Nicolas, et s'appela Guy XVI, comte de Laval-Montfort,
etc. Il fut fait par François Ier gouverneur et amiral de Bretagne, et
mourut en mai 1531. Il avait épousé\,: 1° Charlotte, fille de Frédéric
d'Aragon, roi de Naples\,; 2° une soeur du connétable Anne de
Montmorency\,; 3° Anne, fille de Jean de Daillon, seigneur et baron du
Lude, sénéchal de Poitou\,; de la première, il eut trois fils qu'il
perdit jeunes, sans alliances, dont un tué au combat de la Bicoque, et
deux filles, Catherine, mariée en 1518 à Claude, sire de Rieux, comte
d'Harcourt, etc., et Anne qui épousa François, seigneur de La Trémoille,
vicomte de Thouars. La dame de Rieux, comtesse d'Harcourt, devint
héritière de Laval-Vitré\,; etc. Elle n'eut que deux filles, Renée, qui
épousa en 1540 Louis de Précigny, dit de Sainte-Maure, maison dont était
le duc de Montausier. Elle mourut sans enfants. Cl., sa soeur et son
héritière, épousa le célèbre François de Coligny, seigneur d'Andelot,
colonel général de l'infanterie, frère du fameux amiral Gaspard,
seigneur de Châtillon-sur-Loing. De ce mariage, vint Paul de Coligny
qui, héritant de sa tante, se nomma Guy XVIII, comte de Laval, etc.\,;
son fils, Guy XIX fut tué en Hongrie en 1605, à la fleur de son âge,
sans postérité, par quoi ce grand héritage vint à celle de François de
La Trémoille, et d'Anne de Laval-Montfort susdits. Or voici comment cet
héritage tomba en ces héritières.

On vient de voir la postérité de Guy XVI et de Charlotte d'Aragon sa
première femme\,; il n'en eut point de la Montmorency sa seconde
femme\,; mais de la troisième, Anne de Daillon, il eut trois filles bien
mariées, dont la dernière le fut à l'amiral de Coligny, ou de Châtillon
dont on vient de parler, et un fils Guy XVII, comte de Laval, etc., mort
en 1547, sans enfants de Charlotte, soeur de la dame de Châteaubriant
dont il a été parlé ci-dessus. En lui finit cette maison de Montfort qui
avait pris le nom et les armes de Laval en quittant les siennes, et qui
laissa cette grande succession aux héritières dont on vient de parler.
Il ne faut rien oublier\,: Jean de Montfort, qui épousa l'héritière de
Montmorency-Laval, eut aussi deux filles. La cadette épousa Guy de
Chauvigny, seigneur de Châteauroux, et l'aînée, en 1489, Louis de
Bourbon, comte de Vendôme, dont le fils Jean II de Bourbon, comte de
Vendôme, épousa l'héritière de Beauvau de qui sort toute la maison ou
branche de Bourbon aujourd'hui régnante. On a vu qu'à la mort de
Monseigneur, Voysin obtint du feu roi qu'il fût permis à M. de
Châtillon, son gendre, longtemps depuis fait duc et pair, de draper à
cause des alliances fréquentes et directes de la maison de Châtillon
avec la maison royale, et que, sur cet exemple, M\textsuperscript{me} la
princesse de Conti, qui s'honorait fort avec raison de l'alliance des La
Vallière avec la maison de Beauvau, obtint pour M. de Beauvau la même
permission sur ce que toute la famille régnante descend d'Isabelle de
Beauvau, et qu'il n'y a plus personne de vivant de la maison de qui elle
descend immédiatement. C'eût bien été le cas où par même raison MM. de
Laval n'eussent pu être refusés de la même distinction, si une fille de
leur maison eût été la mère de Jean II de Bourbon, comte de Vendôme, qui
épousa cette héritière de Beauvau\,; mais ce n'était pas le temps de
hasarder d'en faire accroire au feu roi et de prendre tout le monde pour
dupe\,; mais à sa mort, lorsque M. le duc d'Orléans prostitua la
draperie jusqu'au premier président, M. de Laval saisit la conjoncture,
et donna les Laval-Montfort pour les Laval-Montmorency avec d'autant
plus de facilité qu'on était lors occupé de trop de choses pour en
éplucher la généalogie. C'était le même nom et les mêmes armes de
Laval-Montmorency\,; les nom et armes des Montfort étaient éclipsés dès
le mariage de Montfort avec l'héritière de Laval-Montmorency, et le
dernier de Laval-Montfort avait éteint cette race dès 1547. M. de Laval
ne balança donc pas depuis la mort du roi de revêtir sa branche de
toutes les grandeurs qui avaient illustré les Laval-Montfort.

Depuis que l'héritière de la branche aînée de Laval-Montmorency était
entrée dans les Montfort, et y avait porté ses grands biens avec son nom
et ses armes, les branches cadettes de Laval-Montmorency étaient, pour
ainsi dire, demeurées à sec jusqu'à nos jours\,; et en cent
quarante-deux ans qu'a duré la maison de Montfort, depuis le mariage de
l'héritière de Laval-Montmorency, c'est-à-dire depuis 1405 jusqu'en
1547, cette heureuse maison a presque atteint toutes les grandeurs de la
maison Montmorency, en charges, emplois, distinctions, alliances et
grandes terres, sans avoir presque rien eu de médiocre, même dans les
cadets et dans les filles. Ce n'est pas qu'on puisse ignorer
l'essentielle et foncière différence qui est entre ces deux maisons,
dont l'une, peu connue auparavant, ne s'est élevée à ce point que par
l'alliance et l'héritage de l'aînée de la dernière branche de l'autre\,;
mais cette vérité n'empêche pas que ce que j'avance ici ne soit vrai de
l'extrême illustration en tous genres de cette maison de Montfort depuis
qu'elle est devenue. Laval jusqu'à son extinction, et de
l'obscurcissement en tous genres aussi où est tombée la branche de
Laval-Montmorency depuis le mariage de son héritière aînée dans la
maison de Montfort jusqu'à aujourd'hui. On n'y trouve que des alliances
communes, peu de fort bonnes, quantité de basses, peu de biens, point de
terres étendues, point de charges, d'emplois\,; nulles distinctions, si
on en excepte Gilles de Laval, seigneur de Raiz, etc., maréchal de
France en 1437, pendu et brûlé juridiquement à Nantes, pour
abominations, 25 décembre 1440, et Urbain de Laval, seigneur de
Boisdauphin, etc., maréchal de France en 1599, dont le fils n'a point
laissé d'enfants, et le maréchal de Raiz une fille unique, mariée au
maréchal de Loheac-Laval-Montfort, morts tous deux sans enfants. On voit
ainsi que rien n'est si essentiellement différent, ni plus étranger l'un
à l'autre, quoique avec le même nom et les mêmes armes que ces deux
maisons de Laval, l'une cadette de Montmorency, l'autre du nom de
Montfort en Bretagne, qui quitta son nom et ses armes pour porter
uniquement le nom de Laval et les armes de Montmorency brisées de cinq
coquilles d'argent sur la croix, en épousant la riche et unique
héritière de la branche aînée de Laval-Montmorency, et la facilité qu'a
eue la hardiesse de M. de Laval de revêtir les branches de
Laval-Montmorency des plumes d'autrui, et de s'attribuer toutes les
grandeurs, alliances et distinctions des Laval-Montfort éteints depuis
si longtemps. Il drapa donc à la mort du roi, et tous les Laval ont
toujours depuis drapé sur ce fondement si évidemment démontré faux par
ce qui vient d'en être mis au net.

Mais cette mensongère usurpation n'est pas la seule imposture dont le
même M. de Laval ait voulu s'avantager, et que son audace ait alors
persuadée à l'ignorance du monde, et à son incurie et à sa paresse
d'examiner. Il publia que sa maison avait eu la préséance sur le
chancelier de France, et sur sa périlleuse parole, on eut la bonté de
n'en pas douter. La vérité est qu'il se contenta d'avancer cette
fausseté ainsi en général, et qu'il se garda bien de s'enferrer dans
aucune particularité d'occasion ou de date. Le célèbre André du Chêne,
qui a donné une histoire fort étendue de la maison de Montmorency, où il
n'oublie rien pour la relever\,; et qu'il dédia à M. le Prince, fils
d'une fille du dernier connétable de cette maison, n'en dit pas un mot,
et il n'est pas croyable que ses recherches lui eussent laissé ignorer
un fait aussi singulier, ou qu'il eût voulu l'omettre. Ni les
Laval-Montfort n'ont eu cette préséance dans toute la durée de leur
grandeur\,; ni les cadets Laval-Montmorency de cette héritière de leur
branche aînée, dont le mariage avec le Montfort lui apporta et à sa
postérité tant de splendeur, et à ces mêmes cadets Laval-Montmorency un
obscurcissement qui, de degré en degré, les a fait tomber dans un état
où, même dans les temps les plus voisins du mariage de leur héritière
aînée avec le Montfort, ils ne se sont jamais trouvés en situation de
rien prétendre au delà de tous les gens de qualité ordinaire. Je
n'allongerai point cette digression, déjà trop longue, d'une
dissertation sur le rang, les prétentions et leurs divers degrés de
l'office de chancelier. Je me contenterai de dire que je ne vois qu'un
seul exemple de cette préséance dans la maison de Montmorency, non de
Laval-Montmorency. Personne n'ignore la violence extrême faite par Henri
II et par le connétable Anne de Montmorency au maréchal de Montmorency
son fils aîné, pour lui faire épouser sa bâtarde légitimée, veuve
d'Horace Farnèse, duc de Castro, sans enfants.

Le maréchal de Montmorency était amoureux de M\textsuperscript{lle} de
Piennes, Jeanne d'Halluyn, soeur de Charles d'Halluyn, seigneur de
Piennes, marquis de Maignelets, gendre de l'amiral Chabot, tous deux
enfants d'une Gouffier, fille de l'amiral de Bonnivet, lequel Charles
d'Halluyn, Henri III fit duc et pair en 1587. Le maréchal de Montmorency
avait donné une promesse de mariage à M\textsuperscript{lle} de Piennes,
qui, comme on voit, était de naissance très sortable à l'épouser. Le
connétable, très absolu dans sa famille, voulait disposer de ses
enfants, encore plus s'il se peut de cet aîné. Il attendait l'occasion
de quelque grand mariage, et son fils celle de lui parler de celui qu'il
voulait faire, et de l'y faire consentir. Dans l'intervalle, la duchesse
de Castro perdit son mari, et Henri II, qui aimait fort sa fille, et
auprès duquel le connétable était alors dans la plus grande faveur, lui
demanda son fils aîné pour sa fille, et le connétable ébloui, non de
l'alliance bâtarde légitimée, mais de la faveur et de la fortune qui en
serait la longue dot, conclut à l'instant avec beaucoup de joie. Elle
fut bien troublée quand il parla à son fils. L'histoire des regrets des
deux amants et de leur résistance est touchante, et la violence qu'ils
éprouvèrent ne fait pas honneur à ceux qui l'employèrent. Je n'ai pas
dessein de la copier ici. Je dirai seulement qu'ils n'eurent de défense
contre l'autorité royale et paternelle tout entière déployée contre eux,
ni d'autres armes pour se défendre que leur conscience et leur honneur.
M\textsuperscript{lle} de Piennes fut mise et resserrée dans un couvent,
et le maréchal de Montmorency forcé d'aller à Rome solliciter en
personne la dispense de sa promesse, qu'il y sollicita en homme qui ne
la voulait pas obtenir. En même temps Henri II fit l'édit célèbre contre
les mariages clandestins, avec clause rétroactive expressément mise pour
l'affaire du maréchal de Montmorency, lequel fut la seule cause de
l'édit. Finalement il fallut obéir. Il épousa la duchesse de Castro. Ce
fut pour le consoler, et en considération de ce mariage qu'Henri II lui
donna, dans son conseil, la préséance sur le chancelier, n'étant encore
que maréchal de France, mais avec de grands emplois. On voit combien ce
fait personnel et singulier est étranger à la branche de
Montmorency-Laval, et combien M. de Laval fut prodigue de mensonges pour
s'en avantager. J'ajouterai pour la simple curiosité que
M\textsuperscript{lle} de Piennes fut longtemps dans la douleur et dans
la solitude.

Bien des années après, les Guise, méditant la Ligue et ce qu'ils furent
si près d'exécuter, s'attachèrent le marquis de Maignelets\,; ce furent
eux qui le firent faire duc et pair dans la suite. Ils s'attachèrent
tant qu'ils purent les ministres, et M\textsuperscript{lle} de Piennes
se trouvant très difficile à marier après une aventure si éclatante, où
son honneur pourtant n'était point intéressé, mais par la délicatesse de
ces temps-là sur les mariages, les Guise, pour flatter les ministres, et
qui avaient les Robertet tout à fait à eux, firent le mariage de
M\textsuperscript{lle} de Piennes avec Florimond Robertet, seigneur
d'Alluye, secrétaire d'État, et ministre alors important, qui avait le
gouvernement d'Orléans. M\textsuperscript{lle} de Piennes, devenue
M\textsuperscript{me} d'Alluye, belle encore et pleine d'esprit et
d'intrigues, figura fort dans celles de la cour, et même de l'État,
depuis ce mariage qui est le premier exemple d'un pareil avec un
secrétaire d'État, qui après assez de lacune n'a que trop été imité.

\hypertarget{chapitre-xv.}{%
\chapter{CHAPITRE XV.}\label{chapitre-xv.}}

1717

~

{\textsc{Six conseillers d'État nommés commissaires, et l'un d'eux
rapporteur de l'affaire des princes du sang et bâtards au conseil de
régence, et temps court fixé aux deux partis pour lui remettre leurs
papiers.}} {\textsc{- Extrême embarras du duc et de la duchesse du
Maine.}} {\textsc{- Leurs mesures forcées.}} {\textsc{- Requête de
trente-neuf personnes, se disant la noblesse, présentée par six d'entre
eux au parlement pour faire renvoyer l'affaire des princes du sang et
des bâtards aux états généraux du royaume.}} {\textsc{- Réflexion sur
cette requête.}} {\textsc{- Le premier président avec les gens du roi
portent la requête au régent et lui demandent ses ordres.}} {\textsc{-
Digression sur la fausseté d'un endroit, entre autres, concernant cette
affaire, des Mémoires manuscrits de Dangeau.}} {\textsc{- Courte
dissertation sur les porteurs de la requête de la prétendue noblesse au
parlement, et sur cette démarche.}} {\textsc{- Les six porteurs de la
requête au parlement arrêtés par des exempts des gardes du corps, et
conduits à la Bastille et à Vincennes.}} {\textsc{- Libelle très
séditieux répandu sur les trois états.}} {\textsc{- Le régent travaille
avec le rapporteur et avec les commissaires.}} {\textsc{- Formation d'un
conseil extraordinaire de régence pour juger.}} {\textsc{- Lettre sur le
dixième et la capitation de force gentilshommes de Bretagne au comte de
Toulouse, pour tocsin de ce qui y suivit bientôt.}} {\textsc{-
Députation du parlement au roi pour lui rendre compte de ce qui s'y
était passé sur l'affaire des princes du sang et bâtards, et recevoir
ses ordres.}} {\textsc{- Arrêt en forme d'édit rendu au conseil de
régence, enregistré au parlement, qui prononce sur l'affaire des princes
du sang et des bâtards\,; adouci par le régent, et aussitôt après adouci
de son autorité contre la teneur de l'arrêt.}} {\textsc{- Rage de la
duchesse du Maine.}} {\textsc{- Douleur de M\textsuperscript{me} la
duchesse d'Orléans.}} {\textsc{- Scandale du monde.}} {\textsc{- Les six
prisonniers très honorablement remis en liberté\,; leur hauteur.}}
{\textsc{- Misère du régent.}} {\textsc{- Il ôte néanmoins la pension et
le logement qu'il donnait à M. de Châtillon, qui va s'enterrer pour
toujours en Poitou.}} {\textsc{- Conduite des ducs en ces mouvements, et
la mienne particulière.}} {\textsc{- Motifs et mesures des bâtards et du
duc de Noailles, peut-être les mêmes, peut-être différents, pour faire
convoquer les états généraux.}} {\textsc{- Occasion de la pièce
suivante, qui empêche la convocation des états généraux.}} {\textsc{-
Raisons de l'insérer ici, et après coup.}}

~

Les gens du roi du parlement, à qui l'arrêt préparatoire du conseil de
régence avait renvoyé les princes du sang et les bâtards pour leur
remettre leurs mémoires et pièces respectives, ayant refusé de s'en
charger, il fut résolu, au conseil de régence du dimanche 6 juin, d'en
charger six commissaires. Les princes du sang et les bâtards sortirent
du conseil lorsque M. le duc d'Orléans mit cette affaire sur le tapis.
Je sortis incontinent après eux, et les autres ducs du conseil me
suivirent. Je ne crus pas qu'il nous convînt d'être juges dans cette
affaire, où nous devions désirer que justice fût faite aux princes du
sang contre les bâtards, après avoir présenté au roi une requête pour la
restitution de notre rang contre ces derniers. Les commissaires nommés
furent six conseillers d'État\,: Pelletier de Sousy, Amelot, Nointel,
Argenson, La Bourdonnaye et Saint-Contest, nommé rapporteur, à qui tous
les mémoires et papiers respectifs durent être remis dans le 20 juin
pour tout délai, pour être vus par les six commissaires, puis en leur
présence, être rapportés au conseil de régence, où le régent se réserva
d'appeler qui il jugerait à propos pour remplir les places des princes
du sang, bâtards et ducs du conseil de régence, qui n'en devaient pas
être juges.

M. et M\textsuperscript{me} du Maine, pressés de la sorte, se trouvèrent
dans le dernier embarras. Leur déclaration de ne reconnaître pour juges
que le roi majeur ou les états généraux avait mis M. le duc d'Orléans
dans la nécessité de les juger, ou de perdre toute l'autorité de la
régence. Ils avaient espéré de si bien étourdir sa faiblesse de cette
hardiesse, et des manèges d'Effiat, de Besons et des autres gens à eux
qui obsédaient le régent, qu'ils avaient compté l'arrêter tout court.
Mais lorsque l'arrêt préparatoire intervenu si peu de jours après leur
eut appris qu'ils s'étaient trompés, et que cette audace, qu'ils avaient
cru leur salut, était une faute capitale qui précipiterait leur
jugement, ils se trouvèrent dans une angoisse qui fut coup sur coup
portée au comble par l'arrêt intervenu sur cette prétendue noblesse dont
M. le duc d'Orléans avait refusé de recevoir le mémoire ou la requête,
qu'il n'avait renvoyée à personne, qui était ainsi tombée dans l'eau\,;
et par la défense de l'arrêt du conseil de régence à tous nobles de la
signer, et celle de M. le duc d'Orléans à tous nobles de s'assembler,
sous peine de désobéissance. La débandade qui avait suivi de cette
prétendue noblesse, l'impossibilité de faire plus subsister à son égard
le prétexte des ducs, et de continuer ainsi à l'ameuter et à la
grossir\,; la nécessité de prendre promptement un parti devenait
extrême\,; il ne leur restait que celui de se servir de l'aveuglement de
ce qui était resté de cette noblesse fascinée, pour essayer, par un coup
de désespoir, d'en faire peur au régent et aux princes du sang, en
flattant le parlement et en les unissant ensemble. Il fallut pour cela
sortir de derrière le rideau à l'ombre duquel ils s'étaient tenus cachés
tant qu'avait pu durer le prétexte des ducs, et se montrer à découvert.
Ils persuadèrent donc tumultuairement à ce reste de noblesse enivrée
qu'il y allait de tout pour elle de souffrir que l'affaire entre eux et
les princes du sang fût jugée par le régent et par un conseil qu'il
choisirait sous le nom de conseil extraordinaire de régence, et la
firent tumultuairement résoudre à la requête la plus folle et dont
l'audace fut pareille à l'ineptie.

Trente-neuf personnes portant l'épée à titres fort différents, sans
élection, sans députation, sans mission, sans autorité que d'eux-mêmes,
soi-disant l'ordre de la noblesse, signèrent et présentèrent comme telle
une requête au parlement pour demander que l'affaire d'entre les princes
du sang et bâtards fût renvoyée aux états généraux du royaume, parce
que, s'y agissant du droit d'habilité à la succession à la couronne, il
n'y avait en cette matière de juges compétents que les états généraux du
royaume, et entre ces trois états, le seul second ordre qui est celui de
la noblesse. L'audace était sans exemple. C'étaient des gens ramassés,
sans titre et sans pouvoir, qui usurpaient le respectable nom de la
noblesse, qui, n'ayant point été convoquée par le roi, ne pouvait faire
corps, s'assembler, députer, donner des instructions, ni autoriser
personne\,; ainsi, dès là, très punissables. Usurpation pourquoi
faite\,? Pour attenter à l'autorité du régent, et sans être, sans
existence, sans consistance, lui arracher une cause si majeure pour s'en
saisir eux-mêmes, sans autre droit que leur bon plaisir. L'ineptie
n'était pas moindre. Dans leur folle prétention, ils étaient la noblesse
en corps, par conséquent le second ordre de l'État\,; et ce second ordre
de l'État, si auguste et si grand, se prostitue à cette bassesse sans
exemple de présenter une requête à autre qu'au roi, de la présenter à un
tribunal de justice qui, si relevé qu'il soit, n'est que membre, et non
pas ordre de l'État, et non seulement membre d'un ordre, mais du
troisième, qui est le tiers état, si disproportionné de l'ordre de la
noblesse, et ce prétendu ordre de la noblesse, encore présente à ce
simple tribunal de justice, membre du tiers état, une requête
intitulée\,: \emph{À nos seigneurs de parlement, supplient, etc}. Ce
n'est pas la peine d'être si glorieux, si fous et si enivrés de sa
naissance, et de l'état que l'orgueil et la vanité insensée lui veut
attribuer, que de la mettre ainsi sous les pieds d'une compagnie de gens
de loi, et d'invoquer son autorité pour user, par sa protection et son
prétendu pouvoir, de celui qu'on prétend ne tenir que de sa naissance,
en chose si capitale que la décision sur la succession à la couronne. Si
jamais on voyait des états généraux assemblés, ces messieurs de la
requête auraient bien à craindre le châtiment du second ordre des trois
états du royaume, et qu'il ne voulût plus reconnaître pour siens, des
nobles qui, en tant qu'il a été en eux, l'ont avili et dégradé jusqu'à
les jeter dans la poussière aux pieds de nos seigneurs membres du tiers
état. Ni l'audace ni l'ineptie, quoique l'une et l'autre au plus haut
comble, ne se présentèrent point à l'esprit ni au jugement de ces
messieurs. Ils se laissèrent fasciner d'une démarche hardie, qui mettait
au jour une si belle prétention, sans s'apercevoir qu'ils étaient d'une
part dépourvus de tout titre, et qu'ils se déshonoraient complètement de
l'autre par ce recours au parlement.

Cette compagnie plus sage qu'eux, et qui savait mieux mesurer ses
démarches, eut plus d'envie de rire de celle-là que de s'en
enorgueillir. Cette rare requête, ou plutôt unique depuis la monarchie,
n'eut pas été plutôt présentée que, quelque abandonné que fût le premier
président à M. et à M\textsuperscript{me} du Maine, sans qui cette folie
ne s'était pas tentée dans l'espérance pour dernière ressource
d'effrayer M. le duc d'Orléans par cet éclat\,; et l'empêcher de passer
outre au jugement, le premier président, dis-je, n'osa branler, et
l'alla porter au régent accompagné des gens du roi et lui demander ses
ordres.

Avant d'aller plus loin, la nécessité de constater la vérité des faits
m'oblige ici à une digression nouvelle. Dangeau, dont je me réserve à
parler ailleurs, écrivait depuis plus de trente ans tous les soirs
jusqu'aux plus fades nouvelles de la journée. Il les dictait toutes
sèches, plus encore qu'on ne les trouve dans la \emph{Gazette de
France}\footnote{On peut aujourd'hui apprécier le \emph{Journal de
  Dangeau} dans l'édition que publient MM. Didot\,: \emph{Journal du
  marquis de Dangeau avec les additions de Saint-Simon}, etc. (Paris
  1854 et ann. suiv.) Les éditeurs ont mis en tête une \emph{Notice sur
  la vie de Dangeau}, où ils s'efforcent de le défendre contre les
  attaques de Voltaire et de Saint-Simon.}. Il ne s'en cachait point, et
le roi l'en plaisantait quelquefois. C'était un honnête homme et un très
bon homme, mais qui ne connaissait que le feu roi et
M\textsuperscript{me} de Maintenon dont il faisait ses dieux, et
s'incrustait de leurs goûts et de leurs façons de penser quelles
qu'elles pussent être. La fadeur et l'adulation de ses Mémoires sont
encore plus dégoûtantes que leur sécheresse, quoiqu'il fût bien à
souhaiter que, tels qu'ils sont, on en eût de pareils de tous les
règnes. J'en parlerai ailleurs davantage. Il suffit seulement de dire
ici que Dangeau était très pitoyablement glorieux, et tout à la fois
valet, comme ces deux choses se trouvent souvent jointes, quelque
contraires qu'elles paraissent être. Ses Mémoires sont pleins de cette
basse vanité, par conséquent très partiaux, et quelquefois plus que
fautifs par cette raison. Il y est très politique autant que la
partialité le lui permet, et toujours en adoration du roi même depuis sa
mort, de ses bâtards, de M\textsuperscript{me} de Maintenon, et très
opposé à M. le duc d'Orléans, au gouvernement nouveau, et singulièrement
aux ducs, surtout de l'ignorance la plus crasse qui se montre en mille
endroits de ses Mémoires.

On a vu en son temps qu'il avait marié son fils à la fille unique de
Pompadour. Pompadour était des plus avant dans le secret du parti de M.
et de M\textsuperscript{me} du Maine, comme on verra en son temps, et
dès lors par là des plus avant avec cette prétendue noblesse.
M\textsuperscript{me} de Pompadour était soeur de la duchesse douairière
d'Elboeuf mère de la feue duchesse de Mantoue\,; il vivait intimement
avec eux. Cette alliance de son fils lui avait tourné la tête, et ces
deux soeurs, filles de feu M\textsuperscript{me} de Navailles, étaient
sous la protection déclarée de M\textsuperscript{me} de Maintenon. C'en
est assez pour ce qui va suivre. Tant que le roi vécut, Dangeau, qui ne
bougeait de la cour, qui était son unique élément, y tenait une maison
honorable, et vivait là et ailleurs avec la bonne compagnie, et avec les
gens les plus à la mode. Il avait grand soin d'être bien informé des
choses publiques, car d'ailleurs il ne fut jamais de rien. Depuis la
mort du roi ses informations n'étaient plus les mêmes\,; l'ancienne cour
se trouvait éparpillée et ne savait plus rien\,; lui-même retiré chez
lui, touchant à quatre-vingts ans, ne voyait plus que des restes
d'épluchures, et il y paraît bien à la suite de ses Mémoires depuis la
mort du roi. À propos de cette requête au parlement de la prétendue
noblesse sur l'affaire des princes du sang et des bâtards, il dit sur le
samedi 19 juin \emph{que le duc du Maine et le comte de Toulouse
allèrent au parlement, et firent leurs protestations contre tout ce qui
serait réglé dans l'affaire qu'ils ont avec les princes du sang\,;} et
sur le lundi 21 juin, il dit \emph{que M. le Duc et M. le prince de
Conti allèrent au parlement, qu'ils demandèrent que la protestation des
princes légitimes ne fût pas reçue, et que M. le prince de Conti lut un
petit mémoire lui-même}. Voilà qui est bien précis sur la date, et bien
circonstancié sur les faits.

Je n'eus occasion de voir ces Mémoires que depuis la mort de Dangeau, et
cet endroit me surprit au dernier point. Je n'en avais aucune idée. Je
ne pouvais comprendre qu'un fait de cet éclat fût sitôt effacé de ma
mémoire, surtout avec la part que j'avais prise à toute cette affaire,
par rapport à l'intérêt des ducs. D'un autre côté, je ne pouvais
imaginer que Dangeau eût mis dans ses Mémoires une fausseté de cette
espèce, et tellement datée et circonstanciée. Cela me tourmenta quelques
jours\,; enfin, je pris le parti d'aller trouver le procureur général
Joly de Fleury, et de lui demander ce qui en était. Il m'assura qu'il
n'y en avait pas un mot, qu'il était très certain que jamais le duc du
Maine et le comte de Toulouse n'étaient venus faire ces protestations au
parlement, ni M. le Duc et M. le prince de Conti non plus demander
qu'elles ne fussent pas reçues, qu'il avait cela très présent à la
mémoire, et qu'un fait de tel éclat ne lui aurait pas échappé de la
mémoire dans la place qu'il remplissait dès lors, et qui le mettait
{[}en état{]} d'en être bien et promptement informé, s'il y en eût eu
seulement la moindre, chose, de ce que le parlement y eût fait ou voulu
faire, et des suites que cela y aurait eues et au
Palais-Royal\footnote{Une note, écrite sur la marge du manuscrit
  autographe de Saint-Simon est ainsi conçue\,: «\,Le fait rapporté par
  Dangeau est vrai, je viens de le vérifier sur le Journal du
  parlement.\,» Cette note est probablement de M. Le Dran, qui était
  principat commis des affaires étrangères en 1761, lorsque les
  manuscrits de Saint-Simon y furent déposés.}. Il est vrai aussi que
Dangeau n'en marque aucune, quoiqu'il fût impossible que cela n'en eût
eu de façon ou d'autre, quoiqu'il soit exact à n'en omettre aucune.
Reste à voir si c'est une fausseté qu'il ait faite exprès, et qu'à faute
de mieux, le duc du Maine ait désirée, pour qu'il restât au moins
quelque part, et quelque part qui bien que sans plus d'autorité que les
gazettes, serait un jour comme elles entre les mains de tout le monde,
pour qu'il restât, dis-je, un témoignage qu'il avait conservé son
prétendu droit aussi authentiquement qu'il, avait pu le faire, et qu'il
l'avait mis de la sorte à couvert contre tout jugement selon lui
incompétent, par un acte si solennel, et qui n'avait reçu ni
condamnation ni contradiction. (En effet elle en était bien à couvert,
puisque jamais elle n'a été faite) et après prétendre, que ne se
trouvant pas dans les registres du parlement, elle en aura été ou omise
par ordre exprès du régent, ou tirée par la même autorité de ces
registres si elle y avait été d'abord mise. Peut-être aussi Dangeau
l'aura-t-il cru et mis sur la parole de Pompadour, avec la circonstance
de M. le Duc et de M. le Prince deux jours après, pour mieux appuyer et
assurer le premier mensonge, dont ce vieillard renfermé chez lui aura
été la dupe. Quoi qu'il en soit, il est sûr que la chose est fausse, et
que le, procureur général Joly de Fleury, dont la mémoire ni la personne
en cela ne peuvent être suspectes, me l'a très certainement et très
nettement assurée telle. De même conséquence et fausseté, et ce que le
procureur général m'a certifié être également faux, c'est ce que Dangeau
ajoute du même samedi 19 juin, jour qu'il raconte cette protestation
faite dans la grand'chambre par les deux bâtards en personne, \emph{que
le parlement résolut de se rassembler le lundi matin pour répondre à la
protestation des bâtards, et qu'en attendant, ils envoyèrent recevoir
les ordres de M. le duc d'Orléans là-dessus}. Puis de ce lundi 21 juin,
jour où il marque l'entrée des deux princes du sang au parlement pour
lui demander de ne pas recevoir la protestation des bâtards, il ajoute
\emph{que le parlement envoie les gens du roi au roi pour recevoir ses
ordres sur ce qu'ils auront à faire sur la protestation des bâtards}.
Après quoi il n'en parle plus, non plus que de chose non avenue. Or de
façon ou d'autre il y aurait eu des ordres au parlement là-dessus, et le
parlement eût envoyé au régent pour les avoir, car au roi qui n'était
pas d'âge à en donner, ce n'eût été qu'une forme, et du samedi il
n'aurait pas attendu au lundi pour cela, ni s'il avait envoyé dès le
samedi au régent comme il l'insinue, il aurait encore moins envoyé au
roi deux jours après. Après cet éclaircissement nécessaire, revenons.

MM\hspace{0pt}. de Châtillon, de Rieux, de Clermont et de Baufremont
qui, avec les quatre autres qu'on a nommés ci-dessus, avaient été au
Palais-Royal présenter au régent le mémoire ou requête dont on a parlé,
qui ne l'avait pas voulu recevoir, furent aussi ceux qui allèrent
présenter au parlement la requête sur l'affaire des princes du sang et
bâtards accompagnés de MM. de Polignac et de Vieuxpont. On a fait
connaître les quatre premiers. À l'égard des deux autres, Polignac était
un petit bilboquet qui n'avait pas le sens commun, conduit et nourri par
son frère le cardinal de Polignac, à vendre et à dépendre, qui était de
tout temps de M. et de M\textsuperscript{me} du Maine, et leur plus
intime confident. Le pauvre petit Polignac obéit et ne sut pas seulement
de quoi il s'agissait\,; je dis l'écorce même, car il en était
entièrement incapable jamais deux frères ne furent si complètement
différents en tout. Vieuxpont était un assez bon officier général, qui
ne connaissait que cela, et qui logeait chez son beau-père, le premier
écuyer, où il vivait dans la plus aveugle dépendance. On a vu ailleurs
ce que c'était que M\textsuperscript{me} de Beringhen et le duc d'Aumont
son frère, à quel point ils étaient vendus au premier président, et le
premier écuyer d'ailleurs son ami intime, et d'ancienneté tout aux
bâtards. Son gendre, sottement glorieux d'ailleurs et fort court
d'esprit, goba aisément ce prestige de noblesse, crut figurer, et obéit
à beau-père et à belle-mère, et aux jargons du duc d'Aumont. Le crime
était complet, 1° de se prétendre être la noblesse ne pouvant être que
des particuliers, par toutes les raisons palpables qu'on en a vues
ci-dessus\,; 2° de s'assembler contre la défense expresse à eux faite
par le régent\,; car faire une requête souscrite de trente-neuf
signatures et présentée au parlement par six seigneurs en personne n'est
pas chose qui se puisse sans s'être concertés, et pour cela
nécessairement assemblés\,; 3° se mêler de choses supérieures à tous
particuliers comme tels\,; 4° d'oser implorer l'autorité du parlement
pour arrêter le jugement d'une affaire dont le régent du royaume est
saisi, qu'il a déclaré qu'il va juger, qui s'y est engagé par des
démarches juridiques et publiques, pour lui en ôter la connaissance,
comme si le parlement pouvait plus que le régent, et pour la faire
renvoyer à un tribunal qui n'existe point. Le régent sentit qu'il
fallait opter entre lâcher tout à fait les rênes du gouvernement et
faire une punition exemplaire. Il porta cette requête au conseil de
régence, où elle nous fut lue avec les signatures. On en raisonna sans
opiner, et le régent en parut fort altéré\,; mais ceux qui l'obsédaient,
aidés de sa faiblesse et de sa facilité, de plus contredits de personne,
car moi ni pas un autre duc n'en dîmes pas un seul mot, trouvèrent moyen
de tourner cette punition de la manière la plus singulière.

On fit l'honneur à ces six messieurs qui avaient été au parlement
présenter la requête, de les faire arrêter par des exempts des gardes du
corps, le samedi matin 19 juin, qui les conduisirent partie à la
Bastille, partie à Vincennes, où ils furent comblés de civilités et de
toutes sortes de bons traitements, sans pourtant voir personne. Cet
emprisonnement fit grand bruit parce qu'on n'en attendait pas tant de
l'infatigable débonnaireté de M. le duc d'Orléans\,; mais la manière si
distinguée en fit encore davantage, et tant de ménagements, si fort
déplacés, firent triompher la prétendue noblesse, et envier publiquement
l'honneur d'être des prisonniers. Trois jours après, il courut un
libelle extrêmement insolent et séditieux, intitulé \emph{Écrit des
trois états}, qui ramena le souvenir des écrits les plus emportés de la
Ligue. Il ne parut que manuscrit, mais dix mille copies à la fois, qui
se multiplièrent bien davantage.

Parmi tout ce bruit, Saint-Contest travaillait souvent avec M. le duc
d'Orléans, et il travaillait en même temps avec les six commissaires,
qui allèrent aussi deux fais tous six travailler avec M. le duc
d'Orléans. Outre ceux du conseil de régence qui n'étaient point parties
ni ducs, et qui demeuraient juges de l'affaire des princes du sang et
bâtards, le maréchal d'Huxelles, MM. de Bordeaux, de Biron\,; et
Beringhen, premier écuyer, leur forent joints des conseils de
conscience, de guerre, des affaires étrangères et du dedans. Cela ne fut
déclaré que le dimanche matin 27 juin, au conseil de régence,
c'est-à-dire après qu'il fut levé en sortant. Le lendemain lundi, le
comte de Toulouse, qui se tenait fort à part dans tous ces mouvements
qui n'étaient point du tout de son goût, rendit compte à M. le duc
d'Orléans qu'il avait reçu une lettre, souscrite de quantité de
gentilshommes de Bretagne, sur l'impossibilité où était cette province
de payer le dixième, et de la sage réponse qu'il leur avait faite. Je
remarque cette lettre comme le premier coup de tocsin de ce qu'on verra
dans la suite en Bretagne. Le mercredi 30 juin, le premier président,
tous les présidents à mortier et les gens du roi allèrent à onze heures
aux Tuileries, députés pour venir rendre compte au roi de ce qui s'était
passé sur l'affaire des princes du sang et légitimés, lui remettre la
requête et protestation de la prétendue noblesse, et recevoir ses ordres
M. le duc d'Orléans présent, et le chancelier, à qui le roi remit de la
main à la main ce que le premier président lui avait présenté\,; le
chancelier leur dit que le roi leur ferait savoir sa volonté.

L'après-dînée du même jour se tint le conseil de régence extraordinaire
pour le jugement, qui fut continué le lendemain matin jeudi 1er juillet.
L'arrêt ne fut pas tout d'une voix. Saint-Contes fit un très beau
rapport et fut en entier pour les princes du sang ainsi que la plupart
des juges. La rare bénignité de M. le duc d'Orléans, que tant de
criminels et d'audacieux manèges n'avaient pu émousser, sa facilité, sa
faiblesse pour ceux qui l'obsédaient et qui étaient aux bâtards, quelque
vapeur de crainte, et cette politique favorite \emph{divide et impera},
le mit en mouvement pour faire revenir les juges à quelque chose de plus
doux. La succession à la couronne fut totalement condamnée, le rang des
enfants supprimé, celui des deux bâtards modéré. L'arrêt, tourné en
forme d'édit, fut trouvé trop doux au parlement, et pour cette raison
enregistré avec difficulté le mardi 6 juillet. Et malgré la teneur de
l'édit, M. le duc d'Orléans, de pleine autorité, le modéra de fait
encore, en sorte que les bâtards n'y perdirent que l'habilité de
succéder à la couronne, et le traversement du parquet au parlement. M.
le Duc défendit aux maîtres d'hôtel du roi de lui laisser présenter la
serviette par les enfants du duc du Maine\,; le duc de Mortemart premier
gentilhomme de la chambre d'année, leur refusa le service de prince du
sang, et il y eut difficulté dans les salles des gardes de prendre les
armes pour eux. M. le duc d'Orléans ordonna sur-le-champ qu'ils fussent
traités en princes du sang à l'ordinaire, et comme avant l'arrêt ce
qu'il fit exécuter. Cette très étrange bonté n'empêcha pas
M\textsuperscript{me} du Maine de faire les hauts cris comme une
forcenée, ni M\textsuperscript{me} la duchesse d'Orléans de pleurer jour
et nuit, et d'être deux mois sans vouloir voir personne, excepté ses
plus familières et en très petit nombre, et encore sur la fin. M. du
Maine avait le don de ne montrer jamais que ce qui lui convenait, et ses
raisons pour en user en cette occasion. Il ne vint pourtant pas au
premier conseil de régence, il fit dire qu'il était incommodé, mais il
se trouva au second à son ordinaire. Le comte de Toulouse parut toujours
le même, et ne s'absenta de rien. Excepté les enrôlés avec M. du Maine,
le reste du monde fut étrangement mécontent, et les princes du sang
encore davantage, d'une si démesurée mollesse, mais n'en pouvant plus
tirer mieux, ils triomphèrent de ce qu'ils avaient obtenu.

Les six prisonniers, bien servis et bien avertis par d'Effiat,
écrivirent au bout d'un mois à M. le duc de Chartres, qui envoya leur
lettre à M. le duc d'Orléans par Cheverny, son gouverneur, de même nom
que Clermont-Gallerande l'un d'eux. M. le duc d'Orléans fit espérer leur
prochaine liberté. Le samedi 17 juillet, le premier écuyer alla par
ordre du régent prendre les trois qui étaient à Vincennes, et Cheverny
les trois qui étaient à la Bastille, et les amenèrent chez M. le duc de
Chartres, qui alla les mener à M. le duc d'Orléans. Le régent leur dit
qu'ils connaissaient assez qu'il ne faisait du mal que lorsqu'il s'y
croyait fortement obligé. Pas un des six ne prit la peine de lui dire
une seule parole, et se retirèrent aussitôt. Cette sortie de prison eut
tout l'air d'un triomphe, et par le choix des conducteurs, et par la
hauteur et le silence des prisonniers rendus libres. Il sembla qu'ils
faisaient grâce au régent de lui épargner les reproches, et que ce
prince avait tâché de mériter cette modération de leur part par une si
étonnante façon de les mettre en liberté. Il le sentit après coup, et se
repentit de sa mollesse, comme il lui arrivait souvent après des fautes
dont après il ne se corrigeait pas plus. Il éprouva bientôt après le
fruit d'une si faible conduite, et l'effet qu'elle avait fait sur tous
ceux qui, avec dérision et mépris, en avaient su profiter. Il eut
pourtant le courage d'ôter le même jour à M. de Châtillon la pension de
douze mille livres qu'il lui donnait, et son logement au Palais-Royal.
Comme il était fort pauvre, et depuis bien des années fort obscur, il
alla bientôt après s'enterrer dans une très petite terre qu'il avait
auprès de Thouars, où il est presque toujours demeuré jusqu'à sa mort.

Les ducs ne prirent aucune part à pas un de tous ces mouvements et
demeurèrent parfaitement tranquilles\,; ils n'avaient rien ni à perdre
ni à gagner, et laissèrent bourdonner et aboyer. À l'égard des bâtards,
contents des requêtes qu'ils avaient présentées au roi, et portées au
régent sur la restitution de leur rang à cet égard, ils n'avaient pas
trouvé assez de fermeté, de justice, ni de parole dans le régent sur le
bonnet et les autres choses concernant le parlement, pour s'en promettre
davantage contre des personnes si proches, si grandement établies, et si
fortement soutenues d'intrigues et d'obsessions près de lui. Ils
estimèrent donc qu'après avoir mis leur droit à couvert par leurs
requêtes au roi, le repos et la tranquillité était le seul parti qu'ils
eussent à prendre, en attendant des conjonctures plus favorables, si
tant était qu'il en arrivât, et les surprenants adoucissements que, de
pleine autorité, le régent apporta à l'arrêt en forme d'édit beaucoup
trop doux encore aux yeux des juges et du parlement qui l'enregistra,
témoigna bien la sagesse de cette prévoyance. À mon égard en
particulier, je continuai dans mon même silence avec le régent par les
mêmes raisons que je viens de dire, et pour lui montrer aussi une sorte
d'indifférence sur une conduite que je ne pouvais ni approuver ni
changer, et je me contentai de lui répandre froidement et laconiquement,
lorsque rarement il ne put s'empêcher de me parler de ces deux affaires
qui, n'ayant qu'une même source, marchèrent en même temps. Elles m'ont
paru mériter d'être rapportées tout de suite, et sans mélange d'aucune
autre. C'est cette raison qui m'a fait remettre ici après coup ce qui en
aurait trop longuement interrompu la narration. C'est une pièce que je
crois convenir mieux ici malgré son étendue, que parmi les autres
pièces, par la connexité qu'elle a avec la matière de ces Mémoires et
l'éclaircissement naturel qu'elle y pourra donner.

Dans les commencements que l'affaire s'échauffa entre les princes du
sang et les bâtards au point que M. le duc d'Orléans sentit qu'il ne
pourrait éviter de la juger, les bâtards qui désespérèrent de le pouvoir
échapper et qui n'établissaient leurs ressources que dans l'éloignement
de ce jugement, le firent sonder par d'Effiat sur le renvoi aux états
généraux, pour s'en délivrer. C'était toujours plusieurs mois de délais
avant qu'ils fussent assemblés, car ils sentaient bien qu'en les y
renvoyant, les princes du sang ne souffriraient pas que ce fût un renvoi
de temps indéfini et sans bout. Les mesures qui leur réussissaient si
bien avec cette foule de toute espèce qui se disait la noblesse, et
celles qu'ils prenaient sourdement de loin dans les provinces, leur
persuadaient que, jugés pour jugés, il valait encore mieux pour eux
hasarder cette voie où leurs cabales leur donnaient du jeu pour faire
mille querelles dans les états, leur faire mettre mille prétentions en
avant pour les rompre, si le vent du bureau ne leur était pas favorable,
que de se laisser juger par un conseil formé par M. le duc d'Orléans,
que M. du Maine avait tant et si cruellement et dangereusement et
monstrueusement offensé, et dont le fils unique, premier prince du sang,
avait contre eux un intérêt pareil et commun avec M. le Duc et M. le
prince de Conti. En cadence de d'Effiat, le duc de Noailles, soit qu'il
fût dans la même bouteille comme les mouvements de la prétendue noblesse
à qui il avait donné l'être et le ton par lui-même, par Coetquen son
beau-frère, et par d'autres émissaires à la mort du roi, comme je l'ai
raconté en son lieu\,; soit qu'en effet à bout et en crainte sur la
gestion des finances dont il avait embrassé seul toute l'autorité, par
conséquent les suites et le poids, et sujet en toutes choses à voler
d'idée en idée et de passer subitement aux plus contradictoires sans
autre cause que sa singulière mobilité, il se fût avisé de souhaiter à
contretemps ce qu'il avait seul empêché si fort à temps, il se mit à
déployer toute son éloquence auprès de M. le duc d'Orléans pour lui
persuader qu'il n'y avait plus de remède à l'état déplorable des
finances, que d'assembler les états généraux. Le régent en fut d'autant
plus susceptible que d'Effiat le touchait par son endroit sensible qui
était l'incertitude et la timidité. Il commençait par se donner du temps
et se délivrer de poursuites, et se déchargeait de l'embarras et de
l'iniquité d'un jugement qui ne pouvait qu'exciter violemment la partie
condamnée dans une affaire sans milieu, comme était le droit maintenu ou
supprimé de succéder à la couronne, d'où dépendaient mille suites
poignantes\,; et du côté des finances, plus il avait résolu d'assembler
pour les régler les états généraux à la mort du roi, plus le seul duc de
Noailles l'en avait empêché, comme je l'ai raconté en son temps, plus
l'avis du même Noailles de les assembler maintenant, pour trouver
ressource aux finances, avait de poids à ses yeux.

Dans l'irrésolution où il se trouvait sur une chose de conséquences si
importantes, il s'en ouvrit à moi, et m'en demanda mon avis, comme il
faisait toujours dans ce qui l'embarrassait, où dans ce qui était
important. Je me récriai d'abord sur un si dangereux parti. Il m'opposa
mon propre avis lors de la dernière année et de la mort du roi. Je
répandis que ce qui était excellent alors se trouverait pernicieux
aujourd'hui que tout avait changé de face. Il voulut discuter, je coupai
court, et lui dis que la matière valait bien d'y penser, et de lui
mettre devant les yeux beaucoup de choses, qui s'oublient ou se
déplacent dans les conversations, au lieu qu'un écrit se fait plus
mûrement, se trouve toujours ensuite sous la main sans rien perdre, et
se livre plus parfaitement à la balance. Il me dit que je le fisse donc,
mais qu'il était pressé de prendre son parti, et ce parti, je vis qu'on
l'entraînait au précipice. La crainte que j'eus de l'y voir rapidement
enlevé m'engagea à lui promettre mon écrit dans deux jours, et en effet
je le lui apportai le troisième sans avoir eu presque le temps de
relire. Pour le montrer à personne, sa teneur fera comprendre que je ne
l'imaginai pas. On y verra la mesure d'un écrit fait pour ce prince, et
adressé à lui, fort différente comme de raison de la liberté des
conversations autorisée par la familiarité de toute notre vie, et des
temps pour lui les plus abandonnés et les plus périlleusement orageux.
Le voici.

\hypertarget{chapitre-xvi.}{%
\chapter{CHAPITRE XVI.}\label{chapitre-xvi.}}

1717

~

{\textsc{Projet d'états généraux fréquents de Mgr le Dauphin, père du
roi.}} {\textsc{- Je voulais des états généraux à la mort du roi.}}
{\textsc{- Embarras des finances et subsidiairement de l'affaire des
princes.}} {\textsc{- Motifs de vouloir les états généraux.}} {\textsc{-
Trait sur le duc de Noailles.}} {\textsc{- Introduction à l'égard des
finances.}} {\textsc{- État de la question.}} {\textsc{- Grande
différence d'assembler d'abord, et avant d'avoir touché à rien, les
états généraux, ou après tout entamé et tant d'opérations.}} {\textsc{-
Chambre de justice, mauvais moyen.}} {\textsc{- Timidité, artifice et
malice du duc de Noailles sur le duc de La Force, très nuisible aux
affaires.}} {\textsc{- Banque du sieur Law.}} {\textsc{- Première
partie\,: raisons générales de l'inutilité des états.}} {\textsc{-
Malheur du dernier gouvernement.}} {\textsc{- Choc certain entre les
fonciers et les rentiers.}} {\textsc{- Premier ordre divisé
nécessairement entre les rentiers et les fonciers, quoique bien plus
favorables aux derniers.}} {\textsc{- Second ordre tout entier contraire
aux rentiers.}} {\textsc{- Éloge et triste état du second ordre.}}
{\textsc{- Troisième ordre tout entier pour les rentes.}} {\textsc{-
Choc entre les deux premiers ordres et le troisième sur les rentes,
certain et dangereux.}} {\textsc{- Pareil choc entre les provinces sur
les rentes, auxquelles le plus grand nombre sera contraire.}} {\textsc{-
Ce qu'il paraît de M. le duc d'Orléans sur l'affaire des princes.}}
{\textsc{- Ses motifs de la renvoyer aux états généraux.}} {\textsc{-
Certitude du jugement par les états généraux et de l'abus des vues de
Son Altesse Royale à son égard.}} {\textsc{- États généraux parfaitement
inutiles pour le point des finances et pour celui de l'affaire des
princes.}} {\textsc{- Deuxième partie\,: inconvénients des états
généraux.}} {\textsc{- Rangs et compétences.}} {\textsc{- Autorité et
prétentions.}} {\textsc{- Difficulté de conduite et de réputation pour
M. le duc d'Orléans.}} {\textsc{- Danger et dégoût des promesses sans
succès effectif.}} {\textsc{- Fermeté nécessaire.}} {\textsc{- Demandes
des états.}} {\textsc{- Propositions des états.}} {\textsc{- Nulle
proportion ni comparaison de l'assemblée des états généraux à pas une
autre.}} {\textsc{- Deux moyens de refréner les états, mais pernicieux
l'un et l'autre.}} {\textsc{- Refus.}} {\textsc{- Danger de formation de
troubles.}} {\textsc{- Autorité royale à l'égard du jugement de
l'affaire des princes.}} {\textsc{- Troisième partie\,: premier ordre.}}
{\textsc{- La constitution Unigenitus.}} {\textsc{- Juridiction
ecclésiastique.}} {\textsc{- Deuxième ordre.}} {\textsc{- Le deuxième
ordre voudra seul juger l'affaire des princes.}} {\textsc{- Trait sur
les mouvements de la prétendue noblesse et sur le rang de prince
étranger.}} {\textsc{- Partialités et leurs suites.}} {\textsc{-
Situation du second ordre, d'où naîtront ses représentations et ses
propositions.}} {\textsc{- Choc entre le second ordre et le troisième
ordre inévitable, sur le soulagement du second.}} {\textsc{-
Mécontentement du militaire.}} {\textsc{- Troisième ordre et ce qui le
compose.}} {\textsc{- Troisième ordre en querelle et en division.}}
{\textsc{- Confusion intérieure en laquelle le second ordre prendra
partie\,; et {[}troisième ordre{]} commis d'ailleurs entre les deux
premiers ordres.}} {\textsc{- Grande et totale différence de la tenue
des états généraux, à la mort du roi, d'avec leur tenue à présent.}}
{\textsc{- Tiers état peu docile, et dangereux en matière de finance.}}
{\textsc{- Péril de la banque du sieur Law.}} {\textsc{- Trait sur le
duc de Noailles.}} {\textsc{- Exemples qui doivent dissuader la tenue
des états généraux.}} {\textsc{- États généraux utiles, mais suivant le
temps et les conjonctures.}} {\textsc{- Courte récapitulation des
inconvénients d'assembler les états généraux.}} {\textsc{- Conclusion.}}
{\textsc{- Trait sur le duc de Noailles.}} {\textsc{- Vues personnelles
à moi répandues en ce mémoire.}}

~

MÉMOIRE ADRESSÉ À S. A. R. MONSEIGNEUR LE DUC D'ORLÉANS, RÉGENT DU
ROYAUME, SUR UNE TENUE D'ÉTATS GÉNÉRAUX (MAI 1717)\,:

Monseigneur, l'honneur que me fait Votre Altesse Royale de m'ouvrir ses
pensées sur l'avantage et les inconvénients d'assembler les états
généraux de ce royaume dans les embarras présents du gouvernement de
l'État dont, vous êtes chargé, et de m'ordonner d'y bien penser pour
vous en dire mon avis, m'engage, pour répondre dignement à la grandeur
et à l'importance de la matière, d'écrire plutôt que de parler, comme un
moyen contre les défauts de mémoire, et ceux de la promptitude du
discours, et de la confusion de la conversation.

Avant d'entrer en matière, Votre Altesse Royale se souviendra s'il lui
plaît, par deux faits trop graves pour lui être échappés, que de tous
ceux qui ont eu l'honneur de l'approcher dans tous les temps aucun n'a
plus d'estime, ni, pour ainsi parler, de goût naturel pour les états
généraux que j'en ai toujours eu. L'un est que, travaillant sous les
yeux de feu Mgr le Dauphin, père du roi, aux projets dont vous avez pris
quelques parties\,; le principal des miens était des états généraux de
cinq ans en cinq ans, et de les simplifier de manière qu'ils se pussent
assembler sans cette confusion qui les a si souvent rendus inutiles\,;
que ces états généraux fussent en grand et en corps le surintendant des
finances pour les dons, les impôts, leur répartition, leur recette, et
leur dépense\,; qu'il fût compté de tout devant eux\,; qu'entre chaque
tenue il en subsistât une députation d'un personnage de chacun des trois
ordres pour faire dans l'intervalle les choses journalières et d'autres
pressées, jusqu'à certaines bornes, par une administration dont ils
seraient comptables aux états prochains\,; qu'ils eussent durant net
exercice un rang et des privilèges, qui vous ont montré jusqu'où va mon
respect pour la nation représentée\,; et que ce qui serait mis à part
pour les dépenses particulières du roi, comme une espèce de liste
civile, fût géré par un trésorier, qui n'en compterait qu'au roi par sa
chambre des comptes.

L'autre est celui d'assembler les états généraux aussitôt après la mort
du feu roi, dont Votre Altesse Royale se peut souvenir combien j'ai pris
la liberté de l'en presser, qu'elle l'avait résolu, et que, si elle a
depuis changé d'avis, ç'a été constamment contre le mien.

Il n'est pas question ici de s'arrêter à ces deux faits, qu'il suffit de
représenter à votre mémoire en deux mots. Le premier ne pouvait être
d'usage que sous un roi majeur et selon le coeur de Dieu, né pour être
le père de ses peuples, le restaurateur de l'ordre, et un modérateur
incorruptible par un discernement exquis de la justice et de ses
intérêts véritables. L'explication de ce projet ne vous apprendrait rien
de nouveau, m'écarterait de mon sujet, renouvellerait inutilement ma
douleur amère de la perte d'un tel prince, et de l'inutilité de ce que
j'avais conçu et digéré avec plus de joie encore que de travail pour
l'honneur et l'avantage solide de la France. L'autre a été si fort agité
avec Votre Altesse Royale avant et après la mort du roi, et cette époque
est si récente, qu'elle ne peut être échappée de votre mémoire.

Ce qui fait présentement naître la pensée d'une tenue d'états généraux
est, par ce que Votre Altesse Royale m'a fait l'honneur de m'en dire,
subsidiairement l'état d'engagement et de difficulté où en est l'affaire
des princes, mais effectivement le terme d'embarras où se trouvent les
finances\,; et puisque c'est de ce dernier point qu'il s'agit réellement
ici, c'est celui qu'il faut traiter le plus solidement qu'il me sera
possible par rapport au remède des états généraux, en y faisant entrer
après en son temps celui des princes.

Beaucoup de raisons m'empêcheront d'entrer en aucun détail sur
l'administration des finances. J'évite toujours avec soin de traiter des
choses passées, où il n'y a plus de remède à proposer. Je me suis rendu
une si exacte justice sur mon incapacité spéciale en ce genre que Votre
Altesse Royale sait que je n'ai pu être vaincu ni par son choix, ni par
ses bontés, pour m'en charger. J'ai pris la liberté de lui en proposer
un autre, comptant sur son esprit, sur son application, sur son
désintéressement et naturel et fondé sur les biens et les établissements
infinis dont il est environné. Si de profonds détours, si des desseins
artificieusement amenés à leur période, en ont été pour moi un fruit
amer aussi surprenant qu'imprévu et subit, ce m'est un nouveau motif de
silence, quelque impartial que je me sente quand il est question du bien
de l'État, ou même de traiter d'affaires. J'ose même en attester Votre
Altesse Royale, qui a eu souvent occasion d'en être témoin, soit en
particulier, soit dans le conseil. Je n'ai que des grâces infinies à lui
rendre de ce que ses bontés ont seules excité tout cet effet d'ambition,
et de ce qu'elles sont demeurées invulnérables à toutes les étranges
machines conjurées et l'assemblées contre moi durant ma plus juste et ma
plus profonde confiance.

Quel que soit l'état des finances, que, jusqu'à ce mois-ci, Votre
Altesse Royale m'avait toujours assuré devoir sûrement prendre une bonne
consistance, je suis persuadé qu'il y a du remède, si on veut le
chercher avec docilité, et se départir de même de ce que l'expérience
montre avoir été mal commencé. Encore une fois, je le répète, je ne
prétends point, blâmer une administration dont je me suis senti
incapable, que je ne puis ni ne voudrais examiner, et dans laquelle je
me persuade qu'on a fait du mieux qu'{[}on{]} a pu. Mais sans tomber sur
une gestion inconnue, et raisonnant seulement sur l'effet de cette
gestion dans une matière que le feu roi a laissée dans un état
infiniment difficile et violent, je dis que la bonté des peuples de ce
royaume, et l'habitude du gouvernement monarchique, ne doit faire
chercher le remède qu'entre les mains de Votre Altesse Royale, et dans
les conseils des personnes intelligentes en cette matière qu'elle en
voudra consulter par elle-même, ou par ceux qui, sous elle, conduisent
les finances.

La difficulté consiste en la continuation de deux impôts extraordinaires
que l'autorité du feu roi et l'extrémité de ses affaires firent établir
l'un après l'autre sous le nom de capitation et de dixième, avec les
paroles les plus authentiques de les supprimer à la paix, et sans
lesquels nonobstant la paix et toute la diminution de dépense qui
résulte de la mort de nos premiers princes, et de l'âge du roi, le
courant ne peut se soutenir\,; et en ce que ces mêmes impôts sont
insupportables par leur nature et par leur poids à la plupart des
contribuables, réduits à l'impossibilité de payer.

Plusieurs questions se présentent à l'esprit tout à la fois sur le genre
du remède des états généraux, mais qui se réduisent à deux principales,
desquelles naîtront les subdivisions\,: 1° si on doit espérer le remède
par les états généraux\,; 2° si les états généraux ne produiront pas de
plus fâcheux embarras que ne sont ceux pour l'issue desquels on
réfléchit si on les assemblera.

Plût à Dieu, Monseigneur, que vous n'eussiez point été détourné de la
sainte et sage résolution que vous aviez si mûrement prise de les
indiquer à la mort du roi\,; c'est-à-dire dans la séance de la
déclaration de votre régence, pour en signer les lettres de convocation
le jour même, et les assembler deux mois après\,; deux autres mois de
prolongation pour donner plus de loisir aux choix et aux délibérations
des assemblées particulières pour la députation à la générale, et autres
deux mois pour la tenue des états généraux, n'auraient fait que six
mois, huit au plus, pendant quoi la finance eût roulé bien ou mal de
l'impulsion précédente, mais sans rien du vôtre. De dire, comme on le
fit avec trop de succès, qu'il fallait vivre en attendant, est-ce en
vérité, que, si le feu roi fût encore demeuré huit mois au monde, on
n'eût pas vécu ces huit mois\,? Les états généraux auraient trouvé tout
en entier à votre égard, et n'auraient eu ni excuse, ni désir d'excuse
de chercher et de proposer des remèdes à l'épuisement, charmés d'une
marque si prompte de l'honneur de votre confiance, et par cela même
prêts à tout sacrifier pour vous. Pardonnez ce mot à mes regrets, il ne
se trouvera pas inutile pour la suite.

À présent tout est entamé sur la finance\,: monnaies, taxes,
liquidations, suppressions, retranchements, billets de l'État,
conversions et décris de papiers, ordres de comptables. Il en est
résulté une diminution de dépenses par l'extinction d'un grand nombre de
capitaux en tout ou en partie, et de beaucoup d'arrérages accumulés, et
en outre il en doit être rentré de gros fonds extraordinaires dans les
coffres du roi. Tout cela néanmoins est insuffisant\,; et il n'est pas
malaisé d'en conclure qu'il en faut venir à frapper de plus grands
coups, dont la bonté de Votre Altesse Royale ne peut que difficilement
se résoudre à donner les ordres et que ceux qui par leurs emplois les
lui peuvent suggérer, et les doivent exécuter, craignent de prendre
l'événement sur eux.

Ceux-là sentent maintenant la faute qu'ils ont faite de vous avoir
détourné de la convocation des états généraux à la mort du roi. Ils
avaient compté sur des arrangements et des ressources qui leur ont
manqué, après avoir assuré Votre Altesse Royale que la finance se
rétablirait aisément en suite de certaines opérations nécessaires, et
l'en avoir persuadée par, leur propre confiance. Mais la principale de
ces opérations est celle qui cause le plus de désordre dans les
finances. Ce n'est point par l'avoir prévu, et m'y être constamment
opposé autant que le respect pour vous me l'a permis, que je fais ici
mention de la chambre de justice, mais parce que les suites en sont
telles qu'il n'est pas possible de n'en pas dire un mot. Je me garderai
bien de retoucher aucune des raisons que j'eus l'honneur de vous
représenter contre cet établissement, dès le premier moment que vous me
fîtes celui de m'en parler, et que j'ai pris la liberté de vous répéter
souvent. Mais en même temps qu'il était juste et nécessaire de punir les
excès des gens d'affaires d'une manière qui remplit les coffres du roi
au soulagement du peuple, ce qui est arrivé de l'interruption du
commerce était infiniment à craindre de la voie qui a été prise, et d'un
manque de confiance dont le remède est impossible tant que les suites en
seront subsistantes, et que les états généraux ne paraissent pas propres
à fournir.

En effet, bien que le tribunal de la chambre de justice ait terminé ses
séances, l'examen de ce qu'elle a laissé imparfait se continue chez M.
le duc de La Force. Il a eu peine à s'en charger sans un nombre de
personnes suffisantes pour expédier promptement les matières et pour
s'entr'éclaircir les uns les autres. Votre Altesse Royale avait
elle-même jugé sa demande si raisonnable qu'elle avait destiné un bureau
à ce travail. Mais d'autres raisons ont fait borner ce bureau à un seul
homme avec M. le duc de La Force, qui tous deux y suffiront à peine en
un an. Par cette lenteur un grand nombre de fortunes demeurent
suspendues\,; et tant qu'elles ne seront point assurées de leur état, et
par un cercle inévitable, beaucoup d'autres avec elles, il n'y a pas de
circulation à espérer. M. le duc de La Force court risque de partager la
haine des taxes avec les premiers auteurs par ce genre de travail tête à
tête\,; mais la confiance en de meure nécessairement arrêtée, et avec
elle, tout le mouvement de l'argent, et le salut de l'État pour ce qui
concerne les finances.

La seule chose qui les soulage, en remédiant aux désordres du change, et
en facilitant les payements, est l'établissement de la banque du sieur
Law, à laquelle j'avoue que j'ai été très contraire, et dont je vois le
succès avec une joie aussi sincère que si j'en avais été d'avis, encore
que je n'y aie voulu prendre aucun intérêt. Mais puisque ce soulagement
ne promet pas assez pour se passer d'autres remèdes, voyons enfin, après
tout cet exposé, ce qui se peut attendre d'une tenue d'états généraux.

Cette assemblée, infiniment respectable, et qui représente tout le corps
de la nation, forme un conseil très nombreux. Chaque député y est chargé
des plaintes et des griefs de son pays et de son état, dont il est
ordinairement plus instruit que des remèdes qu'il vient y demander au
roi. Chacun y sent son mal d'autant plus vivement que c'est de l'effet
de ce sentiment qu'il espère le soulagement qu'il est venu demander.
Avec les maux généraux il y en a beaucoup de particuliers qui suivent la
nature des productions et du genre de commerce de chaque province, et
encore la nature de chacun des trois ordres qui composent les états
généraux\,; et l'homme est fait de manière qu'il est bien plus touché de
son mal particulier que de celui qu'il souffre en commun avec tous les
autres, conséquemment porté à se reposer sur qui il appartiendra du
remède à ces maux généraux, et à n'agir vivement que sur ce qui en
particulier le regarde. C'est ce qu'il est à craindre de voir arriver
dans une assemblée tirée de tous les divers pays du royaume et des trois
ordres de chaque pays, que chacun n'y pense qu'à sa propre chose, sans
se mettre beaucoup en peine de la générale, ni de celle de son voisin,
sinon par rapport à la sienne, et que cet intérêt particulier ne
remplisse l'assemblée d'une foule de propositions de remèdes différents,
contradictoires les uns aux autres, sans qu'il en résulte rien qui ait
une application certaine au mal, général pour la guérison duquel elle
aura été convoquée. En ce cas, quelle confusion\,! et quel fruit de ces
états généraux\,?

Mais parmi ceux qui y seront députés, peut-on espérer qu'il s'y en
trouve de bien versés dans la science des finances, qui en aient fait
une étude suivie et principale, qui s'y soient perfectionnés par
l'expérience\,? Tous ceux de ce genre sont sûrement connus, et il n'est
pas besoin d'une telle assemblée pour les avoir sous sa main et pour les
consulter. Il est, au contraire, à présumer que, faisant un nombre, pour
ainsi dire, imperceptible parmi la foule des députés et parlant une
langue étrangère à la plupart, ils leur deviendront aisément suspects,
qu'ils en seront peut-être méprisés, et que leurs avis y deviendront au
moins inutiles. Or ce succès ne vaut pas une tenue des états généraux.

Que si l'on objecte que c'est être hardi que de penser qu'une telle
assemblée ne soit pas capable des bonnes raisons, et de goûter les bons
remèdes que quelques députés y pourront proposer, et de n'espérer pas de
cette foule un bon nombre de bonnes têtes remplies d'expédients de la
discussion desquels il se puisse tirer d'excellents remèdes, il est aisé
de répondre que tel est le malheur, non la faute, de la nation gouvernée
depuis tant d'années sans avoir presque le temps ni la liberté de
penser, que chacun a ses affaires domestiques, et encore avec les
entraves qui ne sont pas cessées depuis un assez long temps pour qu'on
ait pu les oublier. Il est difficile d'espérer qu'il se soit formé dans
ce long genre de gouvernement un assez grand nombre de gens pour
l'administration des affaires publiques à travers les périls attachés à
cette sorte d'application, d'où il ne se peut qu'il n'étincelle toujours
quelque chose, et dans le dégoût de l'inutilité qui s'y trouvait jointe.
Je dis donc, et à Dieu ne plaise que je, pense autrement de ma nation,
et d'une nation qui s'est toujours si fort distinguée parmi toutes les
autres en tout genre\,! je dis donc qu'elle abonde en esprit et en
talents, mais que cet esprit et ces talents ayant été si longuement
enfouis à l'égard de ce dont il s'agit maintenant, ce serait comme une
création subite, si on voyait le talent et l'art de l'administration, et
en chose si difficile, paraître en un nombre suffisant de députés pour
former avec succès des délibérations heureuses, et qui pussent remédier
aux maux généraux pour lesquels on les aurait assemblés\,; que c'est un
malheur, qu'on ne peut jamais assez déplorer, et qui ne peut être assez
fréquemment et assez fortement inculqué au roi, que d'avoir rendu
inutiles tant d'excellents esprits, qui font maintenant un si grand
besoin, par les avoir continuellement gouvernés sans aucune liberté
d'application, et d'avoir commis cette faute dans une nation unique
peut-être dans le monde, en théorie et en pratique, par sa fidélité, son
obéissance, son attachement, son amour pour sa patrie et pour ses rois.
Mais le mal est fait par une longue suite d'années écoulées sur le même
ton. Il ne se peut réparer que par un autre espace de temps où il soit
permis de s'instruire, de penser et de raisonner\,; et il s'agit
présentement que ce temps ne fait que commencer sous les heureux
auspices {[}de la régence{]} de toutes les régences la plus douce et la
moins contredite, de se servir de ce que la nation peut offrir, et non
de ce qu'on a ci-devant comme éteint en elle. Or, ce qui y sera toujours
subsistant est un fonds d'esprit, de pénétration, d'activité,
d'application, qui, ayant la liberté de germer dans les suites, produira
les fruits excellents que la conduite passée a rendus si rares, au grand
dommage de l'État, du roi, de Votre Altesse Royale, et en attendant
{[}produira{]} cette fidélité, cette obéissance, cet attachement, cet
amour du roi et de la patrie qu'on ne peut suffisamment exalter, et dont
Votre Altesse Royale peut faire de sages et d'excellents usages.

Par ces tristes raisons, mais si sensiblement vraies, il me paraît,
Monseigneur, qu'il n'y a point de remède à attendre des états généraux
pour les finances. Si vous appelez remèdes ces grands coups que vous ne
m'avez point encore confiés, mais qu'il est impossible, de ne pas
entrevoir dans la situation violente qui fait penser aux états généraux
ceux peut-être dont l'emploi les éloigne le plus, il est bien à craindre
que cette grande assemblée, essentiellement divisée d'intérêt, ne se
divise en troubles à cette occasion. En effet ce qui tombe le plus
aisément dans la pensée dès qu'il est question des grands coups, c'est
l'abolition, ou le retranchement peu différent, des rentes de la ville
et suivant le besoin des autres pareilles créées sur le roi. Sans que
Votre Altesse Royale sonde là-dessus les états généraux, ce qui serait
d'un danger infini pour elle, on peut se persuader que la proposition y
en sera faite par tous les députés de la campagne, et vivement
contredite par tous ceux des villes. Je m'exprime ainsi par rapport à
l'intérêt contradictoire de ces deux espèces de personnes, et j'entends
sans distinction d'ordres par députés de la campagne tous ceux des trois
ordres qui n'ont rien ou très peu sur le roi, et de même par ceux des
villes, ceux dont la principale fortune roule sur ces sortes de rentes.
De ce genre sont tous les magistrats de la haute et basse robe, et tout
ce qu'on peut nommer suppôts de justice, comme avocats, procureurs,
huissiers, payeurs des gages des compagnies, et avec eux tous les
bourgeois et gens dont le patrimoine n'est point en terres. De tous
ceux-là, qui sont en grand nombre, et qui par leur profession sont les
plus en état de bien parler et de se faire entendre, la ruine est
attachée à cette suppression. Les députés de la campagne, avec raison, y
croiront trouver leur salut, parce que cette immense diminution de
dépense, donnant lieu à une grande diminution de charges
extraordinaires, les soulagera beaucoup sans rien entamer de leur fonds
de biens qui, au contraire, profitera d'autant plus qu'ils se trouveront
plus en état de faire valoir leurs terres. À ce grand intérêt se joindra
la jalousie de ceux-ci contre les autres, qui a déjà sourdement paru en
bien des rencontres. Ils regardent comme le malheur et la ruine de
l'État ces établissements de biens factices qui, par la facilité de leur
perception, donnent occasion à un si grand nombre de personnes d'y
placer leur bien pour en vivre à l'ombre et dans le repos, aux dépens
des sueurs des gens de la campagne, dont presque tout le travail
retourne au roi par l'excès des impôts dont il a besoin pour suffire aux
rentes dont il s'est chargé, et qui par ce moyen met en sa main tout le
bien de son royaume ceux des terriens par ce qui vient d'être dit, ceux
des rentiers en ouvrant ou fermant la main comme il lui plaît.

D'un intérêt aussi pressant et aussi contradictoire que peut-on se
promettre qu'une division, dont le moindre mouvement sera de ne plus
trouver assez de tranquillité dans l'assemblée générale pour en espérer
les remèdes aux maux pour la cure desquels elle aura été convoquée\,?
division d'autant plus grande que les ordres mêmes se trouveront dans un
intérêt opposé. Le premier sera le moins désuni des trois sur ce point,
excepté un petit nombre d'ecclésiastiques riches de patrimoine, et dont
le patrimoine consistera pour la plus grande partie en rentes\,; tous
les autres ou nés pauvres ou cadets de famille, ne vivent que de leurs
bénéfices, c'est-à-dire des terres qui on font la consistance, et seront
pour la suppression ou le retranchement des rentes. Le second
{[}ordre{]} se portera avec rapidité au même avis. C'est de tous les
trois le plus opprimé, celui qui a le moins de ressources, le seul
néanmoins qui existât dans les temps reculés, celui qui a été
constamment la ressource de l'État, le salut de la patrie, la gloire des
rois, qui a mis sur le trône la branche régnante, et dont le zèle,
l'amour de la vertu, de la patrie, de ses légitimes souverains, n'a
point cessé, depuis la fondation de la monarchie jusqu'à maintenant,
d'être en exemple illustre à toutes les nations, et de soutenir la
sienne par les flots de son sang.

J'avoue, monseigneur, que j'ai besoin de me faire violence pour me
retenir sur la situation cruelle où le dernier gouvernement a réduit
l'ordre duquel je tire mon être et mon honneur. Votre Altesse Royale a
souvent été témoin de l'amour et du respect que je lui porte, et des
élans qui m'ont trop souvent échappé aux traitements qui lui ont été
faits. Réduit pour vivre à des alliances affligeantes, et à manger
bientôt après pour s'avancer ce que ces alliances avaient produit, peu
de cet ordre auront intérêt à soutenir les rentes\,; beaucoup moins le
voudront faire, liés par vertu à l'intérêt général\,; moins encore
l'oseront par rapport à tant d'autres qui, n'ayant point de cette sorte
de bien, tomberaient rudement sur ce petit nombre. Les terres et l'épée,
voilà tout le bien de la noblesse. Les rentes sont très opposées au bien
foncier\,; elles ne le sont pas moins à celui qui se peut acquérir par
la récompense des armes. Plus le roi a de rentes à payer, moins il a de
pensions et de grâces pécuniaires à répandre sur la noblesse qui sert,
qui ruine ses terres en servant, et y contracte nécessairement des
dettes qui transportent ses terres aux paisibles rentiers\,; et ces
rentiers, qui ne font aucune dépense de cour ni de guerre, profitent
doublement du sang de la noblesse, et par la conservation de leur
patrimoine, et par la ruine de ceux qui suivent les armes. On doit donc
compter que tout notre ordre sera contraire aux rentes, avec ce feu
français qui est si utile à la guerre, mais qu'il n'est pas à propos
d'allumer au milieu de la paix et de la régence.

Le troisième ordre sera d'un avis entièrement et tout aussi vivement
différent, si la bonne manière de juger de ce que feront les hommes, et
en choses de ce genre, se doit prendre par l'intérêt. Or l'intérêt de
cet ordre est double à maintenir les rentes\,: premièrement elles font
presque tout son bien, en total du plus grand nombre, en la plus grande
partie de beaucoup, en quelque partie au moins de tous. D'ailleurs tout
cet ordre est appliqué à des emplois, et tourné à un genre de vie qui ne
lui permet guère de changer de goût et de méthode sur la nature de son
bien. Ceux qui suivent l'administration de la justice et l'étude des
lois n'ont pas le loisir de se détourner à la régie de leurs biens
fonciers. La perception de leurs rentes ne les tire ni des tribunaux ni
de leur cabinet. Le commerce des charges entre eux en puise toute sa
facilité. L'augmentation de leur bien se fait de même d'une manière
aisée, et la commodité de le partager dans leur famille s'y trouve toute
pareille. Je ne parle point d'un petit nombre de cet ordre qui, portés
aux armes par une élévation de courage, et soutenus de beaucoup
d'application et de mérite, sont arrivés à faire honneur à la noblesse,
et quelques-uns même à la commander avec réputation et gloire pour eux
et pour l'État, ni d'un plus grand nombre de paresseux et libertins qui
se sont comme fondus ou dans les troupes ou dans l'oisiveté. Les
premiers, inscrits dans l'ordre de la noblesse par leur vertu, ne se
sépareront point de l'intérêt de ceux dont ils tirent tout leur lustre,
mais ce nombre est si petit qu'il n'est pas à compter\,; beaucoup moins
ces libertins, la plupart ignorés jusque dans leurs familles. Les
négociants se trouvent par leur état aussi attachés aux rentes\,; et
pour ce qui est des bourgeois proprement dits, gens vivant de leur bien,
presque tout est en rentes, et de ceux-là il n'y en a presque aucun qui
{[}ne{]} songe à élever sa famille par quelque charge. Voilà pour la
première raison.

La seconde n'est pas moins forte, parce que c'est celle de l'ambition.
Nul moyen à cet ordre de se mêler avec le second que l'abondance de l'un
et le malaise de l'autre\,; et comme de ce mélange résulte un honneur et
un avantage dont le troisième ordre est très jaloux, il est à présumer
qu'il ne s'en laissera pas aisément fermer la porte, beaucoup moins
celle que le dernier gouvernement lui a si largement ouverte, cette
domination que le riche a toujours sur le pauvre, de quelque extraction
qu'ils soient, et qu'il appuie par des emplois d'autorité où on n'arrive
que par les charges vénales, dont les prix sont excessifs par rapport à
leur revenu. Ces voies de s'égaler à la noblesse ne s'abandonneront pas
aisément, d'autant plus qu'elles se terminent à quelque chose de plus
fort, par le besoin continuel où la noblesse se trouve, depuis la plus
illustre jusqu'à la moindre, des biens et de la protection (car il en
faut dire le mot) des particuliers riches et en charge du troisième
ordre, dont il est presque tout entier composé. Ce n'est pas que je
pense que tout le troisième ordre soit riche\,; mais je dis que, à la
réserve d'un très petit nombre, tous sont considérables à la noblesse ou
par les biens ou par les emplois. En effet, pour un créancier du second
ordre, on en trouverait mille du troisième, et au contraire un débiteur
du troisième pour mille du second. À l'égard des charges, outre que le
nombre de celles de judicature, de plume et de finances, est infini,
c'est qu'il n'en est aucune qui n'ait une autorité et un pouvoir direct
ou indirect, qui ne souffre aucune comparaison avec quelque charge
militaire que ce soit dont la proportion puisse être faite.

Par ce court détail il paraît que presque tout le premier et le second
ordre seront très animés contre les rentes, et le troisième, au
contraire, très ardent et très attentif à les soutenir. De ce débat, qui
est fondé sur la destruction de la fortune des uns et des autres, on ne
peut attendre qu'aigreurs, cabales, animosités. Les \emph{mezzo-termine}
auront en ce genre, plus qu'en aucun autre, le sort d'amuser le tapis,
de nourrir les intrigues, d'aiguiser les haines, et de demeurer
inutiles. Aucun foncier ne voudra renoncer à une si belle occasion de se
délivrer de ce qui l'opprime. Aucun rentier ne donnera son fonds, ni
partie de son fonds, au bien public ni à l'avantage de la paix et de la
tranquillité. Dans ce contraste que fera Votre Altesse Royale entre le
clergé et la noblesse d'une part, et les parlements et autres cours, les
négociants, tout le tiers état de l'autre\,? Ce mal sera en sus de tous
les autres. N'est-il pas plus sage de le prévoir et de l'avoir de moins,
puisque, au lieu d'un remède que vous voulez demander, et que vous
voulez espérer des états généraux, non seulement vous n'en aurez point,
mais vous vous procurerez cette division de plus qui peut devenir très
embarrassante\,? Mais, après avoir examiné la chose par les ordres,
recherchons-la par les provinces. Cela n'apprendra pas beaucoup de
choses nouvelles, puisque les députations des provinces ne sauraient
être que des trois ordres\,; mais cette manière achèvera d'approfondir.

Je pense qu'on n'y trouvera que peu de différence. Les provinces
d'états\footnote{Voy., sur les provinces ou pays d'états, la note IV à
  la fin du volume.} seront partagées. Les unes voudront se continuer la
douceur de l'administration, les autres celle de la perception facile de
ces rentes créées sur les États\,; d'autres, qui n'en sentent que le
poids, et qui ont jalousie de l'autorité que cette gestion donne à ceux
qui l'ont en quelque degré que ce soit, désireront s'en affranchir.
Quelques gens voisins de Paris seront aussi pour les rentes\,; mais
toutes les provinces qui n'ont point d'États y seront très contraires,
et, comme elles sont en plus grand nombre, le parti des fonciers contre
les rentiers en sera d'autant plus fort. Ainsi, de quelque manière que
cette affaire puisse être considérée, on ne peut la regarder que comme
la pomme de discorde qui rendra la tenue des états généraux longue,
difficile, infructueuse pour l'objet qu'on s'en propose, et périlleuse
pour la division qui seule en résultera. En voilà suffisamment pour la
première partie, quant aux finances. Voyons si on s'en peut
raisonnablement promettre un meilleur succès par rapport à l'affaire des
princes.

Avant de mettre une affaire sur le tapis, il faudrait être bien d'accord
avec soi-même pour savoir précisément quelle issue on lui désire d'une
manière définitive. Par tout ce qui s'est passé (car je n'en puis juger
que par là, et Votre Altesse Royale me pardonnera bien si je le lui dis
avec franchise), il me paraît que l'événement lui en importe peu, pourvu
qu'il ne roule pas sur elle. Par politique vous voulez une balance\,;
par nature une indécision entre si proches, et c'est ce qui incruste
cette balance à vos yeux\,; par sentiment M\textsuperscript{me} la
duchesse d'Orléans d'une part, de l'autre M. votre fils et sa postérité,
vous tiennent en suspens\,; d'où il résulte que de votre choix les
choses en demeureraient où elles en sont, sans l'importunité d'une
poursuite qui vous paraît ardente et qui se renouvelle trop souvent à
votre gré. Je me garderai bien d'entrer dans aucun détail du fond de la
question pendante, ni de la manière dont elle a été jusqu'à présent
traitée par Votre Altesse Royale ni par les parties, moi-même j'en suis
une, et c'est pour moi une surabondance de raisons pour m'en taire\,;
mais il s'agit de savoir ce que vous prétendez en renvoyant la cause aux
états généraux, et si ce moyen est bon pour arriver à la fin que vous
vous proposez.

Vous n'en pouvez avoir que deux\,: 1° d'éviter tout jugement, pour
conserver cette balance entre les princes\,; 2° de vous décharger de la
haine de ce qui sera décidé. Mais si vous vous trompez dans l'une et
dans l'autre de ces vues, certainement vous ne devez pas déférer cette
affaire aux états généraux.

Portez-la-leur pour en attendre le jugement et l'avis\,? la chose est
égale. Si c'est en apparence pour en avoir le jugement, ne comptez ni
sur votre adresse ni sur votre autorité pour l'empêcher. Un tel
jugement, proposé à une pareille assemblée, ne lui échappera jamais.
C'est un monument trop important aux états généraux pour que rien
l'emporte auprès d'eux sur cette sorte de conquête, et après une
interruption si longue et si irritante, et dans un temps si affranchi.
La multitude ne craint point la haine que redoutent les particuliers\,;
et plus cette grande affaire a été présentée à différents juges, moins
toutes sortes de jugements ont paru compétents, et plus, encore une
fois, il sera du goût des états généraux de la décider nette et précise.
Si vous vous contentez d'une consultation simple, peut-être ne s'en
satisferont-ils pas\,; mais à tout le moins ils répondront à votre
consultation d'une manière claire et publique. Ainsi Monseigneur, au
lieu d'échapper par cette voie, vous verrez très certainement un
jugement rendu, ou un avis si décisif et si public qu'il ne vous restera
plus de refuites pour éviter de le tourner en jugement et de le
prononcer vous-même. Vous n'éviterez donc point un jugement aux états
généraux\,; et cette première vue vous la devez réputer fausse.

À l'égard de vous décharger de la haine du jugement, espérez-le aussi
peu que d'éviter le jugement même par le moyen des états généraux. Je ne
m'engagerai pas à détailler des personnes respectables\,; mais bien
dirai-je à Votre Altesse Royale que vous avez affaire à des yeux très
perçants, qui voient très bien que rien du dehors ni du dedans rie vous
engage à convoquer une pareille assemblée\,; conséquemment que, dès que
vous la convoquerez pour les juger, ou dès que le jugement s'ensuivra,
comme je crois l'avoir démontré, qui ne s'en prendront qu'à votre
volonté, laquelle, laissée à elle-même par la situation des choses, se
sera librement déterminée de son plein gré à ce parti, conséquemment à
vous de ce qui en résultera à l'égard de la question qui y est décidée.
Eh\,! que Votre Altesse Royale perde en ceci toute confiance aux
adresses, aux négociations, aux interpositions. Tout se mesurera par la
décision, et dans cette décision tout n'est qu'accessoire, hors un point
unique qui est celui de la question.

De la manière dont cette question sera déterminée, tout dépendra donc
pour vous, c'est-à-dire la haine certaine des uns, le gré médiocre des
autres, qui à travers tout pénétreront, se porteront, ne considéreront
que vous comme convocateur et moteur de l'assemblée\,: convocateur
certain et d'autant plus assuré que vous l'aurez fait en toute
liberté\,; moteur, personne n'en saurait répondre que le dépit de ceux
qui auront perdu leur procès\,; mais à l'égard de qui l'aura gagné, peu
de gré à vous, un médiocre à l'assemblée, beaucoup à la nature de leur
cause ou à celle de leurs établissements, non peut-être sans quelque
indignation de tant de circuits et de peines à se voir enfin au bout des
leurs. Au contraire, la haine et le dépit de qui l'aura perdu, n'osant
et ne pouvant mordre sur une telle assemblée avec laquelle il serait
trop imprudent de rompre toute mesure\,; tombera à plomb sur vous d'une
manière d'autant plus envenimée que la solennité du jugement en aura
infiniment augmenté la douleur et la confusion. Ainsi, Monseigneur,
comptez d'en recueillir une haine d'autant plus dangereuse que cette
voie de finir la question est plus solennelle et publique, conséquemment
plus pénétrante\,; que cette haine sera trop forte pour ne tomber sur
personne, que l'assemblée n'en est pas susceptible, que par les raisons
touchées, et par mille autres, vous êtes le seul à qui elle puisse
s'appliquer.

La double vue qui vous fait penser à porter l'affaire des princes aux
états généraux, ne pouvant que vous faire plus lourdement tomber dans ce
que vous voulez éviter et que vous attendiez de cette voie, la
conclusion n'est pas difficile que les réflexions de Votre Altesse
Royale doivent la porter à l'abandonner sur ce point. Or, celui des
finances n'en tirant aucun secours, et Votre Altesse Royale ne pensant à
une tenue d'états généraux que pour les finances essentiellement, et
subsidiairement pour l'affaire des princes, il me paraît qu'elle ne peut
être conseillée de les assembler. Mais ce n'est pas assez de vous les
avoir démontrés parfaitement inutiles pour les desseins que vous vous en
étiez proposés\,; il faut encore faire faire à Votre Altesse Royale
l'attention nécessaire sur les inconvénients qu'ils pourraient produire
à présent.

On ne peut les prévoir tous, et il est aisé qu'il en arrive de plus
grands que ceux dont on va parler, tant de la combinaison et de
l'entrelacement de ceux-là mêmes que des événements fortuits et de la
nature des choses. Le premier qui se présente à l'esprit est l'embarras
qui naîtra des compétences et des rangs qui seront respectivement
prétendus. On voit maintenant que ceux dont le droit est le plus
certain, et {[}que{]} l'usage le plus constant et le plus suivi devrait
avoir mis hors de toute contestation, deviennent chaque jour l'objet des
plus vives disputes\,; combien plus dans une assemblée aussi générale,
aussi longuement interrompue, dont toutes les relations qui nous restent
de celles qui ont été tenues sont laconiques sur cette matière, parce
qu'autrefois rien n'était mieux établi et observé que les rangs dans ces
grandes solennités, et que personne n'osait ni ne pensait à outrepasser
rien\,! Le temps présent semble tout permettre en ce genre, et le pis
aller d'une mauvaise cause est un \emph{mezzo-termine}, par lequel elle
gagne au moins, pour peu que ce soit, ce qu'elle n'avait pas. Ainsi on
doit s'attendre que les députés personnellement entre eux, que les
députations, au nom de leurs bailliages et de leurs gouvernements, que
les ordres mêmes, quelque décidé que soit celui des trois chambres entre
elles, tous formeront des contestations qui dureront longtemps, et tous
y seront si opiniâtres que Votre Altesse Royale en aura pour plusieurs
mois avant de pouvoir travailler à aucune autre affaire\,; que celle-là
deviendra très importante par les haines, la division, l'esprit de
contention, et que ce qui en résultera portera nécessairement sur toute
la tenue des états généraux. J'abrège cet article, qui pourrait être
prouvé et étendu à l'infini, mais qu'il suffit de présenter tout nu pour
en faire apercevoir, du premier coup d'oeil, toute l'importance à Votre
Altesse Royale, et lui donner à méditer sur ses dangereuses
conséquences.

Personne n'a une idée bien juste des états généraux. Le petit nombre de
ceux qui se sont appliqués à l'examen de la nature de ces assemblées et
de leur autorité, soit par une étude essentielle, soit par une étude
historique par rapport à elles, ne peut être regardé que comme un point
en comparaison de ceux qui en sont membres, dont la multitude n'écoutera
que l'intérêt de son autorité, et par conséquent portera ses prétentions
jusqu'où elles pourront aller. Après ce qui a été touché dans l'article
précédent à l'occasion des rangs, il n'est pas aisé de se flatter, pour
peu qu'on veuille raisonner sans prévention, que les états généraux s'en
tiennent aux simples remontrances, aux demandes, à ne délibérer que sur
les matières qui leur seront proposées par Votre Altesse Royale. Le nom
d'états généraux est d'autant plus grand qu'il n'a paru qu'en
éloignement depuis un grand nombre d'années, qu'il est accru dans
l'esprit du public par l'idée mal approfondie que ces assemblées ne se
sont tenues que dans les cas les plus importants, qu'elles ont toujours
été redoutées par les rois, d'où on infère que rien de grand ne se peut
sans elles et que par elles, et que leur autorité borne, balance, ajoute
à celle des rois. Le bruit qui se répandit, lors des traités depuis
conclus à Utrecht, qu'il s'en allait tenir, ce qui se dit et s'écrit
journellement à l'occasion de l'affaire des princes, grossit infiniment
ces idées, qui flatteront trop ceux qui les composeront pour devoir
s'attendre de leur part à une grande modestie dans un temps de minorité,
sous un prince dont on connaît maintenant avec étendue et par des
exemples la bonté, la facilité, le désir de plaire, sa peine de choquer
le nombre, et qui, étant le premier sous un roi de huit ans, ne laisse
pas de voir en Espagne une branche qui est son aînée et qui se multiplie
tous les ans. Les réflexions que cet article présente sont immenses en
nombre et en poids\,; c'est à vous, Monseigneur, à les faire, et toutes,
et à les pousser dans toute leur étendue. Vous n'êtes que le tuteur et
l'administrateur de l'autorité royale\,; vous aurez un jour à en rendre
un compte exact au jeune prince à qui vous la conservez comme
dépositaire\,; vous devez la lui remettre tout entière, les rois en sont
infiniment jaloux. Vous savez trop pour ignorer quelle est la différence
que mettra entre vous-même et vous-même le jour de la majorité\,; c'est
ce jour qui doit faire sans cesse l'objet de vos méditations. Elles sont
trop hautes pour qu'il m'appartienne autre chose que de vous les
représenter.

Mais, outre ce compte exact de l'autorité souveraine dont vous serez
comptable au roi en ce grand jour, vous l'êtes à vous-même au dedans et
au dehors, aux siècles futurs. Votre réputation dépendra tout entière de
la conduite que vous aurez tenue aux états généraux, et encore plus de
leur issue. Sur ce grand théâtre vous paraîtrez tout entier, et sans
qu'aucune partie de vous-même puisse être cachée à tant d'yeux perçants,
dont vous ferez l'objet et l'étude principale. Là, chacun apprendra à
vous craindre ou à ne vous rendre que de vains respects de rang, à vous
aimer, à aimer votre administration, ou à se lasser d'elle et de vous\,;
et ce dégoût est un malheur que celui des temps a souvent attiré aux
meilleurs princes, à ceux qui étaient le plus expressément nés pour
faire l'amour et les délices des hommes, et qui avaient le mieux
commencé. C'est donc en vain que de ce côté-là Votre Altesse Royale
s'appuierait sur la pureté de ses intentions, de ses desseins, de son
travail, sur son désir et son soin de plaire, ajouterai-je sur son
esprit et sur son industrie. Dans une situation aussi forcée qu'est
celle du royaume depuis tant d'années, on ne peut plaire qu'à mesure
qu'on soulage. Les promesses, les excuses, les espérances, jusqu'à
l'évidence de l'impossibilité, tout est également usé. On en est réduit
à ce point de ne vouloir plus se satisfaire que de réalités présentes et
effectives, parce qu'on est réduit à toute espèce d'impuissance qui, par
son genre de nécessité, passe par-dessus toute espèce de considération.
Les trois états sont presque également sous le pressoir (je dis presque,
car il est vrai que le second y est bien plus durement et on bien plus
de manières que les deux autres), ne crieront pas moins les hauts cris,
et leurs cris ne seront pas moins perçants. La noblesse, accoutumée de
tout temps à postposer tout à l'honneur, à tirer tout le sien de son
sang, et conséquemment à le verser avec prodigalité pour l'État et pour
ses rois, en est moins attachée aux biens, ainsi qu'il n'y paraît que
trop. Les deux autres ordres, dont la vertu et les dignités ne
s'acquièrent point par les armes, sont plus attentifs\,: le premier à un
bien dont il n'est que dépositaire et qui appartient aux autels\,; le
troisième à un patrimoine qui fait toute sa fortune, toute son
élévation, tout son établissement. Persuadez-vous donc, Monseigneur, que
vous ne plairez aux états qu'autant que vous leur donnerez un
soulagement actuel, présent, effectif, solide et proportionné à leurs
besoins et à leur attente. C'est cette juste attente qui a amorti
généralement partout la douleur de la perte du roi.

Vous l'avez promis solennellement et à diverses reprises, depuis que
vous tenez les rênes du gouvernement, ce soulagement si nécessaire et si
désiré. Jusqu'ici, c'est-à-dire depuis vingt mois, nul effet ne s'en est
suivi\,; et il ne faut pas vous le taire, tout a été levé avec plus
d'exactitude et de dureté que sous le dernier gouvernement, jusque-là
que chacun s'en plaint, et avec une comparaison amère. Les provinces en
retentissent. Le temps des états deviendra-t-il enfin celui du
soulagement\,? Vous qui voyez avec tant de pénétration, espérez-vous le
pouvoir donner tel qu'il plaise\,? et si la situation des finances ne le
permet pas, croyez-vous pouvoir empêcher les états de le prendre aux
dépens de ce qui en pourra arriver\,? et combien la lutte, s'il en
naissait une entre Votre Altesse Royale et eux, serait-elle pénible et
douloureuse, et quelles en pourraient être les suites dedans et
dehors\,!

Ce serait vous abuser d'une manière aussi dangereuse que facile
d'espérer contenter en donnant peu et promettant davantage. Je le
répète, et Votre Altesse Royale ne peut trop se persuader cette vérité,
les promesses sont usées, et les vôtres comme toutes les précédentes.
Vous en avez fait de publiques, par des lettres rendues telles par votre
ordre aux intendants à l'entrée de votre régence, et vous n'avez pu les
exécuter. Le haussement des monnaies, que je crois avoir été très
nécessaire, mais dont on devait avoir prévu la nécessité de plus loin,
a, au même temps, suivi de trois semaines une déclaration solennelle qui
assurait le public qu'elles ne seraient point augmentées. Je passe sous
silence d'autres occasions qui, pour n'avoir pas regardé
l'administration générale, n'en ont pas été moins publiques. Concluez de
toutes que rien ne sera agréable ni admis que des soulagements présents,
effectifs, certains, durables par leur nature et leur forme, et que
toutes ces différentes qualités, qui n'y seront pas moins requises que
les soulagements mêmes, ajouteront des embarras infinis à la nature de
la chose, déjà de soi si difficile. De croire après l'issue des états
sortir comme on pourrait des engagements pris avec eux, c'est-à-dire
n'en tenir que le possible, ce serait se précipiter dans les plus
dangereuses confusions, donner lieu aux tumultes, aux refus appuyés du
nom des états, à les voir {[}se{]} rassembler d'eux-mêmes d'une manière
dont l'autorité royale ne pourrait souffrir sans y trop laisser du sien,
ni peut-être l'empêcher sans de grands désordres, {[}sans{]} rompre à
jamais toute confiance avec les trois ordres et avec chacun de ce qui
les compose, et signaler un manquement de foi qui serait un exemple à
toute l'Europe, à profit certain contre vous et contre la France à tous
vos ennemis et à tous les siens, en un mot {[}sans{]} vous diviser de
l'État et de la nation, {[}ce{]} qui serait le comble des plus
irrémédiables malheurs, dont on ne peut trop méditer et craindre les
suites funestes, qui dureraient non seulement autant que votre régence
mais que votre vie, par la juste indignation du roi et de la nation
même. Ce serait encore ici un vaste champ à s'étendre, mais la matière
en est trop triste et trop palpable pour s'y arrêter plus longtemps.

5° Considérez donc bien attentivement, Monseigneur, de ne rien promettre
aux états, soit pour la chose, soit pour la manière que ce que vous
serez en état et en volonté de tenir avec une fidélité exacte et
précise\,; et considérez avec la même application si vous serez en état
et en volonté de leur accorder et tenir ainsi toutes les demandes, même
justes, qu'ils vous pourront faire pour leur soulagement. Pour faire
cette méditation avec fruit, portez d'abord votre vue sur vous-même, et
ensuite sur eux. Sur vous-même, examinez bien si votre bonté naturelle,
votre désir d'accorder et de plaire, la facilité qui en résulte, et le
sérieux qu'imprime toute la nation assemblée, laissera assez de fermeté
en vous pour ne vous point détourner, à leurs demandes, du discernement
mûr que vous aurez fait de ce que vous pourrez et de ce que vous ne
pourrez pas, et pour vous soutenir dans les pas glissants qui se
présenteront souvent. Ne craignez-vous point que, pressé dans ces
moments critiques par le poids du nombre, par l'évidence de injustice,
par l'adresse, la louange, l'espérance, semées dans un beau et solide
discours, par la majesté du spectacle, vous ne puissiez résister à tant
de forces, et que votre imagination, trouvant alors possible ce que vous
aviez bien connu ne l'être pas auparavant, vous ne veniez à accorder ce
que vous aviez résolu de refuser\,; que si vous ne l'accordez pas tout à
la fois, vous ne vous serviez de termes dont la douceur sera tournée
après d'une manière équivoque, qui produira des discussions fâcheuses
auxquelles vous succomberez par les mêmes voies qui les auront
produites\,; enfin que vous ne fassiez souvent par impulsion subite ce
que vous auriez bien résolu de ne faire pas. Alors par où se relever de
ces sortes de chutes dont le principe est excellent, mais dont les
suites peuvent devenir grandes\,? et permettez-moi d'aller plus loin. Je
ne vous rappellerai point les choses\,; je ne ferai que vous les
indiquer. Comparez les états avec l'assemblée du clergé qui était lors
de la mort du roi, et avec une autre assemblée continuelle (le
parlement), qui ne peut avoir de proportion avec celle des états
généraux. Souvenez-vous-en vous-même, et de ce qui s'est passé à leur
égard, et voyez si vous devez espérer de vous-même que l'assemblée de la
nation vous imposera moins que n'ont fait ces deux assemblées
particulières, toutes deux séparément l'une de l'autre.

Sur les états, examinez-en bien la multitude des membres, et que tout y
passe, non au poids des voix, mais à leur pluralité. Or, sans manquer à
l'amour, au respect, ni à l'estime que j'ai pour ma nation, je crois
qu'il serait bien téméraire d'avancer que, après une interruption si
longue de ces sortes d'assemblées, qu'à la suite de tant d'années où il
était si inutile, si difficile, si dangereux même d'être et de paraître
instruit, le plus grand nombre sera le plus mesuré en demandes, et bien
capable des raisons qui se pourront représenter là-dessus. Non,
Monseigneur, le besoin extrême, le désir pareil, la justice du
soulagement, le manque absolu de confiance régleront le fond et la forme
pour les demandes, et c'est vouloir s'abuser que s'attendre à mieux.
Votre Altesse Royale trouvera une foule de gens qui, dans le désir de se
distinguer, lui promettront merveilles de leur crédit dans l'assemblée.
Souvent elle les en payera d'avance, qui n'est pas un léger inconvénient
en soi, et pour l'exemple et les suites, et ces merveilles s'en iront en
fumée, ou parce que ces entremetteurs n'y auront pas le crédit dont ils
auront fait parade, ou parce que, contents du fruit personnel qu'ils en
auront tiré de vous avant l'effet de leurs promesses, ils ne se voudront
pas commettre à l'exécution, ou parce qu'eux-mêmes ne chercheront qu'à
embarrasser les affaires pour avoir le brillant des entremises, un éclat
de confiance et de crédit, et un moyen de se faire valoir aux états et à
vous, comme il n'est pas que Votre Altesse Royale n'ait éprouvé de ces
sortes de conduites en d'autres choses. L'issue de ces embarras n'est
pas aisée à trouver, et il n'est pas facile de prévoir jusqu'à quel
point ils peuvent conduire. C'est néanmoins ce qui mérite la plus
sérieuse méditation.

6° Mais, outre le point capital du soulagement des peuples qui mettra
tout le royaume du côté des états, sans peser ce qui est ou ce qui n'est
pas possible, qui peut s'assurer du nombre et de la nature des
propositions qui seront mises par eux sur le tapis\,? Plus la situation
présente est violente, plus les remèdes sont difficiles, plus l'excuse
en porte sur le gouvernement passé, plus les états se sentiront pressés
de chercher des moyens solides d'en empêcher les retours, et par ce
désir si naturel, si juste, même s'il était de leur ressort, plus ils
essayeront de s'en donner l'autorité. Or, qui peut imaginer, d'une
manière à peu près précise, quels seront ces moyens qui pourront être
proposés\,? Tout ce qu'on en peut prévoir est qu'il n'y en a aucun de
possible qui ne porte à plomb sur l'autorité royale, et qui ne soit mis
en avant pour lui servir de frein.

C'est au prince qui exerce cette autorité d'une manière précaire et
comptable, et qui est né moins éloigné de la couronne que son bisaïeul
qui y est parvenu, à discuter avec soi-même s'il lui convient de
s'embarquer sur une mer si orageuse et si pleine d'écueils de toutes les
sortes, et à se jeter dans la nécessité d'irriter les états en refusant
toutes les propositions de cette nature qui lui seront faites, ou à suer
longtemps parmi les angoisses des négociations pour en diminuer le
nombre et en rendre la forme plus tolérable, avec la majorité et le
compte à rendre de l'autorité royale en perspective, ou, à ce qu'à Dieu
ne plaise\,! la couronne même, que les états se croiront en droit et en
force de faire tomber à ses aînés ou à lui, suivant la satisfaction
qu'ils en auraient eue en leur assemblée et en ce qui en aurait suivi la
tenue. Quelque heureuses que fussent ces négociations, que Votre Altesse
Royale se persuade que les propositions les plus tolérables écorneront
beaucoup le pouvoir des rois, et que, si par les événements elles
cessent d'avoir tout leur effet dans la suite, votre réputation ne
laissera pas d'y demeurer tout entière, sans que le gré, partagé dans la
multitude, vous soit d'aucune consolation contre le mauvais gré que le
roi aura lieu de vous en savoir, ou, à ce qu'à Dieu ne plaise qui
arrive\,! contre le joug d'autant plus pesant et plus embarrassant que
vous vous le serez laissé imposer à vous-même. Mais il y a une autre
considération à faire, et qui ne peut être assez pesée c'est qu'en cette
sorte d'affaires il n'y aurait pour les états que la première de
difficile. Une première proposition, comme que ce soit admise, serait
bientôt suivie d'une seconde, par le refus de laquelle il ne faudrait
pas perdre l'amour et la confiance acquise par la première concession\,;
de là une troisième\,; et votre politique et naturelle bonté, et
l'ardeur et la fécondité des états s'accroissant mutuellement, les
bornes deviendraient bien difficiles.

Et que Votre Altesse Royale se garde bien de tirer les conseils, et ce
qui s'y passe, en exemple pour les états. Nulle proportion, nul
raisonnement, nulle conséquence à tirer des premiers pour les seconds.
Les conseils, vous les avez établis. Quoique très nombreux, ce n'est
qu'un point par rapport à la multitude des députés aux états généraux,
qui ne vous auront point une obligation personnelle de leur députation,
au moins pour le grand nombre, quoi que vous puissiez faire lors de
leurs élections, comme l'ont tous ceux qui de votre seul choix tiennent
des places honorables et permanentes, mais seulement honorables autant
que vos bontés et votre confiance, en quelque degré que ce soit, y est
jointe, et permanente autant qu'il vous plaît\,; tous gens nés ou venus
à la cour, et dont les emplois militaires ou civils ont ployé les
manières à un respect et à une crainte de déplaire, qui pourra être
aussi dans les états, mais différemment tournée, et qui y aura pour
contre-poids l'appui mutuel, le zèle du patrimoine et de la liberté, le
motif de se signaler pour son pays et de se faire un nom, celui du bien
public, prétexte dans les uns, objet réel dans le plus grand nombre,
mais objet d'autant plus dangereux qu'il est à craindre qu'il ne soit
pas bien pris dans l'idée même sincère de ce plus grand nombre, et qu'il
ne soif bien difficile de vaincre sa défiance sur ce point par des
raisons qui le touchent. Alors les plus capables, ceux qui
raisonneraient le plus juste, et qui tempéreraient le mieux par leurs
sages réflexions l'esprit zélateur de l'assemblée, craindront de se
commettre avec elle, et sans réussir d'y laisser trop du leur. Leurs
maux passés et présents sont un aiguillon pressant qui, se joignant à
celui de la liberté maintenant si, à la mode, ou encore à celui de
l'autorité que chacun s'arroge, qui n'y devient pas moins, et qui dans
une pareille assemblée sera dans toute sa force, et n'y sera contredit
d'aucun ou de bien peu de membres\,; la considération puissante, qu'ils
auront toujours devant les yeux, que l'occasion passée, tout
affranchissement est sans retour\,; toutes ces choses feront parler haut
les états, dont aucune ne se trouve dans les conseils, qui se laissent
aisément et doucement conduire à ceux qui leur président, et plus encore
à Votre Altesse Royale, dans les yeux de laquelle sont souvent leurs
avis, par une habitude de dépendance, augmentée par le respect pour sa
personne, et par la conviction de la justesse de ses sentiments et de la
pureté de ses intentions. Là personne n'a de nom à se faire, de liberté
ni d'autorité à acquérir, de foule où se dérober, ni, pour ainsi dire,
la nation en croupe pour asile. Il ne s'y agit que de voir les affaires
qui y sont portées, point du tout de s'en former, ni de proposer des
plans, des réformations, des prétentions. Tous, et chacun de ceux qui
les composent, ne peuvent tirer de considération que de la portion de
l'autorité royale que l'emploi qu'ils tiennent de vous leur donne à
exercer\,; et messieurs de la régence, devant qui les affaires discutées
ailleurs se rapportent, et qui en ont la voix définitive, n'exercent
eux-mêmes aucune portion de l'autorité royale, mais opinent seulement de
quelle manière ils croient qu'elle doit être employée sur chaque
affaire, sans en avoir l'exécution. Rien n'est donc en tout genre si
dissemblable que les conseils et les états\,; et ce serait se perdre que
de raisonner et de conclure des uns par les autres.

7° Deux moyens sautent aux yeux pour couper la racine à ces propositions
fâcheuses\,: le premier d'empêcher les états d'en mettre aucune sur le
tapis, et de les réduire à la seule délibération de ce qui leur sera
donné à discuter par Votre Altesse Royale\,; l'autre de refuser si
fermement la première proposition qu'ils oseront vous porter, que cette
conduite les empêche de s'y commettre une seconde fois. Rien, en effet,
de si aisé à penser, mais rien aussi de plus difficile dans l'exécution,
et de plus pernicieux dans la pratique. Assembler les états généraux
après une interruption si longue, dans une minorité, au commencement
d'une régence, non d'une mère, mais d'un prince cadet de la branche
d'Espagne, au milieu d'une profonde paix, pour les consulter sur l'état
fâcheux des finances, après y avoir inutilement essayé vingt mois et
plus toute espèce de remède, et ne leur permettre pas de rien proposer
d'eux-mêmes, c'est une contradiction dont l'évidence frappe, et
frapperait encore plus les états, contre qui elle porterait tout
entière, et avec une indécence qui les blesserait vivement et justement.
Nous ne sommes point en Angleterre, et Dieu garde un tuteur et un
conservateur de l'autorité royale en titre aussi éclairé que l'est Votre
Altesse Royale, de donner occasion aux usages de ce royaume voisin, dont
nos rois se sont affranchis depuis bien des siècles, et dont le nôtre
vous redemanderait un grand compte\,! Nulle nécessité des états pour
obtenir des secours des peuples de France\,; le roi y pourvait lui seul
par ses édits et déclarations enregistrés. Il ne pourra donc s'y en agir
aux états, mais bien et principalement des remèdes pour les finances. Si
leur difficulté a mis à bout vos lumières soutenues de tout votre
pouvoir, après tant de moyens tentés, il est clair qu'on n'assemble les
états que pour consulter un plus grand nombre de personnes éclairées et
intéressées en cette matière, dont vous n'auriez pas en besoin si vous
aviez pu trouver des solutions par vous-même\,; par conséquent qu'il
doit être moins question de leur en proposer là-dessus que de leur
exposer l'état des affaires pour en recevoir leur avis après qu'ils en
auront délibéré. Or quoi de plus contradictoire à cela que les empêcher
de rien proposer\,? Quoi même de plus illusoire\,? qualité dans, les
affaires a constamment été l'écueil fatal de presque toutes les tenues
d'états généraux. Et quoi encore de plus injurieux que de refuser si
fermement la première proposition qui vous sera faite par eux qu'ils
n'osent plus se commettre à vous en faire aucune\,? Ce moyen est bien
plus propre à en faire naître d'étranges, et à roidir les états contre
tout ce qui viendrait de Votre Altesse Royale, qu'à les lui soumettre.
Ils se lasseront moins des refus que vous de refuser\,; et si après un
premier refus commencé vous vous laissiez entamer, où ne pourrait-il pas
vous mener\,? Ce serait alors qu'irrités du refus, sans être apaisés par
ce qui leur aurait été accordé, fiers de la conquête qu'ils croiraient
ne devoir qu'à eux-mêmes, ils en essayeraient d'autres avec plus de
chaleur, dont le refus et l'acquiescement auraient d'égaux dangers, et
qui commenceraient la funeste lutte que j'ai touchée plus haut, sans
qu'on en pût prévoir les suites. Concluez donc de cet article,
Monseigneur, que vous ne pouvez employer sagement les deux moyens qui le
forment pour empêcher les propositions des états, comme vous devez avoir
conclu de l'article précédent que les états en feront, sans qu'il soit
possible d'en prévoir la nature ni le nombre, mais qu'il n'y en peut
avoir aucune qui ne porte coup sur l'autorité royale.

8° J'ai eu l'honneur de vous observer, dès l'entrée de ce mémoire,
qu'après tout ce qui a été tenté de différents remèdes sur la finance,
Votre Altesse Royale résolue, puis détournée à mon cuisant regret, de
convoquer les états généraux au moment de la déclaration de votre
régence, ne peut revenir à cette pensée que par la nécessité de frapper
de grands coups, par la peine que sa bonté et son équité en ressentent,
et ceux qui sous elle gèrent les finances pour éviter d'en prendre les
événements sur eux. Je le répéterai ici sans répugnance, Votre Altesse
Royale ne m'a point fait l'honneur de me rien faire entendre sur la
nature de ces grands coups, ainsi je n'en puis raisonner qu'en général,
et trois mots suffiront à cet article.

Souvenez-vous de ce que je vous ai représenté, dans la première partie
de ce mémoire, sur la suppression ou la diminution des rentes sur le
roi. Considérez que la nature des choses est telle que, malgré vous,
tous les remèdes que vous avez employés sont très durs, et par
conséquent très peu propres à vous avoir bien disposé une assemblée
aussi grande, et qui ne souffre pas moins de votre administration, pour
ne rien dire de plus, que de celle qui l'ont précédée, malgré toutes les
grandes et justes espérances conçues. Pesez avec tout ce que vous avez
de pénétration s'il n'y a rien à craindre ni apparent, ce dernier terme
n'est point trop fort, que la proposition que vous ferez de ces grands
coups aux états n'y soit mal prise et refusée, ou par des instances et
des supplications ardentes, fortes, réitérées, ou d'une manière encore
plus fâcheuse\,; et en ce cas méditez infiniment quelles en peuvent être
les suites au dedans et au dehors\,: l'affaiblissement de l'autorité
royale entre vos mains, l'accroissement de vos embarras sur les
finances, des difficultés sur toutes sortes d'affaires et de matières,
la manifestation authentique d'impuissance et d'épuisement, sans y faire
voir à côté aucun remède. Le nombre des paroles ne ferait qu'énerver
cette expression, que Votre Altesse Royale est plus capable
d'approfondir que personne. Son intérêt y est tout entier\,; elle ne
trouverait pas les mêmes ressources qui en peuvent attendre d'autres.

9° La bonne opinion qu'on doit avoir de tout le monde me persuade
aisément que personne ne désire des cabales, ni moins encore des
troubles. Ceux néanmoins qui, après de tranquilles commencements, ont
agité toutes les régences, et qui ont donné lieu à la fixation de la
majorité de nos rois à quatorze ans, puis à quatorze ans commencés, loi
dont la louange se perpétue par l'expérience constante, ces troubles,
dis-je, doivent être prévus. Dans la situation présente du royaume il
serait assez difficile d'en exciter. Rien n'y est ensemble, rien
d'organisé. L'embarras serait à qui s'adresser dans cette pernicieuse
vue. Le dernier règne en a comme arraché toutes les racines, et il ait
bien important de ne les pas voir renaître. Mais lorsque toute la nation
serait assemblée en états généraux, on conçoit aisément que les
assemblées nécessaires des divers membres dans chaque province pour
faire l'instruction et la députation à l'assemblée générale, que la
relation indispensable de ces députations à leurs provinces et des
provinces à eux, que celle de tous les députés aux états généraux les
uns avec les autres durant la tenue, forment des liaisons, découvrent
les gens qui, par le crédit qu'ils y acquièrent, peuvent devenir ceux à
qui s'adresser, et qui, pour conserver leur considération, peuvent
succomber à des tentations qui, dans l'organisement qu'on ne peut éviter
qui ne résulte entre les provinces, et dans chacune d'elles, après la
tenue des états généraux, peuvent devenir dangereuses au royaume,
tristes à Votre. Altesse Royale, et fâcheuses à l'autorité royale. Ce
dernier article mérite toutes vos réflexions, et a peut-être autant ou
plus de poids qu'aucun des autres qui l'ont précédé en ordre.

10° Avant de quitter la considération des états généraux pris en entier
pour venir au particulier des ordres qui les composent, il faut dire
quelque chose de l'affaire des princes qui en regarde le gros, et qui
reviendra après avec le détail.

Le dernier écrit abrégé, ou par réflexions signé de M. le Duc et de M.
le prince de Conti, dit tout à cet égard à Votre Altesse Royale. Encore
une fois, je n'entre point par ce mémoire dans la question, je me
souviens trop que j'y suis partie pour n'y faire pas une entière
abstraction d'intérêt particulier\,; mais ceci regarde la matière du
mémoire\,: c'est à cela seul que j'ose rappeler votre attention. Les
princes du sang vous disent qu'il ne faut pas une force différente, pour
détruire, de celle dont il a été besoin pour édifier\,; que le feu roi a
donné par des édits et des déclarations émanées de lui seul, et ensuite
solennellement enregistrées, ce qui est maintenant en contestation\,;
que c'est au roi à juger de la justice de ce qui est respectivement
prétendu, et d'autant plus au roi qu'il s'agit de laisser subsister ou
de casser un effet de la puissance royale dont nul autre que le roi
n'est compétent\,; que la minorité empêchant le roi de décider par
lui-même, c'est au dépositaire d'une autorité qui ne connaît en France
que la maturité de l'âge, et qui n'est sujette à aucun affaiblissement,
à juger pour le roi, ou à nommer des juges qu'ils offrent de
reconnaître\,; que ces juges nommés par Votre Altesse Royale, quels
qu'ils soient, exerceront en ce point l'autorité royale\,; et semblables
à la vraie mère du jugement de Salomon, qui aime mieux donner son fils à
l'étrangère que d'en souffrir le partage, ces enfants de la couronne
insistent à être jugés par l'autorité seule de celui qui la porte.

C'est à Votre Altesse Royale à peser les grandes suites d'un tel procès
déféré par un régent à des états généraux. Est-ce que le roi mineur n'a
pas le même pouvoir que le roi majeur\,? Mais en Angleterre où les rois
ont un pouvoir si limité en comparaison des nôtres, on a vu des
échafauds dressés sur cette question, et des tètes coupées pour avoir
contesté cette maxime d'égalité de pouvoir à tout âge, qui y a passé
jusqu'en ce jour en loi, et qui, en France, n'a jamais été disputée.
Cette déférence aux états ne peut donc rouler que sur leur supériorité
de puissance à celle des rois en ces matières, et alors, Monseigneur, où
en êtes-vous et que faites-vous\,? Que si c'est seulement une
consultation plus étendue que vous désirez, pensez-vous qu'un jugement
de cette importance échappe aux états, comme je vous l'ai représenté à
la fin de la première partie de ce mémoire, et que cette consultation à
tout le moins ne passe pas pour un point de droit en ces matières, qui y
met dès lors l'autorité des états au-dessus de celle du roi même. Or, si
elle y est reconnue supérieure en quelque point que ce soit, où la
bornerez-vous dans le reste, et quel frein lui pourrez-vous donner
durant la tenue des états, à l'âge du roi et dans la situation
personnelle où vous êtes\,? Quelles partialités ne feront point les
princes mécontents dans les états\,? Quelles autres la constitution n'y
excitera-t-elle pas\,? Mais ces matières appartiennent à la
considération des états prise en particulier. C'est à Votre Altesse
Royale à faire à ce dixième article toute l'attention qu'il mérite, et à
moi à passer au détail de la considération des trois ordres qui
composent les états généraux.

Le premier des trois est maintenant dans une agitation si grande à
l'occasion de la constitution \emph{Unigenitus}, qu'il est bien à
craindre que ce mouvement d'ébullition ne s'étende aux matières
temporelles dont il sera traité dans l'assemblée des états, et que
beaucoup de ceux de cet ordre ne s'y conduisent par rapport aux préjugés
et aux intérêts de sentiment où ils sont sur la huile. On ne peut jamais
s'assurer jusqu'où porte l'esprit de contention lorsqu'il est poussé au
point où on le voit sur cette matière, ni si ce grand nombre de prélats
et d'autres ecclésiastiques se trouvant ensemble ne voudraient pas se
tourner en manière de concile national, et commencer par cette affaire
avant de traiter d'aucune autre. Vous savez, Monseigneur, à quel point
M. le cardinal de Bissy le désire\,; vous êtes instruit des sentiments
de ceux que ces mouvements ont fait connaître sous le nom de
Sulpiciens\,; vous n'ignorez pas la division qui commence à se glisser
entre le premier et le second ordre de ce premier ordre de
l'État\footnote{Cette phrase, qui a été changée dans les éditions
  précédentes, s'entend parfaitement\,: Saint-Simon distingue dans le
  clergé, premier ordre de l'État, deux partis celui du haut clergé
  (cardinaux, archevêques, évêques), et celui du clergé inférieur. La
  phrase avait été ainsi modifiée par les anciens éditeurs\,: «\,La
  division qui commence à se glisser entre le premier et le second ordre
  et \emph{quant à} ce premier ordre, combien, etc.\,» Ils avaient
  supposé que Saint-Simon voulait parler des divisions entre le clergé
  et la noblesse, et nuit de scission dans l'ordre même du clergé.}\,;
combien l'esprit d'indépendance s'y introduit, et vous en serez encore
plus convaincu, si vous vous faites rendre compte de l'écrit qui vient
de paraître sous le titre de \emph{Réponse au mémoire} qui vous a été
présenté par plusieurs cardinaux, archevêques et évêques. Des prélats,
touchés par les deux points les plus sensibles à des gens de leur
profession, l'autorité et la doctrine\,; liés depuis longtemps par la
nécessité de l'affaire, et dont fort peu ont des familles qui les
retiennent\,; d'ailleurs appuyés de Rome et de cette clameur \emph{à
l'hérésie}, si bienséante dans la bouche des évêques lorsqu'elle est
fondée, et qui devient maintenant si à la mode sur la question présente,
ces prélats, dis-je, seront puissamment tentés d'user de l'occasion. Il
vient d'échapper à M. le cardinal de Bissy, dans la douleur du dernier
arrêt rendu contre M. l'archevêque de Reims, qu'il se fallait unir à la
noblesse\,; et à M. de Nîmes, qu'il n'y a qu'un mot à dire et une chose
à faire\,: \emph{anathème}, et rompre de communion. Dans ces
dispositions, qui peut vous assurer que les députés de cet ordre
n'auront pas une double procuration dans leur poche, et qu'ils ne
commencent par en tirer celle qui les autorise pour le concile
national\,? Je sais combien elle serait informe, en ce que votre
autorité n'y aurait pas donné lieu. Je suis également instruit de toutes
les répugnances de Rome à cet égard\,; mais ces répugnances n'ont point
jusqu'à présent retenu tous ceux qui lui sont les plus attachés. Eh\,!
qui sait si ce que le pape a refusé si opiniâtrement du temps du feu
roi, par l'autorité duquel il espérait de tout emporter de haute lutte,
il ne le désirerait pas maintenant par l'expérience qu'il a acquise
depuis sur cette affaire, pourvu qu'il n'y parût pas, et qu'au fond il
se pût assurer du succès du concile. Pour le manque de forme et de
pouvoir, parce que vous ne l'auriez ni convoqué ni permis il s'y
trouverait tout entier, mais votre embarras n'en serait pas moins grand
à ce coup imprévu entre refuser un si grand nombre, et en chose si
sensible et si prétextée de la couleur de la religion, et par ce refus,
d'indisposer de la manière la plus certaine et la plus forte une telle
quantité de membres et des plus principaux du premier ordre avec
lesquels vous auriez incontinent à compter, et dans cette première
chaleur aux états généraux, ou accorder par une brèche si hors de tout
exemple à l'autorité royale un concile ainsi frauduleusement convoqué et
assemblé tout à coup, si justement suspect, pour ne pas dire odieux à
tout l'autre parti, d'une si médiocre canonicité, et qui, outre la
longueur et cependant la suspension des états tous assemblées, pourrait
avoir de si grandes suites, dans lesquelles toute cette multitude de
membres des deux autres ordres prendrait sûrement plus de part que vous
ne voudriez. Il est inutile d'allonger la dissertation sur les
inconvénients et très aisément les troubles qu'on en verrait naître. Il
suffit d'en avoir montré la possibilité à Votre Altesse Royale, pour que
toutes les suites lui en deviennent présentes.

Mais, sans pousser les choses si loin, sans concile peut-on espérer que
le premier ordre, ainsi assemblé, n'en profite pas tout d'abord pour
cette matière de la constitution qui se trouve maintenant de plus en
plus échauffée. Chacun y voudra faire un personnage et y être compté
dans l'un et dans l'autre parti\,: les évêques en plus grand nombre pour
Rome, les autres députés presque tous contre, aigris de part et d'autre
sur le point qui commence à paraître sur la scène, et que les prélats
traitent de sentiments presbytériens. Quelle division dans un corps qui
doit l'arrêter dans les autres par son exemple et par ses instructions,
et quelle part tout le reste des états n'y prendra-t-il point, puisque
déjà, sans être assemblés, il y a si peu de gens neutres\,! Combien de
médiateurs dont la sincérité et l'amour de la paix de l'Église, de la
patrie, ne sera point {[}à{]} l'épreuve de l'amour-propre, et qui,
peut-être sans le vouloir expressément, fomenteront plus qu'ils
n'apaiseront\,! Et si, à l'exemple du cardinal du Perron aux états de la
minorité de Louis XIII, dont Votre Altesse Royale ne peut trop lire la
relation\footnote{La relation des états généraux de 1614, à laquelle
  renvoie Saint-Simon, est probablement celle de Florim. Rapine
  intitulée\,: \emph{Recueil très exact et très curieux de tout ce qui
  s'est fait et passé de singulier aux état tenus à Paris en l'année}
  1614 (Paris, 1651, in-4).}, quelque grand prélat s'avise de faire une
harangue à la romaine, quelles en peuvent être les conséquences si on la
laisse passer, ou si on prend le parti d'en réprimer les maximes et les
abus\,! Rome, en ce temps-là, ne partageait pas tous les esprits par une
bulle adorée des uns, abhorrée des autres, suspecté au moins à nos
libertés parmi toutes les personnes neutres sur le fond des propositions
dogmatiques, mais qui sont instruites de nos maximes et de quelle
importance en est la conservation\,; et cependant ce discours du
cardinal du Perron scandalisa, troubla l'assemblée, et, jusqu'à la fin
du dernier règne, ceux de son sentiment pour Rome ont su en tirer de
grands avantages. Si quelque chose d'approchant arrivait aux états,
comme il est difficile que la nature de l'affaire ne le produise, quel
embarras pour Votre Altesse Royale entre les deux partis dont l'un
relèverait vivement l'autre\,! Et si les parlements, singulièrement
destinés à veiller au maintien des libertés de l'Église gallicane se
portaient à quelque démarche à ces occasions, et que les états vinssent
à prétendre que c'est attenter à la dignité et à la liberté de leur
assemblée, quelle division dans le troisième ordre, et quelles nouvelles
difficultés pour vous\,!

Si, après ces considérations, on se renferme uniquement dans la matière
qui forme celle des délibérations des états, n'est-il pas à craindre
qu'il n'y résulte de la division entre un grand nombre de députés du
premier et du troisième ordre, de l'aigreur que les procédures de
plusieurs prélats et les arrêts de plusieurs parlements ont fait naître,
et que des personnes qui se croient avoir été réprimées mal à propos ne
soient disposées à s'élever dans les délibérations d'autres matières
contre les avis de celles des jugements desquelles elles sont encore
mécontentes. C'est le moins qui puisse arriver et une faiblesse de
l'humanité qui ne se rencontre que trop partout, et qui néanmoins
pourrait apporter une grande longueur et de grands mouvements aux
affaires. Il y aurait bien d'autres considérations à représenter sur le
premier ordre aux réflexions de Votre Altesse Royale. Celle de la
juridiction ecclésiastique, trop bornée à son gré par les parlements,
pourrait former ici un article long et important. On peut aisément
prévoir que le premier ordre en fera un de demande là-dessus, qu'il
pressera d'autant plus vivement que l'affaire de la constitution a donné
lieu à renouveler ses désirs d'une autorité plus étendue. Cette même
affaire a pu aussi faire sentir à Votre Altesse Royale la nécessité du
contre-poids, et les parlements ne seront pas moins ardents à soutenir
l'usage présent à cet égard, s'il vient à être attaqué par des demandes
du premier ordre, nouvelles épines pour vous, et nouvelles longueurs
pour terminer les affaires pour lesquelles vous auriez convoqué les
états généraux. Il serait donc infini de rapporter tout dans un mémoire.
Il suffit d'y toucher les choses principales. C'est à l'excellent esprit
de Votre Altesse Royale à suppléer au reste. Examinons maintenant le
second ordre, autrefois le seul des états.

Oui, monseigneur, le seul de l'État. Ce n'a été qu'en vertu de grands
fiefs et de la qualité de grands feudataires que les prélats ont
commencé à être admis avec la noblesse aux délibérations de l'État. Les
ecclésiastiques, dépourvus de cette libéralité de la piété de notre
ordre, ne s'y mêlaient point. Peu à peu la quantité des fiefs, jointe à
celle du sacerdoce, sépara les grands feudataires ecclésiastiques d'avec
les grands feudataires laïques, et fit des premiers le premier ordre par
le respect de leur caractère, qui dans la suite admirent parmi eux
d'autres ecclésiastiques moins considérables pour le temporel. Ces deux
ordres subsistèrent seuls jusqu'après le malheur de la bataille de
Poitiers\footnote{Le tiers état figure déjà aux états généraux de 1302,
  sons le règne de Philippe le Bel. Voy. Note V à la fin du volume.},
que les nécessités de l'État épuisé firent recourir à ceux qui le purent
secourir et qui, en cette considération, furent consultés et furent
admis en troisième ordre avec les deux premiers, ce qui a continué
depuis Charles V. Je ne puis me refuser un souvenir si précieux de notre
origine, une avec la monarchie, dans l'état d'abjection, de décadence,
d'oppression où notre ordre se voit réduit, tandis que les deux autres,
que nous avons vus naître, conservent une dignité que celle de l'autel
communique au premier, et une autorité que notre ignorance, notre
faiblesse, notre désunion, voilées du nom de la gloire et des armes, a
laissé usurper au troisième, appuyé de la longueur du dernier règne et
de l'esprit qui y a continuellement dominé. Mais, indépendamment d'un
souvenir si cher, il n'est point étranger à la matière présente, et ma
déférence pour ce troisième ordre, puisqu'il en fait un des trois qui
composent l'État, m'aurait fait supprimer ce que j'ai dit et ce que j'ai
encore à dire là-dessus, sans la nécessité qui va en être développée.

Le troisième ordre ne paraît que sous le quatorzième règne de la race
capétienne\footnote{Il est impossible de concilier cette assertion avec
  celle de la page précédente, où Saint-Simon déclare qu'il n'y eut que
  deux ordres jusqu'à la bataille de Poitiers, c'est-à-dire jusqu'au
  règne de Jean, en 1356. Ici au contraire, Saint-Simon place
  l'apparition du tiers état beaucoup plus tôt, puisque le quatorzième
  roi de la dynastie capétienne est Charles IV le Bel, ou même son frère
  Philippe V, si l'on compte Jean Ier, fils de Louis X qui ne vécut que
  quelques jours.}, et il n'existe solidement que depuis\,; il est donc
clair qu'il n'a eu aucune part à aucun des trois changements des trois
maisons qui ont porté l'une après l'autre la couronne de France, encore
moins au choix des rois qui s'est fait plus d'une fois dans les deux
premières races, ni à la fixation des aînés sur le trône, en vigueur non
contredite depuis le roi Robert, fils de Hugues Capet, en faveur de
Henri Ier. La célèbre querelle pour la couronne, et sur la loi salique,
entre Philippe de Valois et le roi d'Angleterre, Édouard III, lequel
Philippe de Valois était le grand-père de Charles V, a donc été jugée
avant que le troisième ordre eût pris naissance, et il ne s'est point
depuis présenté de contestation sur la couronne où il ait eu part. Vous
en avez maintenant deux idéales qui, s'il plaît à Dieu, ne se
réaliseront jamais\,: l'une regarde Votre Altesse Royale\,; l'autre MM.
du Maine et de Toulouse et leur postérité. Cette dernière est portée en
jugement, et les légitimés demandent les états généraux. Je n'entre
point en raisonnement du droit. J'ignore ce que vous vous proposez sur
cette grande affaire, mais elle sera jugée et restera indécise avant la
tenue des états. Si vous les assemblez cette cause restant pendante, il
n'est pas douteux que les parties ne la portent devant les états, et que
tous auront la même ardeur d'être jugés que de juger. Alors qui seront
les juges\,? Le troisième ordre pourra-t-il souffrir que sa compétence
soit agitée si celle des deux autres ordres est reconnue\,; et les juges
de Philippe de Valois, pour en demeurer au dernier exemple et à celui
dont il reste des preuves moins obscures, voudront-ils prendre pour
associés des serfs de ce temps-là\,? si les princes du sang disent
nettement, dans le dernier mémoire qu'ils viennent de signer et de
présenter, et de rendre public, qu'ils se croiraient déshonorés de
souffrir les légitimés dans le même ordre de succession, conséquemment
dans les mêmes rang et honneurs qu'eux-mêmes en tiennent que de cette
faculté innée en eux de succéder à la couronne, ceux qui en ont jugé de
tout temps, ceux qui, non plus que les princes du sang pour la
succession à la couronne et ce qui y est attaché, n'ont point de
compagnon dans ces sortes de jugements si célèbres et si honorables, et
qui tiennent cette faculté de juger ces grandes questions de leur
naissance, comme les princes du sang tiennent leur faculté de succéder à
la couronne, de leur tige et de leur descendance de mâle en mâle en
légitime mariage, est-il à présumer que ces juges naturels consentent à
partager leur pouvoir en ce genre, si éclatant et si unique, avec ceux
qui n'ont jamais été dans le cas de prétendre à le partager avec eux, et
que ces juges originaires ne s'en estimeraient pas déshonorés\,? si ce
débat s'émeut, quelles en seront les suites, quelle la fin qui le
terminera\,? Vous n'y pourrez prononcer sans vous rendre
irréconciliables ceux que vous condamnerez. Point de milieu entre être
ou n'être pas juges, entre souffrir une égalité inconnue à nos pères et
jusqu'à aujourd'hui, et une disparité si humiliante pour le tiers état.
Et point de ressource dans l'exemple du lit de justice, car c'est un
tribunal tout singulier, animé par la majesté royale, et qui sous sa
présidence n'a d'existence que par la présence des pairs, quoi qu'on ait
essayé depuis cette régence. Le roi y mène qui bon lui semble, ceux
qu'il y mène y sont sans voix s'ils ne sont pas officiers de la
couronne, ou en effet de son conseil d'État\,; ainsi rien de plus
distinct des états, ni qui y ait moins d'influence et de rapport.

Que ce débat s'émeuve, très assurément Votre Altesse Royale n'en peut
douter. Elle voit les mouvements de plusieurs de la noblesse sur des
prétextes où je suis trop intéressé pour en vouloir parler. Mon tendre
amour pour mon ordre, je n'en crains point le terme, mon respect pour
lui me fera regarder sa division avec larmes, et me ferait déplorer en
secret, mais sans en venir jusqu'aux plaintes, s'il venait à être séduit
jusqu'au point de renoncer, en faveur du désordre et de la confusion, à
la seule récompense solide qu'il puisse prétendre, et à ce qui a
toujours existé dans la monarchie, et à ce qui n'est pas moins en usage
de tous les temps, dans tous les autres États que le nôtre, de quelque
genre de gouvernement qu'ils soient chacun en leur manière, au lieu de
s'unir tous ensemble comme frères au pied du trône, comme en 1649 par un
si différent exemple, contre les excressences qui n'ont et ne prétendent
que contre notre ordre, et comme n'étant d'aucun des trois ou hors de
l'ordre naturel et commun des trois qui composent et forment la nation.
Mais ce mouvement même si peu de la convenance d'un arrêt du conseil,
s'il m'est permis que ce mot m'échappe, doit faire sentir à Votre
Altesse Royale que le second ordre, poussé à bout de toutes les manières
avant que vous soyez arrivé à la régence, a dessein et une grande
volonté de travailler à son rétablissement\,; et que, d'accord en
certaines matières, que quelques-uns d'eux ont avidement saisies, avec
quelques notables du tiers état qui les leur ont artificieusement
présentées, dans l'appréhension d'une union utile à l'État et à Votre
Altesse Royale, mais propre aux vues particulières de ces notables,
cette union ne peut durer parmi des intérêts si essentiels et si fort
contradictoires qui se développeront chaque jour dans une tenue d'états,
qui causeront un choc entre le droit d'une part et l'autorité accoutumée
de l'autre, qui ne peut enfanter que des angoisses pour vous et des
malheurs pour l'État.

Mais je dis plus, et me renfermant dans l'affaire des princes, vous ne
pouvez ignorer l'extrême désir de la noblesse d'en être juge, et je
m'étendrais inutilement à vous convaincre d'une chose dont vous l'êtes.
De là à prétendre juger seule, il n'y a plus qu'un pas, et ce pas est si
naturel que tout en persuade, et singulièrement tout ce qui se passe
depuis ces mouvements commencés. Que si la tenue des états trouve
l'affaire jugée, comptez, Monseigneur, que les mécontents du jugement
rendu, et que la noblesse, qui ne le sera pas moins qu'une telle affaire
lui ait échappé, voudront également la remettre sur le tapis, et que,
quand notre ordre serait convaincu de l'équité de ce que vous auriez
prononcé, et ne pourrait que prononcer de même, il agira de concert avec
ceux qui auront été condamnés pour arriver à revoir l'affaire, dût-il
encore une fois y prononcer en mêmes termes qu'il aurait été fait. Nul
plus grand intérêt ne se peut présenter à lui. Vous voyez à quel point
plusieurs se montrent touchés de ce qu'ils devraient regarder avec
d'autres yeux. Concluez du moins que ceux-là mêmes, et tous les autres
avec eux, verront clair sur celui-ci qui porte avec soi toute la vérité
et la solidité du plus grand et du plus sensible intérêt, et qu'ils ne
se détourneront pour quoi que ce soit ni à droite ni à gauche.

Vous connaissez, Monseigneur, les princes du sang et les légitimés, la
naissance des uns, les établissements des autres, le mérite de tous.
Quelles partialités ne formeront-ils point parmi le second ordre, et
encore parmi les deux autres\,! quels mouvements jusqu'à la décision
entre eux\,! Quelles suites de cette décision\,! Quel ralliement des
esprits remuants et mécontents avec ceux de ces célèbres plaidants qui
auront perdu leur cause\,! En envisagez-vous bien les conséquences et
les suites durant et après les états\,? Pouvez-vous espérer quelque
fruit heureux de leur tenue avec des accompagnements si turbulents\,?
J'avoue pour moi qu'ils m'effrayent. Je les laisse à toutes les
réflexions de Votre Altesse Royale, pour achever de lui présenter en
raccourci quelques autres inconvénients qui peuvent arriver de notre
ordre.

Plus vous avez fait de grâces, moins il vous en reste à faire\,; par
conséquent peu d'espérance d'en obtenir, encore moins de tout ce que
l'espérance fait faire. Cette considération, qui tombera dans l'esprit
de tout le monde, en est une de plus, et puissante sur notre ordre, pour
lui faire sentir plus vivement, en particulier, ce que tous les trois
ordres sentiront en général, qu'il faut user de l'occasion des états,
après laquelle plus de ressource, et qui vous privera de la plupart des
instruments dont vous auriez pu espérer de vous servir avec succès pour
aller au-devant des demandes embarrassantes. Nul des trois ordres plus
opprimé que celui de la noblesse. Tous ses privilèges sont non seulement
blessés, mais anéantis, et il est exactement vrai de dire qu'elle paye
la taille et tous les autres impôts autant et plus réellement que les
roturiers\,: la taille et fort peu d'autres tributs par d'autres mains
et sous d'autres noms, mais de sa bourse\,; tout le reste sans aucune
distinction. C'est sur quoi vous devez vous attendre à des
représentations aussi fortes que justes, et à des propositions pour les
formes aussi embarrassantes à rejeter qu'à accorder.

L'autorité des gens de plume et de finance ne s'est appesantie sur nul
autre ordre à l'égal du nôtre. Le premier est en possession de s'imposer
presque pour tout, lui-même, et le troisième a tant de rapport et de
réciproque avec ces messieurs d'autorité, que l'expérience journalière
et actuelle montre quels sont leurs ménagements, et combien à plomb ces
ménagements retombent sur la noblesse, parce qu'il ne faut pas que le
roi ni ses bien-tenants y perdent rien. De là, et de ce que la noblesse
n'a nulle autre ressource ni métier en France que les armes, où, elle se
ruine encore, est arrivé le malaise des seigneurs les plus distingués,
la chute des plus grandes maisons, et la pauvreté affreuse d'une
infinité de noblesse. Le mépris qui en résulte achève d'accabler les uns
et d'outrer les autres, et cette horrible extrémité ne peut manquer de
produire des remontrances d'une justice infinie, mais qui, pour le fond
et la forme, ne seront pas d'un moindre embarras.

Outre ceux qui naîtront du fonds général d'épuisement en matière de
soulagement, c'est qu'il est impossible que le rejet des uns ne retombe
en partie sur les autres, et que les formes proposées, tant sur le fonds
du soulagement que sur sa forme, par rapport aux privilèges de la
noblesse et à l'autorité qui s'exerce tyranniquement sur elle, ne la
commettent avec le tiers état, qui ne voudra point payer le soulagement
d'autrui, ni aussi peu perdre les moyens auxquels il se trouve arrivé
peu à peu de la tenir dans sa dépendance. Des intérêts si pressants et
si contradictoires ne se poursuivent pas longtemps sans aigreur, que le
temps et les circonstances présentes ne semblent pas trop en état
{[}de{]} réprimer suffisamment. Nouvelles difficultés pour Votre Altesse
Royale, et toutes plus fâcheuses les unes que les autres.

Le militaire, nerf de l'État, élite de la noblesse, a infiniment
souffert dans les dernières années du feu roi, et non depuis votre
régence. Vos moyens à cet égard n'ont pu être d'accord avec votre
inclination\,; mais ne comptez pas, Monseigneur, que le mécontentement
en soit moindre. Les gens de guerre, remplis d'espérances proportionnées
à leurs besoins, ont vu avec une extrême joie passer entre les mains de
ceux de leur métier l'administration de tout ce qui le regarde sous un
régent qui en a fait sa gloire, mais ce régent guerrier, ni ses
ministres pris des armées, n'ont pu répondre à ces justes désirs, et ces
désirs déçus causent un chagrin que l'espérance ne soutient plus, et
qu'il n'est pas même permis de vous taire. Les conséquences de ce
malheur, c'est à votre prudence à les prévenir\,; mais dans une telle
situation je douterais beaucoup si ce ne serait pas une raison de plus,
et bien forte, contre une convocation d'états généraux, qui n'en
seraient pas au moins plus dociles, ni peut-être moins hasardeux.

Le tiers état ne sera pas plus aisé que les deux premiers ordres. Après
ce qui a été examiné sur ceux-là, la matière de celui-ci est dégrossie.
Il ne laisse pas de présenter des réflexions qui lui sont particulières,
et qui ne méritent pas moins d'attention que les précédentes.

Ceux dont il est composé forment une assemblée diverse. La magistrature
en a si constamment qu'elle ne le peut nier, et que tous les exempts y
sont précis. Quoique les dignités, les offices et les charges excitent
plus que jamais de la contention dans les esprits, la règle est si
certaine en France en leur faveur, au préjudice de toute autre
considération, que sans nul égard pour l'extraction noble, dès que ceux
qui en sont se trouvent revêtus de quelque magistrature que ce soit, et
députés aux états généraux, ce n'est jamais que pour le troisième ordre.
Je ne parle pas du chancelier qui y est dans son rang particulier
d'officier de la couronne, ni du garde des sceaux qui, bien que
commission amovible, a l'honneur d'y participer à cause de celui du
dépôt dont il est chargé. Mais, nul autre magistrat n'en est excepté,
sur quoi il y aurait des remarques à faire dans des usages hors des
états, qu'il est inutile d'expliquer ici, parce que la vérité qu'on
avance n'a pas besoin de preuves. Il est pourtant vrai que cette
identité d'ordre avec de simples bourgeois a quelquefois déplu à la
première magistrature, et qu'elle a quelquefois voulu s'en séparer. Mais
l'État n'étant composé que de trois ordres, et la magistrature ne
pouvant entrer dans les deux premiers, il ne lui reste que le troisième.
L'autorité qu'elle s'est acquise sous le dernier règne, et ce qui en
paraît depuis la régence, ne laisse pas présumer que sa répugnance ait
diminué à figurer dans le tiers état. Quelques assemblées rares et
informes lui pourront donner lieu à prétendre diviser ce dernier ordre
en deux distincts, et à en composer seuls la première partie\,; premier
sujet de contestation dans tout cet ordre, qui aura droit de s'y
opposer, et de soutenir les règles anciennes, et qui ont été suivies
dans tous les vrais états. Les deux premiers ordres le voudront-ils
souffrir, et n'y va-t-il pas du leur de laisser intervertir l'ordre
ancien et ordinaire\,? La noblesse, qui voit introduire des compétences
inouïes jusqu'au milieu du dernier règne entre elle et la première
magistrature, et qui les sent maintenant se tourner en des préférences
encore plus nouvelles, n'aura-t-elle pas lieu de craindre enfin pour
tout son ordre en corps\,? si cette prétention a lieu, second sujet de
dispute. Enfin quelle sera la manière d'opiner aux états lorsque ce sera
par ordre, comme cela s'y pratique souvent en certaines affaires\,?
troisième difficulté dont la solution ne paraît pas. Comme ce que Votre
Altesse Royale traite volontiers légèrement l'est d'ordinaire avec
ardeur par les parties intéressées, je la supplie de compter pour
quatrième, et non moindre embarras, ceux du cérémonial de cette espèce
d'ordre nouveau, également contestable et sûrement contesté par tous les
trois ordres des états généraux\,; et pour cinquième, où poser les
bornes de ce qui entrerait dans cet être nouveau\,? Voilà donc le tiers
état divisé en lui-même si cette question est mue, divisé encore si la
constitution donne lieu aux parlements d'agir durant la tenue des états
à l'occasion des discours que les prélats attachés à Rome y pourraient
faire, divisé de plus, ou commis avec le premier ordre, sur la
juridiction ecclésiastique, divisé avec le second ordre sur les
propositions qu'il pourra faire tant sur le fond que plus encore sur la
forme de son juste soulagement, enfin commis avec les deux premiers
ordres sur le jugement de l'affaire des princes, comme il a été expliqué
plus haut sur tout cet article. Certainement, Monseigneur, en voilà
beaucoup pour s'en tirer avec adresse et bonheur.

C'est en traitant ce qui regarde le tiers état qu'il faut
particulièrement réfléchir sur ce que j'ai pris la liberté de vous
représenter à l'entrée de ce mémoire, de la différence d'avoir assemblé
les états généraux en prenant les rênes du gouvernement, ou de le faire
maintenant que tout est entamé sur la finance. Je n'ai garde d'en
vouloir presser le raisonnement en faveur de l'avis persévérant dont
j'ai été là-dessus. Mais il est impossible de ne pas effleurer l'un pour
venir plus utilement à l'autre. Je prévoyais ce qui arriverait, et qu'on
ne pourrait se tirer d'une matière si épuisée par le dernier
gouvernement que par des coups également douloureux au dedans et
éclatants au dehors. J'appréhendais que, sans le mériter, Votre Altesse
Royale n'en recueillit toute la haine\,; et, tandis que vous étiez tout
neuf encore, je voulais, par une exposition et une consultation toute
sincère aux états généraux, leur faire frapper ces grands coups
inévitables, dont la promptitude de votre confiance en eux n'eût reçu
des applaudissements, sans avoir rien à craindre pour la suite des
exécutions dont les résolutions ne seraient point émanées de vous, ni
ensuite d'aucune gestion de votre part\,; et si, par un triste
événement, les remèdes proposés par les états, et fidèlement employés
ensuite sans les outrepasser, avaient été insuffisants, rien à craindre
d'une nouvelle convocation d'états généraux, qui n'eût été qu'une suite
de votre première confiance, un gage réitéré de votre amour pour la
nation, et une solide confirmation du lien entre vous et elle, pour
prendre ensemble des moyens plus efficaces\,: grand et rare exemple pour
toute l'Europe, qui eût fondé votre sûreté au dehors par le concert du
dedans, et qui eût comblé votre gloire jusque par les malheurs du
dernier gouvernement.

Mais présentement les choses n'en sont plus dans ces termes\,; et,
quoique les bons desseins, la droiture des intentions, l'application et
le travail de Votre Altesse Royale méritent toutes sortes de louanges,
il n'est pourtant que trop vrai que le peuple, qui sent ses justes
espérances tournées en augmentation de douleurs, n'est pas disposé à des
jugements favorables, s'irrite de ce qu'il ignore, et peut-être encore
de ce qu'il devrait ignorer. Ce n'est plus l'air de confiance ni la
confiance même qui conduit aux états, ce sont les mêmes nécessités qui
ont donné occasion à d'autres tenues dont le succès n'a pas été heureux.
À bout de remèdes, vous y en voulez chercher\,; eux-mêmes n'ont plus
rien à vous offrir en ce genre qui puisse être à leur goût, après avoir
souffert tous ceux que vous avez tentés, mais que, convaincus de la
nécessité publique, eux-mêmes, d'abord consultés, vous eussent peut-être
proposés plus forts et plus utiles, avec un succès plus heureux, parce
que le mal qu'on se fait à soi-même est infiniment moins douloureux et
moins sensible.

Ces remèdes ont tous porté sur le tiers état d'une manière directe\,; et
si les deux autres en ont souffert, ce n'a été que du rejaillissement de
celui-ci. Ensuite ç'a été le militaire sur le prix de son sang et de ses
travaux, dans les différentes révolutions des papiers du roi qu'il a été
forcé de recevoir pour sa solde. Après des opérations si sensibles, se
doit-on flatter que le tiers état le soit assez d'une consultation qu'il
croira forcée par la pure nécessité pour chercher à présenter des
remèdes à ses dépens, ou pour consentir sans émotion à ceux qui lui
pourraient être proposés\,? Tels sont ceux qui portent sûr les rentes,
que j'ai suffisamment traités plus haut, et de même nature tout ce qui
est sur le roi. N'y a-t-il point plutôt à craindre que, comme la
consultation emporte un raisonnement nécessaire, il ne mette sur le
tapis des questions embarrassantes, et que, l'humeur s'y joignant, on ne
se contente pas aisément des réponses les plus solides\,? Je doute, par
exemple, que, quelques avantages qu'on puisse montrer de la banque du
sieur Law et des arrangements qu'on y a mis, tant de membres, alliés de
parenté ou de bourse avec tout ce qu'il y a de banquiers et de
commerçants d'argent que cet établissement ruine, s'en accommodent,
aussi peu d'un étranger de pays et de religion pour un emploi si
considérable, et moins encore de ce que tout l'argent du roi passe par
ses mains, sur un simple arrêt du conseil, au préjudice d'édits
enregistrés, non révoqués, qui le défendent sous de si grosses peines.
Or, si cette banque générale devient l'aversion des états, c'est-à-dire
du tiers ordre, à qui ces discussions seront familières, elle se
décréditera. Si elle se décrédite, elle tombe, et sa chute ne peut être
que bien importante. Dérobez-la par autorité aux yeux des états\,; que
ne ferez-vous point dire\,? Elle en tombera plus tard\,; mais cette
chute ne sera que différée. Alors, Monseigneur, tout le fruit que vous
en avez déjà recueilli, et que vous en espérez pour l'avenir, sera perdu
sans ressource\,; et, si cette banque en a fait une des principales
depuis son établissement, c'est ici mieux qu'à la mort du roi, pour le
changement de résolution sur l'assemblée des états, qu'il faut appliquer
le raisonnement qui vous fut suggéré, faux alors, vrai aujourd'hui\,:
\emph{De quoi vivrez-vous en attendant l'effet des remèdes des États\,?}
Moins vous aurez de quoi les attendre, plus vous dépendrez d'eux\,; et,
s'ils aperçoivent ce genre de dépendance, pouvez-vous, après ce qui a
été dit, croire qu'ils ne voudront pas en profiter\,; et qui osera en
poser les bornes\,?

Il n'y a point maintenant de duc de Guise\,; mais aussi n'êtes-vous pas
roi. Henri IV l'était par son droit, par sa vertu, par son épée,
lorsqu'il assembla les notables à Rouen. On ne peut lire le discours
qu'il leur fit sans sentir tout à la fois une admiration et un amour
pour ce grand prince qui émeut jusqu'aux larmes. Rien de si rempli de
majesté, en même temps de tendresse pour son peuple, et d'une estime
pour la nation, qui faisait leur gloire réciproque, après leurs travaux
communs qui avaient achevé de l'établir sur le trône. Chéri et révéré de
tous ses sujets, il crut pouvoir leur faire des consultations et des
demandes. Il n'avait alors à leur montrer que la gestion d'un
surintendant dont on admire encore les lumières et la droiture. Qu'en
arriva-t-il\,? Des propositions qu'on eut grand'peine à modérer, et qui,
dans toute la considération qu'on put obtenir par adresse, touchèrent
sensiblement Henri IV, l'obligèrent à tout éluder et à congédier
l'assemblée, dont il ne recueillit que ce seul fruit. C'est à vous,
Monseigneur, à en faire l'application, et de cet exemple et de celui des
états de la minorité de Louis XIII, sur lesquels vous ne pouvez
suffisamment méditer. Craignez de vous voir obligé à supprimer beaucoup
d'impôts tout d'un coup, et spécialement ceux de la capitation et du
dixième, sans avoir en même temps d'autres ressources présentes, et
peut-être peu à espérer des états. C'est le moins peut-être qui puisse
arriver de leur tenue. Mais, pour dernier inconvénient, que serait-ce si
vous aviez à les vouloir dissoudre, comme Henri IV l'assemblée des
notables, et comme il est arrivé à plusieurs tenues d'états\,? Que
dirait le dedans, et que ne ferait point le dehors avec lequel vous êtes
maintenant dans une situation si heureuse et si différente de votre
avènement à la régence\,? Profitez-en, Monseigneur, et ne la troublez
point par une résolution qui ne vous apportera pour tous remèdes que des
embarras et des dangers.

Ce n'est pas que je voulusse m'engager à soutenir qu'il ne faut jamais
plus d'états généraux\,; je les ai ardemment souhaités et conseillés à
l'entrée de votre régence, et il se pourra trouver des conjonctures où
il sera bon et utile de les assembler\,; mais ce ne sont pas celles
d'aujourd'hui, où tout est enflammé, où tout est entamé sur les
finances, où sans états vous avez tous ceux que vous pouvez consulter,
et qui seraient peu écoutés dans cette assemblée, laquelle fournirait
autant de remèdes contradictoires qu'il s'y trouverait d'intérêts
d'ordres et de provinces différents, et produirait une funeste dispute
entre les fonciers et les rentiers, où certainement les princes seraient
jugés, ou bien Votre Altesse Royale réduite à les juger sur l'avis des
états qui n'en auraient rien à craindre, et vous à recueillir seul la
haine des perdants, sans gré aucun de ceux qui auraient gagné leur
cause.

Dans des circonstances, dis-je, où tous les inconvénients ne peuvent
être prévus, ni l'effet de la combinaison de ceux qu'on aperçait, le
cérémonial, le danger de l'autorité royale\,; la nécessité du
soulagement effectif, le précipice de promettre sans tenir, le péril
d'accorder plus qu'il n'est possible le hasard des propositions que les
états pourraient faire sans moyens de les en empêcher qui ne soient
pernicieux, les apparences évidentes d'y trouver des maux et des
embarras nouveaux pour tout remède à ceux dont on se trouve déjà
chargé\,; la faculté qui résulterait de cette assemblée pour qui
voudrait cabaler et troubler le royaume, la manifestation également
inutile et dangereuse au dedans et au dehors d'un état d'impuissance, et
par le bruit qui arriverait nécessairement de division qui, bien connu
des mauvais sujets et des étrangers, pourrait avoir de si grandes
suites\,; la volonté sûre et suivie d'effet certain de juger ou rejuger
les princes, volonté qui marquerait la supériorité des états sur les
rois, sont des inconvénients si naturels à la situation présente qu'on
ne peut leur refuser toute l'attention qu'ils méritent par rapport aux
états en général.

À l'égard des états par parties, le premier ordre présente ceux de sa
division sur la constitution\,; le péril d'un concile national à
souffrir ou à empêcher, celui de l'imitation du cardinal du Perron
inévitable, et de ses suites en elles-mêmes, et à l'égard du
parlement\,; enfin, ce qui naîtrait par rapport à la juridiction
ecclésiastique parmi les états et avec les parlements.

Le second ordre, qui voudra juger ou rejuger les princes, dont rien ne
le fera départir, qui se commettra très possiblement avec le troisième
ordre en ne voulant pas l'admettre à ce jugement, et très certainement
sur le fond et la forme de son soulagement, et du rétablissement solide
de ses privilèges anéantis, sans possibilité de compatir ensemble avec
des intérêts si grands et si opposés, malgré l'union qui paraît
maintenant entre quelques membres de ces deux ordres, et qui
n'embarrassera pas moins à refuser qu'à accorder ce soulagement avec le
mécontentement général de tous les gens de guerre.

Le troisième ordre en scission en soi-même, et commis avec les deux
autres ordres, pour de ce dernier ordre en faire comme deux, avec toutes
les difficultés et les contentions qui en naîtraient, et séparément sur
les points qu'on vient de voir avec chacun des deux autres ordres et
avec les parlements\,; le danger de la banque du sieur Law\,; enfin, les
exemples des notables de Rouen sous Henri IV, roi d'effet alors comme de
droit, et des états tenus sous la minorité de Louis XIII.

Voilà, Monseigneur, en peu de lignes une vaste et sérieuse matière à vos
réflexions. J'ai essayé de la développer avec le moins de confusion et
de choses inutiles ou étrangères que j'ai pu dans le tissu de ce
mémoire. Je l'aurais bien désiré plus court, et {]}e dégoût de sa
matière ne m'y a que trop convié\,; mais son étendue, plus propre à un
volume qu'à un simple mémoire, ne me l'a pas permis\,; et je me suis
souvenu que Votre Altesse Royale, chargée de tout le poids d'un
gouvernement pénible, n'a pas le temps de faire toutes les réflexions
nécessaires. J'ai donc cru y devoir suppléer en lui mettant sous les
yeux celles qui me sont venues dans l'esprit. L'excellence du vôtre en
fera un juste discernement, et la bonté de Votre Altesse Royale excusera
la disproportion du mien. Qu'elle me permette de lui protester de
nouveau le désintéressement entier avec lequel je l'ai fait, et la peine
que j'ai eue à des remarques que j'aurais omises si elles n'avaient pas
été essentielles au sujet. Quoiqu'il ne soit que pour vous seul, on ne
peut répondre absolument du secret d'un écrit. Celui-ci n'est pas fait
de manière à pouvoir blesser personne, j'ai tâché d'y apporter une
particulière attention\,; mais j'ai si cruellement éprouvé, et dès
l'entrée de votre régence, que mes intentions les plus droites, et les
plus soutenues par mes discours et par mes actions, n'en avaient pas
moins été détournées à des interprétations et à des suppositions
entières les plus éloignées de mon coeur et de mon esprit, malgré toute
évidence et les preuves publiques, par un art que j'aimerai toujours
mieux éprouver qu'employer, que j'avoue ingénument à Votre Altesse
Royale que, ayant affaire aux mêmes personnes, je crains jusqu'aux
choses les plus indifférentes et les plus innocentes, et qu'il ne m'a
pas fallu des raisons moins fortes que le bien de l'État, l'importance
de la matière et mon attachement à Votre Altesse Royale, pour lui obéir
en cette occasion.

En effet ces états généraux étaient un abîme ouvert sous les pieds du
régent dans les conjonctures où on se trouvait de toutes parts, et qui
par leurs divers rapports auraient jeté l'État dans la dernière
confusion, avec la facilité, la mollesse et la timidité de celui qui en
tenait le gouvernail, en prise à tous les gens qui en auraient voulu
profiter dans leurs divers intérêts. C'est ce qui me pressa de jeter ce
mémoire sur le papier en si peu de temps, et de le porter tout de suite
à M. le duc d'Orléans, pour l'arrêter par une première lecture, et
barrer à temps les engagements que les propos spécieux du duc de
Noailles sur les finances, et d'Effiat sur l'affaire des bâtards, lui
pouvaient faire prendre avec eux à tous moments, et qu'ils auraient
sur-le-champ rendus publics, et si subitement enfourner la chose qu'il
n'y eût plus eu moyen de s'en dédire. Je compris bien aussi que si le
mémoire réussissait, comme je l'espérais bien, ces deux hommes en
seraient enragés, et les bâtards avec toute leur cabale et leur
prétendue noblesse\,; et qu'ils feraient retomber sur moi l'empêchement
de la tenue des états généraux, avec tout le vacarme qu'ils en
pourraient exciter, et que la nature de la chose exciterait d'elle-même.
C'est ce qui m'engagea à y faire mention des états généraux proposés par
moi à la mort du roi, résolus sur mes vives raisons, empêchés par le duc
de Noailles, et d'appuyer sur la différence de les avoir tenus alors à
les tenir aujourd'hui. C'est aussi ce qui m'engagea à faire mention du
projet là-dessus auquel j'avais travaillé sous Mgr le Dauphin, père du
roi, pour bien mettre en évidence que, si j'étais contraire aux états
généraux pour aujourd'hui, ce n'était qu'à cause des conjonctures, et
non par aversion pour l'assemblée nationale, que j'avais voulue et fait
résoudre en d'autres, et mettre par là à bout là-dessus la malignité de
ceux dont j'en avais éprouvé les plus noires et les plus profondes.

Il est vrai que je n'ai pu m'y refuser quelques traits sur le duc de
Noailles, tant pour remettre sous les yeux de M. le duc d'Orléans les
horreurs gratuites qu'il me fit à la mort du roi, que ses opiniâtres
méprises dans sa gestion des finances, et l'abus de son crédit pour
affubler le duc de La Force d'une besogne odieuse, pour s'en ôter la
haine à ses dépens et la détourner toute sur lui par la longueur d'une
besogne qui tenait toutes les fortunes des particuliers en l'air, au
grand détriment des affaires publiques. Je me doutais bien que M. le duc
d'Orléans n'aurait pas la force de lui cacher mon mémoire, et je me
proposais de lui ôter l'envie de tenir des propos sur moi en cette
occasion par la crainte de voir courir ce mémoire, comme je l'avais bien
résolu au premier mot qu'il aurait osé lâcher.

C'est dans la pensée d'en faire cet usage que j'ai adouci et enveloppé
le plus qu'il m'a été possible ce qu'il n'y avait pas moyen de
dissimuler à M. le régent sur sa faiblesse et sa facilité, parce que ce
défaut était un inconvénient capital qui eût grossi tous les autres, et
donné naissance à quantité\,; et c'est aussi, outre ce que je devais à
sa personne et à son rang en lui écrivant des choses si principales, ce
qui m'a engagé à y employer plus de louanges et de tours pleins de
respect.

Cette même faiblesse que les ducs avaient si cruellement éprouvée, les
étranges conjonctures, et nos requêtes pour la restitution de notre rang
à l'égard des bâtards, ne me permirent pas de faire aucune mention du
droit des pairs sur le jugement de l'affaire des princes\,; c'est ce qui
a fait que je me suis contenté de glisser sur cette matière avec une
sage réticence, mais telle qu'elle-même ni rien qui soit dans le mémoire
y puisse faire de tort. Du reste, j'ai tâché de ne rien dire qui pût
blesser aucun corps ni aucun particulier, et à ne rapporter que des
vérités connues et des inconvénients tels que, en y réfléchissant, on ne
puisse disconvenir qu'ils sautent tous aux yeux. D'ailleurs on ne peut
trouver mauvais ce que je dis à la louange et de l'oppression de la
noblesse, ni de ce peu que j'ai laissé échapper sur le gouvernement du
feu roi à cet égard, que j'ai même exprimé moins que je ne l'ai fait
entendre. À l'égard du petit mot qui se trouve glissé sur la conduite de
cette prétendue noblesse et sur le rang de prince étranger, par
opposition à ce qu'on a vu qui se passa en 1649, il me semble qu'on n'en
peut blâmer la ténuité, et, si j'ose le dire, la délicatesse\,; et que
c'eût été une affectation de n'en point faire mention du tout qui aurait
été très susceptible d'être mal interprétée. Je m'explique toujours ici
dans l'esprit où j'étais en faisant ce mémoire, quoique fort
brusquement, de le rendre public, si je m'y trouvais forcé.

Heureusement je n'en eus pas besoin\,; car je hais les scènes et les
plaidoyers publics.

\hypertarget{chapitre-xvii.}{%
\chapter{CHAPITRE XVII.}\label{chapitre-xvii.}}

1717

~

{\textsc{M. le duc d'Orléans, prêt à se rendre sur les états, se trouve
convaincu par le mémoire, et on n'entend plus parler d'états généraux.}}
{\textsc{- Mémoire sur les finances annoncé par le duc de Noailles.}}
{\textsc{- M. le duc d'Orléans me parle du mémoire\,; d'un comité pour
les finances\,; me propose à deux reprises d'en être, dont je m'excuse
fortement.}} {\textsc{- Le duc de Noailles lit son mémoire en plusieurs
conseils de régence.}} {\textsc{- Quelle cette pièce.}} {\textsc{- Je
suis bombardé du comité, au conseil de régence, où, malgré mes excuses,
je reçois ordre d'en être.}} {\textsc{- M. de Fréjus obtient
personnellement l'entrée du carrosse du roi, où jamais évêque non pair,
ni précepteur, ni sous-gouverneur n'était entré, lesquels
sous-gouverneurs l'obtiennent aussi.}} {\textsc{- Dispute sur la place
du carrosse entre le précepteur et le sous-gouverneur, qui la perd.}}
{\textsc{- Mariage de Fresnel avec M\textsuperscript{lle} Le Blanc\,; de
Flamarens avec M\textsuperscript{lle} de Beauvau\,; de La Luzerne avec
M\textsuperscript{me} de La Varenne\,; du marquis d'Harcourt avec
M\textsuperscript{lle} de Barbezieux, dont le duc d'Albert veut épouser
la soeur et y trouve des obstacles.}} {\textsc{- Arouet à la Bastille,
connu depuis sous le nom de Voltaire.}} {\textsc{- Mort du vieux prince
palatin de Birkenfeld.}} {\textsc{- Mort de la duchesse douairière
d'Elboeuf.}} {\textsc{- Mort de M. de Montbazon.}} {\textsc{- Mort de la
fameuse M\textsuperscript{me} Guyon.}} {\textsc{- Six mille livres de
pension au maréchal de Villars.}} {\textsc{- Dix mille livres au duc de
Brissac.}} {\textsc{- Six mille livres de pension à Blancménil, avocat
général.}} {\textsc{- Canillac lieutenant général de Languedoc.}}
{\textsc{- Duel à Paris de Contade et de Brillac, dont il n'est autre
chose.}} {\textsc{- Je fais acheter ce diamant unique en tout, qui fut
nommé le Régent.}}

~

Je portai mon mémoire dès qu'il fut achevé, et tel de ma main que je
l'avais écrit, tant j'étais pressé, par la raison que j'en ai dite, de
le montrer à M. le duc d'Orléans. Le volume le surprit par la
promptitude. Je le lui lus tout entier, nous arrêtant à chaque point
pour en raisonner. Cela prit toute l'après-dînée jusque fort tard. Il
convint qu'il s'allait jeter dans un profond précipice, et me remercia
fort de mon travail, et de l'en empêcher. Il lui échappa même dans le
raisonnement qu'il était si pressé de l'embarras des finances et de
celui de l'affaire des princes, et si rebattu par ceux qui voulaient les
états, qu'il y était intérieurement rendu comme à sa seule ressource et
à son repos, d'où je jugeai que de cette résolution intérieure à
l'extérieure le pas était bien court, et bien facile avec les gens à qui
il avait affaire, et qu'il n'y avait eu en effet rien de si pressé que
mon mémoire pour l'en détourner. Ses yeux ne pouvaient lire ma petite
écriture courante et pleine d'abréviations, quoique fort peu sujette aux
ratures et aux renvois. Il me pria de lui faire faire une copie du
mémoire, et de la lui donner dès qu'elle serait faite. Il me parut si
convaincu que je lui demandai sa parole que le pied ne lui glisserait en
aucune façon sur les états avant que je lui eusse remis cette copie, et
qu'il se fût donné le temps de la lire à reprises, et d'y réfléchir à
loisir. Je fis donc travailler, dès le lendemain matin, à une copie
unique, car c'est sur mon original que je l'ai copié ici\,; et, dès que
cette copie fut faite, je la portai à M. le duc d'Orléans. Nous
raisonnâmes encore là-dessus, mais sans détail, parce qu'il me parut que
son parti était bien pris de ne vouloir point d'états.

Je ne sais quel usage il fit de mon mémoire\,; mais, au bout de sept ou
huit jours, il ne se parla plus du tout d'états généraux, dont le bruit
avait été fort grand et fort répandu, et, ce qui me fit grand plaisir
encore, c'est qu'il ne se dit pas un mot du mémoire ni de moi à cette
occasion.

Ce qui m'a le plus convié à ne pas rejeter ce mémoire, malgré sa
longueur, parmi les Pièces, c'est qu'il s'y trouve plusieurs choses sur
les finances qui donnent une idée de leur état, de leur gestion et des
embarras qui s'y trouvaient, dont il n'est guère parlé ailleurs ici\,;
et de même de quelque chose sur la constitution, qui seront toujours à
éclaircir, et qui sont deux matières dont on a vu, il y a longtemps que
je me suis expliqué de n'en point parler ici d'une manière expresse et
suivie.

L'espérance des états évanouie, les bâtards ne songèrent plus qu'à
retarder, embarrasser et accrocher leur affaire\,; les princes du sang à
presser le régent de la juger\,; et ce prince, piqué enfin de voir son
autorité si hardiment mise en compromis par la hardie déclaration de M.
et de M\textsuperscript{me} du Maine de ne reconnaître pour juges que le
roi majeur ou les états généraux, prit le parti de juger\,: c'est ce qui
a été raconté.

Le duc de Noailles, de son côté, chercha aussi d'autres expédients sur
les finances, mais surtout pour mettre sa gestion à couvert. Il fit
travailler à un long mémoire, pour être lu par lui au conseil de
régence, où il fût longuement annoncé. J'ai déjà fait remarquer, et par
des exemples évidents, qu'avec tout son esprit, la multitude et la
continuelle mobilité de ses idées et de ses vues qui se succédaient et
se chassaient successivement ou en total ou en partie sur toutes sortes
de sujets, de choses et de matières, le rendaient incapable d'aucun
travail par lui-même, ni d'être jamais content de ceux qu'il faisait
faire et qu'il faisait refondre (c'était son terme) jusqu'à désoler ceux
dont il se servait. C'est ce qui fit attendre si longtemps ce mémoire
après l'avoir annoncé et, autant qu'il le put, préparé à l'admirer.

Huit ou dix jours avant qu'il parût au conseil de régence, M. le duc
d'Orléans m'en parla et me le vanta comme en ayant vu des morceaux, puis
me dit qu'il formerait un comité (car on ne parlait plus qu'à
l'Anglaise) de quelques-uns du conseil de régence, où le duc de Noailles
voulait avec plus de loisir et d'étendue exposer sa gestion et l'état
des finances, et consulter ce comité sur les choses qu'il y proposerait
pour en suivre leur avis\,; que ce comité s'assemblerait chez le
chancelier, et qu'il voulait que j'en fusse.

Je témoignai au régent ma surprise et ma répugnance\,; je le fis
souvenir de mon incapacité sur les finances, de mon dégoût pour cette
matière, de ma situation avec le duc de Noailles. Je l'assurai que je ne
pourrais être de ce comité que comme une {[}personne{]} nulle, qui
n'entendrait rien, à qui on ferait accroire tout ce qu'on voudrait, que
j'y serais parfaitement inutile, que j'y perdrais un temps infini, et
que je le suppliais de m'en dispenser. Il insista, et moi aussi, me dit
force louanges sur mon esprit et ma capacité quand je voudrais bien
prendre la peine de vouloir m'appliquer et entendre, et sur mon
impartialité avec le duc de Noailles quand il s'agissait de traiter
affaires avec lui, dont il avait été souvent témoin et charmé. Je
répondis brusquement que ces louanges étaient belles et bonnes, mais que
je n'étais pas encore assez sot pour m'en laisser engluer, et qu'en deux
mots, il ne me persuaderait pas d'aller ouvrir la bouche et de grands
yeux pour n'entendre rien à ce qui se dirait et proposerait, et que ce
n'était pas la peine d'avoir refusé les finances aussi opiniâtrement que
j'avais fait pour m'aller après fourrer dans un comité de finances, où
je ne comprendrais rien du tout. Le régent me vit si résolu qu'il ne
répliqua point, et me mit sur d'autres affaires.

Quatre jours après, travaillant avec lui, il me reparla encore du
comité, et qu'il voulait que j'en fusse. Je répondis que je croyais
avoir dit de si bonnes raisons, auxquelles même, à la fin, il n'avait
plus répondu, que j'avais compté n'en plus ouïr parler\,; que je n'avais
que les mêmes à lui alléguer, dont je ne me départirais pas. J'ajoutai,
qu'étant avec le duc de Noailles hors de toutes mesures, même de la
moindre bienséance, je ne comprenais pas quel plaisir il trouvait à nous
mettre vis-à-vis l'un de l'autre dans un examen de sa conduite et des
propositions qui serait long, et qui nous exposerait très aisément à des
choses qui embarrasseraient la compagnie, et qui peut-être
l'embarrasseraient lui-même\,; et comment il voulait donner cette
contrainte au duc de Noailles, qui sûrement y en aurait plus que moi.
Mais, me dit-il, c'est le duc de Noailles lui-même qui désire que vous
en soyez, qui m'en a prié et qui m'en presse. --- Monsieur, repris-je,
voilà la dernière folie. A-t-il oublié, et vous aussi, comme je l'ai
mené et traité, je ne sais combien de fois, tant en particulier devant
vous qu'en plein conseil de régence\,? Quel goût peut-il prendre à des
scènes où il a toujours ployé le dos et fait un si misérable personnage,
et vous de donner lieu à les multiplier\,?» Je parlai tant et si bien,
du moins si fort, que cela finit comme la première fois. Le régent me
parla d'autres choses, et je m'en crus enfin quitte et débarrassé.

Mais je fis mes réflexions sur la singularité de ce désir du duc de
Noailles que je fusse de te comité, et tout ce que j'en pus comprendre,
c'est que l'ivresse de la beauté de ce qu'il comptait d'y exposer
emporterait mon suffrage, dont il se parerait plus qu'aucun autre par la
manière dont nous étions ensemble. Il avait affecté plusieurs fois de se
louer de mon impartialité en affaires quand je m'étais trouvé de son
avis, et quand il m'était arrivé quelquefois de le soutenir, même contre
d'autres au conseil de régence, ou en particulier entre quatre ou cinq
chez M. le duc d'Orléans. Je crus donc que l'espérance du même succès,
et du poids que ce manque total de ménagement que j'avais pour lui
donnerait à sa besogne, {[}était le motif de sa conduite{]}\,; mais
comme une funeste expérience m'avait appris jusqu'où pouvait aller la
noirceur et la profondeur de cette caverne, je me sus extrêmement bon
gré d'avoir su m'en préserver.

Trois ou quatre jours après cette dernière conversation, le duc de
Noailles commença la lecture de son mémoire. Il dura plusieurs conseils
de régence\,; il y en eut même d'extraordinaires pour l'achever. C'était
une apologie de toute sa gestion avec beaucoup de tour pour l'avantager
de tout, et beaucoup de louanges mal voilées d'une gaze de modestie.

Cette première partie était prolixe\,; l'autre roulait sur la
proposition d'un comité où il pût exposer sa gestion avec plus
d'étendue, et ses vues sur ce qu'il serait à propos de faire ou de
rejeter. Ce fut là où la fausse modestie n'oublia rien pour capter les
auditeurs par un air de désir de chercher à exposer ses fautes et ses
vues à l'examen et à la correction du comité, et à profiter de ses
lumières. Rien de si humble, de si plein de flatterie, de si
préparatoire à l'admiration qu'il espérait damer au comité, ni de plus
désireux d'en enlever l'approbation. Cette partie ne fut pas mains
diffuse que l'autre, mais le spécieux le plus touchant y brillait
partout.

Quand il eut fini, M. le duc d'Orléans et presque tous les auditeurs,
dans le nombre desquels étaient les présidents ou chefs des conseils,
lui donnèrent des louanges. Ensuite M. le duc d'Orléans, passant les
yeux sur toute la compagnie, dit qu'il ne s'agissait plus que de nommer
le comité. C'était un samedi après-midi, 26 juin. Il y avait un mais que
je vivais là-dessus dans une parfaite confiance, lorsque M. le duc
d'Orléans déclara le comité tout de suite, qu'il se tiendrait toutes les
semaines chez le chancelier autant de fois qu'à chaque comité il serait
jugé nécessaire, et que tout à coup je m'entendis nommer le premier.

Dans ma surprise, j'interrompis et je suppliai M. le duc d'Orléans de se
souvenir de ce que j'avais eu l'honneur de lui représenter toutes les
deux fais qu'il m'avait fait l'honneur de m'en parler\,; il me répondit
qu'il ne l'avait pas oublié\,; mais que je lui ferais plaisir d'en être.
Je répliquai que j'y serais entièrement inutile, parce que je
n'entendais rien du tout aux finances, et que je le suppliais très
instamment de m'en dispenser. «\, Monsieur, reprit M. le duc d'Orléans
d'un ton honnête, mais de régent, et c'est l'unique fois qu'il l'ait
pris avec moi, encore une fois, je vous prie d'en être, et s'il faut
vous le dire, je vous l'ordonne.\,» Je m'inclinai sur la table
intérieurement fort en colère, et lui répartis. «\,Monsieur, vous êtes
le maître\,; il ne me reste qu'à obéir\,; mais au moins vous me
permettrez d'attester tous ces messieurs de ma répugnance et de l'aveu
public que je fais de mon ignorance et de mon incapacité sur les
finances, par conséquent de mon inutilité dans le comité.\,»

Le régent me laissa achever, puis, sans me rien dire davantage, nomma le
duc de La Force, le maréchal de Villeroy, le duc de Noailles, le
maréchal de Besons, Pelletier-Sousy, l'archevêque de Bardeaux et le
marquis d'Effiat, qui tous s'inclinèrent à leur nom et ne dirent rien.

Mon colloque avec le régent avait attiré sur moi les yeux de tous, et je
remarquai de l'étonnement sur leurs visages. M. de Noailles eut l'air
fort content, et bavarda un peu sur le bon choix et sur ce qu'il
espérait de ces assemblées, puis se mit à rapporter, car le samedi était
un jour de finances à la régence.

N'ayant pu éviter cette bombe, par tout ce que j'avais fait pour m'en,
garantir, je ne crus pas devoir en montrer de chagrin, et donner ce
plaisir au duc de Noailles, ni me faire tirer misérablement l'oreille
pour l'assiduité au comité et l'exactitude aux heures.

Il s'assemblait trois fois la semaine au moins, entre trois et quatre
heures, et durait rarement moins de trois heures\,; on se mettait en
rang des deux côtés de la table, ou plutôt du vide d'une table longue
comme au conseil de régence, mais dans des fauteuils, le chancelier seul
au, bout, et vis-à-vis de lui une table carrée pour les papiers du duc
de Noailles, et lui assis derrière. Comme ce comité dura au moins trois
mois, il n'est pas temps d'en dire ici davantage, mais bien de revenir
au courant, depuis si longtemps interrompu par des matières qui ne
pouvaient comporter de l'être.

C'était plus que jamais le temps des entreprises les plus étranges et
les plus nouvelles. M. de Fréjus et les sous-gouverneurs prétendirent
entrer dans le carrosse du roi où jamais en aucun temps ils n'avaient
mis le pied. Ils se fondèrent sur ce que les sous-gouverneurs, un à la
fois, entraient dans le carrosse des princes fils de Monseigneur. Cela
était vrai, mais jamais M. de Fénelon ne l'imagina ni M. de Beauvilliers
pour lui, quoique tous deux dans l'intimité que l'on a vue. Saumery,
insolent, entreprenant, cousin germain du duc de Beauvilliers, avait
commencé à y entrer en son absence, et alors le sous-gouverneur y est de
telle nécessité que, sans préséance sur aucun, il y monterait de
préférence à qui que ce fût\,; mais le gouverneur présent, il est effacé
et la nécessité est remplie. Néanmoins Saumery y monta, le duc de
Beauvilliers présent, mais tellement à la dernière place qu'il faisait à
chaque fais des excuses, et souvent le duc de Beauvilliers pour lui, de
ce qu'il ne pouvait se mettre à la portière à cause de son ancienne
blessure au genou, qui ne lui permettait pas de le ployer. J'ai vu cela
maintes fois, moi dans le carrosse. Je n'y ai jamais vu que lui des
trois sous-gouverneurs. Le hasard apparemment a fait cela\,; et toujours
avec cette excuse ne montrait que le pénultième pour se mettre au
devant, et le dernier remplissait de son côté la portière, où il ne se
pouvait pas mettre. Entrer dans le carrosse et manger avec le prince est
de même droit, mais comme il n'y avait point d'occasion où les princes
fils de Monseigneur mangeassent avec personne, cela facilita
l'effronterie de Saumery. M. de Fénelon était bien de qualité à l'un et
à l'autre, mais il était précepteur, qui partait l'exclusion, et comme
il n'a rien à faire auprès du prince que pour l'étude, et qu'il n'y en a
point en carrosse, point de nécessité pour lui d'y entrer comme pour le
sous-gouverneur en l'absence du gouverneur\,; de plus il était prêtre,
puis archevêque, autres exclusions\,; parce qu'il n'y a que les
cardinaux et les évêques pairs, au ceux qui ont rang de princes
étrangers qui entrent dans les carrosses et qui mangent. M. d'Orléans,
depuis cardinal de Coislin, et M. de Reims, l'un premier aumônier,
l'autre maître de la chapelle, charges bien inférieures, ont fait
maintes campagnes avec le roi, et je les ai vus au siège de Namur.
Jamais M. d'Orléans, bien mieux avec le roi que M. de Reims, n'a eu
l'honneur de manger avec lui, tandis que l'archevêque de Reims, duc et
pair, l'avait souvent et tant qu'il lui plaisait. Ainsi, nul exemple
pour le précepteur d'entrer dans le carrosse, et un très faible du
sous-gouverneur, parce que quelque grands que soient les fils de France,
il y a bien loin encore du roi à eux.

Néanmoins M. le duc d'Orléans, qui faisait litière de toutes choses,
accorda l'entrée du carrosse à un sous-gouverneur et à M. de Fréjus. Il
est vrai qu'il eut le courage de lui dire que ce n'était que
personnellement et point comme précepteur ni comme évêque. Dieu sait à
quels excès et à quelle lie ce carrosse et l'honneur de manger avec le
roi ont été depuis étendus.

De cette grâce s'ourdit une dispute de préférence et de préséance dans
le carrosse entre le précepteur et le sous-gouverneur. Comme ils n'y
étaient jamais entrés en aucun temps, la question était toute nouvelle
et sans exemple. Il est vrai que le précepteur n'a rien à dire au
sous-gouverneur, et que les fonctions sont toutes indépendantes et
séparées\,; mais le précepteur au moins est en chef à l'étude, et le
sous-gouverneur ne se trouve en chef nulle part. Sa dépendance du
gouverneur est totale en tout et partout\,; celle du précepteur est fort
légère, lequel a sous lui des sous-précepteurs\,; et le sous-gouverneur
n'a personne\,: aussi M. de Fréjus le gagna-t-il.

En même temps le maréchal de Villeroy cessa pour toujours d'étouffer le
roi en troisième. Il se mit à la portière de son côté\,; mais
l'indécence de M. du Maine à côté du roi demeura toujours, que, tout
fils favori du feu roi qu'il était, ce monarque n'eût pas soufferte.

Fresnel épousa la fille de Le Blanc, lors du conseil de guerre, dont il
fut bien parlé dans les suites\,; et Flamarens épousa une fille de M. de
Beauvau, frère de l'évêque de Nantes. La fille aînée du maréchal de
Tessé, veuve de La Varenne petit-fils ou arrière-petit-fils du La
Varenne de Henri IV, et qui passai sa vie à la Flèche, épousa le jeune
La Luzerne, son voisin, dont elle était éprise. Elle était fort riche,
il avait du bien et la naissance tout à fait sortable. Le marquis
d'Harcourt, fils aîné du maréchal, épousa une fille de feu M. de
Barbezieux et de la fille aînée de M. d'Alègre, qui fit la noce, et le
duc d'Albret, qui voulut épouser la soeur de cette mariée, trouva des
oppositions dans la famille, qui durèrent longtemps avec beaucoup de
bruit.

Je ne dirais pas ici qu'Arouet fut mis à la Bastille pour avoir fait des
vers très effrontés, sans le nom que ses poésies, ses aventures et la
fantaisie du monde lui ont fait. Il était fils du notaire de mon père,
que j'ai vu bien des fois lui apporter des actes à signer. Il n'avait
jamais pu rien faire de ce fils libertin, dont le libertinage a fait
enfin la fortune sous le nom de Voltaire, qu'il a pris pour déguiser le
sien.

Le prince palatin de Birkenfeld mourut chez lui en Alsace, à près de
quatre-vingts ans, peu riche, et le meilleur homme du monde. Il avait
fort servi. Il était lieutenant général, et avait des pensions. Il
venait rarement à la cour, où il était toujours fort bien reçu du roi et
fort accueilli du monde. Son fils avait été fort de mes amis. Il avait
eu le Royal-allemand et est mort assez jeune, retiré chez lui, laissant
deux fils, dont l'aîné par succession est devenu duc des Deux-Ponts
depuis quelques années. Il n'y a plus que cette branche des palatins
outre les deux électorales.

En même temps mourut la duchesse douairière d'Elboeuf d'une longue suite
de maux qu'elle avait gagnés de son mari, mort depuis longtemps. J'ai
assez souvent parlé d'elle, pour qu'il ne me reste plus rien à en dire.
Elle n'était pas fort âgée.

M. de Montbazon, fils aîné de M. de Guéméné, et gendre sans enfants de
M. de Bouillon, mourut, jeune et brigadier d'infanterie, de la petite
vérole.

Une autre personne, bien plus illustre par les éclats qu'elle avait
faits, quoique d'étoffe bien différente, ne fit pas le bruit qu'elle
aurait fait plus tôt. Ce fut la fameuse M\textsuperscript{me} Guyon.
Elle avait été longtemps exilée en Anjou depuis le fracas et la fin de
toutes les affaires du quiétisme. Elle y avait vécu sagement et
obscurément sans plus faire parler d'elle. Depuis huit ou dix ans elle
avait obtenu d'aller demeurer à Blois, où elle s'était conduite de même,
et où elle mourut sans aucune singularité, comme elle n'en montrait plus
depuis ses derniers exils, fort dévote toujours et fort retirée, et
approchant souvent des sacrements. Elle avait survécu à ses plus
illustres protecteurs et à ses plus intimes amis.

Le maréchal de Villars, gorgé de toutes espèces de biens, n'eut pas
honte de prendre ni M. le duc d'Orléans de lui donner six mille livres
de pension pour le dédommager de ses prétentions sur la vallée de
Barcelonnette, disputée au gouvernement de Provence par La Feuillade,
comme gouverneur de Dauphiné, qui fut jugée devoir être de ce dernier
gouvernement.

Le maréchal de Villeroy obtint en même temps pour le duc de Brissac, qui
était fort mal à son aise, dix mille livres de pension. Quelque temps
après, Blancménil, avocat général, frère du président Lamoignon, eut
aussi une pension de six mille livres\,; et Canillac eut pour rien la
lieutenance générale de Languedoc, de vingt mille livres de rente,
vacante par la mort de Peyre, qui n'avait point de brevet de retenue.

Contade et Brillac, l'un major, l'autre capitaine aux gardes, avaient
passé leur vie dans ce corps, sans avoir pu se souffrir l'un l'autre.
Contade bien plus brillant, l'autre ne laissait pas d'avoir des amis.
Son frère était premier président du parlement de Bretagne, mais fort
peu estimé. Je ne sais ce qui arriva de nouveau entre deux officiers
généraux de cet âge\,; mais, le samedi 12 juin, Brillac vint, sur les
quatre heures du matin, chez Contade, dans la rue Saint-Honoré,
l'éveilla, le fit habiller et sortit avec lui. Ils entrèrent tout auprès
dans une petite rue inhabitée, qui va de la rue Saint-Honoré vers le
bout du jardin des Tuileries, près de l'orangerie, et là se battirent
bel et bien. Brillac fut légèrement blessé, et disparut aisément.
Contade le fut dangereusement, et il fallut le reporter chez lui. Ce fut
un grand vacarme. Un cordier et sa femme, qui profitaient de la
commodité de cette rue pour leur métier, étaient déjà levés pour leur
travail, et furent témoins du combat. Ils babillèrent\,; cela embarrassa
beaucoup\,; on les enleva\,; on cacha Contade dans le fond de l'hôtel de
Noailles, là tout auprès, et comme il avait beaucoup d'amis
considérables, tout se mit en campagne pour lui. Les Grammont, les
Noailles, les Villars, le premier président et bien d'autres en firent
leur propre affaire\,; et le régent n'avait pas moins d'envie qu'eux de
l'en tirer. Il en coûta du temps, des peines et de l'argent\,; et
l'affaire s'en alla en fumée. Pendant tout cela, Contade guérit. À la
fin de tout, Contade et Brillac parurent une fois au parlement pour la
forme, et il ne s'en parla plus. Néanmoins on voulut séparer deux hommes
si peu compatibles, et qui se rencontraient si souvent par la nécessité
de leurs emplois. Le gouvernement de l'île d'Oléron vaqua. Il est bon\,;
mais il demande résidence. Cela le fit donner à Brillac.

Par un événement extrêmement rare, un employé aux mines de diamants du
Grand-Mogol trouva le moyen de s'en fourrer un dans le fondement, d'une
grosseur prodigieuse, et, ce qui est le plus merveilleux, de gagner le
bord de la mer, et de s'embarquer sans la précaution qu'on ne manque
jamais d'employer à l'égard de presque tous les passagers, dont le nom
ou l'emploi ne les en garantit pas, qui est de les purger et de leur
donner un lavement pour leur faire rendre ce qu'ils auraient pu avaler,
ou se cacher dans le fondement. Il fit apparemment si bien qu'on ne le
soupçonna pas d'avoir approché des mines ni d'aucun commerce de
pierreries. Pour comble de fortune, il arriva en Europe avec son
diamant. Il le fit voir à plusieurs princes, dont il passai les forces,
et le porta enfin en Angleterre, où le roi l'admira sans pouvoir se
résoudre à l'acheter. On en fit un modèle de cristal en Angleterre, d'où
on adressa l'homme, le diamant et le modèle parfaitement semblable à
Law, qui le proposa au régent pour le roi. Le prix en effraya le régent,
qui refusa de le prendre.

Law, qui pensait grandement en beaucoup de choses, me vint trouver
consterné, et m'apporta le modèle. Je trouvai comme lui qu'il ne
convenait pas à la grandeur du roi de France de se laisser rebuter par
le prix d'une pièce unique dans le monde et inestimable, et que plus de
potentats n'avaient osé y penser, plus on devait se garder de le laisser
échapper. Law, ravi de me voir penser de la sorte, me pria d'en parler à
M. le duc d'Orléans. L'état des finances fut un obstacle sur lequel le
régent insista beaucoup. Il craignait d'être blâmé de faire un achat si
considérable, tandis qu'on avait tant de peine à subvenir aux nécessités
les plus pressantes, et qu'il fallait laisser tant de gens dans la
souffrance. Je louai ce sentiment\,; mais je lui dis qu'il n'en devait
pas user pour le plus grand roi de l'Europe comme pour un simple
particulier, qui serait très répréhensible de jeter cent mille francs
pour se parer d'un beau diamant, tandis qu'il devrait beaucoup et ne se
trouverait pas en état de satisfaire\,; qu'il fallait considérer
l'honneur de la couronne et ne lui pas laisser manquer l'occasion unique
d'un diamant sans prix, qui effaçait ceux de toute l'Europe\,; que
c'était une gloire pour sa régence, qui durerait à jamais, qu'en tel
état que fussent les finances, l'épargne de ce refus ne les soulagerait
pas beaucoup, et que la surcharge en serait très peu perceptible. Enfin
je ne quittai point M. le duc d'Orléans, que je n'eusse obtenu que le
diamant serait acheté.

Law, avant de me parler, avait tant représenté au marchand
l'impossibilité de vendre son diamant au prix qu'il l'avait espéré, le
dommage et la perte qu'il souffrirait en le coupant en divers morceaux,
qu'il le fit venir enfin à deux millions avec les rognures en outre qui
sortiraient nécessairement de la taille. Le marché fut conclu de la
sorte. On lui paya l'intérêt des deux millions jusqu'à ce qu'on lui pût
donner le principal, et en attendant pour deux millions de pierreries en
gage qu'il garderait jusqu'à entier payement des deux millions.

M. le duc d'Orléans fut agréablement trompé par les applaudissements que
le public donna à une acquisition si belle et si unique. Ce diamant fut
appelé \emph{le Régent}. Il est de la grosseur d'une prune de la reine
Claude, d'une forme presque ronde, d'une épaisseur qui répond à son
volume, parfaitement blanc, exempt de toute tache, nuage et paillette,
d'une eau admirable, et pèse plus de cinq cents grains. Je m'applaudis
beaucoup d'avoir résolu le régent à une emplette si illustre.

\hypertarget{chapitre-xviii.}{%
\chapter{Chapitre XVIII.}\label{chapitre-xviii.}}

1717

~

{\textsc{Le czar vient en France, et ce voyage importune.}} {\textsc{-
Origine de la haine personnelle du czar pour le roi d'Angleterre.}}
{\textsc{- Kurakin ambassadeur de Russie en France\,; quel.}} {\textsc{-
Motifs et mesures du czar qui veut, puis ne veut plus être catholique.}}
{\textsc{- Courte réflexion sur Rome.}} {\textsc{- Il est reçu à
Dunkerque par les équipages du roi, et à Calais par le marquis de
Nesle.}} {\textsc{- Il est en tout défrayé avec toute sa suite.}}
{\textsc{- On lui rend parfois les mêmes honneurs qu'au roi.}}
{\textsc{- On lui prépare des logements au Louvre et à l'hôtel de
Lesdiguières, qu'il choisit.}} {\textsc{- Je propose au régent le
maréchal de Tessé pour le mettre auprès du czar pendant son séjour, qui
l'attend à Beaumont.}} {\textsc{- Vie que menait le maréchal de Tessé.}}
{\textsc{- Journal du séjour du czar à Paris.}} {\textsc{- Verton,
maître d'hôtel du roi, chargé des tables du czar et de sa suite, gagne
les bonnes grâces du czar.}} {\textsc{- Grandes qualités du czar\,; sa
conduite à Paris.}} {\textsc{- Sa figure\,; son vêtement\,; sa
nourriture.}} {\textsc{- Le régent visite le czar.}} {\textsc{- Le roi
visite le czar en cérémonie.}} {\textsc{- Le czar visite le roi en toute
pareille cérémonie.}} {\textsc{- Le czar voit les places du roi en
relief.}} {\textsc{- Le czar visite Madame, qui l'avait envoyé
complimenter\,; puis {[}va{]} à l'Opéra avec M. le duc d'Orléans, qui là
lui sert à boire.}} {\textsc{- Le czar aux Invalides.}} {\textsc{-
M\textsuperscript{me} la duchesse de Berry et M\textsuperscript{me} la
duchesse d'Orléans, perdant espérance d'ouïr parler du czar, envoient
enfin le complimenter.}} {\textsc{- Il ne distingue les princes du sang
en rien, et trouve mauvais que les princesses du sang prétendissent
qu'il les visitât.}} {\textsc{- Il visite M\textsuperscript{me} la
duchesse de Berry.}} {\textsc{- Dîne avec M. le duc d'Orléans à
Saint-Cloud, et visite M\textsuperscript{me} la duchesse d'Orléans au
Palais-Royal.}} {\textsc{- Voit le roi comme par hasard aux Tuileries.}}
{\textsc{- Le czar va à Versailles.}} {\textsc{- Dépense pour le czar.}}
{\textsc{- Il va à Petit-Bourg et à Fontainebleau\,; voit en revenant
Choisy, et par hasard M\textsuperscript{me} la princesse de Conti un
moment, qui y était demeurante.}} {\textsc{- Le czar va passer plusieurs
jours à Versailles, Trianon et Marly\,; voit Saint-Cyr\,; fait à
M\textsuperscript{me} de Maintenon une visite insultante.}} {\textsc{-
Je vais voir le czar chez d'Antin tout à mon aise sans en être connu.}}
{\textsc{- M\textsuperscript{me} la duchesse l'y va voir par
curiosité.}} {\textsc{- Il en est averti\,; il passe devant elle, la
regarde, et ne fait ni la moindre civilité, ni semblant de rien.}}
{\textsc{- Présents.}} {\textsc{- Le régent va dire adieu au czar,
lequel va dire adieu au roi sans cérémonie, et reçoit chez lui celui du
roi de même.}} {\textsc{- Départ du czar, qui ne veut être accompagné de
personne.}} {\textsc{- Il va trouver la czarine à Spa.}} {\textsc{- Le
czar visite le régent.}} {\textsc{- Personnes présentées au czar.}}
{\textsc{- Maréchal de Tessé commande tous les officiers du roi servant
le czar.}} {\textsc{- Le czar, en partant, s'attendrit sur la France et
sur son luxe.}} {\textsc{- Il refuse le régent qui, à la prière du roi
d'Angleterre, désirait qu'il retirât ses troupes du Mecklenbourg.}}
{\textsc{- Il désire ardemment de s'unir avec la France, sans pouvoir
réussir, à notre grand et long dommage, par l'intérêt de l'abbé Dubois
et l'infatuation de l'Angleterre funestement transmise à ses
successeurs.}}

~

Pierre Ier, czar de Moscovie, s'est fait avec justice un si grand nom
chez lui et par toute l'Europe et l'Asie, que je n'entreprendrai pas de
faire connaître un prince si grand, si illustre, comparable aux plus
grands hommes de l'antiquité, qui a fait l'admiration de son siècle, qui
sera celle des siècles suivants, et que toute l'Europe s'est si fort
appliquée à connaître. La singularité du voyage en France d'un prince si
extraordinaire m'a paru mériter de n'en rien oublier, et la narration de
n'être point interrompue. C'est par cette raison que je la place ici un
peu plus tard qu'elle ne devrait l'être dans l'ordre du temps, mais dont
les dates rectifieront le défaut.

On a vu en son temps diverses choses de ce monarque\,; ses différents
voyages en Hollande, Allemagne, Vienne, Angleterre et dans plusieurs
parties du nord\,; l'objet de ces voyages et quelques choses de ses
actions militaires, de sa politique, de sa famille. On a vu aussi qu'il
avait voulu venir en France dans les dernières années du feu roi, qui
l'en fit honnêtement détourner. N'ayant plus cet obstacle, il voulut
contenter sa curiosité, et il fit dire au régent par le prince Kurakin,
son ambassadeur ici, qu'il allait partir des Pays-Bas où il était pour
venir voir le roi.

Il n'y eut pas moyen de n'en pas paraître fort aise, quoique le régent
s'en fût bien volontiers passé. La dépense était grande à le défrayer\,;
l'embarras pas moins grand avec un si puissant prince et si
clairvoyant\,; mais plein de fantaisies, avec un reste de moeurs
barbares et une grande suite de gens d'une conduite fort différente de
la commune de ces pays-ci, pleins de caprices et de façons étranges, et
leur maître et eux très délicats et très entiers sur ce qu'ils
prétendaient leur être dû ou permis.

Le czar de plus était avec le roi d'Angleterre en inimitié ouverte qui
allait entre eux jusqu'à l'indécence et d'autant plus vive qu'elle était
personnelle\,; ce qui ne gênait pas peu le régent dont l'intimité avec
le roi d'Angleterre était publique, et que l'intérêt personnel de l'abbé
Dubois portait fort indécemment aussi jusqu'à la dépendance. La passion
dominante du czar était de rendre ses États florissants par le commerce.
Il y avait fait faire quantité de canaux pour le faciliter. Il y en eut
un pour lequel il eut besoin du concours du roi d'Angleterre, parce
qu'il traversait un petit coin de ses États d'Allemagne. La jalousie du
commerce empêcha Georges d'y consentir. Pierre, engagé dans la guerre de
Pologne, puis dans celle du Nord, dans laquelle Georges l'était aussi,
négocia vainement. Il en fut d'autant plus irrité, qu'il ne le trouvait
pas en situation d'agir par la force, et que ce canal, extrêmement
avancé, ne put être continué. Telle fut la source de cette haine, qui a
duré toute leur vie et dans la plus vive aigreur.

Kurakin était d'une branche de cette ancienne maison des Jagellons, qui
avait longtemps porté les couronnes de Pologne, de Danemark, de Norvège
et de Suède. C'était un grand homme bien fait, qui sentait fort la
grandeur de son origine, avec beaucoup d'esprit, de tour et
d'instruction. Il parlait assez bien français et plusieurs langues\,; il
avait fort voyagé, servi à la guerre, puis été employé en différentes
cours. Il ne laissait pas de sentir encore le russe, et l'extrême
avarice gâtait fort ses talents. Le czar et lui avaient épousé les deux
soeurs, et en avaient chacun un fils. La czarine avait été répudiée et
mise dans un couvent près de Moscou, sans que Kurakin se fût senti de
cette disgrâce. Il connaissait parfaitement son maître avec qui il avait
conservé de la liberté, de la confiance et beaucoup de considération\,;
en dernier lieu, il avait été trois ans à Rome, d'où il était venu à
Paris ambassadeur. À Rome, il était sans caractère et sans affaires que
la secrète pour laquelle le czar l'y avait envoyé comme un homme sûr et
éclairé.

Ce monarque, qui se voulait tirer lui et son pays de leur barbarie et
s'étendre par des conquêtes et des traités, avait compris la nécessité
des mariages pour s'allier avec les premiers potentats de l'Europe.
Cette grande raison lui rendait nécessaire la religion catholique, dont
les grecs se trouvaient séparés de si peu qu'il ne jugea pas son projet
difficile à faire recevoir chez lui en y laissant d'ailleurs la liberté
de conscience. Mais ce prince instruit l'était assez pour vouloir être
auparavant éclairci sur les prétentions romaines. Il avait envoyé pour
cela à Rome un homme obscur, mais capable de se bien informer, qui y
passa cinq ou six mois, et qui ne lui rapporta rien de satisfaisant. Il
s'en ouvrit, en Hollande, au roi Guillaume, qui le dissuada de son
dessein, et qui lui conseilla même d'imiter l'Angleterre, et de se faire
lui-même chef de la religion chez lui, sans quoi il n'y serait jamais
bien le maître. Ce conseil plut d'autant plus au czar que c'était par
les biens et par l'autorité des patriarches de Moscou, ses grand-père et
bisaïeul, que son père était parvenu à la couronne, quoique d'une
condition ordinaire parmi la noblesse russienne.

Ces patriarches dépendaient pourtant de ceux du rite grec de
Constantinople, mais fort légèrement. Ils s'étaient saisis d'un grand
pouvoir et d'un rang prodigieux, jusque-là qu'à leur entrée à Moscou, le
czar leur tenait l'étrier et conduisait à pied leur cheval par la bride.
Depuis le grand-père de Pierre, il n'y avait point eu de patriarche à
Moscou. Pierre Ier, qui avait régné quelque temps avec son frère aîné,
qui n'en était pas capable, et qui était mort sans laisser de fils, il y
avait longtemps, n'avait jamais voulu de patriarche non plus que son
père. Les archevêques de Novogorod y suppléaient en certaines choses
comme occupant le premier siège après celui de Moscou, mais sans presque
d'autorité que le czar usurpa tout entière, et plus soigneusement encore
depuis le conseil que le roi Guillaume lui avait donné, en sorte que peu
à peu il s'était fait le véritable chef de la religion dans ses vastes
États.

Néanmoins la passion de pouvoir ouvrir à sa postérité la facilité de
faire des mariages avec des princes catholiques, l'honneur surtout de
les allier à la maison de France et à celle d'Autriche, le fit revenir à
son premier projet. Il se voulut flatter que celui qu'il avait envoyé
secrètement à Rome n'avait pas été bien informé, ou qu'il avait mal
compris\,; il résolut donc d'approfondir ses doutes, de manière qu'il ne
lui en restât plus sur le parti qu'il aurait à prendre.

Ce fut dans ce dessein qu'il choisit le prince Kurakin, dont les
lumières et l'intelligence lui étaient connues, pour aller à Rome sous
prétexte de curiosité, dans la vue qu'un seigneur de cette qualité
s'ouvrirait l'entrée chez ce qu'il y avait de meilleur, de plus
important et de plus distingué à Rome, et qu'en y demeurant, sous
prétexte d'en aimer la vie et de vouloir tout voir à son aise et admirer
à son gré toutes les merveilles qui y sont rassemblées en tant de
genres, il aurait loisir et moyen de revenir parfaitement instruit de
tout ce qu'il voulait savoir. Kurakin y demeura, en effet, trois ans
mêlé avec les savants d'une part, et avec la meilleure compagnie de
l'autre, d'où peu à peu il tira ce qu'il voulut apprendre avec d'autant
plus de facilité que cette cour triomphe de ses prétentions temporelles,
de ses conquêtes en ce genre, au lieu de les tenir dans le secret. Sur
le rapport long et fidèle que Kurakin en fit au czar, ce prince poussa
un soupir en disant qu'il voulait être maître chez lui, et n'y en pas
mettre un plus grand que soi, et oncques depuis ne songea à se faire
catholique.

Tels sont les biens que les papes et leur cour font à l'Église, et
qu'ils procurent aux âmes dont ce vicaire de Jésus-Christ, qui les a
rachetées, est le grand pasteur, et dont sur la sienne il répondra au
souverain Pasteur, qui a déclaré à saint Pierre comme aux autres apôtres
que son royaume n'est pas de ce monde, et qui demanda à ces deux frères,
qui le voulurent prendre pour juge de leur différend sur leur héritage,
qui l'avait établi sur eux en cette qualité\,? et qui ne s'en voulut
point mêler quoique ce fût une bonne oeuvre que d'accorder deux frères,
pour enseigner aux pasteurs et aux prêtres par un si grand exemple et si
précis, qu'ils n'ont aucun pouvoir ni aucun droit sur le temporel par
quelque raison que ce puisse être, et qu'ils sont essentiellement exclus
de s'en mêler.

Ce fait du czar sur Rome, le prince Kurakin ne s'en est pas caché. Tout
ce qui l'a connu le lui a ouï conter\,; j'ai mangé chez lui et lui chez
moi, et je l'ai fort entretenu et ouï discourir avec plaisir sur
beaucoup de choses.

Le régent averti par lui de la prochaine arrivée du czar en France, par
le côté maritime, envoya les équipages du roi, chevaux, carrosses,
voitures, fourgons, tables et chambres, avec du Libois, un des
gentilshommes ordinaires du roi, dont j'ai quelquefois parlé, pour aller
attendre le czar à Dunkerque, le défrayer jusqu'à Paris de tout et toute
sa suite, et lui faire rendre partout les mêmes honneurs qu'au roi même.
Ce monarque se proposait de donner cent jours à son voyage. On meubla
pour lui l'appartement de la reine mère au Louvre, où il se tenait
divers conseils, qui s'assemblèrent chez les chefs depuis cet ordre.

M. le duc d'Orléans, raisonnant avec moi sur le seigneur titré qu'il
pourrait choisir pour mettre auprès du czar pendant son séjour, je lui
conseillai le maréchal de Tessé comme un homme qui n'avait rien à faire,
qui avait fort l'usage et le langage du monde, fort accoutumé aux
étrangers par ses voyages de guerre et de négociations en Espagne, à
Turin, à Rome, en d'autres cours d'Italie, qui avait de la douceur et de
la politesse, et qui sûrement y ferait fort bien. M. le duc d'Orléans
trouva que j'avais raison, et dès le lendemain l'envoya chercher et lui
donna ses ordres.

C'était un homme qui avait toujours été dans des liaisons fort
contraires à M. le duc d'Orléans et qui était demeuré avec lui fort sur
le pied gauche. Embarrassé de sa personne, il avait pris un air de
retraite. Il s'était mis dans un bel appartement aux Incurables. Il en
avait pris un autre aux Camaldules, près de Grosbois. Il avait dans ces
deux endroits de quoi loger toute sa maison. Il partageait sa semaine
entre cette maison de ville et cette maison de campagne. Il donnait dans
l'une et dans l'autre à manger tant qu'il pouvait, et avec cela se
prétendait dans la retraite. Il fut donc fort aise d'être choisi pour
faire les honneurs au czar, se tenir près de lui, l'accompagner partout,
lui présenter tout le monde. C'était aussi son vrai ballot, et il s'en
acquitta très bien.

Quand on sut le czar proche de Dunkerque, le régent envoya le marquis de
Nesle\footnote{Le marquis d'Argenson rapporte sur ce personnage,
  l'anecdote suivante (\emph{Mémoires}, édit. de 1825, p.~193-194)\,:
  «\,Le marquis de Nesle avait brigué la mission d'aller au-devant du
  czar Pierre et de lui faire les honneurs de la France, lors du voyage
  de ce prince au commencement de ce règne. On sait que le marquis se
  pique d'une extrême magnificence. Il avait si bien pris ses mesures
  qu'il changeait d'habit tous les jours. Toute l'attention que cette
  recherche lui attira du czar fut que ce prince dit à quelqu'un\,:
  «\,En, vérité, je plains M. de Nesle d'avoir un si mauvais tailleur
  qu'il ne puisse trouver un habit fait à sa guise.\,»} le recevoir à
Calais et l'accompagner jusqu'à l'arrivée du maréchal de Tessé, qui ne
devait aller que jusqu'à Beaumont au-devant de lui. En même temps on fit
préparer l'hôtel de Lesdiguières pour le czar et sa suite\,; dans le
doute qu'il n'aimât mieux une maison particulière avec tous ses gens
autour de lui que le Louvre. L'hôtel de Lesdiguières était grand et
beau, touchant à l'Arsenal, et appartenait au maréchal de Villeroy, qui
logeait aux Tuileries. Ainsi la maison était vide, parce que le duc de
Villeroy, qui n'était pas homme à grand train, l'avait trouvée trop
éloignée pour y loger. On le meubla entièrement et très magnifiquement
des meubles du roi.

Le maréchal de Tessé attendit un jour le czar à Beaumont à tout hasard
pour ne le pas manquer. Il y arriva le vendredi 7 mai sur le midi. Tessé
lui fit la révérence à la descente de son carrosse, eut l'honneur de
dîner avec lui, et de l'amener le jour même à Paris.

Il voulut entrer dans Paris dans un carrosse du maréchal, mais sans lui,
avec trois de ceux de sa suite. Le maréchal le suivait dans un autre. Il
descendit à neuf heures du soir au Louvre, entra partout dans
l'appartement de la reine mère. Il le trouva trop magnifiquement tendu
et éclairé, remonta tout de suite en carrosse et s'en alla à l'hôtel de
Lesdiguières, où il voulut loger. Il en trouva aussi l'appartement qui
lui était destiné trop beau, et tout aussitôt fit tendre son lit de camp
dans une garde-robe. Le maréchal de Tessé, qui devait faire les honneurs
de sa maison et de sa table, l'accompagner partout et ne point quitter
le lieu où il serait, logea dans un appartement de l'hôtel de
Lesdiguières, et eut beaucoup à faire à le suivre et souvent à courir
après lui. Verton, un des maîtres d'hôtel du roi, fut chargé de le
servir et de toutes les tables tant du czar que de sa suite. Elle était
d'une quarantaine de personnes de toutes les sortes, dont il y en avait
douze ou quinze de gens considérables par eux-mêmes ou par leurs
emplois, qui mangeaient avec lui.

Verton était un garçon d'esprit, fort d'un certain monde, homme de bonne
chère et de grand jeu, qui fit servir le czar avec tant d'ordre, et sut
si bien se conduire, que le czar le prit en singulière amitié ainsi que
toute sa suite.

Ce monarque se fit admirer par son extrême curiosité toujours tendante à
ses vues de gouvernement, de commerce, d'instruction, de police\,; et
cette curiosité atteignit à tout et ne dédaigna rien dont les moindres
traits avaient une utilité suivie, marquée, savante, qui n'estima que ce
qui méritait l'être, en qui brilla l'intelligence, la justesse, la vive
appréhension de son esprit. Tout montrait en lui la vaste étendue de ses
lumières et quelque chose de continuellement conséquent. Il allia d'une
manière tout à fait surprenante la majesté la plus haute, la plus fière,
la plus délicate, la plus soutenue, en même temps la moins embarrassante
quand il l'avait établie dans toute sa sûreté avec une politesse qui la
sentait, et toujours et avec tous et en maître partout, mais qui avait
ses degrés suivant les personnes. Il avait une sorte de familiarité qui
venait de liberté\,; mais il n'était pas exempt d'une forte empreinte de
cette ancienne barbarie de son pays qui rendait toutes ses manières
promptes, même précipitées, ses volontés incertaines, sans vouloir être
contraint ni contredit sur pas une. Sa table, souvent peu décente,
beaucoup moins ce qui la suivait, souvent aussi avec un découvert
d'audace et d'un roi partout chez soi, ce qu'il se proposait de voir ou
de faire toujours dans l'entière indépendance des moyens qu'il fallait
forcer à son plaisir et à son mot. Le désir de voir à son aise,
l'importunité d'être en spectacle, l'habitude d'une liberté au-dessus de
tout lui faisait souvent préférer les carrosses de louage, les fiacres
mêmes, le premier carrosse qu'il trouvait sous sa main de gens qui
étaient chez lui et qu'il ne connaissait pas. Il sautait dedans et se
faisait mener par la ville ou dehors. Cette aventure arriva à
M\textsuperscript{me} de Matignon, qui était allée là bayer, dont il
mena le carrosse à Boulogne et dans d'autres lieux de campagne, qui fut
bien étonnée de se trouver à pied. Alors c'était au maréchal de Tessé et
à sa suite, dont il s'échappait ainsi, à courir après, quelquefois sans
le pouvoir trouver.

C'était un fort grand homme, très bien fait, assez maigre, le visage
assez de forme ronde\,; un grand front\,; de beaux sourcils\,; le nez
assez court sans rien de trop gros par le bout\,; les lèvres assez
grosses\,; le teint rougeâtre et brun\,; de beaux yeux noirs, grands,
vifs, perçants, bien fendus\,; le regard majestueux et gracieux quand il
y prenait garde, sinon sévère et farouche, avec un tic qui ne revenait
pas souvent, mais qui lui démontait les yeux et toute la physionomie, et
qui donnait de la frayeur. Cela durcit un moment avec un regard égaré et
terrible, et se remettait aussitôt. Tout son air marquait son esprit, sa
réflexion et sa grandeur, et ne manquait pas d'une certaine grâce. Il ne
portait qu'un col de toile, une perruque ronde brune, comme sans poudre,
qui ne touchait pas ses épaules, un habit brun juste au corps, uni, à
boutons d'or, veste, culotte, bas, point de gants ni de manchettes,
l'étoile de son ordre sur son habit et le cordon par dessous, son habit
souvent déboutonné tout à fait, son chapeau sur une table et jamais sur
sa tête, même dehors. Dans cette simplicité, quelque mal voituré et
accompagné qu'il pût être, on ne s'y pouvait méprendre à l'air de
grandeur qui lui était naturel.

Ce qu'il buvait et mangeait en deux repas réglés est inconcevable, sans
compter ce qu'il avalait de bière, de limonade et d'autres sortes de
boissons entre les repas, toute sa suite encore davantage\,; une
bouteille ou deux de bière, autant et quelquefois davantage de vin, des
vins de liqueur après, à la fin du repas des eaux-de-vie préparées,
chopine et quelquefois pinte. C'était à peu près l'ordinaire de chaque
repas. Sa suite à sa table en avalait davantage, et {[}ils{]} mangeaient
tous à l'avenant à onze heures du matin et à huit du soir. Quand la
mesure n'était pas plus forte, il n'y paraissait pas. Il y avait un
prêtre aumônier qui mangeait à la table du czar, plus fort de moitié que
pas un, dont le czar, qui l'aimait, s'amusait beaucoup. Le prince
Kurakin allait tous les jours à l'hôtel de Lesdiguières\,; mais il
demeura logé chez lui.

Le czar entendait bien le français, et, je crois, l'aurait parlé s'il
eût voulu\,; mais, par grandeur, il avait toujours un interprète. Pour
le latin et bien d'autres langues, il les parlait très bien. Il eut chez
lui une salle des gardes du roi, dont il ne voulut presque jamais être
suivi dehors. Il ne voulut point sortir de l'hôtel de Lesdiguières,
quelque curiosité qu'il eût, ni donner aucun signe de vie, qu'il n'y eût
reçu la visite du roi.

Le samedi matin, lendemain de son arrivée, le régent alla voir le czar.
Ce monarque sortit de son cabinet, fit quelques pas au-devant de lui,
l'embrassa avec un grand air de supériorité, lui montra la porte de son
cabinet, et, se tournant à l'instant sans nulle civilité, y entra. Le
régent le suivit, et le prince Kurakin après lui, pour leur servir
d'interprète. Ils trouvèrent deux fauteuils vis-à-vis l'un de l'autre\,;
le czar s'assit en celui du haut bout, le régent dans l'autre. La
conversation dura près d'une heure, sans parler d'affaires, après quoi
le czar sortit de son cabinet, le régent après lui, qui, avec une
profonde révérence médiocrement rendue, le quitta au même endroit où il
l'avait trouvé en entrant.

Le lundi suivant 10 mai, le roi alla voir le czar, qui le reçut à sa
portière, le vit descendre de carrosse, et marcha de front à la gauche
du roi jusque dans sa chambre où ils trouvèrent deux fauteuils égaux. Le
roi s'assit dans celui de la droite, le czar dans celui de la gauche, le
prince Kurakin servit d'interprète. On fut étonné de voir le czar
prendre le roi sous les deux bras, le hausser à son niveau, l'embrasser
ainsi en l'air, et le roi à son âge, et qui n'y pouvait pas être
préparé, n'en avoir aucune frayeur. On fut frappé de toutes les grâces
qu'il montra devant le roi, de l'air de tendresse qu'il prit pour lui,
de cette politesse qui coulait de source, et toutefois mêlée de
grandeur, d'égalité de rang, et légèrement de supériorité d'âge\,; car
tout cela se fit très distinctement sentir. Il loua fort le roi, il en
parut charmé, et il en persuada tout le monde. Il l'embrassa à plusieurs
reprises. Le roi lui fit très joliment son petit et court compliment, et
M. du Maine, le maréchal de Villeroy, et ce qui se trouva là de
distingué fournirent la conversation. La séance dura un petit quart
d'heure. Le czar accompagna le roi comme il l'avait reçu, et le vit
monter en carrosse.

Le mardi 11 mai, le czar alla voir le roi entre quatre et cinq heures.
Il fut reçu du roi à la portière de son carrosse, et conduit de même,
eut la droite sur le roi partout. On était convenu de tout le
cérémonial, avant que le roi l'allât voir. Le czar montra les mêmes
grâces et la même affection pour le roi, et sa visite ne fut pas plus
longue que celle qu'il en avait reçue\,; mais la foule le surprit fort.

Il était allé dès huit heures du matin voir les places Royale, des
Victoires et de Vendôme, et le lendemain il fut voir l'Observatoire, les
manufactures des Gobelins et le Jardin du Roi des simples. Partout là il
s'amusa beaucoup à tout examiner et à faire beaucoup de questions.

Le jeudi 13 mai, il se purgea, et ne laissa pas l'après-dînée d'aller
chez plusieurs ouvriers de réputation. Le vendredi 14, il alla dès six
heures du matin dans la grande galerie du Louvre voir les plans en
relief de toutes les places du roi, dont Asfeld avec ses ingénieurs lui
fit les honneurs. Le maréchal de Villars s'y trouva aussi pour la même
raison avec quelques lieutenants généraux. Il examina fort longtemps
tous ces plans\,; il visita ensuite beaucoup d'endroits du Louvre, et
descendit après dans le jardin des Tuileries, dont on avait fait sortir
tout le monde. On travaillait alors au Pont-Tournant. Il examina fort
cet ouvrage, et y demeura longtemps. L'après-dînée, il alla voir Madame
au Palais-Royal, qui l'avait envoyé complimenter par son chevalier
d'honneur. Excepté le fauteuil, elle le reçut comme elle aurait fait le
roi. M. le duc d'Orléans l'y vint, prendre pour le mener à l'Opéra dans
sa grande loge, tous deux seuls sur le banc de devant avec un grand
tapis. Quelque temps après, le czar demanda s'il n'y aurait point de
bière. Tout aussitôt on en apporta un grand gobelet sur une soucoupe. Le
régent se leva, la prit, et la présenta au czar, qui, avec un sourire et
une inclination de politesse, prit le gobelet sans aucune façon, but et
le remit sur la coupe, que le régent tint toujours. En la rendant, il
prit une assiette qui portait une serviette, qu'il présenta au czar,
qui, sans se lever, en usa comme il avait fait pour la bière, dont le
spectacle parut assez étonné. Au quatrième acte il s'en alla souper, et
ne voulut pas que le régent quittât la loge. Le lendemain samedi, il se
jeta dans un carrosse de louage, et alla voir quantité de curiosités
chez les ouvriers.

Le 16 mai, jour de la Pentecôte, il alla aux Invalides, où il voulut
tout voir et tout examiner partout. Au réfectoire, il goûta de la soupe
des soldats et de leur vin, but à leur santé, leur frappant sur
l'épaule, et les appelant camarades. Il admira beaucoup l'église,
l'apothicairerie et l'infirmerie, et parut charmé de l'ordre de cette
maison. Le maréchal de Villars lui en fit les honneurs. La maréchale de
Villars y alla pour le voir comme bayeuse. Il sut que c'était elle, et
lui fit beaucoup d'honnêtetés.

Lundi 17 mai, il dîna de bonne heure avec le prince Ragotzi, qu'il en
avait prié, et alla après voir Meudon, où il trouva des chevaux du roi
pour voir les jardins et le parc à son aise. Le prince Ragotzi l'y
accompagna.

Mardi 18, le maréchal d'Estrées le vint prendre à huit heures du matin
et le mena, dans son carrosse, à sa maison d'Issy, où il lui donna à
dîner, et l'amusa fort le reste de la journée avec beaucoup de choses
qu'il lui fit voir touchant la marine.

Mercredi 19, il s'occupa de plusieurs ouvrages et ouvriers.
M\textsuperscript{me} la duchesse de Berry et M\textsuperscript{me} la
duchesse d'Orléans, à l'exemple de Madame, envoyèrent le matin
complimenter le czar par leurs premiers écuyers. Elles en avaient toutes
trois espéré un compliment ou même une visite. Elles se lassèrent de
n'en point entendre parler, et à la fin se ravisèrent. Le czar répondit
qu'il irait les remercier. Des princes et princesses du sang, il ne s'en
embarrassa pas plus que des premiers seigneurs de la cour, et ne les
distingua pas davantage. Il avait trouvé mauvais que les princes du sang
eussent fait difficulté de l'aller voir, s'ils n'étaient assurés qu'il
rendrait une visite aux princesses du sang, ce qu'il rejeta avec grande
hauteur, tellement qu'aucune d'elles ne le vit que par curiosité, en
voyeuse, excepté M\textsuperscript{me} la princesse de Conti, par
hasard. Tout cela s'expliquera dans la suite.

Jeudi 20 mai, il devait aller dîner à Saint-Cloud, où M. le duc
d'Orléans l'attendait avec cinq ou six courtisans seulement, mais un peu
de fièvre qu'il eut la nuit l'obligea le matin de s'envoyer excuser.

Vendredi 21, il alla voir M\textsuperscript{me} la duchesse de Berry au
Luxembourg, où il fut reçu comme le roi. Après sa visite il se promena
dans les jardins. M\textsuperscript{me} la duchesse de Berry s'en alla
cependant à la Muette pour lui laisser la liberté de voir toute sa
maison, qu'il visita fort curieusement. Comptant partir vers le 16 juin,
il demanda des bateaux pour ce temps-là à Charleville, dans le dessein
de descendre la Meuse.

Samedi 22, il fut à Bercy, chez Pajot d'Ons-en-Bray, principal directeur
de la poste, dont la maison est pleine de toutes sortes de raretés et de
curiosités, tant naturelles que mécaniques. Le célèbre P. Sébastien,
carme, y était. Il s'y amusa tout le jour, et y admira plusieurs belles
machines.

Le dimanche 23 mai, il fut dîner à Saint-Cloud, où M. le duc d'Orléans
l'attendait\,; il vit la maison et les jardins, qui lui plurent fort\,;
passa, en s'en retournant, au château de Madrid, qu'il visita, et alla
de là voir M\textsuperscript{me} la duchesse d'Orléans au Palais-Royal,
où, parmi beaucoup de politesses, il ne laissa pas de montrer un grand
air de supériorité, ce qu'il avait bien moins marqué chez Madame et chez
M\textsuperscript{me} la duchesse de Berry.

Lundi 24, il alla aux Tuileries de bonne heure, avant que le roi fût
levé. Il entra chez le maréchal de Villeroy, qui lui fit voir les
pierreries de la couronne. Il les trouva plus belles et en plus grand
nombre qu'il ne pensait, mais il dit qu'il ne s'y connaissait guère. Il
témoignait faire peu de cas des beautés purement de richesses et
d'imagination, de celles surtout auxquelles il ne pouvait atteindre. De
là, il voulut aller voir le roi qui, de son côté, venait le trouver chez
le maréchal de Villeroy. Cela fut compassé exprès pour que ce ne fût
point une visite marquée, mais comme de hasard. Ils se rencontrèrent
dans un cabinet, où ils demeurèrent. Le roi, qui tenait un rouleau de
papier à la main, le lui donna, et lui dit que c'était la carte de ses
États. Cette galanterie plut fort au czar, dont la politesse et l'air
d'amitié et d'affection furent les mêmes, avec beaucoup de grâce, mais
de majesté et d'égalité.

L'après-dînée il alla à Versailles où le maréchal de Tessé le laissa au
duc d'Antin, chargé de lui en faire les honneurs. L'appartement de
M\textsuperscript{me} la Dauphine était préparé pour lui, et il coucha
dans la communication de Mgr le Dauphin, père du roi, qui fait à cette
heure des cabinets pour la reine.

Mardi 25, il avait parcouru les jardins, et s'était embarqué sur le
canal dès le grand matin, avant l'heure qu'il avait donnée à d'Antin
pour se rendre chez lui. Il vit tout Versailles, Trianon et la
Ménagerie. Sa principale suite fut logée au château. Ils menèrent avec
eux des demoiselles qu'ils firent coucher dans l'appartement qu'avait
M\textsuperscript{me} de Maintenon tout proche de celui où le czar
couchait. Bloin, gouverneur de Versailles, fut extrêmement scandalisé de
voir profaner ainsi ce temple de la pruderie, dont la déesse et lui qui
étaient vieux l'auraient été moins autrefois. Ce n'était pas la manière
du czar ni de ses gens de se contraindre.

Mercredi 26, le czar, qui s'amusa fort tout le jour à Marly et à la
machine, manda au maréchal de Tessé à Paris qu'il y arriverait le
lendemain matin à huit heures à l'hôtel de Lesdiguières, où il comptait
le trouver, et qu'il le mènerait en lieu de voir la procession de la
Fête-Dieu. Le maréchal lui fit voir celle de Notre-Dame.

Le défrai de ce prince coûtait six cents écus par jour, quoiqu'il eût
beaucoup fait diminuer sa table dès les premiers jours. Il eut un moment
envie de faire venir à Paris la czarine qu'il aimait beaucoup\,; mais il
changea bientôt d'avis. Il la fit aller à Aix-la-Chapelle ou à Spa, à
son choix, pour y prendre des eaux en l'attendant.

Dimanche 30 mai, il partit avec Bellegarde, fils et survivancier de
d'Antin pour les bâtiments, et beaucoup de relais pour aller dîner chez
d'Antin à Petit-Bourg, qui l'y reçut et le mena l'après-dînée voir
Fontainebleau où il coucha, et le lendemain à une chasse du cerf de
laquelle le comte de Toulouse lui fit les honneurs. Le lieu lui plut
médiocrement, et point du tout la chasse où il pensa tomber de cheval\,;
il trouva cet exercice trop violent, qu'il ne connaissait point. Il
voulut manger seul avec ses gens au retour dans l'île de l'Étang de la
cour des Fontaines. Ils s'y dédommagèrent de leurs fatigues. Il revint à
Petit-Bourg seul dans un carrosse avec trois de ses gens. Il parut dans
ce carrosse qu'ils avaient largement bu et mangé.

Mardi le 1er juin, il s'embarqua au bas de la terrasse de Petit-Bourg
pour revenir par eau à Paris. Passant devant Choisy, il se fit arrêter,
et voulut voir la maison et les jardins. Cette curiosité l'obligea
d'entrer un moment chez M\textsuperscript{me} la princesse de Conti qui
y était. Après s'être promené il se rembarqua, et il voulut passer sous
tous les ponts de Paris.

Jeudi 3 juin, octave de la Fête-Dieu, il vit de l'hôtel de Lesdiguières
la procession de la paroisse de Saint-Paul. Le même jour il alla coucher
encore à Versailles, qu'il voulut revoir avec plus de loisir\,; il s'y
plut fort, et voulut aussi coucher à Trianon, puis trois ou quatre nuits
à Marly dans les pavillons les plus près du château qu'on lui prépara.

Vendredi 11 juin, il fut de Versailles à Saint-Cyr où il vit toute la
maison et les demoiselles dans leurs classes. Il y fut reçu comme le
roi. Il voulut aussi voir M\textsuperscript{me} de Maintenon qui dans
l'apparence de cette curiosité s'était mise au lit, ses rideaux fermés
hors un qui ne l'était qu'à demi. Le czar entra dans sa chambre, alla
ouvrir les rideaux des fenêtres en arrivant, puis tout de suite tous
ceux du lit, regarda bien M\textsuperscript{me} de Maintenon tout à son
aise, ne lui dit pas un mot ni elle à lui, et sans lui faire aucune
sorte de révérence, s'en alla. Je sus qu'elle en avait été fort étonnée
et encore plus mortifiée\,; mais le feu roi n'était plus. Il revint le
samedi 12 juin à Paris.

Le mardi 15 juin, il alla de bonne heure chez d'Antin à Paris.
Travaillant ce jour-là avec M. le duc d'Orléans, je finis en une
demi-heure\,; il en fut surpris et voulut me retenir. Je lui dis que
j'aurais toujours l'honneur de le trouver, mais non le czar qui s'en
allait, que je ne l'avais point vu, et que je m'en allais chez d'Antin
bayer tout à mon aise. Personne n'y entrait que les conviés et quelques
dames avec M\textsuperscript{me} la Duchesse et les princesses ses
filles qui voulaient bayer aussi. J'entrai dans le jardin où le czar se
promenait. Le maréchal de Tessé qui me vit de loin vint à moi, comptant
me présenter au czar. Je le priai de s'en bien garder et de ne point
s'apercevoir de moi en sa présence, parce que je voulais le regarder
tout à mon aise, le devancer et l'attendre tant que je voudrais pour le
bien contempler, ce que je ne pourrais plus faire si j'en étais connu.
Je le priai d'en avertir d'Antin, et avec cette précaution je satisfis
ma curiosité tout à mon aise. Je le trouvai assez parlant mais toujours
comme étant partout le maître. Il rentra dans un cabinet où d'Antin lui
montra divers plans et quelques curiosités, sur quoi il fit plusieurs
questions. Ce fut là où je vis ce tic dont j'ai parlé. Je demandai à
Tessé si cela lui arrivait souvent\,; il me dit plusieurs fois par jour,
surtout quand il ne prend pas garde à s'en contraindre. Rentrant après
dans le jardin, d'Antin lui fit raser l'appartement bas, et l'avertit
que M\textsuperscript{me} la Duchesse y était avec des dames qui avaient
grande envie de le voir. Il ne répondit rien et se laissa conduire. Il
marcha plus doucement, tourna la tête vers l'appartement où tout était
debout et sous les armes, mais en voyeuses. Il les regarda bien toutes
et ne fit qu'une très légère inclination de la tête à toutes à la fois
sans la tourner le long d'elles, et passa fièrement\,; je pense à la
façon dont il avait reçu d'autres dames qu'il aurait montré plus de
politesse à celles-ci, si M\textsuperscript{me} la Duchesse n'y eût pas
été, à cause de la prétention de la visite. Il affecta même de ne
s'informer pas laquelle c'était ni du nom de pas une des autres. Je fus
là près d'une heure à ne le point quitter et à le regarder sans cesse.
Sur la fin je vis qu'il le remarquait\,: cela me rendit plus retenu dans
la crainte qu'il ne demandât qui j'étais. Comme il allait rentrer, je
passai en m'en allant dans la salle où le couvert était mis. D'Antin
toujours le même avait trouvé moyen d'avoir un portrait très ressemblant
de la czarine qu'il avait mis sur la cheminée de cette salle, avec des
vers à sa louange, ce qui plut fort au czar dans sa surprise. Lui et sa
suite trouvèrent le portrait fort ressemblant.

Le roi lui donna deux magnifiques tentures de tapisseries des Gobelins.
Il lui voulut donner aussi une belle épée de diamants laquelle il
s'excusa d'accepter\,; lui, de son côté, fit distribuer environ soixante
mille livres aux domestiques du roi qui l'avaient servi, donna à d'Antin
et aux maréchaux d'Estrées et de Tessé à chacun son portrait enrichi de
diamants, cinq médailles d'or et onze d'argent des principales actions
de sa vie. Il fit un présent d'amitié à Verton et pria instamment le
régent de l'envoyer auprès de lui, chargé des affaires du roi, qui le
lui promit.

Mercredi 16 juin, il fut à cheval à la revue des deux régiments des
gardes, des gens d'armes, chevau-légers et mousquetaires. Il n'y avait
que M. le duc d'Orléans\,: le czar ne regarda presque pas ces troupes
qui s'en aperçurent. Il fut de là dîner-souper à Saint-Ouen, chez le duc
de Tresmes où il dit que l'excès de la chaleur de la poussière et de la
foule de gens à pied et à cheval lui avait fait quitter la revue plus
tôt qu'il n'aurait voulu. Le repas fut magnifique\,; il sut que la
marquise de Béthune qui y était en voyeuse était fille du duc de
Tresmes\,; il la pria de se mettre à table\,; ce fut la seule dame qui y
mangea avec beaucoup de seigneurs. Il y vint plusieurs dames aussi en
voyeuses à qui il fit beaucoup d'honnêtetés, quand il sut qui elles
étaient.

Jeudi 17, il alla pour la seconde fois à l'Obervatoire, et de là souper
chez le maréchal de Villars.

Vendredi 18 juin, le régent fut de bonne heure à l'hôtel de Lesdiguières
dire adieu au czar. Il fut quelque temps avec lui, le prince Kurakin en
tiers. Après cette visite, le czar alla dire adieu au roi aux Tuileries.
Il avait été convenu qu'il n'y aurait plus entre eux de cérémonies. On
ne peut montrer plus d'esprit, de grâces ni de tendresses pour le roi
que le czar en fit paraître en toutes ces occasions, et le lendemain
encore que le roi alla lui souhaiter à l'hôtel de Lesdiguières un bon
voyage, où tout se passa ainsi sans cérémonies.

Dimanche 20 juin, le czar partit et coucha à Livry, allant droit à Spa
où il était attendu par la czarine, et ne voulut être accompagné de
personne, pas même en sortant de Paris. Le luxe qu'il remarqua le
surprit beaucoup\,; il s'attendrit en partant sur le roi et sur la
France, et dit qu'il voyait avec douleur que ce luxe la perdrait
bientôt. Il s'en alla charmé de la manière dont il avait été reçu, de
tout ce qu'il avait vu, de la liberté qu'on lui avait laissée, et dans
un grand désir de s'unir étroitement avec le roi, à quoi l'intérêt de
l'abbé Dubois et de l'Angleterre fut un funeste obstacle dont on a
souvent eu et on a encore grand sujet de repentir.

On ne finirait point sur ce czar si intimement et si véritablement
grand, dont la singularité et la rare variété de tant de grands talents
et de grandeurs diverses, feront toujours un monarque digne de la plus
grande admiration jusque dans la postérité la plus reculée, malgré les
grands défauts de la barbarie de son origine, de son pays et de son
éducation. C'est la réputation qu'il laissa unanimement établie en
France, qui le regarda comme un prodige dont elle demeura charmée.

Je suis certain que le czar alla voir M. le duc d'Orléans dès les
premiers jours, qu'il ne lui rendit que cette unique visite au
Palais-Royal\,; que M. le duc d'Orléans le reçut et le conduisit à son
carrosse, que leur conversation s'y passa dans un cabinet, seuls avec le
prince Kurakin en tiers, et qu'elle dura assez longtemps. J'en ai oublié
le jour.

Ce monarque fut très content du maréchal de Tessé et de tout le service.
Ce maréchal commandait à tous les officiers de la maison du roi de tout
genre qui servirent le czar.

Beaucoup de gens se firent présenter à lui, mais de considération.
Beaucoup aussi ne se soucièrent pas de l'être\,; aucune dame ne le fut,
et les princes du sang ne le virent point, dont il ne témoigna rien que
par sa conduite avec eux, quand il en vit chez le roi. En partant il
s'attendrit sur la France, et dit qu'il voyait avec douleur que son
grand luxe la perdrait bientôt.

Il avait des troupes en Pologne et beaucoup dans le Mecklembourg\,; ces
dernières inquiétaient fort le roi d'Angleterre qui avait eu recours aux
officiers de l'empereur et à tous les moyens qu'il avait pu pour engager
le czar à les en retirer. Il pria instamment M. le duc d'Orléans de
tâcher de l'obtenir de ce prince tandis qu'il était en France. M. le duc
d'Orléans n'y oublia rien, mais sans succès.

Néanmoins le czar avait une passion extrême de s'unir avec la France.
Rien ne convenait mieux à notre commerce, à notre considération dans le
Nord, en Allemagne et par toute l'Europe. Ce prince tenait l'Angleterre
en brassière par le commerce, et le roi Georges en crainte pour ses
États d'Allemagne. Il tenait la Hollande en grand respect et l'empereur
en grande mesure. On ne peut nier qu'il ne fit une grande figure en
Europe et en Asie, et que la France n'eût infiniment profité d'une union
étroite avec lui. Il n'aimait point l'empereur, il désirait de nous
déprendre peu à peu de notre abandon à l'Angleterre, et ce fut
l'Angleterre qui nous rendit sourds à ses invitations jusqu'à la
messéance, lesquelles durèrent encore longtemps après son départ. En
vain je pressais souvent le régent sur cet article, et lui disais des
raisons dont il sentait toute la force, et auxquelles il ne pouvait
répondre. Mais son ensorcellement pour l'abbé Dubois, aidé encore alors
d'Effiat, de Canillac, du duc de Noailles, était encore plus fort.

Dubois songeait au cardinalat et n'osait encore le dire à son maître.
L'Angleterre, sur laquelle il avait fondé toutes ses espérances de
fortune, lui avait servi d'abord à être de quelque chose par le leurre
de son ancienne connaissance avec Stanhope. De là il s'était fait
envoyer en Hollande le voir à son passage, puis à Hanovre\,; enfin il
avait fait les traités qu'on a vus, et s'en était fait conseiller
d'État, puis fourré dans le conseil des affaires étrangères. Il avait
été, puis {[}était{]} retourné en Angleterre. Les Anglais qui voyaient
son ambition et son crédit, le servaient à son gré pour en tirer au
leur. Son but était de se servir du crédit du roi d'Angleterre sur
l'empereur qui était grand et de sa liaison alors intime et personnelle,
pour se faire cardinal par l'autorité de l'empereur qui pouvait tout à
Rome, et qui faisait trembler le pape.

Cette riante perspective nous tint enchaînés à l'Angleterre avec la
dernière servitude, qui ne permit rien au régent qu'avec sa permission,
que Georges était bien éloigné d'accorder à la liaison avec le czar,
tant à cause de leur haine et de leurs intérêts, que par ménagement pour
l'empereur deux points si capitaux pour l'abbé Dubois que le czar se
dégoûta enfin de notre surdité pour lui, et de notre indifférence qui
alla jusqu'à ne lui envoyer ni Verton, ni personne de la part du roi.

On a eu lieu depuis d'un long repentir des funestes charmes de
l'Angleterre, et du fol mépris que nous avons fait de la Russie. Les
malheurs n'en ont pas cessé par un aveugle enchaînement, et on n'a enfin
ouvert les yeux que pour en sentir mieux l'irréparable ruine scellée par
le ministère de M. le Duc, et par celui du cardinal Fleury ensuite,
également empoisonnés de l'Angleterre, l'un par l'énorme argent qu'en
tira sa maîtresse après le cardinal Dubois, l'autre par l'infatuation la
plus imbécile.

\hypertarget{chapitre-xix.}{%
\chapter{CHAPITRE XIX.}\label{chapitre-xix.}}

1717

~

{\textsc{Mort du palatin de Livonie.}} {\textsc{- Nouveaux manèges
d'Albéroni pour sa promotion.}} {\textsc{- Giudice à Gènes, misère de
ses neveux.}} {\textsc{- Effet à Madrid de la promotion de Borromée.}}
{\textsc{- Patiño depuis premier ministre et grand.}} {\textsc{-
Vanteries d'Albéroni.}} {\textsc{- Le roi de Sicile inquiet désire être
compris dans le traité projeté de l'Espagne avec la Hollande.}}
{\textsc{- Réponse d'Albéroni.}} {\textsc{- Albéroni change tout à coup
de système et en embrasse un fort peu possible, et encore avec
d'étranges variations.}} {\textsc{- Ses ordres à Beretti là-dessus.}}
{\textsc{- Les Hollandais désirent l'union avec l'Espagne.}} {\textsc{-
Ils craignent la puissance et l'ambition de l'empereur et les mouvements
du roi de Prusse.}} {\textsc{- Plaintes et dépit du roi de Prusse contre
le roi d'Angleterre.}} {\textsc{- Cabales et changements en
Angleterre.}} {\textsc{- Beretti propose d'attacher à l'Espagne
plusieurs membres principaux des états généraux, qu'il nomme, par des
pensions.}} {\textsc{- Lettre d'Albéroni à Beretti suivant son nouveau
système, pour être montrée au Pensionnaire et à quelques autres de la
république, et parle en même sens à Riperda.}} {\textsc{- Riperda
découvre un changement dans le dernier système d'Albéroni, et prévoit le
dessein sur la Sicile.}} {\textsc{- Esprit continuel de retour à la
succession de France.}} {\textsc{- Double friponnerie d'Albéroni et
d'Aubenton sur la constitution.}} {\textsc{- Artifices d'Albéroni pour
sa promotion\,; ses éclats et ses menaces.}} {\textsc{- Mauvais état des
finances d'Espagne.}} {\textsc{- Propos des ministres d'Angleterre et de
Hollande à celui de Sicile, en conformité du dernier système d'Albéroni,
et lui font une proposition étrange.}} {\textsc{- Il élude d'y répondre
et fait une curieuse et importante découverte.}} {\textsc{- Albéroni,
sous le nom de la reine, éclate en menaces, ferme l'Espagne à
Aldovrandi, fait un reproche et donne une leçon à Acquaviva, avec l'air
de le ménager.}} {\textsc{- Nouveaux efforts d'Albéroni pour sa
promotion.}} {\textsc{- Rare bref du pape au P. Daubenton.}} {\textsc{-
Le roi d'Espagne parle trois fois à Riperda suivant le système
d'Albéroni.}} {\textsc{- L'ambassadeur de Sicile, alarmé sur la cession
de cette île, élude de répondre aux propositions de l'ambassadeur de
Hollande.}} {\textsc{- Albéroni change de batteries et veut plaire au
pape pour obtenir sa promotion.}} {\textsc{- Embarras du pape.}}
{\textsc{- Vénitiens mal avec la France et avec l'Espagne.}} {\textsc{-
Acquaviva veut gagner le cardinal Ottobon.}} {\textsc{- Vil intérêt des
Romains.}} {\textsc{- Réflexion sur les cardinaux français.}} {\textsc{-
Changement de plus en plus subit de la conduite d'Albéroni sur sa
promotion.}} {\textsc{- Ses raisons.}} {\textsc{- Conduite et ordres
d'Albéroni à Beretti suivant son dernier système.}} {\textsc{-
Raisonnements de Beretti.}} {\textsc{- Agitations intérieures de la cour
d'Angleterre.}}

~

On apprit en même temps la mort du palatin de Livonie, qui avait
accompagné le prince électeur de Saxe dans tous ses voyages, qui avait
toute la confiance du père et du fils, et qui acquit par son esprit, par
ses lumières et par sa conduite et celle de ce prince en France tant de
réputation. Il était catholique, il eût été ravi de voir ce prince sur
le trône de Pologne, et bien étonné s'il eût pu deviner que la fille de
Stanislas serait reine de France et celle de son jeune prince Dauphine
par le contraste le plus étrangement singulier.

Le pape était toujours en des frayeurs mortelles des préparatifs du
Turc, et se réjouissait de la diligence qu'on lui faisait valoir de ceux
de l'Espagne pour envoyer l'escadre promise en Levant, et Acquaviva en
profitait pour presser la promotion d'Albéroni, qui perdrait, disait-il
au pape, toute sa grâce s'il ne l'accordait qu'avec toutes les
précautions qu'il y voulait apporter, c'est-à-dire que l'escadre fût
dans les mers du Levant, la nonciature rouverte en Espagne et tous les
différends entre les deux cours terminés. Giudice était encore à Gênes.
Son neveu le prélat, témoin des exclamations de tous les cardinaux,
lorsqu'ils entendaient parler de la promotion d'Albéroni, tremblait que
la conduite de son oncle à Rome ne nuisit à sa fortune. Cellamare n'en
avait pas moins de frayeur pour lui-même, tous deux bien résolus de s'en
tenir aux plus légères bienséances avec leur oncle, et se servir
eux-mêmes en servant Albéroni. Ce dernier avait reçu la nouvelle de la
promotion de Borromée avec beaucoup de fermeté\,; il parut qu'elle lui
faisait affecter de se montrer comme l'arbitre des affaires et de la
cour d'Espagne\,; mais donnant toujours sa promotion comme l'affaire
uniquement de la reine. Elle était lors en couches. On affecta de lui
cacher la nouvelle de peur de nuire à sa santé, mais deux heures après
l'arrivée du courrier qui l'apporta, il en fut dépêché un au prince Pio,
vice-roi de Catalogne à Barcelone, avec ordre d'empêcher Aldovrandi
d'entrer en Espagne, et de l'en faire sortir sur-le-champ s'il y était
déjà entré. En chemin ce nonce avait reçu une lettre du cardinal
Paulucci, par ordre du pape, qui lui donnait pouvoir d'assurer Albéroni
que sa promotion suivrait de près, pourvu que l'accommodement entre les
deux cours se fît aux conditions proposées par le pape et comme
acceptées, et qu'avant la conclusion la nonciature fût rouverte et
l'escadre à la voile. C'était vendre et acheter un chapeau bien cher\,:
aussi ces conditions furent-elles trouvées en Espagne d'une insolence
extrême\,: ce terme n'y fut pas ménagé, et toutes les autres expressions
mêlées de raisonnements qui y répondirent\,; on menaça de la fureur de
la reine quand elle en serait informée, et des plus grandes extrémités.
Le roi écrivit cependant au pape, en termes respectueux mais forts.
Aldovrandi fut accusé à Madrid d'avoir suggéré au pape cette résolution
par le désir qu'il avait de faire rouvrir sa nonciature et de n'y être
pas trompé.

Néanmoins Albéroni regardait l'envoi de l'escadre comme le seul moyen
d'opérer sa promotion. Il s'était rendu maître des fonds de l'armement,
et pour être plus assuré de la diligence, il en avait confié le soin à
Patino, avec le titre d'intendant général de la marine. C'était l'unique
Espagnol qu'il eût jamais jugé digne de sa confiance et capable de bien
servir. Il avait été dix-huit ans jésuite\,; il figura depuis de plus en
plus, et est mort enfin grand d'Espagne et premier ministre, avec autant
de pouvoir et de probité qu'en avait eu Albéroni. Il se vantait, en
attendant, d'avoir anéanti les conseils, rétabli le commerce et la
marine, réparé les places et l'artillerie, construit et augmenté des
ports, détruit la contractation\footnote{Ce mot espagnol signifie ici
  \emph{chambre de commerce}.} et le consulat de Séville, bridé pour
toujours l'Aragon et la Catalogne, par la construction de la citadelle
de Barcelone, et {[}il se vantait{]} de la santé du roi d'Espagne,
suffisamment raffermie pour ne ralentir plus l'empressement des
puissances étrangères de prendre des engagements avec lui.

Le roi de Sicile, toujours en crainte et mal avec l'empereur, fit
presser Albéroni de le comprendre dans le traité de ligue dont il se
parlait fort alors entre l'Espagne et la Hollande. Albéroni répondit à
l'abbé del Maro, son ambassadeur, que la conclusion n'en était pas
prochaine\,; que s'il y avait apparence de traiter, il serait averti\,;
que le motif de cette proposition avait été de rompre le traité de ligue
que l'empereur avait proposé aux États généraux avec lui, et que le roi
d'Espagne avait été bien aise de trouver une occasion de déclarer que si
l'empereur attaquait l'Italie, il prendrait ses mesures pour conserver
ses droits et ceux de ses amis\,; enfin que toutes les fois que les
Hollandais seraient raisonnables le roi d'Espagne serait disposé à
traiter avec eux, et qu'en ce cas les intérêts du roi de Sicile ne
seraient pas oubliés.

Il dit assez vrai pour cette fois\,; car, dès qu'il fut assuré de
n'avoir plus de traité à craindre entre l'empereur et les Hollandais, il
manda à Beretti de semer soigneusement la défiance entre eux, et de se
contenter de maintenir sur pied la négociation commencée, sans en
presser la conclusion, parce que, dans l'heureuse situation du roi
d'Espagne, il était en état d'être recherché de tout côté et n'avait
rien à craindre pour ses royaumes\,:, d'où il concluait qu'il fallait
aussi aller très lentement dans la négociation commencée avec
l'Angleterre, en quoi on verra bientôt l'ignorance de sa politique. Il
prescrivit donc à Beretti de mander à Stanhope que nul accommodement
avec l'empereur ne convenait à l'Espagne si on ne réglait, comme un
préliminaire, le point de la sûreté de l'Italie, dont il pouvait se
rendre maître en vingt-quatre heures, et que l'Angleterre, ayant
inutilement versé tant de sang et d'argent pour soutenir la dernière
guerre, ne devait rien oublier pour que les engagements qu'elle
prendrait pour assurer le repos de l'Europe eussent un effet certain.
Mais il voulut que Beretti écrivît en ce sens, comme de lui-même et sans
ordre, seulement comme très sûrement informé de l'intention de l'Espagne
de maintenir l'équilibre de l'Europe.

Elle n'y pouvait être selon lui, quelque précaution qu'on pût prendre
contre les changements des temps et des conjonctures, tant que
l'empereur posséderait des États en Italie, surtout une place comme
Mantoue. Il ne regardait plus que comme des dispositions trop éloignées
et trop casuelles pour y faire une attention sérieuse, l'offre du roi
d'Angleterre d'obliger l'empereur de promettre aux enfants de la reine
d'Espagne les successions de Parme et de Toscane, à faute d'enfants de
ces deux maisons. Il prétendait que Stanhope, qu'il avait vu en Espagne,
était fin et adroit. Il croyait voir de l'artifice dans ses lettres.
Pour le fixer il voulait un engagement positif des Anglais d'obliger
l'empereur à sortir d'Italie, et de Parme surtout. On ne peut s'empêcher
d'admirer ici qu'un premier ministre d'Espagne, quelque peu habile qu'il
pût être dans la connaissance des affaires, pût imaginer possible une
pareille vision.

Il ne laissait pas de prévoir que Stanhope se retrancherait sur le
traité d'Utrecht, auquel cette demande serait une infraction,
{[}traité{]} confirmé depuis par la ligue nouvellement faite entre la
France, l'Angleterre et la Hollande. Mais cela n'arrêtait point Albéroni
qui, sans l'engagement qu'il désirait, ne voyait point d'utilité pour
l'Espagne à traiter avec l'empereur, parce que des affaires d'Italie
dépendait, selon lui, l'équilibre de l'Europe, qui ne pouvait jamais
subsister tant qu'il y aurait un Allemand en Italie. Cela pouvait être
vrai. Mais comment obliger l'empereur, puissant comme il était et les
forces en main, de renoncer à l'Italie, qui faisait un des plus beaux et
des plus riches fleurons de sa couronne, et un des principaux fondements
de son autorité en Europe, et comment, persuader les Anglais, de tous
temps liés avec lui et le roi d'Angleterre, lors son ami personnel et
intime, et qui avait tant d'intérêt de le ménager pour ses États
d'Allemagne, de lui faire une proposition si folle et encore sans
équivalent, et de le forcer à cet abandon qui, par leur situation, ne
leur était à eux d'aucune considération\,?

Albéroni comptait dire merveilles en protestant que le roi d'Espagne,
content de ce qu'il possédait, ne prétendait rien en Italie pour
lui-même, et se contentait de ce qui devait appartenir au fils de son
second lit, par toutes les lois divines et humaines. Ce leurre en sus
était aussi par trop grossier. C'était néanmoins en ce sens que Beretti
reçut ordre d'écrire et de parler si la négociation se portait à
Londres.

Albéroni ne jugeait pas convenable de céder tant de droits et d'États
usurpés pour une promesse vague garantie par l'Angleterre et la
Hollande, qui pour leur intérêt propre, à ce qu'il se figurait, seraient
obligées d'empêcher l'empereur de se rendre maître des États du
grand-duc, si la succession s'en ouvrait sans héritiers\,; par
conséquent que l'Espagne ne gagnerait rien, et perdrait tout, en faisant
ce traité avec l'empereur. Il en parla en ce sens au secrétaire
d'Angleterre, toutefois dans l'intention d'entretenir le traité sans le
rompre.

Le naturel froid et temporiseur d'Heinsius servait Albéroni contre les
empressements que Beretti redoublait sans cesse pour le traité, avant
que d'avoir reçu ses derniers ordres. Ce Pensionnaire l'assurait de la
bonne disposition de toutes les provinces\,; mais il ajoutait qu'avant
de traiter et de conclure, il fallait voir ce que produiraient les soins
de l'Angleterre et de la république, pour moyenner la paix entre
l'empereur et l'Espagne\,; que si cette paix ne réussissait point, la
république s'unirait avec l'Espagne par une alliance, soit que les
Anglais y voulussent entrer ou non. Amsterdam paraissait le désirer\,;
Beretti s'en applaudissait comme du fruit de ses soins, et comptait
aussi sur les provinces d'Utrecht et de Gueldre. Les principaux membres
de la république rejetaient sur l'Angleterre la lenteur de la
négociation de la paix entre l'empereur et l'Espagne. Duywenworde se
plaignait de ces délais, qui laissaient perdre la conjoncture si
favorable de la guerre de Hongrie pour rendre l'empereur plus facile. Il
convenait de l'intérêt commun que l'empereur ne se rendît pas maître de
l'Italie, et assurait que les États généraux l'abandonneraient s'il ne
se rendait pas raisonnable, et traiteraient avec l'Espagne pour leurs
intérêts particuliers. Il se vanta, pour prouver ses bonnes intentions,
d'avoir parlé très fermement, en dernier lieu, dans l'assemblée des
États de Hollande, sur les contraventions de l'empereur au traité de la
Barrière, et prétendait l'avoir engagé d'écrire au roi d'Angleterre,
pour lui demander l'interposition de ses bons offices à Vienne, d'où il
arriverait qu'en le faisant la république aurait ce qu'elle désirait, ou
s'il l'en refusait, sa mauvaise foi serait reconnue, et la république
serait en pleine liberté de traiter avec l'Espagne.

Elle venait de réformer cinq régiments écossais. Albéroni en voulait
prendre deux à son service\,; mais Beretti qui en avait écrit à Londres,
n'en ayant point de réponse, augurait mal du succès de cette demande.

Malgré cette réforme de troupes, que le mauvais état des affaires des
Hollandais les avait obligés de faire, ils étaient inquiets des
nouvelles levées que le roi de Prusse faisait\,: il voulait avoir
soixante-cinq mille hommes sur pied, sans que ses ministres, ni
peut-être lui-même, sût ce qu'il en voulait faire. Ces troupes faisaient
des mouvements dans le pays de Clèves. Il remplissait ses magasins, et
donna tant d'alarme aux Hollandais, qu'ils firent travailler aux
fortifications de Nimègue et de Zutphen, et lui payèrent cent vingt
mille florins des subsides qu'ils lui devaient de la dernière guerre.

Le roi de Prusse inquiétait aussi le roi d'Angleterre, son beau-père,
par les plaintes qu'il faisait de lui et par ses liaisons étroites avec
le czar. Le gendre se déclarait vivement piqué de trouver son beau-père
opposé partout à ses intérêts, difficile sur les moindres bagatelles\,;
dans son dépit, il protestait qu'il ne tiendrait pas à l'empereur de
l'attacher, inviolablement à ses intérêts, parce qu'il était persuadé
que le chef de l'Empire devait être et serait l'arbitre de la paix du
Nord. Il se plaignait qu'une escadre anglaise eût bloqué le port de
Gottembourg, et que Georges fit tenir le baron de Gœrtz si étroitement
dans les prisons de Hollande, qu'il n'y avait eu que le seul
adoucissement d'y faire porter son lit.

En même temps la cour de Londres était si remplie de cabales, que le roi
d'Angleterre n'avait pu conserver ses principaux ministres. Townshend,
secrétaire d'État, avait quitté cette place pour la vice-royauté
d'Irlande, qu'il perdit encore bientôt après. Methwin, aussi secrétaire
d'État, et Walpole, premier commissaire de la trésorerie, furent démis
aussi, ainsi que Pulteney de {[}la place{]} de secrétaire des guerres,
et le duc de Devonshire, leur ami, et de même cabale, ne voulut pas
demeurer président du conseil après leur disgrâce, et remit cette grande
place. Stanhope changea la sienne de secrétaire d'État pour celle de
premier commissaire de la trésorerie.

Parmi ces mouvements, la cour d'Angleterre était médiocrement occupée
des affaires du dehors, et Stanhope encore moins, qui en avait quitté la
direction. Ainsi, ses réponses à Beretti étaient sèches, obscures, et
désolaient l'activité de ce ministre sur une affaire dont il désirait
ardemment la conclusion, pour en avoir l'honneur, et tous ses
raisonnements tendaient à éprouver si Georges agissait sincèrement, ou
se contentait d'amuser\,; ce qui ne se pouvait qu'en le pressant
extraordinairement de faire expliquer l'empereur avant la décision de la
campagne en Hongrie. Il se confirmait dans cette opinion par l'aveu que
faisaient Heinsius et Duywenworde, autrefois impériaux si zélés, qu'ils
ne pouvaient avoir de confiance en la sincérité de l'empereur dans la
négociation commencée, en en éprouvant si peu de sa part sur l'exécution
des conditions du traité de la Barrière.

Le Pensionnaire même si mesuré, s'était emporté contre l'ambition de la
cour de Vienne et le danger de la laisser en état de se rendre maîtresse
de tous côtés, par conséquent de faire les derniers efforts sur le
traité de paix avec l'Espagne pendant la campagne de Hongrie. Beretti
proposait la nécessité d'acquérir des amis encore plus sûrs à l'Espagne,
par des pensions dont on flatterait les plus propres à les recevoir, et
en même temps les plus en état de bien servir, mais qui ne leur seraient
données que lorsque l'alliance avec la république serait comme certaine.
Ceux qu'il nommait pour ces pensions des principaux membres de la
république étaient le comte d'Albemarle, les barons de Reenswonde, de
Norwich et de Welderen. Ce dernier était député pour la Gueldre. Il le
disait fort autrichien, mais sensible à l'argent, et plus encore aux
bons repas.

Albéroni, dans les principes qu'on a vus, était fort ralenti sur ces
alliances. Il écrivit une lettre à Beretti, suivant ces mêmes principes,
avec ordre de la montrer au Pensionnaire et aux bien intentionnés. Il y
insistait sur l'absolue nécessité de l'équilibre, sur son impossibilité
tant que l'empereur conserverait un pouce de terre et un soldat en
Italie, sur l'indifférence du roi d'Espagne, sur la paix à faire avec
l'empereur. Surtout, il y relevait le bon état de l'Espagne, et ses
espérances de le rendre encore meilleur, avant qu'il fût cinq ou six
ans.

En même temps, il manda Riperda, ambassadeur de Hollande. Il lui parla
des propositions de l'Angleterre et de la Hollande, pour la paix entre
l'empereur et l'Espagne, lui dit qu'il fallait compter que ce n'était
que de belles paroles de la cour de Vienne, que la négociation serait
infructueuse, qu'il serait même très dangereux de l'entamer, tant que la
sûreté pour l'équilibre de l'Europe ne serait pas solidement établie\,;
lui expliqua en quoi il le faisait consister, et qu'il fallait que
l'empereur remît tout ce qu'il possédait en Italie entre les mains de
l'Angleterre et de la Hollande, pour en être disposé par ces deux
puissances comme elles le jugeraient à propos, suivant la justice\,; et
que le roi d'Espagne, dont il loua l'amour du bien public, consentait
d'en être parfaitement exclu. Il ajouta des plaintes de l'attachement
des États généraux pour l'empereur\,; qu'il comprenait bien les
ménagements que le roi d'Angleterre avait pour le chef de l'empire, par
rapport à ses États d'Allemagne\,; qu'il ne voyait donc qu'un esprit de
dépendance à ses volontés dans cette conduite de la Hollande\,; que
néanmoins il fallait une balance dans l'Europe. Il proposa comme un
moyen d'y parvenir de procurer aux États généraux les Pays-Bas
catholiques, et promit à Riperda, en lui en demandant le secret, que le
roi d'Espagne ferait là-dessus ce qu'il jugerait à propos. Il finit
comme il avait commencé, sur l'empereur et sur l'Italie.

Riperda sortit de cette conversation persuadé que l'Espagne ne ferait,
jamais la paix avec l'empereur aux conditions proposées par l'Angleterre
et la Hollande. Il croyait avoir découvert que le projet d'Albéroni, qui
pourtant avait insisté au commencement et à la fin de cette conversation
qu'il n'y pouvait avoir d'équilibre tant que l'empereur posséderait un
pouce de terre en Italie\,; Riperda, dis-je, croyait avoir découvert que
son projet était de laisser le Milanais à l'empereur, d'y faire ajouter
Crémone et le Crémonois, donnant en échange Mantoue et le Mantouan à la
république de Venise, de recouvrer pour l'Espagne Naples, Sicile et
Sardaigne, et d'assurer au fils aîné du second lit du roi d'Espagne les
successions de Florence et de Parme. Cet ambassadeur était même persuadé
que l'Espagne recouvrerait la Sicile lorsqu'on s'y attendrait le moins.

Albéroni Était bien aise d'insinuer aux états généraux ces différentes
vues, parce qu'il craignait plutôt qu'il ne souhaitait la paix avec
l'empereur. Dans la prévoyance des événements qui pouvaient arriver, il
évitait d'engager de nouveau le roi d'Espagne, soit en confirmant les
engagements déjà pris, soit par de nouvelles cessions dont l'Europe
deviendrait garante. Il disait que la main de Dieu n'était pas
raccourcie, et par ce discours il laissait assez entendre ce qu'il avait
dans l'esprit. C'est une chose étrange qu'être possédé de l'esprit de
retour, et de n'oser en laisser rien apercevoir ni à la France ni à
l'Espagne.

Dans ce même esprit il profita de la conjoncture de plusieurs écrits
contre la constitution qui avaient été brûlés publiquement à Rome. Il
fit écrire au pape par leur fidèle Aubenton des merveilles de la piété
du roi d'Espagne, et de son inséparable attachement au chef de
l'Église\,; quoiqu'il pût arriver dans cette affaire. Ces mêmes écrits
que Cellamare avait envoyés furent livrés à l'inquisition d'Espagne pour
y être brûlés. Cellamare eut ordre de ne plus envoyer d'écrits faits
contre la constitution, mais tous ceux au contraire qui lui étaient
favorables, tandis que le cardinal Acquaviva reçut ordre d'éviter avec
soin de prendre aucun parti dans ces différends et de se contenter
simplement de rendre compte des suites qu'ils, pourraient avoir\,:
c'est-à-dire qu'Albéroni voulait donner au pape une grande idée de
l'attachement du roi d'Espagne pour la saine doctrine, et de son horreur
pour les nouveautés, en même temps que ce ministre se voulait ménager
soigneusement la France, et ne pas donner aussi trop d'assistance au
pape dans une conjoncture où il en était aussi mécontent.

Toutefois il pressait l'armement de la flotte comme l'instrument unique
de sa promotion, qui ne touchait, disait-il, que la reine. Il continuait
à garder le silence qu'il s'était imposé, et de dire qu'il savait bien
que, s'il proposait quelques tempéraments, ses envieux diraient qu'il ne
songeait qu'à ses intérêts aux dépens de ceux de ses maîtres, jusque-là
qu'il était convaincu de leur cacher les lettres d'Acquaviva\,: c'était
un bon reproche qu'il lui faisait de n'avoir pas été assez ferme à
presser le pape\,; que les lénitifs n'étaient ni selon l'humeur du roi
ni selon celle de la reine\,; qu'à l'avenir Rome serait obligée à plus
d'égards pour eux\,; que Leurs Majestés Catholiques donneraient enfin
des marques de leur ressentiment à une cour pleine de brigands, aisée à
châtier par l'intérêt\,; qu'étant lui-même homme d'honneur et
désintéressé, il serait content d'avoir préféré la décence du service de
ses maîtres à sa propre élévation\,; que s'ils avaient désiré un chapeau
de cardinal, il leur conviendrait enfin de le mépriser, voyant l'étrange
procédé de Rome\,; qu'il ne doutait pas que, si le roi d'Espagne
changeait de résolution sur l'envoi de ses vaisseaux, ce changement ne
fût attribué à son ministère, et que quelque fripon ne répandît qu'il se
serait servi de son crédit pour ôter ce secours à la chrétienté\,; que
le pape seul perdrait la religion, puisque dans le même temps qu'il
accordait aux instances de ses parents la dignité de cardinal pour un
homme vendu aux Allemands, il refusait avec mépris là justice que le roi
d'Espagne lui demandait. Il établissait pour principe (et ce principe
est très vrai, et c'est la seule vérité qu'Albéroni dise ici), il
établissait pour principe qu'il ne fallait pas filer doux avec la cour
de Rome, que tous les remèdes mitoyens étaient mauvais, et que le temps
détromperait enfin de l'orviétan de cette cour\,; il ajoutait que ses
amis les plus dévoués ne pouvaient approuver sa conduite, que le
confesseur même jetait feu et flamme\,; mais Albéroni ne prétendait pas
lui en savoir gré, parce que, si ce jésuite en usait autrement, il s'en
trouverait mal.

Cette flotte, dont Albéroni faisait tant de parade, coûtait
prodigieusement. L'état des affaires n'était pas tel qu'Albéroni
s'efforçait de le montrer. Les dettes étaient en grand nombre et
pressantes, les moyens de les acquitter difficiles\,; lui-même était
contraint de l'avouer à ses confidents, mais il avait le bonheur de
faire accroire le contraire aux ministres étrangers qui étaient à
Madrid. Ceux d'Angleterre et de Hollande qu'il caressait le plus,
assuraient l'ambassadeur de Sicile que le roi d'Espagne trouvait en
argent comptant au delà de l'opinion commune\,; qu'il pouvait aider le
roi de Sicile à devenir le libérateur de l'Italie, puisque le seul moyen
d'empêcher l'empereur de s'en rendre enfin le maître, était d'unir par
un traité le roi d'Espagne, le roi de Sicile et les princes d'Italie.
L'abbé del Maro voulut savoir quel serait à peu près le plan que
l'Angleterre et la Hollande formeraient pour cette union. Les ministres
de ces deux puissances parlèrent de faire céder la Sicile au roi
d'Espagne, et de faire donner au roi de Sicile les États contigus au
Montferrat, et la partie du Milanais dont il était en possession.
Quoique la proposition fût étrange, del Maro jugea qu'elle était faite
de concert avec Albéroni, qui voulait faire sa cour à la reine en
trouvant le moyen de fonder un État pour ses enfants. Il tacha de
pénétrer un point plus important. Il remarquait les ménagements que
l'Angleterre et la Hollande avaient pour le roi d'Espagne. Il voulut
découvrir quel parti prendraient ces puissances au cas d'ouverture à la
succession de France. Mais il jugea par les réponses de leurs ministres
que leurs égards étaient encore plus pour l'Espagne que pour la personne
de Philippe V, et que si jamais il entreprenait de revenir contre les
renonciations, elles emploieraient leur crédit et leurs armes pour
traverser son entreprise.

La reine d'Espagne apprit enfin la promotion de Borromée, Albéroni sous
son nom éclata en menaces. Outre le courrier dépêché à Barcelone dont on
a parlé, il en avait fait envoyer un autre à Alicante pour le même effet
au cas qu'Aldovrandi eût pris la route de la mer pour l'empêcher d'y
mettre pied à terre. Ce nonce avait laissé à Madrid un nommé Giradelli,
son secrétaire, qui était aussi agent du cardinal Acquaviva. Albéroni
fut tenté de le chasser. Mais réfléchissant que cet homme ne pouvait lui
nuire\,; il s'en fit un mérite auprès d'Acquaviva, et lui donna en même
temps une leçon. Le mérite fut de lui mander qu'à sa seule considération
il avait empêché que cet homme fût chassé, mais à condition qu'il ne
ferait aucune fonction d'agent du pape, et qu'il ne parlerait ni ne
présenterait de mémoire au nom de Sa Sainteté.

Pour la leçon, Acquaviva pressait depuis longtemps d'être délivré à Rome
de la critique importune de don Juan Diaz, agent d'Espagne, qui
censurait toutes ses actions avec la liberté la plus outrée. Albéroni
lui avait promis de le rappeler. Le cardinal l'en avait de nouveau
sollicité. Albéroni, mécontent de sa mollesse et d'avoir laissé passer
Borromée sans lui, ajouta, à sa lettre qu'il fallait user de flegme à
l'égard de cet homme, regardé par les Espagnols comme très zélé pour le
service et comme incapable de ménager personne quand il s'agissait de
l'intérêt des maîtres\,; que, de plus, il s'était encore acquis un
nouveau crédit depuis la promotion de Borromée, parce, qu'il avait
constamment assuré qu'elle serait faite, et Leurs Majestés Catholiques
trompées malgré les belles paroles du pape et les espérances dont lui
Acquaviva s'était laissé flatter.

Ce reproche fait au pape et à lui était annoncer la vengeance\,; deux
Italiens n'y pouvaient donner une autre interprétation. Aldovrandi
voyant sa fortune perdue si l'entrée d'Espagne lui demeurait interdite,
demanda instamment la permission de passer à Barcelone ou à Saragosse.
La colère de la reine fut le prétexte de n'écouter aucune proposition
que la promotion d'Albéroni ne fût faite. Mais pour en conserver le
véritable appât, il fit savoir à Rome que l'escadre si désirée se
rendrait incessamment à Gênes, et pourrait même s'avancer jusqu'à
Livourne, mais que dans l'un de ces deux ports, elle attendrait des
nouvelles d'Acquiaviva, d'où elle regagnerait les ports d'Espagne si la
promotion tant de fois promise n'était pas faite, résolution dont Leurs
Majestés Catholiques ne se départiraient jamais quand même le monde
viendrait à tomber, parce que le roi d'Espagne se lassait enfin d'être
depuis seize ans le jouet de la cour de Rome.

Ce prince, dépeint à Rome avec tant de soin comme si soumis au pape pour
le lui faire désirer en France, si malheureusement la succession venait
à s'ouvrir, ne voulait pas qu'il lui fût permis de différer la promotion
d'un si rare sujet, et se portait à toute extrémité. Ainsi il menaça
Rome à cette occasion de former une junte pour examiner les moyens et
les mesures à prendre pour établir de justes bornes à son autorité en
Espagne, et l'y réduire à celle qu'on lui permettait en France et à
Venise. Il ajoutait que Leurs Majestés, Catholiques seraient inflexibles
sur ce point capital\,; que qui que ce soit n'oserait entreprendre de
tenter de les fléchir\,; qu'il aimerait mieux être mort que d'en avoir
ouvert la bouche, parce qu'on ne manquerait pas de l'accuser de préférer
ses intérêts à celui de ses maîtres. Que le confesseur avait d'autant
plus d'intérêt de garder le plus profond silence qu'il lui était très
sévèrement enjoint par le roi sur toutes les affaires de Rome, à
laquelle d'ailleurs il passait pour être vendu. Ainsi Albéroni voulait
que le pape connût tout le danger de différer sa promotion, et qu'il le
regardât comme le seul maître de terminer les différends entre les deux
cours.

Pour le confirmer dans cette pensée, il obtint du roi d'Espagne
d'engager le duc de Parme à promettre au nom de Sa Majesté Catholique de
se rendre garant que l'accommodement se ferait, et que le tribunal de la
nonciature serait rouvert dans le moment que la promotion serait faite
et déclarée.

Cet instant de la promotion d'Albéroni était le point critique de toute
difficulté sur l'accommodement. Albéroni ne le voulait point faire si
cette condition n'était remplie\,; il avait trop de peur d'être laissé
après. Le pape, dans la même défiance qu'on ne se moquât de lui après la
promotion, se tenait ferme à sa promesse de la faire sitôt que
l'accommodement serait fait aux termes convenus déjà par Albéroni, et
que l'escadre serait à la voile sur la route de Corfou. Cette défiance
mutuelle arrêtait tout. Néanmoins le pape voulut d'avance lever toutes
les difficultés préliminaires. Il écrivit à d'Aubenton un bref de sa
main, portant pouvoir d'absoudre le roi d'Espagne de toutes les censures
qu'il avait encourues par les actes faits en son nom et par son autorité
contre les droits du saint siège, mais à condition que ces mêmes actes
seraient annulés, et que Sa Majesté Catholique entrerait dans tous les
projets d'accommodement proposés par Sa Sainteté. On ne peut s'empêcher
de dire ici que les réflexions s'offrent en foule sur ce beau bref et
sur cette rare invention, d'envahir tout comme juge et partie.

Albéroni en même temps, attentif à l'objet qu'il s'était fait pour
l'Italie, procura à Riperda qu'il avait toujours particulièrement
ménagé, trois audiences consécutives du roi d'Espagne en sa présence,
dans lesquelles le roi d'Espagne, louant la candeur du Pensionnaire, dit
qu'il souhaitait qu'il devînt le directeur de la négociation entre lui
et la cour de Vienne, et que les propositions y fussent portées et à
Madrid en même temps par les offices de l'Angleterre et de la Hollande.
Il insista sur la nécessité d'établir avant toutes choses la balance
nécessaire pour la sûreté de l'Italie, et il renouvela ce qui avait déjà
été dit à cet ambassadeur de Hollande, pour exciter ses maîtres à
profiter de l'occasion favorable qu'ils avaient de se rendre maîtres des
Pays-Bas.

Riperda put aisément reconnaître aux conférences particulières qu'il
avait avec Albéroni que l'Italie était son objet principal. Il crut
démêler que les instances que faisait le roi de Sicile pour être compris
dans ce traité n'auraient pas grand succès, et qu'on n'était pas disposé
en Espagne à favoriser l'augmentation de sa puissance. Son ambassadeur
travaillait à persuader le roi d'Espagne qu'une étroite intelligence
entre lui et son maître était nécessaire pour leurs intérêts communs, et
que l'ambassadeur de Hollande appuierait sa pensée de ses offices.
Riperda, en effet, dans une visite qu'il lui avait faite, l'avait fort
entretenu de la nécessité de profiter de la guerre du Turc pour
maintenir la liberté de l'Italie contre les invasions de l'empereur,
d'où dépendait la tranquillité de l'Europe\,; que les rois d'Espagne et
d'Angleterre étoient persuadés de cette vérité, ainsi que les États
généraux\,; qu'il fallait les unir et savoir si le roi de Sicile
concourrait avec eux dans la même union\,; qu'il parlait par ordre du
Pensionnaire, choisi par le conseil secret de la république pour seul
commissaire et interprète dans cette négociation particulière\,; qu'il
demandait une réponse là-dessus du roi de Sicile, lequel ne devait pas
être surpris du silence qui se gardait là-dessus avec son résident à la
Haye, parce que la négociation devait être concertée principalement avec
l'Espagne, et qu'il était absolument nécessaire d'empêcher que le
mystère n'en fût éventé.

Tous ces propos néanmoins furent suspects à del Maro, à qui Riperda
avait déjà tenu quelques discours désagréables sur l'idée de la cession
de la Sicile à l'empereur moyennant un échange. Les offres de Riperda
lui parurent de nouvelles preuves du concert fait entre les trois
puissances de dépouiller son maître de la Sicile et de l'obliger à se
contenter d'un échange tel qu'il leur plairait pour faciliter la paix de
l'empereur avec l'Espagne\,: ainsi il éluda de répondre positivement en
demandant du temps de recevoir les ordres de son maître.

Albéroni, tout occupé de sa promotion qu'il voulait obtenir par toutes
sortes de voies, envoya ordre à Cadix de mettre à la voile pour le
Levant, et avec cette nouvelle Acquaviva eut ordre d'assurer le pape
qu'Aldovrandi serait au plus tôt reçu en qualité de nonce. Le prétexte
de ce changement subit fut de montrer la droiture et la sincérité du
procédé du roi d'Espagne, mais dont il attendait un juste retour de sa
part par la promotion actuelle et déclarée à la réception de sa lettre.
Le pape ne pouvait s'aveugler sur l'indignité de cette promotion qu'il
sentait et voyait. Les clameurs publiques en retentissaient et en
frappaient ses oreilles. Mais de cette promotion, telle qu'elle fût,
dépendaient l'accommodement à l'avantage de Rome, et le secours maritime
contre les Turcs.

Le pape pleurait donc, et les larmes lui coûtaient peu. Il se trouvait
dans les douleurs de l'enfantement. Il se servait de la frayeur commune
des Vénitiens pour agir par leur ambassadeur à Rome auprès, d'Acquaviva,
pour persuader l'Espagne de secourir l'Italie contre les Turcs, sans
attendre la promotion. Ce ricochet était employé, parce que le noble
résident à Madrid n'avait pas encore pris caractère. L'Espagne
prétendait des satisfactions que là république éludait encore sur ce
qu'elle avait reconnu l'archiduc roi d'Espagne, Acquaviva souhaitait que
le roi, d'Espagne, secouant les Vénitiens, obtînt d'eux le
rétablissement entier de la famille Ottoboni dans ses biens et
prérogatives, et dans leurs bonnes grâces, dont elle était privée depuis
que le cardinal Ottobon avait, sans leur congé, accepté la protection de
France. Il considérait qu'il était important pour un conclave d'acquérir
un cardinal tel que celui-là, qui d'ailleurs avait toujours bien mérité
du roi d'Espagne.

On trouve à Rome quantité de gens empressés à témoigner leur zèle, soit
à la France, soit à la maison d'Autriche suivant ce qu'ils appellent
\emph{il genio} qui les partage entre les deux. L'espérance des
bienfaits est un puissant motif, même pour des personnes principales qui
ne peuvent jamais espérer de la cour de Rome des récompenses approchant
de celles qu'ils reçoivent des couronnes en bénéfices ou en pensions.
Quelques-uns même, non contents d'en tirer de modiques d'un côté,
tâchent d'en recevoir aussi de l'autre sous un titre de politiques ou de
nouvellistes. On éprouva cette conduite d'un abbé Juliani, qui
rapportait au palais du pape d'une part, et aux Espagnols, de l'autre,
tout ce qu'il apprenait du cardinal de La Trémoille, dont il avait gagné
la confiance. Il avait une forte pension du roi, et son père en avait
aussi été fort bien payé.

On ne peut ici s'empêcher de déplorer l'aveuglement sur les cardinaux
nationaux toujours inutiles, et c'est marché donné fort à charge, et
impunément très dangereux quand il leur plaît. Deux cent mille livres de
rente est peu de chose en bénéfices pour un cardinal français. Je laisse
à part le rang et la considération personnelle qui porte sur tous les
siens. Il n'y en a jamais qu'un demeurant à Rome pour les affaires du
roi. Les autres virent à Paris et à la cour comme bon leur semble.
Vient-il un conclave, il faut les payer pour y aller\,: encore s'en
excusent-ils tant qu'ils peuvent. En arrivant à Rome, ils trouvent les
cabales formées et les partis pris. Ils n'y connaissent personne\,:
aussi éprouve-t-on qu'on s'y moque d'eux avec force compliments. Le pape
est-il fait, c'est à qui reviendra le plus vite. Tous les crimes leur
sont permis, ceux même de lèse-majesté\,; quoi qu'ils attentent, ils
sont inviolables et vont tête levée. Louis XI n'osa jamais punir les
attentats et les trahisons avérées du cardinal Balue que par la prison,
et encore avec combien de traverses, et on le vit sous son successeur
triompher de son crime dans l'éclat de légat en France. Sixte V approuva
tout ce qui s'était passé à Blois, et détestait les horreurs de la
Ligue\,; mais, lorsque, quelques jours après, il apprit la mort du
cardinal de Guise, pour le moins aussi coupable que son frère, il
excommunia Henri III, et trouva qu'il n'y avait pas d'assez grands
châtiments pour expier ce crime. On a vu le feu roi réduit à traiter
avec le cardinal de Retz, et n'avoir pu châtier les forfaits du cardinal
de Bouillon ni l'éclat de sa désobéissance. Les avantages et les
inconvénients d'avoir des cardinaux français ne se peuvent donc pas
balancer. À l'égard des prétentions de Rome, on ne peut compter sur les
cardinaux français. On sent encore les suites des manèges et de la
séditieuse harangue du cardinal du Perron, en 1614, aux derniers états
généraux qui se soient tenus. Si nos rois ne souffraient jamais de
cardinaux en France, ils éviteraient ces funestes inconvénients et celui
encore d'un attachement à Rome contre leurs intérêts de tous ceux qui se
figurent arriver à la pourpre, et de quelques-uns qui y sont élevés
malgré eux, comme le, fut le cardinal Le Camus, malgré le feu roi, et le
cardinal de Mailly malgré le roi d'aujourd'hui et le régent, à force de
cabales, de sédition, de rage dans l'affaire de la constitution. En
donnant la nomination à des sujets italiens bien choisis, ils auraient à
Rome des cardinaux permanents, à eux, informés et au fait de tout sans
cesse, qui, par eux, par leurs amis et leur famille, seraient
continuellement utiles et infiniment dans les conclaves, et dont trois
ou quatre seraient plus que contents à eux tous des bénéfices qui ne
suffisent pas à un seul cardinal français. L'espérance du cardinalat ne
débaucherait plus d'évêques contre les libertés de l'Église gallicane et
contre l'autorité et la souveraineté temporelle de nos rois, et leur
procurerait, au contraire, les services et l'attachement des plus
considérables maisons et particuliers de Rome et de toute l'Italie, dont
l'utilité se reconnaîtrait tous les jours. C'en est assez sur cet
important article, dont l'évidence saute aux yeux.

Plusieurs cardinaux se flattaient d'avoir depuis peu détourné le pape de
déshonorer leur collège en y mettant un si étrange sujet. Albéroni le
savait, et il reconnut qu'il n'était pas de son intérêt de porter trop
loin le ressentiment du roi et de la reine, parce que, si le nouveau
différend qu'il produirait durait trop longtemps, il en serait la
victime, que ses ennemis en si grand nombre seraient ravis de le voir
embarqué dans une affaire qu'ils regardaient comme la cause inévitable
de sa perte prochaine, à laquelle tous les Espagnols contribueraient à
l'envi. Ces réflexions lui firent changer de conduite. Il pressa le
départ de la flotte. Il manda au duc de Parme qu'elle mettrait à la
voile le 26 mai, et il pressa Aldovrandi de se rendre à Ségovie, où la
cour était, pour y terminer, à la satisfaction du pape, les différends
entre les deux cours. Il laissa entrevoir qu'il sentait toute la
conséquence dont était pour lui de finir au plus tôt l'affaire de sa
promotion et ce qu'il devait craindre de l'empire que les Allemands,
maîtres de l'Italie, prendraient sur l'esprit et sur les résolutions du
pape. Ce fut l'excuse d'un changement si subit de conduite. On en verra
dans la suite d'autres raisons.

Il avait aussi changé de système sur les affairés générales de l'Europe.
Il avait fort désiré unir le roi d'Espagne avec l'Angleterre et la
Hollande, et lui procurer la paix avec l'empereur par le moyen de ces
deux puissances. Ces idées, qui avaient été si avant dans son esprit, ne
subsistaient plus. Il éludait la négociation de cette paix, que
l'Angleterre voulait entamer. Il se fondait sur la situation avantageuse
où ses soins avaient mis, disait-il, l'Espagne, qui n'avait nulle raison
de rechercher l'amitié de personne, et dont le meilleur parti était de
regarder l'embarras des autres puissances d'un œil tranquille et de bien
jouer son jeu. Il s'appuyait sur les troubles intérieurs dont il croyait
l'Angleterre inévitablement menacée, et sur l'épuisement extrême où la
dernière guerre avait laissé la Hollande, qui obligeraient ces deux
puissances à rechercher l'amitié du roi d'Espagne, en sorte que, le prix
en étant connu des nations étrangères, il ne la donnerait qu'à bon
escient à qui il jugerait à propos. Ainsi, au lieu de presser Beretti,
il modérait son ardeur de négocier pour se faire valoir. Il l'occupait à
gagner et à faire passer en Espagne des ouvriers en laine pour des
manufactures très utiles qu'il méditait, mais sur le succès desquelles
il craignait avec raison la paresse naturelle des Espagnols.

Beretti se fondait en grands raisonnements pour persuader Albéroni de
profiter du désir qu'il voyait dans la république de s'unir à l'Espagne,
d'entrer dans les mesures nécessaires à borner l'ambition de la maison
d'Autriche, et de se garantir de l'impression que faisait sur lui
l'humeur vindicative des transfuges espagnols de son conseil. Il disait
que nul traité ne serait solide si on n'établissait préliminairement un
équilibre parfait dans les affaires de l'Europe, sans lequel le roi
d'Espagne ne devait jamais s'engager, mais demeurer spectateur, et il
traitait de vaines les renonciations faites en faveur de la maison
d'Autriche, parce qu'elle-même n'en avait fait aucune en faveur de
l'Espagne. Il convenait qu'un refus absolu d'écouter rien sur la paix
avec l'empereur pouvait alarmer l'Angleterre et la Hollande, mais qu'il
fallait savoir prolonger la négociation, et faire retomber sur la cour
de Vienne l'odieux des délais.

Le fruit qu'il se proposait de cette conduite était que l'Angleterre et
la Hollande, irritées de celle de l'empereur sur la paix, l'en
craindraient encore davantage et solliciteraient elles-mêmes l'alliance
que le roi d'Espagne leur offrait. Il était vrai que l'Angleterre
n'était pas tranquille dans l'intérieur\,: les partis y étaient plus
animés que jamais, le roi et le prince de Galles brouillés jusqu'à ne
plus garder aucunes apparences, les ministres anglais haïs d'une partie
de la nation, les ministres allemands détestés de la nation, entière, et
regardés comme vendus à la cour de Vienne. Ils passaient pour tels au
point que le ministre du roi de Sicile n'osa les solliciter de
travailler à l'accommodement de son maître avec l'empereur.

\hypertarget{note-i.-causes-de-la-disgruxe2ce-de-fouquet.-son-procuxe8s.}{%
\chapter{NOTE I. CAUSES DE LA DISGRÂCE DE FOUQUET. --- SON
PROCÈS.}\label{note-i.-causes-de-la-disgruxe2ce-de-fouquet.-son-procuxe8s.}}

Saint-Simon parlant de la disgrâce de Fouquet, dit que la principale
cause de son malheur fut un \emph{peu trop de galanterie et de
splendeur}. Le jugement de l'histoire est plus sévère. Tout le monde
sait que le château de Vaux\footnote{Vaux-le-Vicomte (département de la
  Marne).} coûta des sommes énormes, et que Louis XIV indigné fut sur le
point de faire arrêter Fouquet au milieu des fêtes qu'il donnait à la
cour. Quant à la \emph{galanterie} de Fouquet, il suffira de rappeler,
que les lettres trouvées dans sa cassette ne furent pas toutes
détruites, comme on l'a souvent répété\,; elles existent encore pour la
plupart, et attestent les folles prodigalités du
surintendant\footnote{Ces lettres ont été conservées par Baluze,
  bibliothécaire de Colbert, et font aujourd'hui partie des manuscrits
  de la Bibliothèque Impériale. C'est de là que j'ai tiré plusieurs des
  pièces citées dans cette note.}. On prétend que Fouquet, enivré de sa
fortune, osa élever ses prétentions jusqu'à M\textsuperscript{lle} de La
Vallière. Cette accusation, reproduite dans quelques Mémoires du
temps\footnote{Voy. principalement les \emph{Mémoires du jeune Brienne}
  (H. L. de Loménie).}, reçoit une nouvelle confirmation de la lettre
suivante qu'une des entremetteuses de Fouquet lui écrivait\footnote{La
  copie de cette lettre se trouve dans les manuscrits Conrart,
  bibliothèque de l'Arsenal, in-4°, t. XI, p.~152. On comprend que
  l'original d'une pareille lettre ait été détruit\,; mais, comme on
  retrouve dans les papiers de Fouquet, plusieurs lettres dont les
  copies, données par Conrart, reproduisent l'esprit, sinon les
  expressions, il n'y a pas de motif suffisant pour rejeter cette lettre
  comme apocryphe. La copie est de la main de Conrart.}\,:

«\,Je ne sais plus ce que je dis ni ce que je fais lorsqu'on résiste à
vos intentions. Je ne puis sortir de colère lorsque je songe que la
petite demoiselle de La Vallière a fait la capable avec moi. Pour
captiver sa bienveillance, je l'ai assurée sur sa beauté, qui n'est
pourtant pas grande\footnote{C'est aussi l'avis de plusieurs écrivains
  contemporains.}\,; et puis lui ayant fait connaître que vous
empêcheriez qu'elle manquât jamais de rien, et que vous avez vingt mille
pistoles pour elle, elle se gendarma contre moi, disant que deux cent
cinquante mille livres n'étaient pas capables de lui faire faire un faux
pas\,; et elle me répéta cela avec tant de fierté, quoique je n'aie rien
oublié pour l'adoucir avant de me séparer d'elle, que je crains fort
qu'elle n'en parle au roi, de sorte qu'il faut prendre des devants pour
cela\footnote{On a prétendu, en effet, que Louis XIV fut instruit de la
  passion du surintendant pour aille de La Vallière, et que ce fut une
  des causes de l'acharnement avec lequel il poursuivit Fouquet.}. Ne
trouvez-vous pas à propos de dire, pour la prévenir, qu'elle vous a
demandé de l'argent et que vous lui en avez refusé\,? Cela rendra
suspectes toutes ses plaintes. Pour la grosse femme\footnote{La reine
  mère Anne d'Autriche.}, Brancas et Grave vous en rendront bon
compte\,; quand l'un la quitte, l'autre la reprend. Enfin je ne fais
point de différence entre vos intérêts et mon salut. En vérité, on est
heureux de se mêler des affaires d'un homme comme vous\,; votre mérite
aplanit tous les obstacles. Si le ciel vous faisait justice, nous vous
verrions un jour la couronne fermée.\,»

La couronne fermée était un signe de souveraineté, et on peut se figurer
l'indignation du jeune roi à la lecture d'une lettre qui lui montrait
dans son ministre un rival d'amour et de puissance. Je n'insiste pas sur
les expressions injurieuses dont l'entremetteuse se servait pour
désigner la mère de Louis XIV. Cette princesse avait encore une haute
influence, et Fouquet s'était efforcé de la gagner peu de temps avant la
mort du cardinal Mazarin, qui arriva en mars 1661.

Une lettre écrite de la main même de Fouquet renferme les propositions
qu'il lui faisait adresser\footnote{La reine mère n'est pas nommée dans
  ces propositions\,; mais il est très vraisemblable qu'elles devaient
  lui être soumises, puisqu'elles sont jointes à la lettre suivante
  écrite par une des personnes que Fouquet avait chargées de surveiller
  et de gagner Anne d'Autriche\,: «\, J'attendais toujours d'avoir
  l'honneur de vous entretenir pour vous dire bien des choses. Je ne
  sais si vous savez le pouvoir que la mère de la Miséricorde a sur la
  reine et l'intrigue secrète qui s'y fait. M. Le Tellier et M. de
  L'Estrade la voient tous les jours. On m'en a dit bien des choses avec
  le secret. Si cela vous est utile, faites-le-moi savoir\,; vous savez
  que je suis tout à vous et qu'il n'y a rien que je ne fasse pour vous
  le témoigner.\,»}.

«\,On ne veut point, disait le surintendant, que la bonté qu'elle a lui
soit à charge\,; on aime mieux prendre tout sur soi que de la commettre.
Si on a quelques sentiments ou quelque conduite qu'elle n'approuve pas,
on lui demande en grâce de le dire. Un mot réglera tout sur le pied
qu'il lui plaira. On conjure d'accorder sa confiance et de faire
connaître toutes les choses qu'elle affectionnera, de quelque nature
qu'elles soient, et celles qu'elle voudra faire réussir sans y paraître,
et on demande cela avec la plus grande instance du monde, n'ayant point
de plus forte passion que de rendre quelque service agréable, et le zèle
n'empêchera pas que l'on ait la discrétion nécessaire. Tout le monde
appréhende la domination nouvelle de M. le Prince (Louis de Bourbon), et
que Son Éminence ne puisse résister à ses flatteries\footnote{Le prince
  de Condé avait quitté la Belgique pour rentrer en France le 29
  décembre 1659\,; Mazarin mourut le 9 mars 1661\,; c'est entre ces deux
  dates, probablement vers le commencement de 1660, que cette lettre de
  Fouquet a dû être écrite. Quant aux flatteries de Condé envers
  Mazarin, on en trouve la preuve dans une lettre que le prince écrivait
  au cardinal le 24 décembre 1659, peu de jours avant de quitter
  Bruxelles\,: «\,Pour vous, monsieur, lui disait-il, quand je vous
  aurai entretenu une heure, vous serez bien persuadé que je veux être
  votre serviteur, et je pense que vous vous voudrez bien aussi
  m'aimer.\,»}, et que l'on ait le déplaisir de le voir, sous divers
prétextes, triompher de ceux qui ont servi longtemps contre lui. Secret
et dissimulation, sans exception, à tout le monde. M. Le Tellier vit
fort honnêtement, en apparence, mais peut avoir jalousie et craindre que
la faveur n'aille d'un autre côté. Si elle trouve bon qu'on lui rende
compte de ce qu'on apprend, ou s'il y a quelque chose dont elle désire
savoir la vérité, en s'ouvrant un peu, on tâchera de la satisfaire.\,»

Fouquet ne paraît pas avoir réussi à gagner Anne d'Autriche. Il prit
alors les plus minutieuses précautions pour pénétrer ses secrets il
l'entoura d'espions et gagna jusqu'à son confesseur. Nous avons les
lettres d'un anonyme qui servait d'intermédiaire entre Fouquet et le
cordelier confesseur de la reine. En voici quelques extraits\,:

«\,Le cordelier dit hier\footnote{Cette lettre est du 22 avril 1661.} à
la personne dont j'ai parlé à monseigneur que la reine mère lui avait
conté un mécontentement qu'elle avait eu du roi, sur ce que l'autre
jour, entrant fort brusquement dans sa chambre, il lui fit reproche de
ce qu'elle avait prié M. de Brienne\footnote{Secrétaire d'État chargé
  des affaires étrangères.} de quelque affaire, et qu'il lui dit en
propres termes et fort en colère\,: \emph{Madame, ne faites plus de
pareilles choses sans m'en parler\,;} qu'à cela la reine ne répondit
rien et ne fit que rougir. Il a encore dit que Monsieur\footnote{Philippe
  d'Orléans, frère de Louis XIV.} se plaignait, et qu'il avait dit
depuis à quelqu'un que le roi le traitait comme un chien. Au reste, il
assure que la reine mère croit que M. le Prince\footnote{Louis de
  Bourbon dont il était question dans la pièce citée précédemment.}
pense fort à se mettre dans les affaires\,; qu'elle dit avoir remarqué
une patience extrême en lui pour faire sa cour\,; que le roi l'estime
fort, et que, sur toutes les choses qu'il fait, il demande aux gens si
M. le Prince les approuve. Il est même très constant qu'il tâche à
cabaler. Il a été voir ce bonhomme de cordelier\,; et la reine mère,
quoiqu'elle ait une furieuse défiance de lui, l'aimerait encore mieux
que rien\,; car il la recherche. Je tâcherai d'écrire quelque chose à
monseigneur du P. Annat\footnote{Jésuite confesseur de Louis XIV.}\,;
mais comme c'est un homme fort réservé, je n'ose rien promettre.\,»

Peu de jours après, le même espion écrivait à Fouquet\,:

«\,Je n'ai point osé m'empresser ce matin à vous suivre pour vous
apprendre, monseigneur, ce que le bon religieux que vous savez me dit
hier. J'en appris, entre autres choses, qu'il croyait qu'il pourrait
bien n'y avoir plus de conseil de conscience\,; et qu'il y avait deux
jours que quelqu'un donna avis et envie au roi de voir une lettre que
ces messieurs du conseil de conscience écrivaient à Rome par son ordre.
Le paquet étant déjà entre les mains du courrier, fut reporté au roi,
qui trouva que, dans cette lettre qu'il n'avait point vue, ces messieurs
écrivaient qu'ils tenaient le roi dans l'obéissance exacte qu'il devait
au saint-siège, et s'attribuaient comme la gloire de le gouverner. Cela
le choqua extrêmement, et, jaloux comme il est de son autorité, il parut
si irrité qu'il protesta qu'il ne les assemblerait plus.

«\,Au reste, M\textsuperscript{me} de Chevreuse\footnote{Marie de Rohan,
  née en décembre 1600\,; elle avait épousé en 1617 Charles d'Albert,
  duc de Luynes\,; veuve en 1621, elle se remaria l'année suivante avec
  Claude de Lorraine, duc de Chevreuse\,; elle mourut le 12 août 1679.
  M\textsuperscript{me} de Chevreuse, dont le nom reparaît plusieurs
  fois dans ces lettres était une des ennemies de Fouquet.} continue
toujours à faire de grandes recherches à ce bonhomme-ci\,; mais
assurément cela ne servira de rien, et vous apprendrez précisément tout
ce qu'elle lui dira. Il persiste à croire ce que je vous ai écrit du roi
et de M\textsuperscript{lle} de La Vallière, et pense que ce\,: qu'il en
dit il y a quelque temps est absolument vrai.

«\,Comme j'ai appris depuis peu que le P. Leclerc, que je pensais qui
devait être confesseur du roi après le P. Annat, le sera de Monsieur, je
puis vous assurer que si cela est de quelque chose, j'aurai des
habitudes et des liaisons aussi étroites avec lui que j'en ai auprès du
bon père.\,»

L'influence de M\textsuperscript{me} de Chevreuse inquiétait
particulièrement Fouquet, et il chargea la personne qui lui transmettait
ces renseignements de découvrir les projets de cette dame. Il en reçut,
le 21 juillet 1661, la réponse suivante\,:

«\,Je n'ai pu rien savoir de plus particulier de chez
M\textsuperscript{me} de Chevreuse\,; mais depuis peu le bonhomme de
confesseur est venu ici pour voir la personne dont j'ai eu l'honneur de
vous parler autrefois. Il lui a conté tout ce qu'il savait, et, entre
autres choses, lui a dit que depuis quelque temps M\textsuperscript{me}
de Chevreuse lui avait fait de grandes recherches\,; qu'elle lui avait
envoyé Laigues\footnote{Laigues était le \emph{mari de conscience} de
  M\textsuperscript{me} de Chevreuse. Voy. les \emph{Mémoires du jeune
  Brienne}.} plusieurs fois\,; qu'il lui avait parlé fort dévotement
pour le gagner, mais surtout qu'il lui avait parlé contre vous,
monseigneur. Je ne m'étendrai point de quelle sorte\,; car ce
bonhomme-ci a dit qu'il l'avait conté à M. Pellisson \footnote{Pellisson
  était un des principaux commis de Fouquet.}**. Il me suffira donc de
vous faire savoir sur cela que le bonhomme de cordelier se plaint un peu
de ce qu'en faisant un éclaircissement à la reine mère, vous l'aviez
comme cité\,; et que lui disant qu'elle allait à Dampierre\footnote{Château
  de M\textsuperscript{me} de Chevreuse.} parmi vos ennemis, et qu'on
lui avait dit des choses contre vous, comme elle niait qu'on lui eût
jamais parlé de cette sorte, vous lui dites de le demander au père
confesseur\,; que le lendemain la reine lui avait dit qu'elle ne pouvait
comprendre comment vous saviez toutes choses, et que vous aviez des
espions partout.

«\,La reine a encore dit qu'elle voyait une cabale dans la cour fort
méchante qu'elle ne connaissait point et qu'elle ne pouvait encore
pénétrer\footnote{Il s'agit probablement de la cabale de la comtesse de
  Soissons.}\,; qu'elle a su que depuis peu on avait fait coucher le roi
avec une jeune personne, de laquelle ce bonhomme n'a pu redire le nom\,;
et que la reine avait encore ajouté que le roi se relâchait fort sur la
dévotion\,; qu'il ne se confessait ni ne communiait pas si souvent, et
que le P. Annat était un pauvre homme, et si timide qu'il n'osait dire
aucune chose au monde au roi, de peur que cela n'allât contre ses
intérêts.

«\,Il a encore dit que la reine mère, en parlant des mécontentements
qu'elle avait sur Madame\footnote{Henriette d'Angleterre, femme de
  Philippe d'Orléans.}, lui avait assuré qu'elle était une profonde
coquette et une artificieuse\,; mais qu'aussi la jeune reine\footnote{Marie-Thérèse
  d'Autriche.} lui donnait bien de la peine avec ses larmes et toutes
ses façons de faire.

«\,Elle a ajouté encore que depuis peu le roi lui avait dit que M. le
cardinal, en mourant lui avait protesté, en lui parlant contre elle,
\emph{qu'elle ne se passerait jamais d'homme\footnote{Ces mots sont
  soulignés dans le manuscrit.}\,;} qu'il prît garde à elle, et
qu'assurément elle ferait un mariage de conscience avec quelqu'un. Au
reste, ce bonhomme assure que la reine mère reçoit tous les jours des
avis contre tous les ministres, et que tantôt vous ôtes bien et tantôt
mal dans son esprit\,; qu'on vous y rend souvent de très méchants
offices, et que dans ces temps-là elle est fort déchaînée contre
vous.\,»

Ce correspondant de Fouquet lui donnait quelquefois de bons conseils. Il
lui écrivait le 2 août 1661\,:

«\,Le zèle, et la passion extrême que j'ai pour voire service,
monseigneur, m'avaient fait penser en général, comme à plusieurs de vos
serviteurs, qu'il ne vous serait point avantageux en aucune sorte de
vous défaire de votre charge de procureur général. Cependant, par la
connaissance et par l'admiration que j'ai pour votre prudence et pour
votre jugement, j'étais entièrement, persuadé qu'il n'y avait rien de
mieux, et que personne ne pouvant aller si loin ni juger si bien par ses
propres lumières que vous, vous ne deviez prendre conseil que de
vous-même. Cependant, monseigneur, j'ai appris aujourd'hui que vos
ennemis sont ceux-là mêmes qui souhaitent avec passion que vous fassiez
ce que vous avez résolu en cette rencontre\,; que ce sont eux qui vous y
portent sous main, et que vous devez même vous défier du bon accueil et
du bon visage que vous fait le roi, et des vues qu'on vous donne sur
d'autres choses.

«\,M\textsuperscript{me} de Chevreuse a été ici, et l'on m'a promis de
m'apprendre des choses qui vous sont de la dernière conséquence sur
cela, sur le voyage de Bretagne\footnote{Le voyage de Bretagne et
  l'arrestation de Fouquet eurent lieu au commencement de septembre.},
sur certaines résolutions très secrètes du roi, et sur des mesures
prises contre vous. Comme je n'ai pas voulu paraître fort empressé pour
savoir ce qu'on avait à me dire, je n'ai pas osé presser la personne qui
m'a parlé, ni m'opiniâtrer à demander une chose que je saurai demain
naturellement et sans affectation.

«\,La reine mère dit dimanche dernier, sur vous, que M. le cardinal
avait dit au roi que si l'on pouvait vous ôter les bâtiments et les
femmes de la tête, vous étiez capable de grandes choses\,; mais que
surtout il fallait prendre garde à votre ambition\,; et c'est par là
qu'on prétend vous nuire. J'ai compris aussi que, de plusieurs personnes
qui vous rapportent ce qu'ils peuvent attraper, il y en a beaucoup qui
s'y gouvernent étourdiment, et qui font les choses d'une manière qui
fait voir qu'ils ne veulent savoir que pour vous rapporter ce qu'ils
savent. Ce qui a fait dire à la reine mère, encore depuis peu, que vous
aviez des espions partout.\,»

La lettre suivante contenait encore des avis menaçants sur les
dispositions du roi\,:

«\,L'on me dit hier qu'il y a peu de jours la reine mère, en parlant de
vous, monseigneur, dit\,: «\,Il se croit à cette heure bien mieux que M.
D. à la charge de maître de la chapelle du roi, qu'on a achetée trois
fois plus qu'elle ne valait\,; il verra, il verra à quoi cela lui a
servi, et ce qu'a fait sur l'esprit du roi tout l'argent qu'il a baillé
de sa propre bourse pour le marquis de Créqui\footnote{François de
  Créqui avait épousé la fille de M\textsuperscript{me} du
  Plessis-Bellière, qui avait une grande influence sur le surintendant.}.
Le roi aime d'être riche, et n'aime pas ceux qui le sont plus que lui,
puisqu'ils entreprennent des choses qu'il ne saurait faire lui-même, et
qu'il ne doute point que les grandes richesses des autres ne lui aient
été volées.\,»

«\,M\textsuperscript{me} de Chevreuse, lorsqu'elle fut ici, fut voir
deux fois le confesseur de la reine mère. Cependant ce bonhomme cacha
cela à M. Pellisson, qui, l'ayant été voir, lui demanda s'il ne l'avait
point vue\,; ce qu'il lui nia, comme il a dit depuis. Il a encore dit
ici des choses qu'il a données sous un fort grand secret, et qui sont de
très grande conséquence. La personne qui les sait fait difficulté de me
les dire, parce que M\textsuperscript{me} de Chevreuse y est mêlée, et
que lui étant aussi proche, elle a peine à me les dire. Je ne manquerai
point de vous les apprendre lorsque je les saurai, ne doutant point
qu'on ne me les dise enfin. Si M. Pellisson voit le bonhomme, il ne faut
pas qu'il fasse l'empressé avec lui, ni qu'il témoigne savoir ce qu'il
n'a pas voulu lui dire.\,»

Ces avis n'arrêtèrent point Fouquet dans la voie qui le menait à
l'abîme. Il crut, après la mort de Mazarin (9 mars 1661), que la
puissance du cardinal allait passer tout entière entre ses mains. La
plupart de ses partisans l'entretenaient dans cette illusion\,; leurs
lettres apprennent qu'ils le nommaient l'\emph{Avenir}, et voyaient déjà
en lui l'arbitre de la France. L'un d'eux lui écrivait (le Bordeaux, le
29 août 1661, quelques jours avant son arrestation\,: «\,Si les ennemis
de monseigneur ont fait courir des bruits à son désavantage, ils sont
bien punis. Tout le monde présentement, dans ces provinces, ne parle que
du crédit qu'il a sur l'esprit du roi, et dit cent choses avantageuses
que je ne puis mettre sur ce papier.\,»

Jusqu'à quel point Fouquet porta-t-il ses vues ambitieuses\,? Voulut-il,
comme on l'a souvent répété, faire de Belle-Ile une forteresse, où il
aurait pu, en cas de disgrâce, braver l'autorité du roi\,? On ne peut
nier l'authenticité du plan trouvé dans ses papiers pour fortifier cette
île et prendre toutes les mesures nécessaires afin de se mettre à l'abri
de la vengeance du roi. Jamais ni Fouquet ni ses défenseurs n'ont
prétendu que ce plan eût été inventé par leurs ennemis. On voit
d'ailleurs, par les lettres adressées au commandeur de
Neuchèse\footnote{Ce commandeur de l'ordre de Malte avait été nommé
  vice-amiral et intendant général de la marine le 7 mai 1661, en
  remplacement de Louis Foucault de Saint-Germain.}, que Fouquet
comptait sur les galères de cet amiral, et que Neuchèse faillit être
compromis dais son procès\footnote{Un des amis du commandeur de Neuchèse
  lui écrivait le 19 octobre 1661\,: «\,On vous a servi ici de bonne
  manière, \emph{et en vérité vous en aviez grand besoin}. On n'a jamais
  vu une telle rage que celle de M. Fouquet\,; car il a fait tout son
  possible pour perdre amis et indifférents. À La lettre se termine par
  le post-scriptum suivant\,: «\,Assurément on fera le procès à M.
  Fouquet. Si vous aviez le temps, on vous pourrait bien mander de venir
  \emph{ici dire votre projet\,;} mais n'y songez pas, si on ne vous
  l'ordonne.\,»}. Il se tint même caché pendant quelque temps, comme le
prouve la lettre suivante, que lui adressait le duc de Beaufort à la fin
d'octobre 1661\,: «\,Monsieur, vous vous tenez fort caché sur tous les
bruits qui ont couru à la cour, et les démarches de votre secrétaire
sont cause que ces bruits se confirment, Pour moi, comme votre ami,
lorsqu'on parle, je réponds des épaules, et je ne sais que dire, puisque
vous vous êtes caché de moi comme des autres. Vous êtes bon et sage,
mais la Toussaint vous trouve encore non embarqué. Croyez que cela vous
fait grand tort, et plus que je ne vous le saurais dire. Remédiez-y
promptement\footnote{Voy. plus haut, le récit de l'arrestation de
  Fouquet.}.\,» L'affaire du commandeur de Neuchèse fut étouffée\,; mais
les lettres que nous venons de citer confirment les soupçons qu'avait
inspirés le plan trouvé à Saint-Mandé, dans la maison de Fouquet.
Neuchèse y est indiqué comme s'étant engagé à servir le surintendant
\emph{envers et contre tous}.

D'ailleurs les dilapidations de Fouquet étaient parfaitement établies,
et Louis XIV n'avait que trop de motifs pour le livrer à la rigueur de
la chambre de justices\,; mais la violence que l'on mit dans la
poursuite, les efforts des amis de Fouquet, la pitié qui s'attache
naturellement au malheur, la longueur même dû procès, concilièrent peu à
peu au surintendant l'opinion publique. On voulut exercer sur les juges,
et principalement sur l'un des rapporteurs, Olivier d'Ormesson, une
influence inique. D'Ormesson lui-même raconte, dans son journal inédit,
la démarche que fit Colbert auprès de son père\footnote{André
  d'Ormesson, doyen du conseil d'État.} pour se plaindre de la longueur
du procès. Voici ce passage important\,:

«\,Samedi {[}3 mai 1664{]}, étant après le dîner avec mon père dans son
cabinet, et le P. d'Ormesson\footnote{Nicolas Lefèvre d'Ormesson,
  religieux minime, était frère d'Olivier d'Ormesson.}, auquel je
faisais entendre qu'il ne devait plus avoir aucun commerce avec
Berryer\footnote{Berryer, un des commis de Colbert, avait été chargé de
  l'inventaire des pièces du procès de Fouquet, et accusé de les avoir
  falsifiées.}, parce qu'il abusait de sa franchise, et lui faisait dire
bien des choses au delà de celles qu'il avait dites, et en prenait
avantage, et ayant fait entendre à mon père l'injustice de leur
conduite, l'on nous vint dire que M. Colbert entrait. Nous étant
retirés, il resta seul avec mon père près d'une demi-heure. Étant sorti
avec un visage fort sérieux, mon père nous dit qu'après les premières
civilités, il lui avait dit qu'il avait ordre du roi de lui venir dire
qu'il reconnaissait que je n'apportais pas toutes les facilités que je
pouvais pour terminer le procès de M. Fouquet, et qu'il semblait, que
j'affectais la longueur\,; que le roi était persuadé que je ferais
justice au fond, et ne prétendait pas contraindre mes sentiments\,; mais
qu'il voulait faire finir ce procès\,; que la chambre de justice ruinait
toutes les affaires, et qu'il était fort extraordinaire qu'un grand roi,
craint et le plus puissant de toute l'Europe, ne pût pas faire achever
le procès à un de ses sujets, comme M. Fouquet\,; qu'à cela il (mon
père) lui avait répondu qu'il était bien fâché que le roi ne fût pas
satisfait de ma conduite\,; qu'il savait que je n'avais que de bonnes
intentions\,; qu'il m'avait toujours recommandé la crainte de Dieu, le
service du roi, et la justice sans acception de personnes\,; que la
longueur du procès ne venait pas de moi, mais parce qu'il était fort
grand, et qu'on l'avait rempli de trente, ou quarante chefs
d'accusation, où il n'en fallait que deux ou trois\,; qu'un prédicateur
qui prêchait la passion n'était pas trop long parlant trois heures, et
quoique les autres sermons ne fussent que d'une heure\,; qu'il faudrait
que j'eusse perdu le sens de chercher à plaire à M. Fouquet, dont la
fortune était abîmée, et déplaire au roi, qui avait toutes les grâces en
ses mains\,; mais que je ne cherchais que la justice ; que tous mes avis
étaient suivis dans la chambre\,; que ceux mêmes qui ne l'avoient pas
été d'abord l'avoient été depuis\,; que même il apprenait de tous côtés
que je me conduisais de sorte que l'on ne pouvait découvrir mes
sentiments\,; que sur, cela M. Colbert lui avait dit que l'on remarquait
pourtant que je disais plus fortement et plus gaiement les raisons de M.
Fouquet que celles du procureur général\,; qu'il lui avait répliqué
qu'un rapporteur était obligé de faire valoir toutes les raisons\,; que
l'on m'avait ôté l'intendance de Soissons, mais que je ne m'en
plaindrais pas, et que cela ne m'empêcherait pas de faire justice\,;
qu'il avait peu de biens et moi aussi, mais que nous les avions de, nos
pères et que nous en étions contents\,; qu'il m'avait toujours conseillé
de faire justice sans acception de personnes et sans considération
d'intérêt et de fortune\,; et qu'ayant parlé des personnes qui me
faisaient visite, M. Colbert avait dit qu'on n'était pas en peine de
cela, et qu'on savait bien que je ferais justice, mais qu'on ne désirait
que l'expédition\,; qu'il lui avait répliqué que je faisais tout ce qui
dépendait de moi, travaillant soir et matin, et ne faisant autre
chose\,; et ainsi, après plusieurs discours de cette qualité, il s'était
retiré.

«\,Je fus ravi que mon père lui eût, parlé si bien et si généreusement,
et j'en allai faire aussitôt la relation à M. Le Pelletier\footnote{Claude
  Le Pelletier fut contrôleur général des finances en 1683, après la
  mort de Colbert.}, pour en informer M. Le Tellier, afin qu'il prît
garde à la manière dont M. Colbert en parlerait. Nous fûmes ensemble le
soir voir M. le premier président\footnote{Guillaume de Lamoignon.}, qui
était avec M. Colbert, et entretint ensuite M. le maréchal de Villeroy.
Il fut fort surpris d'apprendre cette visite, qui est contre toutes les
règles de la prudence. Là j'appris que M. Berryer était conseiller
d'État ordinaire\,; que le roi lui avait donné une abbaye de six mille
livres, et voulait qu'il donnât le nom de ses enfants pour obtenir de
Rome une dispense de tenir des bénéfices avant l'âge, et qu'il avait
mandé les procureurs généraux de la chambre\footnote{Il y avait alors
  deux procureurs généraux de la chambre de justice, Hotman et
  Chamillart, tous deux maîtres des requêtes.} pour leur dire qu'il
voulait que M. Berryer eût connaissance de toutes les affaires de la
chambre de justice, et qu'ils ne prissent aucunes conclusions que par
son avis, et qu'il sollicitât tous les juges de la chambre de justice
pour ses intérêts.

«\,Une conduite si bizarre et si extraordinaire m'oblige à dire ici les
sentiments qu'on en a. Tout le monde blâme M. Colbert de se charger
lui-même de messages désagréables\,; d'avoir voulu voir lui-même M.
Boucherat\footnote{Louis Boucherat, conseiller d'État\,; il devint
  chancelier de France après la mort du maréchal Le Tellier.} pour faire
plus d'éclat et augmenter l'injure, vu que la même chose se pouvait
faire doucement, sans bruit, et M. Le Tellier s'étant offert de lui
parler\,; d'avoir voulu venir encore lui-même parler à mon père, par le
même principe\,; que d'ôter {[}de la chambre de justice{]} M. Boucherat,
homme de bien et de réputation, c'était faire connaître que ses
intentions étaient mauvaises\,; que de m'avoir ôté l'intendance de
Soissons, étant rapporteur, c'était me faire honneur et se charger de
honte, et faire croire qu'il désirait de moi des choses injustes, et que
j'avais assez d'honneur pour y résister\,; que c'était achever de gâter
le procès en faisant injure au rapporteur, et me mettant hors d'état de
leur être favorable, quand j'en aurais le dessein\,; car l'on
attribuerait mes sentiments à crainte ou à intérêt, et non pas à
justice\,; et, pour comble, d'élever Berryer et le faire conducteur
public de toutes les affaires de la chambre de justice, c'était faire
gloire d'infamie et de honte\,; car Berryer est le plus décrié des
hommes. »

Cette intervention de Colbert avait produit un effet plus défavorable
qu'utile à la cause qu'il voulait faire triompher. Les lettres de
M\textsuperscript{me} de Sévigné suffiraient pour prouver à quel point
l'opinion publique se déclarait en faveur de Fouquet. Le Tellier
lui-même en convint dans une visite que lui fit Olivier
d'Ormesson\footnote{\emph{Journal d'Olivier d'Ormesson}, à la date du 2
  mai 1664.}\,:

«\,Je fus dire adieu à M. Le Tellier, qui me fit entrer dans son jardin,
et lui ayant témoigné lui avoir obligation de là manière dont je savais
qu'il avait parlé, il me dit mille civilités\,; que tout ceci ne serait
rien, et qu'il ne fallait pas que je témoignasse aucun ressentiment\,;
mais que j'allasse toujours le même chemin, sans faire ni plus ni moins,
afin que l'on ne crût pas que je fisse rien par crainte, ni aussi que je
me voulusse venger. Il me parla ensuite du procès, des fautes qu'on y
avait faites, entra dans le détail, dit qu'on avait fait la corde trop
grosse\,; qu'on ne pouvait plus la serrer\,; qu'il ne fallait qu'une
chanterelle\footnote{Corde de luth ou de violon, fort mince.}\,; me
parla fort que M. le cardinal (Mazarin) n'avait jamais pris un quart
d'écu par le moyen de M. Fouquet\,; mais qu'il avait des
prêts\footnote{On voit par ce passage que Mazarin faisait des avances à
  l'État et se remboursait sur les deniers publics. Le Tellier avoue que
  Mazarin prêtait à l'État à gros intérêts, \emph{et ainsi gagnait
  beaucoup}. C'est à peu près ce que dit Saint-Simon (p.~112 du t. XIV).},
et, pour son remboursement, avait pris des recettes, sur lesquelles on
lui donnait la remise comme aux traitants, et lui n'en donnait que peu,
et ainsi avait gagné beaucoup.\,»

Louis XIV lui-même eut occasion de s'expliquer avec les rapporteurs sur
le procès, dont il blâmait la lenteur. Il le fit avec une dignité
qu'Olivier d'Ormesson s'empresse de reconnaître\footnote{\emph{Journal
  d'Olivier d'Ormesson}, à la date du 8 juillet 1664.}\,:

«\,A trois heures, je fus avec M. de Sainte-Hélène\footnote{C'était le
  second rapporteur du procès de Fouquet.} Il au château\footnote{Château
  de Fontainebleau, où la cour résidait alors. La chambre de justice y
  avait été transférée. Fouquet était enfermé à Moret.}**. Nous
trouvâmes le roi dans son cabinet avec MM. Colbert et Lyonne\footnote{Hugues
  de Lyonne, secrétaire d'État chargé des affaires étrangères.}, et
s'étant avancé près de la fenêtre, il nous dit ces mêmes paroles, autant
que j'ai pu m'en souvenir\,:

«\,Lorsque j'ai trouvé bon que Fouquet eût un conseil libre, j'ai cru
que son procès durerait peu de temps\,; mais il y a plus de deux ans
qu'il est commencé, et je souhaite extrêmement qu'il finisse. Il y va de
ma réputation. Ce n'est pas que ce soit une affaire de grande
conséquence\,; au contraire je la considère comme une affaire de rien\,;
mais dans les pays étrangers, où j'ai intérêt que ma puissance soit bien
établie, l'on croirait qu'elle ne serait pas grande si je ne pouvais
venir à bout de faire terminer une affaire de cette qualité contre, un
misérable. Je ne veux néanmoins que la justice, mais je souhaite voir la
fin de cette affaire de quelque manière que ce soit. Quand la chambre a
cessé d'entrer, et qu'il a fallu transférer M. Fouquet à Moret, j'ai dit
à Artagnan de ne plus lui laisser parler avec les avocats, parce que je
ne voulais pas qu'il fût averti du jour de son départ. Depuis qu'il a
été à Moret, je lui ai dit de ne les laisser communiquer avec lui que
deux fois la semaine, et en sa présence, parce que je ne veux\,: pas que
ce conseil soit éternel\,; et j'ai su que les avocats avoient excédé
leur fonction, avoient porté et reporté des paquets et tenu un autre
conseil au dehors, quoiqu'ils s'en défendent fort\,; et puis dans ce
projet, par lequel il voulait bouleverser l'État\footnote{Ce projet,
  trouvé dans la maison de Fouquet à Saint-Mandé, a été publié par M. P.
  Clément, \emph{Histoire de Colbert}, introduction.}, il doit faire
enlever le procès et les rapporteurs. C'est ce qui m'a fait donner cet
ordre, et je crois que la chambre y ajoutera. Je m'en remets néanmoins à
ce qu'elle fera sur la requête de M. Fouquet\footnote{Par cette requête,
  Fouquet demandait à communiquer librement avec ses défenseurs.}. Je ne
veux sur tout cela que la justice, et je prends garde à tout ce que je
vous dis\,; car, quand il est question de la vie d'un homme, je ne veux
pas dire une parole de trop. La chambre donc ordonnera ce qu'elle jugera
à propos. J'aurais pu vous dire mes intentions dès hier\,; mais j'ai
voulue voir la requête, et je me la suis fait lire avec application, et
on est bien aise de savoir ce que l'on a à dire. Je vous ai dit mes
intentions, et je vous rends la requête, afin que la chambre y
délibère.\,»

«\,Après ce discours, le roi m'ayant donné la requête, je lui dis que
nous ferions rapport à la chambre de ce qu'il avait plu à Sa Majesté de
nous dire, et nous nous retirâmes. Je ne veux pas omettre une
circonstance qui me parut fort belle au roi, c'est qu'étant demeuré
court au milieu de son discours, il demeura quelque temps à songer pour
se reprendre, et nous dit\,: «\,J'ai perdu ce que je voulais dire,\,» et
songea encore assez de temps\,; et ne retrouvant point ce qu'il avait
médité, il nous dit\,: «\,Cela est fâcheux quand cela arrive\,; car en
ces affaires, il est bon de ne rien dire que ce qu'on a pensé.\,»

Il y a loin de cette parole mesurée et sérieusement réfléchie aux
anecdotes que l'historien protestant La Hode a recueillies\footnote{\emph{Hist.
  de Louis XIV}, liv. LXXVII, p.~162.}, et que M. de Sismondi a
reproduites\footnote{\emph{Hist. des Français}, t. XXV, p.~75.}. D'après
ces écrivains, Louis XIV aurait personnellement sollicité Olivier
d'Ormesson pour ce qu'il aurait appelé \emph{son affaire}, et d'Ormesson
lui aurait répondu\,: «\,Sire, je ferai ce que mon honneur et ma
conscience me suggéreront.\,» Et pour rendre l'anecdote plus piquante,
les inventeurs ont eu soin d'ajouter que d'Ormesson sollicitant dans la
suite une grâce pour son fils. Louis XIV lui dit, comme parodiant les
paroles du magistrat\,: «\,Je ferai ce que mon honneur et ma conscience
me suggéreront.\,» Rien de plus faux que ces anecdotes. Il n'était ni
dans le caractère de Louis XIV de descendre à des sollicitations
personnelles, ni dans celui d'Olivier d'Ormesson de répondre au roi avec
une hauteur insolente. Ce magistrat savait concilier l'intègre
observation de la justice et le respect pour l'autorité souveraine. Le
résumé qu'il fait du procès en est une nouvelle preuve\,:

«\,Voilà ce grand procès fini, qui a été l'entretien de toute la France
du jour qu'il a été commencé jusqu'au jour qu'il a été terminé. Il a été
grand bien moins par la qualité de l'accusé et l'importance de
l'affaire, que par l'intérêt des subalternes, et principalement de
Berryer, qui y a fait entrer mille choses inutiles et tous les
procès-verbaux de l'épargne, pour se rendre nécessaire, le maître de
toute cette intrigue, et avoir le temps d'établir sa fortune\,; et
comme, par cette conduite, il agissait contre les intérêts de M.
Colbert, qui ne demandait que la fin et la conclusion, et qu'il le
trompait dans le détail de tout ce qui se faisait, il ne manquait pas de
rejeter les fautes sur quelqu'un de la chambre\,: d'abord ce fut contre
les plus honnêtes gens de la chambre, qu'il rendit tous suspects, et les
fit maltraiter par des reproches publics du roi\,; ensuite il attaqua M.
le premier président, et le fit retirer de la chambre et mettre en sa
place M. le chancelier. Après il fit imputer toute la mauvaise conduite
de cotte affaire à M. Talon\footnote{Denis Talon, fils d'Omer Talon,
  avait d'abord été procureur général de la chambre de justice.}, qu'on
ôta de la place de procureur général avec injure\,; et enfin, la
mauvaise conduite augmentant, les longueurs affectées par lui
continuant, il en rejeta tout le mal sur moi\,; il me fit ôter
l'intendance de Soissons, il obligea M. Colbert à venir faire à mon père
des plaintes de ma conduite\,; et enfin l'expérience ayant fait
connaître qu'il était la véritable cause de toutes les fautes, et les
récusations ayant fait voir ses faussetés, les procureurs généraux
Hotman et Chamillart lui firent ôter insensiblement tout le soin de
cette affaire, et dans les derniers mois il ne s'en mêlait plus, et pour
conclusion il est devenu fou\footnote{Voy. M\textsuperscript{me} de
  Sévigné, lettre du 17 décembre 1664.}, et ainsi le procès s'est
terminé\,; et je puis dire que les fautes importantes dans les
inventaires, les coups de haine et d'autorité qui ont paru dans tous les
incidents du procès, les faussetés de Berryer et les mauvais traitements
que tout le monde, et même les juges, recevaient dans leur fortune
particulière\footnote{Olivier d'Ormesson fait allusion à la réduction
  des rentes opérée par Colbert en 1664.}, ont été de grands motifs pour
sauver M. Fouquet de la peine capitale\,; et la disposition des esprits
sur cette affaire a paru par la joie publique, que les plus grands et
les plus petits ont fait paraître du salut de M, Fouquet, jusques à un
tel excès qu'on ne le peut exprimer, tout le monde donnant des
bénédictions aux juges qui l'ont sauvé, et à tous les autres des
malédictions et toutes les marques de haine et de mépris, les chansons
contre eux commençant à paraître\footnote{On trouve en effet de ces
  chansons dans les recueils de la Bibliothèque impériale et de
  l'Arsenal\,; mais elles ne valent pas la peine d'être citées.}\,; et
je suis surpris qu'y ayant quinze jours passés que cette histoire est
finie, le discours n'en finit point encore, et l'on en parle par toutes
les compagnies comme le premier jour.\,»

~

{\textsc{Les assertions d'Olivier d'Ormesson ne sont pas confirmées
seulement par M\textsuperscript{me} de Sévigné, dont le témoignage
pourrait paraître suspect, mais même par Gui Patin, dont on connaît
l'esprit peu charitable, surtout à l'égard des financiers. Il n'a que
des louanges pour Olivier d'Ormesson. Il écrit à son ami Falconet}}

~

\footnote{T. III, p.~499\,; édit. Reveillé-Parise.} \,: «\,M. d'Ormesson
a dit son avis, et, après de belles choses, a conclu à un bannissement
perpétuel et à la confiscation de tous les biens.\,» Quelques jours
après, il disait dans une lettre adressée au même Falconet \footnote{\emph{Ibid}.,
  p.~501.} \,: «\,On dit que M. Fouquet est sauvé, et que, de vingt-deux
juges, il n'y en a que neuf à la mort, les treize autres au bannissement
et à la confiscation de ses biens. On en donne le premier honneur à
celui qui a parlé le premier, qui était le premier rapporteur, M.
d'Ormesson, qui est un homme d'une intégrité parfaite. »

\hypertarget{note-ii.-charles-xii.-projets-quil-avait-formuxe9s-dans-les-derniers-temps-de-son-ruxe8gne.-ses-relations-avec-le-ruxe9gent.}{%
\chapter{NOTE II. CHARLES XII. --- PROJETS QU'IL AVAIT FORMÉS DANS LES
DERNIERS TEMPS DE SON RÈGNE. --- SES RELATIONS AVEC LE
RÉGENT.}\label{note-ii.-charles-xii.-projets-quil-avait-formuxe9s-dans-les-derniers-temps-de-son-ruxe8gne.-ses-relations-avec-le-ruxe9gent.}}

Saint-Simon parle dans ce volume des projets d'alliance entre Charles
XII et Pierre le Grand, pour renverser du trône d'Angleterre la maison
de Hanovre et y replacer les Stuarts. Voltaire donne aussi quelques
détails sur ce plan dans son dernier livre de \emph{l'Histoire de
Charles XII\,;} mais ils ne disent rien des négociations que le roi de
Suède entretint avec le régent. Ce curieux complément des histoires les
plus célèbres de Charles XII se trouve dans les \emph{Mémoires inédits
du marquis d'Argenson}. Il tenait les détails qu'il donne du Suédois qui
avait servi d'intermédiaire entre Charles XII et le régent, du banquier
Hoggers ou Hogguer\,:

«\,Personne, dit-il, ne possède plus au juste les desseins du roi de
Suède que Hogguer, qui me les a contés ainsi qu'il suit\,: Charles XII
faisait la paix avec le czar, et en même temps formait avec lui une
alliance offensive et défensive, pour eux deux, s'emparer du pays, à
leur convenance dans le Nord, anéantir le pouvoir du Danemark, détrôner
Auguste\footnote{Frédéric-Auguste, roi de Pologne depuis 1697.} et
maltraiter le roi de Prusse, rétablir la liberté germanique et donner de
furieuses affaires à l'Angleterre chez elle. Il s'appuyait de l'Espagne,
où régnait alors, pour ainsi dire, Albéroni, ministre à desseins
vastes\,; il procurait à l'Espagne le recouvrement de ses anciens
domaines d'Italie, et il engageait la France, dès qu'elle voudrait, dans
ses desseins, en lui procurant les Pays-Bas\,; et par cette alliance, le
régent était sûr d'un appui bien puissant pour monter sur le trône de
France, si la succession en devenait vacante pour lui\,; car cet
appui-là était bien plus fort que celui du traité de Londres ou
quadruple alliance\footnote{Voy. sur ce traité les \emph{Mémoires de
  Saint-Simon}.}, qui n'entrait que dans un médiocre tourbillon de
desseins, en sorte que le roi Georges n'étant pas inquiété pour son
usurpation, il se souciait peu des inquiétudes qu'on ferait essuyer au
duc d'Orléans\,; et même si le roi d'Espagne savait alors opter pour la
France et abandonner l'Espagne, l'Angleterre se faisait un mérite auprès
de toute l'Europe d'assurer si bien l'équilibre général, et y sacrifiait
les intérêts de son allié le duc d'Orléans. Mais le héros du Nord,
Charles XII, homme à parole inviolable et poussant la magnanimité
jusques à la folie, aurait plutôt manqué à tout qu'à son allié. Il eût
plutôt déféré aux intérêts de la France, plus voisine de lui et plus
concourante à ses vastes desseins, que pourvu aux desseins de l'Espagne
contre le régent, d'autant que les intérêts d'Espagne de ce côté-là
n'entraient pour rien dans leurs projets communs, et qu'il rendait assez
de services à l'Espagne en lui procurant l'Italie.

«\,À l'égard du czar, celui-ci trouvoit un grand avantage à dominer
ainsi dans tout le Nord conjointement avec la Suède\,; il voyait son
empire mieux établi que la puissance suédoise\,; celle-ci ne tenant qu'à
la vie seule et au grand mérite de son roi, ne se soutiendrait pas après
lui comme la sienne. Il voyait toujours les Sarmates et les Goths se
répandre de nouveau, donner la loi comme autrefois au reste de
l'Europe\,; il aguerrissait ses troupes. Ainsi il eût marché d'un
parfait concert avec Charles XII à ces desseins\,; et quelle puissance
c'eût été, les deux extrémités de l'Europe étant jointes ensemble,
savoir Suède et Moscovie avec Espagne et France\,! Par leur position,
nul concours d'intérêt, nulle rivalité ne les eût mis en jalousie et en
défiance, et on eût été jusques au bout si la mort ne fût venue rompre
leurs desseins dès leur principe, en abattant la tète de l'auteur, qui
s'exposait aussi avec trop de prodigalité de son bonheur.

«\,Par ce projet, la Suède cédait à la Russie l'Ingrie\footnote{Aujourd'hui
  partie de la province de Saint-Pétersbourg.}, l'Estonie et la
Livonie\,; mais de cette dernière province, la Suède se réservait Riga
et dépendances. Elle cédait encore à la Russie un canton de Finlande. La
Suède faisait la conquête entière de la Norvège sur le Danemark, et cela
était déjà bien avancé quand Charles XII fut tué\,; ensuite Charles XII
tombait en Danemark et abolissait le droit du Sund \footnote{C'est-à-dire
  le droit que l'on prélevait sur les navires qui traversaient le Sund.}.
Pour en fermer le passage et obvier aux secours des Anglais, le czar
mettait sur pied une flotte formidable, qui se combinait avec celle de
Suède, alors sur un bon pied. On conquérait sur la Pologne, à frais
communs, une petite province fort à la convenance de la Russie. On
donnait à la Suède la Poméranie et le Mecklembourg. On dédommageait le
duc de Mecklembourg, alors en querelle avec ses sujets, comme il y est
resté depuis\,; on lui donnait une province qu'on prenait sur la Prusse.
On attaquait le roi de Prusse pour le punir de s'être mêlé, comme il
avait fait, de la précédente guerre de Pologne. On lui montrait que
toutes ses belles troupes\footnote{On sait que Frédéric-Guillaume Ier,
  alors roi de Prusse, s'attachait à organiser des régiments dont les
  hommes étaient remarquables par leur haute taille.} n'étaient
composées que de faquins. Et qui est-ce qui eût pu ni voulu le
secourir\,? On le privait, comme j'ai dit, de ce qu'on donnait en
indemnité au duc de Mecklembourg, et de quelques postes à la convenance
de la Russie. De là on entrait en Saxe et en Pologne\,; on détrônait une
seconde fois le roi Auguste pour replacer le roi Stanislas\footnote{Stanislas
  Leczinski.} sur le trône de Pologne. On ôtait encore au roi Auguste
son électorat de Saxe, et on y mettait la branche aînée de Saxe-Gotha.

«\,Le traité était déjà signé avec l'Espagne par les travaux qu'y avait
faits le cardinal Albéroni\,: l'Espagne envoyait vingt vaisseaux de
guerre au Sund, pour se joindre à ceux de Russie et de Suède et prévenir
les Anglais. L'Espagne fournissait cinq cent mille piastres par mois.

«\,De Danemark, Charles XII descendait à Hambourg, obtenait aisément de
cette riche république de gros secours en argent, et la déchargeait de
toute tyrannie du Danemark. Bientôt le Danemark, pris de tous côtés,
demandait grâce, et on lui accordait une paix dont on était bien sûr de
la durée\footnote{Nous laissons, au texte de d'Argenson, comme à celui
  de Saint-Simon, les irrégularités grammaticales que d'autres éditeurs
  ont cru devoir rectifier.}.

«\,Charles XII, avec six mille braves Suédois, gens fort aguerris et
enflés de leurs anciennes victoires, descendaient Allemagne, tandis que
le czar agissait aussi avec une armée formidable dans cette même partie
de l'Europe, où il a à cœur d'avoir pied. Là on agissait offensivement
contre l'électeur de Hanovre, qui est aussi roi d'Angleterre. On faisait
venir alors le Prétendant\footnote{Jacques Stuart qui prenait le nom de
  Jacques III.} en Angleterre, et on le rétablissait\,; ce qui donnait
trop d'ouvrage audit électeur de Hanovre pour lui laisser le temps de se
mêler des affaires d'Allemagne. Pour lors on faisait la loi à
l'empereur, à qui on donnait les affaires que je vais dire\,: on faisait
éclore les liaisons prises avec l'électeur de Bavière, la maison
palatine et les électeurs ecclésiastiques\,; on recueillait toutes leurs
prétentions et les griefs du corps germanique, sans augmenter aucunes
jalousies entre les catholiques et les protestants, et on renouvelait le
traité de Westphalie pour la liberté germanique. Les Turcs étaient déjà
en guerre avec l'empereur\,; on animait cette guerre, et on faisait du
prince Ragotsky un roi de Hongrie et de Transylvanie. En même temps
l'Espagne descendait en Italie et y reprenait le Milanais et les
Deux-Siciles, ce qui, comme je l'ai dit, donnait assez d'ouvrage à
l'empereur tout à la fois.

«\,C'était alors l'occasion à la France de paraître ayant armé
puissamment jusque-là sans se déclarer\,; et pour lui donner part au
gâteau et à la dépouille universelle de l'empereur, on nous donnait les
dix provinces des Pays-Bas catholiques\footnote{Ces provinces qui
  répondent à peu près au royaume de Belgique actuel, sont\,: le
  Hainaut, la Flandre occidentale, la Flandre orientale, le Brabant
  méridional, le Brabant septentrional, les provinces d'Anvers, de
  Namur, de Liège, de Luxembourg, de Limbourg.}\,; ce qui remplirait
notre beau dessein de n'avoir au nord et au nord-est que le Rhin pour
barrière.

«\,La puissance de cette ligue et l'affaiblissement total de l'empereur
nous vengeait assez de nos pertes précédentes par le traité d'Utrecht.
L'Angleterre, si occupée par le Prétendant et la Hollande, sans
l'Allemagne et sans l'empereur, n'osait nous traverser\,; et de plus on
garantissait à M. le duc d'Orléans la future succession de France, si
elle venait à s'ouvrir, et cela par un traité particulier entre elle, la
Suède et le czar, sans en avoir rien communiqué avec l'Espagne.

«\,Charles XII, semblable et surpassant le grand Gustave-Adolphe, au
milieu de l'Allemagne avec soixante mille hommes, y faisait la loi, et
tirait de grandes richesses pour soutenir la guerre de Jutland,
Hambourg, Saxe, Prusse et du reste de l'Allemagne. Il réglait en même
temps la future succession de l'empereur entre ses héritiers naturels.

«\,Alors il y avait à Paris un grand seigneur d'Espagne, appelé don
Manuel, envoyé par Albéroni comme simple voyageur, mais pour s'aboucher
avec le sieur Hogguer, dépositaire de tous ces secrets. Ils
s'assemblèrent tous les soirs ensemble chez M\textsuperscript{lle}
Desmares, illustre comédienne et maîtresse d'Hogguer. Ils soupaient
ensemble\,; mais avant souper et pendant la comédie, ils s'enfermaient
ensemble, travaillaient sur des cartes géographiques et écrivaient
beaucoup.

«\,Cependant le baron de Goertz, pour donner de la jalousie et piquer la
curiosité de M. le duc d'Orléans, avait fait cette manœuvre-ci il avait
fait écrire la partie la moins importante et la moins secrète de ces
projets partie en chiffres, en sorte que cette dépêche était tombée
entre les mains de notre résident à Berlin, lequel n'avait pas manqué de
l'envoyer d'abord à M. le duc d'Orléans. On y voyait bien que don Manuel
était à Paris pour cela de la part d'Albéroni, mais on y trouvait qu'il
correspondait pour cela avec un Suédois nommé Sobrissel. On faisait de
grandes perquisitions pour découvrir où était ce Sobrissel à Paris, et
on ne trouvait rien\,; on savait seulement qu'il était fils d'un
sénateur de Suède. Mais ce nom de Sobrissel couvrait celui d'Hogguer,
qui était désigné par là. Mon père, alors garde des sceaux de France,
avait conservé des émissaires de la police\,; il avait mis plus de cent
personnes à cette découverte, et on ne trouvait rien, comme je dis.

«\,Alors M. le duc d'Orléans manda Hogguer pour le savoir. Celui-ci,
fidèle à la France, songea d'abord à la bien servir, mais en ne
trahissant point la cause étrangère dont il était chargé. Il savait que
le régent devait y être admis à de bonnes conditions et à propos, et le
temps en était venu par l'inquiétude et la jalousie dont il était piqué.
Il est vrai qu'il ne pouvait être admis qu'avec dépit de la part de
l'Espagne, qui avait ses intérêts particuliers contre lui\,; mais la
Suède n'était là dedans que pour favoriser le régent, et ce fut cette
admission qui chagrina l'émissaire d'Albéroni, comme je vais dire,
s'imaginant que Hogguer le trahissait totalement après lui avoir fait
signer le traité.

«\,Le régent s'était donné de grands mouvements du côté de Suède, de
Parme et de Madrid, et l'abbé Dubois ne venait à bout de rien sur la
découverte des grands projets qui transpiraient du roi de Suède et
d'Albéroni. Le régent manda donc la Desmares, et l'interrogea sur le
comportement d'Hogguer et de don Manuel, qu'il savait souper chez elle
tous les soirs. Elle lui dit tout ce qu'elle savait, et lui envoya
Hogguer. Celui-ci fit bientôt ses ouvertures au régent, et il lui apprit
{[}que Sobrissel{]} n'était autre chose que lui Hogguer\,; qu'il était
le confident de tout, et qu'il ne tenait qu'à lui régent d'entrer dans
l'alliance. Il lui montra ses pleins pouvoirs, où il y avait carte
blanche sur cela. Le régent se défiait cependant d'Albéroni, et qu'il
n'y eût là dedans quelque article contre lui. Il voulut avant toutes
choses gagner don Manuel\,; il chargea Hogguer de lui offrir la plus
forte récompense s'il voulait quitter l'Espagne et s'attacher à la
France, savoir\,: un million d'argent comptant, une belle terre, le
cordon bleu, le grade de lieutenant général et un gouvernement.

«\,Hogguer s'acquitta de cette négociation en homme d'esprit et
adroit\,; mais il ne put si bien faire que don Manuel ne crût d'abord
qu'il était trahi par Hogguer. Il s'emporta contre lui extrêmement\,; le
lendemain il l'envoya chercher\,; il lui parla avec douceur, lui demanda
même pardon de tout ce qu'il lui avait dit la veille\,; il ajouta qu'il
voyait bien cependant qu'il avait perdu en un moment le fruit, du côté
de l'Espagne, de tous ses travaux\,; qu'il avait le coeur serré\,; qu'il
n'avait plus vingt-quatre heures à vivre, et que pour rien au monde il
ne trahirait sa patrie. En effet, don Manuel tomba dans une grosse
fièvre\,; on lui envoya Chirac\footnote{Chirurgien célèbre dont il est
  souvent question dans les \emph{Mémoires de Saint-Simon}.}\,; il
mourut la nuit suivante.

«\,Albéroni chargea de la suite de cette affaire le marquis
Monti\footnote{Voy. sur ce personnage les t. XV et XVII de Saint-Simon.
  Il y parle du rôle que joua Monti à l'occasion de l'élection du roi de
  Pologne.}, que nous avons gagné depuis, et qui a joué un grand rôle
pour nous à l'élection du roi Stanislas en 1733, et décédé en 1737\,;
mais il n'eut pas tout le secret de cette affaire comme don Manuel.

«\,Le régent continua à perfectionner cette négociation avec Hogguer.
Voyant les pleins pouvoirs qu'il avait de la Suède, il était charmé
d'être si bien tiré d'une intrigue qui lui faisait tarit de peur pour
ses propres intérêts. Il offrit d'abord cinq cent mille écus par mois à
la Suède. Hogguer stipula de conclure sans l'abbé Dubois, puisque par là
le traité de quadruple alliance allait au diable, et qu'on soupçonnait
justement ledit abbé Dubois d'être pensionné par
l'Angleterre\footnote{Saint-Simon l'affirme positivement dans plusieurs
  passages de ses \emph{Mémoires}.}.

«\,Tout étant d'accord entre le régent et Hogguer, le régent manda
l'abbé Dubois, et, en présence d'Hogguer, il le traita de coquin et de
cuistre. \emph{Voilà donc}, dit-il, \emph{quels sont vos travaux pour
découvrir la chose la plus capitale qu'il y eût alors en Europe. J'en ai
plus fait en un quart d'heure avec cet homme, et ici, que vous dans
toute l'Europe en six mois, et votre Angleterre, et le diable qui vous
emporte}.

~

{\textsc{«\,Il fut question de savoir qui on enverrait en Suède pour
ratifier et achever les détails de conclusion. Le régent voulut que ce
fût Hogguer, et qu'il partît la nuit même s'il se pouvait, ou la nuit
d'après. Hogguer demanda des instructions\,; l'abbé Dubois dit qu'il n'y
avait personne qui pût mieux les dresser qu'Hogguer lui-même. Celui-ci y
travailla toute la nuit\,; on les expédia et on les signa sur-le-champ,
et il allait partir, lorsqu'on reçut un courrier de Dunkerque qui apprit
la mort du roi de Suède\,; ce qui finit à l'instant toute l'aventure et
tous ces vastes projets.\,»}}

~

\hypertarget{note-iii.-assembluxe9e-de-la-noblesse-en-1649.}{%
\chapter{NOTE III. ASSEMBLÉE DE LA NOBLESSE EN
1649.}\label{note-iii.-assembluxe9e-de-la-noblesse-en-1649.}}

On peut comparer avec cette partie des \emph{Mémoires de Saint-Simon} la
plupart des ouvrages relatifs à la Fronde, et spécialement les
\emph{Mémoires d'Omer Talon}, à l'année 1649. On y trouvera les actes de
cette assemblée de la noblesse. Il y a quelque différence avec les
signatures que donne Saint-Simon **. Dans Omer Talon, un, des
signataires s'appelle d'Alluye\,; Saint-Simon écrit Halluyes-Schomberg,
et croit qu'il s'agit du duc d'Halluyn-Schomberg. Il est très probable
que la signature donnée par Omer Talon est la véritable, et qu'il s'agit
ici du marquis d'Alluye, fils du marquis de Sourdis. On lit en effet,
dans un journal inédit de l'époque de la Fronde\,: «\,Mardi 5 octobre
{[}1649{]}, encore assemblée de la noblesse opposante chez le marquis de
Sourdis, lui absent, et son fils, le marquis d'Alluye, présent.\,» On a
cité (t. V, p. 438-441 des \emph{Mémoires de Saint-Simon}) la totalité
du passage du \emph{Journal de Dubuisson-Aubenay}. Il en résulte que le,
marquis d'Alluye joua un rôle important dans ces assemblées, et que
c'est très probablement lui qui a signé l'acte de la noblesse.

\hypertarget{note-iv.-pays-ou-provinces-duxe9tats.}{%
\chapter{NOTE IV. PAYS OU PROVINCES
D'ÉTATS.}\label{note-iv.-pays-ou-provinces-duxe9tats.}}

Les pays d'états, ou provinces d'états, étaient ceux qui jouissaient du
privilège d'avoir une assemblée provinciale. Ils se réduisaient depuis
le règne de Louis XIV, au Languedoc, à là Bretagne, à la Bourgogne, à la
Provence, au Hainaut et au Cambrésis (Flandre française), au comté de
Pau (Béarn), au Bigorre, comté de Foix, pays de Gex, Bresse, Bugey,
Valromey, Marsan, Nébouzan, Quatre-Vallées (Armagnac), Soulac et Terre
de Labourd. Les états de Dauphiné, supprimés sous Louis XIII, ne furent
rétablis que peu de temps avant la Révolution. Les pays d'états votaient
eux-mêmes l'impôt qu'ils payaient à la couronne, et qu'on appelait
\emph{don gratuit\,;} ils en faisaient la répartition. La quotité de cet
impôt était le principal sujet du débat dans les états provinciaux, et
l'affaire la plus importante pour les commissaires qui représentaient le
gouvernement. Les états devaient aussi pourvoir aux autres dépenses
provinciales, parmi lesquelles figuraient les frais mêmes qu'entraînait
la session des états, et les gratifications votées aux gouverneurs,
intendants et principaux fonctionnaires de la province. Le don gratuit
n'avait rien d'uniforme\,; il variait de province à province, et, dans
la même province, d'année en année, suivant les besoins du gouvernement
et les ressources du pays.

Si l'on veut étudier les avantages et les inconvénients de ces pays
d'états, il faut surtout consulter la \emph{Correspondance
administrative sous Louis XIV}, publiée dans la collection des
\emph{Documents inédits relatifs à l'histoire de France}. On y suit les
efforts tentés par Louis XIV pour obtenir le concours des états
provinciaux et les soumettre à ses volontés. Dès le commencement de son
gouvernement personnel (le 21 octobre 1661), ce roi écrivait à M. de
Fieubet, premier président du parlement de Toulouse\footnote{Cette
  lettre ne se trouve pas dans les \emph{Oeuvres de Louis XIV}. Je l'ai
  copiée dans le recueil de Rose (ms. de l'Arsenal, n° 199, n° 127-128).}\,:
«\,Dans l'application que je donne à toutes mes affaires généralement,
sans en négliger aucune, je serai bien aise de savoir le nom du
capitoul\footnote{Magistrat municipal de Toulouse.} qui sera député aux
prochains états de ma province de Languedoc, et même ses intentions à
l'égard de mes intérêts. Vous me ferez donc plaisir de m'en informer au
plus tôt\,; et comme vous pouvez beaucoup dans cette députation, il sera
bon de vous prévaloir du crédit que vous y avez pour prendre des
précautions avec ledit capitoul, afin que non seulement il ne se rende
pas chef des avis qui me seront préjudiciables, comme tous ses
prédécesseurs ont fait, mais aussi afin qu'il se joigne aux bien
intentionnés pour favoriser les choses qui seront proposées de ma part.
J'approuve dès à présent tout ce que vous ferez pour cet effet, vous
assurant au surplus que le secret vous sera gardé, et que vous ne me
sauriez rendre un service plus agréable.\,»

Les évêques de Lavaur, d'Albi, de Saint-Papoul et de Viviers, reçurent,
ainsi que l'archevêque de Toulouse, des lettres pressantes pour se
rendre aux états et soutenir le commissaire de Louis XIV\footnote{Même
  ms., n° 160.} . Le zèle des prélats et des principaux membres des
états enleva un vote unanime\footnote{\emph{Correspondance
  administrative}, t. I, p.~54 et 64.}. Dans la suite, l'assemblée
devint de plus en plus docile aux volontés du roi\footnote{\emph{Ibid}.,
  p. 288, 289, 290, 308, 316.}. La Provence fut intimidée par quelques
exils\footnote{\emph{Ibid}., p.~399.} et se montra aussi docile que le
Languedoc\footnote{\emph{Ibid}., p.~403, 405.}\,; il en fut de même en
Bourgogne \footnote{\emph{Ibid}., p.~445-446.}. La Bretagne paraissait
plus obstinée\,; mais elle finit par céder\footnote{\emph{Ibid}., p.~498
  et 500.}. Les plaintes de M\textsuperscript{me} de Sévigné sur le sort
de la Bretagne, jadis «\, toute libre, toute conservée dans ses
prérogatives, aussi considérable par sa grandeur que par situation,\,»
attestent qu'on ne tenait plus compte «\,du contrat de mariage de la
grande héritière\footnote{Lettres de M\textsuperscript{me} de Sévigné du
  6 novembre 1689 et du 18 janvier 1690.}.\,» Les états des petites
provinces n'auraient pu tenter une résistance qui avait été si
facilement vaincue en Languedoc, en Provence, en Bourgogne et en
Bretagne. Colbert songeait à les supprimer, et à faire vivre tous ces
pays sous une loi commune\footnote{\emph{Correspondance administrative
  sous Louis XIV}, t. I, p.~112.}\,; mais il céda aux remontrances de
l'évêque de Tarbes, qui lui représentait que ce changement ne pouvait
«\,rencontrer qu'un consentement forcé de tous ces peuples, qui
regardaient la grande puissance du roi et Sa Majesté armée auprès d'eux,
et ne ressentiraient pas moins la perte de leur liberté et de tant de
glorieuses marques de leurs services que les rois prédécesseurs de Sa
Majesté leur avoient laissées de règne en règne.\,»\footnote{\emph{Ibid}.}

Les pays d'états continuèrent d'exister jusqu'à l'époque de la
Révolution.

\hypertarget{note-v.-tiers-uxe9tat-aux-uxe9tats-guxe9nuxe9raux-de-1302.}{%
\chapter{NOTE V. TIERS ÉTAT AUX ÉTATS GÉNÉRAUX DE
1302.}\label{note-v.-tiers-uxe9tat-aux-uxe9tats-guxe9nuxe9raux-de-1302.}}

Le tiers état, comme on l'a dit, figura aux états généraux de 1302, sous
le règne de Philippe le Bel. On peut citer, entre autres preuves, une
pièce intitulée\,: \emph{La supplication du peuple de France au roi
contre le pape Boniface VIII}\footnote{Voy. Duboulay, \emph{Hist. de
  l'Université de Paris}, t. IV, p.~15\,; P. du Puy, \emph{Différend de
  Philippe le Bel et de Boniface VIII}, p.~214\,:, \emph{Preuves des
  libertés de l'Église gallicane}, t. I. p.~108.}**. Le tiers état
s'adresse au roi comme un corps constitué, et lui demande de défendre
l'indépendance de la couronne de France. Voici quelques extraits de
cette pièce, dont je modifie l'orthographe pour la rendre plus
intelligible\,:

«\,A vous, très noble prince notre sire, par la grâce de Dieu roi de
France, supplie et requiert le peuple de votre royaume, pour ce qu'il
lui appartient que ce soit fait, que vous gardiez la souveraine
franchise de votre royaume, qui est telle que vous ne reconnaissiez de
votre temporel souverain en terre hors Dieu, et que vous fassiez
déclarer, si (de telle sorte) que tout le monde le sache que le pape
Boniface erra manifestement, et fit péché mortel notoirement, en vous
mandant par lettres bullées qu'il était souverain de votre temporel, et
que vous ne pouvez prébendes donner ni les fruits des églises
cathédrales vacants retenir, et que tous ceux qui croient le contraire
il les tient pour héréges (hérétiques)\,; \emph{item}, que vous fassiez
déclarer que l'on doit tenir ledit pape pour hérége, pour ce qu'il ne
veut cette erreur rappeler (abandonner), etc. Ce fut grande abomination
à ouïr que ce Boniface, pour ce que Dieu dit à saint Pierre\,: «\,Ce que
tu lieras en terre sera lié au ciel,\,» cette parole, dite
spirituellement, entendit mallement quant au temporel.

«\,Et pour que aucun autre ne prenne exemple à faire ainsi, et pour ce
que la peine de lui fasse peur aux autres, vous, noble, roi sur tous
autres princes, défenseur de la foi, pouvez et devez, et êtes tenu
requérir et procurer que ledit Boniface soit tenu et jugé pour hérége,
et puni en la manière que l'on le pourra et devra, si (de telle sorte)
que votre souveraine franchise soit gardée.\,»

\end{document}
